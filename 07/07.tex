\documentclass[a4paper,graphics,11pt]{article}
\pagenumbering{arabic}

\usepackage[margin=1in]{geometry}
\usepackage[utf8]{inputenc}
\usepackage[T1]{fontenc}
\usepackage{lmodern}
\usepackage[ngerman]{babel}
\usepackage{amsmath, tabu}
\usepackage{amsthm}
\usepackage{amssymb}
\usepackage{complexity}
\usepackage{mathtools}
\usepackage{setspace}
\usepackage{graphicx,color,curves,epsf,float,rotating}
\usepackage{tasks}
\setlength{\parindent}{0em}
\setlength{\parskip}{1em}

\newcommand{\aufgabe}[1]{\subsection*{Aufgabe #1}}
\newcommand{\up}[2]{\mathrel{\overset{\makebox[0pt]{\mbox{\normalfont\tiny #2}}}{#1}}}

\begin{document}
\noindent Gruppe \fbox{\textbf{11}}             \hfill Tobias Riedel, 379133 \\
\noindent Analysis für Informatiker             \hfill Phil Pützstück, 377247 \\

\begin{center}
	\LARGE{\textbf{Hausaufgabe 7}}
\end{center}
\begin{center}
\rule[0.1ex]{\textwidth}{1pt}
\end{center}



\aufgabe{1}
\textbf{a)}\\[5pt]
Die Folge der Partialsummen lässt sich wie folgt darstellen:
$$
    (s_n)_{n\in \mathbb{N}} := \sum_{k=1}^{n} a_k
$$
\textbf{b)}\\[5pt]
Wir zeigen, dass die Folge $s_n$ der Partialsummen monoton steigt.
Man betrachte $s_{n+1}-s_n\colon$
$$
    s_{n+1} - s_n = \left(\sum_{k=1}^{n+1} a_k\right) - \left(\sum_{k=1}^{n} a_k\right)
    = a_{k+1} + \left(\sum_{k=1}^{n} a_k\right) -\left(\sum_{k=1}^{n} a_k\right)
    = a_{k+1}
$$
Wir wissen weiterhin, dass $a_k \geq 0$ für alle $k \in \mathbb{N}$. Beweis (Induktion):\\
\textbf{(IA)} $k=1$. Es gilt 
$$
    \frac{3^1}{5^1+1} = \frac{3}{5} \geq 0
$$
Also gilt die Aussage für $k=1$.

\textbf{(IS)} Die Behauptung gelte für ein $k \in \mathbb{N}$. $k\mapsto k+1\colon$
$$
    \frac{3^{k+1}}{5^{k+1}+1} = \frac{3^k\cdot3}{5^k\cdot5+1}
    = \frac{3}{5}\cdot \frac{3^k}{5^k+\frac{1}{5}}
    \geq \frac{3}{5}\cdot \frac{3^k}{5^k+1}
$$
Wir wissen, dass $\displaystyle\frac{3^k}{5^k+1}\geq 0$, also folgt auch
$\displaystyle\frac{3}{5} \cdot \frac{3^k}{5^k+1} \geq 0$.
Insgesamt gilt also die Behauptung auch für $k+1$.

Damit ist $a_k$ stets größer 0 und es folgt $s_{n+1}-s_n = a_{k+1} \geq 0$. Also
ist $s_n$ monoton wachsend.

\textbf{c)}\\[5pt]
Es gilt:
$$
    \forall k \in \mathbb{N} \colon a_k = \frac{3^k}{5^k+1} \leq \frac{3^k}{5^k} \leq \frac{1}{5^k}
    = \left(\frac{1}{5}\right)^k
$$
\newpage
\textbf{d)}\\[5pt]
Da $\frac{1}{5}\neq1$ lässt sich die geometrische Summenformel (II Satz 3.5)
wie folgt einsetzen:
$$
    \forall n \in \mathbb{N} \colon s_n = \sum_{k=1}^{n} a_k
    \leq \sum_{k=1}^{n} \left(\frac{1}{5}\right)^k
    = \frac{1-\left(\frac{1}{5}\right)^{n+1}}{1-\frac{1}{5}}
$$
Sei also $(c_n)_{n \in \mathbb{N}} := \sum_{k=1}^{n} \left(\frac{1}{5}\right)^k$.
Durch $\forall k \in \mathbb{N} \colon a_k \geq 0$ folgt $|a_k| = a_k$.\\
Weiterhin gilt $\forall k \in N\colon a_k = |a_k| \leq c_k$. Da $c_k$ nach
Satz 3.5 (geometrische Reihe) konvergiert, folgt nach dem Minorantenkriterium (3.17), dass
auch $a_k$ konvergiert.
\aufgabe{2}
Wir wenden Partialbruchzerlegung auf den gegebenen Bruch an:
$$
    \frac{1}{(2k+1)(2k+5)} = \frac{a}{(2k+1)} + \frac{b}{(2k+5)} \,\Longleftrightarrow\,
    1 = a(2k+5) + b(2k+1)
$$
Nun stellen wir nach $k$ um:
$$
    1 = a(2k+5) + b(2k+1) = 2ak+5a+2bk+b = k(2a+2b) + (5a+b)
$$
Somit haben wir nach Koeffizientenvergleich zwei Gleichungen: $ \mathbb{I}:=(2a+2b = 0)$
und \\$\mathbb{II}:=(5a+b = 1)$. Wir lösen dies durch addieren der Gleichungen:
$$
    \mathbb{I} + \mathbb{II}\cdot(-2) \,\Longrightarrow\, 2a-10a+2b-2b=0-2
    \,\Longleftrightarrow\, -8a = -2
    \,\Longleftrightarrow\, a = \frac{1}{4}
$$
Nun lässt sich $a = \frac{1}{4}$ in die andere Gleichung einsetzen:
$$
    5\cdot \frac{1}{4}+b = 1 \,\Longleftrightarrow\, b = -\frac{1}{4}
$$
Also gilt nach Prinzip der Partialbruchzerlegung nun
$$
    \frac{1}{(2k+1)(2k+5)} = \frac{1}{4(2k+1)} - \frac{1}{4(2k+5)}
$$
Ebenso gilt also
$$
    \sum_{k=2}^{\infty} \frac{1}{(2k+1)(2k+5)}
    \ =\ \sum_{k=2}^{\infty} \frac{1}{4(2k+1)} - \frac{1}{4(2k+5)}
    \ =\ \sum_{k=2}^{\infty} \frac{1}{4(2k+1)} - \sum_{k=2}^{\infty} \frac{1}{4(2k+5)}
$$
Durch Indizienverschiebung erhalten wir
$$
    \sum_{k=2}^{\infty} \frac{1}{4(2k+1)} - \sum_{k=2}^{\infty} \frac{1}{4(2k+5)}
    \ =\ \sum_{k=0}^{\infty} \frac{1}{4(2k+5)} - \sum_{k=2}^{\infty} \frac{1}{4(2k+5)}
$$
Dies entspricht einer teleskopischen Summe, also folgt:
$$
    \sum_{k=0}^{\infty} \frac{1}{4(2k+5)} - \sum_{k=2}^{\infty} \frac{1}{4(2k+5)}
    \ =\ \sum_{k=0}^{1} \frac{1}{4(2k+5)} +
    \sum_{k=2}^{\infty} \frac{1}{4(2k+5)} - \sum_{k=2}^{\infty} \frac{1}{4(2k+5)}\\
$$$$
    \ =\ \sum_{k=0}^{1} \frac{1}{4(2k+5)} = \frac{1}{4\cdot 5} + \frac{1}{4\cdot 7} 
    \ =\ \frac{1}{20} + \frac{1}{28} = \frac{3}{35}
$$
Insgesamt folgt also
$$
    \sum_{k=2}^{\infty} \frac{1}{(2k+1)(2k+5)} = \frac{3}{35}
$$
Also konvergiert die Reihe
$$
    (s_n)_{n\geq 2} := \sum_{k=2}^{\infty} \frac{1}{(2k+1)(2k+5)}\quad\text{mit}\quad
    \lim_{n \to \infty} s_n = \frac{3}{35}
$$
\aufgabe{3}
\textbf{a)}\\[5pt]
Sei $(a_k)_{k \in \mathbb{N}} := (-1)^{k-1}\dfrac{2^k}{k^k}$. Wir betrachten

$$
    \left| \frac{a_{k+1}}{a_k}\right|
    = \left|\frac{\left(\dfrac{(-1)^k\cdot2^{k+1}}{(k+1)^{k+1}}\right)}{\left(\dfrac{(-1)^{k-1}\cdot 2^k}{k^k}\right)}\right|
    = \frac{\left(\dfrac{2^{k+1}}{(k+1)^{k+1}}\right)}{\left(\dfrac{2^k}{k^k}\right)}
    = \frac{2^{k+1}\cdot k^k}{(k+1)^{k+1} \cdot 2^k} = \frac{2k^k}{(k+1)^{k+1}}
$$$$
    =\frac{2}{k+1} \cdot \left(\frac{k}{k+1}\right)^k \quad\up{=}{k\geq 0}\quad
    \frac{2}{k+1} \cdot \frac{1}{\left(\frac{k+1}{k}\right)^k}
    = \frac{2}{k+1} \cdot \frac{1}{\left(1+\frac{1}{k}\right)^k}
$$

Wir wissen aus Hausaufgabe 5, dass
$$
    \lim_{k \to \infty} \left(1+\frac{x}{k}\right)^k = e^x\quad \text{also auch}\quad
    \lim_{k \to \infty} \frac{1}{\left(1+\frac{1}{k}\right)^k} = \frac{1}{e}
$$
Somit lassen sich die Grenzwertsätze wie folgt anwenden:
$$
    \lim_{k \to \infty} \left |\frac{a_{k+1}}{a_k}\right|
    = \lim_{k \to \infty} \frac{2}{k+1} \cdot \frac{1}{\left(1+\frac{1}{k}\right)^k}
    = \lim_{k \to \infty} \frac{2}{k+1} \cdot \lim_{k \to \infty} \frac{1}{\left(1+\frac{1}{k} \right)^k}
    = 0 \cdot \frac{1}{e} = 0
$$
Damit gilt nun $\lim_{k \to \infty}\limits \left|\frac{a_{k+1}}{a_k} \right| < 1$ und nach dem
Quotientenkriterium konvergiert die gegebene Reihe $\sum_{k=0}^{\infty} a_k$ absolut.
Offensichtlich konvergiert die Reihe dann auch (3.14).

\textbf{b)}\\[5pt]
Wir wissen, dass $(a_n)_{n\in \mathbb{N}} = \frac{1}{\sqrt{n}}$ eine monoton fallende und
reelle Nullfolge ist. Nach dem Leibniz-Kriterium (Satz 3.12) folgt dann, dass
$\sum_{k=1}^{\infty} (-1)^ka_k = \sum_{k=1}^{\infty} \frac{(-1)^k}{\sqrt{k}}$ konvergiert.
Weiterhin gilt
$$
    \sum_{k=1}^{\infty} \left|\frac{(-1)^k}{\sqrt{k}}\right|
    = \sum_{k=1}^{\infty} \frac{1}{\sqrt{k}}
$$
Hinzukommend gilt $\forall k \geq 1\colon \sqrt{k}\leq k \,\Longleftrightarrow\, \frac{1}{\sqrt{k}}\geq \frac{1}{k}$.
Somit lässt sich durch das\\
Majorantenkriterium (Satz 3.17) folgendes schließen:

Sei $\sum_{k=1}^{\infty} (d_k)_{k\in \mathbb{N}}= \sum_{k=1}^{\infty} \frac{1}{k}$
die harmonische Reihe. Wir wissen nach Satz 3.6, dass diese\\
bestimmt gegen $\infty$ divergiert.
Weiterhin gilt:
$$
    \forall k \in \mathbb{N}\colon |a_k| = \frac{1}{\sqrt{k}} \geq \frac{1}{k} = d_k \geq 0
$$
Somit divergiert die Reihe $\sum_{k=1}^{\infty}\left|\frac{(-1)^k}{\sqrt{k}}\right|$
ebenfalls. Also ist die Reihe $\sum_{k=1}^{\infty} \frac{(-1)^k}{\sqrt{k}}$ konvergent, aber nicht
absolut konvergent.

\textbf{c)}\\[5pt]
Es sei $\displaystyle(a_k)_{k \in \mathbb{N}} := \frac{k^2}{2^k}$. Es gilt:
$$
    \left|\frac{a^{k+1}}{a_k}\right|
    = \left|\frac{\dfrac{(k+1)^2}{2^{k+1}}}{{\dfrac{k^2}{2^k}}}\right|
    = \left|\frac{(k+1)^2 \cdot 2^k}{2^{k+1}\cdot k^2}\right|
    = \frac{(k+1)^2}{2\cdot k^2}
    = \frac{k^2+2k+1}{2k^2}
    = \frac{1+\dfrac{2}{k}+\dfrac{1}{k^2}}{2}
$$
Wir betrachten nun $\displaystyle\lim_{k \to \infty}\limits \left|\frac{a_{k+1}}{a_k}\right|
= \frac{1+\frac{2}{k}+\frac{1}{k^2}}{2}$. Da 1 und 2 Konstanten sind und
$x_k=\frac{2}{k}$ und $y_k=\frac{2}{k^2}$ Nullfolgen sind (Bsp. 1.11), lassen sich
die Limitenregeln wie folgt anwenden:
$$
    \lim_{k \to \infty} \frac{1+\frac{2}{k}+\frac{1}{k^2}}{2}
    = \frac{\lim_{k\to\infty}\limits1+\frac{2}{k}+\frac{1}{k^2}}{\lim_{k \to \infty}\limits 2}
    = \frac{\lim_{k \to \infty}\limits 1 + \lim_{k \to \infty}\limits \frac{2}{k}+ \lim_{k \to \infty}\limits \frac{1}{k^2}}{2}
    = \frac{1+0+0}{2} = \frac{1}{2}
$$
Daher gilt das Quotientenkriterium für die gegebene Reihe:
$$
    \lim_{k \to \infty} \left|\frac{a_{k+1}}{a_k}\right| = \frac{1}{2} < 1
$$
und somit ist die Reihe $\displaystyle\sum_{k=1}^{\infty} a_k = \sum_{k=1}^{\infty} \frac{k^2}{2^k}$
absolut konvergent. Offensichtlich ist die Reihe dann auch konvergent (Satz 3.16).
\newpage
\textbf{d)}\\[5pt]
\newpage
\aufgabe{4}
Da $\sum_{k=1}^{\infty} a_k$ absolut konvergiert, muss nach Definition auch
$\sum_{k=1}^{\infty} |a_k|$ konvergieren.\\
Nach Korollar 3.9 ist $|a_k|$ also eine Nullfolge.
Also existiert ein $N \in \mathbb{N}$ sodass gilt:
$$
    \forall k > N \colon |a_k| < 1
$$
Weiterhin gilt $x^2 < x$ für $|x| < 1$, also ist $|a_k|$ eine Majorante und es gibt ein
$N \in \mathbb{N}$ sodass das Majorantenkriterium für ${a_k}^2$ gilt:
$$
    \forall k > N \colon |{a_k}^2| \leq |a_k|
$$
Damit gilt nach dem Majorantenkriterium, dass die Reihe $\sum_{k=1}^{\infty} {a_k}^2$
absolut konvergiert und damit auch konvergiert.
\aufgabe{5}
$|e^x - s_n(x)|$ lässt sich wie folgt 
nach oben abschätzen. Wir benutzen den Satz 3.5 der geometrischen Reihe (G) und
dass $x\in \left(-\frac{1}{2},\frac{1}{2}\right)$ gilt (X):
$$
    |e^x-s_n(x)|
    \ =\ \left|\sum_{k=0}^{\infty} \frac{x^k}{k!} - \sum_{k=0}^{n} \frac{x^k}{k!}\right|
    \ =\ \left|\sum_{k=n+1}^{\infty} \frac{x^k}{k!}\right|
    \ <\ \left|\sum_{k=n+1}^{\infty} \frac{x^k}{(n+1)!}\right|
$$$$
    \ =\ \frac{1}{(n+1)!} \left|\sum_{k=n+1}^{\infty} x^k\right|
    \ \leq\ \frac{1}{(n+1)!} \sum_{k=n+1}^{\infty} |x^k|
    \up{\ <\ }{(X)}\frac{1}{(n+1)!} \sum_{k=n+1}^{\infty} \left(\frac{1}{2}\right)^k
$$$$
    \ <\ \frac{1}{(n+1)!} \sum_{k=0}^{\infty} \left(\frac{1}{2}\right)^k
    \up{\ =\ }{(G)}\frac{1}{(n+1)!}\cdot \frac{1}{1-\frac{1}{2}}
    \ =\ \frac{2}{(n+1)!}
$$
Da $e^x$ streng monoton wachsend ist (Satz 3.21) gilt außerdem:
$$
    \forall x \in \left(-\frac{1}{2}, \frac{1}{2}\right)\colon e^{-0.5} < e^{x}
$$
Also lässt sich die gegebene Ungleichung nun wie folgt darstellen. Sei $x \in \left(-\frac{1}{2}, \frac{1}{2} \right)\colon$
$$
    |e^x - s_n(x)| < \frac{2}{(n+1)!} \leq \frac{e^{-1/2}}{10^{16}} < \frac{e^x}{10^{16}}
$$
Durch ausprobieren von ein paar Werten für $n$ lässt sich dies schnell eingrenzen:
$$
    \frac{2}{(18+1)!} < \frac{e^{-1/2}}{10^{16}} < \frac{2}{(17+1)!}
$$
Also kann die Gültigkeit dieser Ungleichung für alle $n\geq 18$ gewährleistet werden.
\end{document}
