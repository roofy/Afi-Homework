\documentclass[a4paper,graphics,11pt]{article}
\pagenumbering{arabic}

\usepackage[margin=1in]{geometry}
\usepackage[utf8]{inputenc}
\usepackage[T1]{fontenc}
\usepackage{lmodern}
\usepackage[ngerman]{babel}
\usepackage{amsmath, tabu}
\usepackage{amsthm}
\usepackage{amssymb}
\usepackage{complexity}
\usepackage{mathtools}
\usepackage{setspace}
\usepackage{graphicx,color,curves,epsf,float,rotating}
\usepackage{tasks}
\usepackage{delarray}
\setlength{\parindent}{0em}
\setlength{\parskip}{1em}


\newcommand{\aufgabe}[1]{\subsection*{Aufgabe #1}}
\newcommand{\up}[2]{\mathrel{\overset{\makebox[0pt]{\mbox{\normalfont\tiny #2}}}{#1}}}
\newcommand{\vect}[5]{\begin{pmatrix}#1\\#2\\#3\\#4\\#5\end{pmatrix}}

\begin{document}
\noindent Gruppe \fbox{\textbf{9}}             \hfill Phil Pützstück, 377247\\
\noindent Lineare Algebra für Informatiker\\
\begin{center}
	\LARGE{\textbf{Hausaufgabe 7}}
\end{center}
\begin{center}
\rule[0.1ex]{\textwidth}{1pt}
\end{center}



\aufgabe{40}
\textbf{a)}
Wir Entwickeln nach der 4. Spalte, dann nach der 2. :

$$
    \chi_A
    = \det(XE_4 - A)
    = \det
        \begin{pmatrix}
            X-7 & 0 & 12 & 0\\
            16 & X-1 & -32 & 0\\
            -3 & 0 & X+5 & 0\\
            3 & 0 & -6 & X-1
        \end{pmatrix}
$$$$
    = (X-1)\cdot \det
        \begin{pmatrix}
            X-7 & 0 & 12 \\
            16 & X-1 & -32 \\
            -3 & 0 & X+5 \\
        \end{pmatrix}
    = (X-1)^2\cdot \det \begin{pmatrix} X-7 & 12\\ -3 & X+5\end{pmatrix}
$$$$
    = (X-1)^2(X^2-2X-35+36) = (X-1)^4
$$
Wir entwickeln nach der 4. Spalte, dann nach der 1. :
$$
    \chi_B
    = \det(XE_4 - B)
    = \det
        \begin{pmatrix}
            X-1 & -2 & -6 & 0\\
            0 & X-4 & -9 & 0\\
            0 & 1 & X+2 & 0\\
            -1 & -1 & -5 & X-1
        \end{pmatrix}
$$$$
    = (X-1)\cdot \det
        \begin{pmatrix}
            X-1 & -2 & -6 \\
            0 & X-4 & -9 \\
            0 & 1 & X+2
        \end{pmatrix}
    = (X-1)^2\cdot \det \begin{pmatrix} X-4 & -9\\ 1 & X+2\end{pmatrix}
$$$$
    = (X-1)^2(X^2-2X-8+9) = (X-1)^4
$$

\textbf{b)}
Nach VL sind die EW eben genau die Nullstellen des Charakteristischen Polynoms.
Folglich haben A und B beide nur den EW 1.

Wir haben
$$
    m_1(A) = m_1(\chi_A) = m_1((X-1)^4) = 4 = m_1(\chi_B) = m_1(B)
$$
Weiter ist
$$
    g_1(A) = \text{dim Eig}_1(A) = \text{dim Sol}(A - E_4, =) = 4 - \text{rk}(A-E_4)
$$
Wir bestimmen also durch eine Spaltenumformung den Rang von $A-E_4 :$
$$
    \begin{pmatrix}
        6 & 0 & -12 & 0\\
        -16 & 0 & 32 & 0\\
        3 & 0 & -6 & 0\\
        -3 & 0 & 6 & 0
    \end{pmatrix}
    \xrightarrow {3.Sp + 2*1.Sp}
    \begin{pmatrix}
        6 & 0 & 0 & 0\\
        16 & 0 & 0 & 0\\
        3 & 0 & 0 & 0\\
        -3 & 0 & 0 & 0
    \end{pmatrix}
$$
Offensichtlich ist damit rk $A - E_4 = 1$. Also $g_1(A) = 4-1 = 3$.

\newpage

Wir gehen analog für rk $B-E_4$ vor :

$$
    \begin{pmatrix}
        0 & 2 & 6 & 0\\
        0 & 3 & 9 & 0\\
        0 & -1 & -3 & 0\\
        1 & 1 & 5 & 0
    \end{pmatrix}
    \xrightarrow {2.Sp - 1.Sp,\ 3.Sp - 5*1.Sp}
    \begin{pmatrix}
        0 & 2 & 6 & 0\\
        0 & 3 & 9 & 0\\
        0 & -1 & -3 & 0\\
        1 & 0 & 0 & 0
    \end{pmatrix}
    \xrightarrow {3.Sp - 3*2.Sp}
    \begin{pmatrix}
        0 & 2 & 0 & 0\\
        0 & 3 & 0 & 0\\
        0 & -1 & 0 & 0\\
        1 & 0 & 0 & 0
    \end{pmatrix}
$$
Dann ist offensichtlich rk $B-E_4 = 2$ also $g_1(B) = 4-2 = 2$.

\textbf{c)}
Angenommen $A$ ist ähnlich zu $B$. Dann gibt es $P \in \text{GL}_4(K)$ mit
$P^{-1}AP = B$.
Es folgt:
$$
    P^{-1}AP = B
    \,\Longrightarrow\, P^{-1}AP - P^{-1}E_4P = B - E_4
    \,\Longrightarrow\, P^{-1}(A-E_4)P = B-E_4
$$
Damit ist also $A-E_4$ ähnlich zu $B-E_4$. Da Ähnlichkeit aber insbesondere Äquivalenz
impliziert, muss
$$
    1 = \text{rk} A-E_4 = \text{rk} B-E_4 = 2
$$
Dies ist aber offensichtlich falsch. Folglich sind $A$ und $B$ nicht ähnlich.

\aufgabe{41}
Die (modifizierte) Linkmatrix entspricht hier der Standard-Linkmatrix, da jede
der 4 Seiten mindestens auf eine andere verlinkt.
Wir haben:
$$
    L := \begin{pmatrix}
        0 & 0 & 0 & 1\\
        0 & 0 & \frac{1}{2} & 0\\
        \frac{1}{2} & 0 & 0 & 0\\
        \frac{1}{2} & 1 & \frac{1}{2} & 0
    \end{pmatrix}
$$
Weiter ist Eig$_1(L) = \text{Sol}(L-E_4, 0)$. Wir lösen dies:
$$
    \begin{array}({@{}cccc|c@{}})
        -1 & 0 & 0 & 1 & 0\\
        0 & -1 & \frac{1}{2} & 0 & 0\\
        \frac{1}{2} & 0 & -1 & 0 & 0\\
        \frac{1}{2} & 1 & \frac{1}{2} & -1 & 0\\
    \end{array}
    \xrightarrow {\text{IV,III + $\frac{1}{2}$I}}
    \begin{array}({@{}cccc|c@{}})
        -1 & 0 & 0 & 1 & 0\\
        0 & -1 & \frac{1}{2} & 0 & 0\\
        0 & 0 & -1 & \frac{1}{2} & 0\\
        0 & 1 & \frac{1}{2} & -\frac{1}{2} & 0\\
    \end{array}
$$$$
    \xrightarrow {\text{II + IV, (-1)*I, (-1)*III}}
    \begin{array}({@{}cccc|c@{}})
        1 & 0 & 0 & -1 & 0\\
        0 & 0 & 1 & -\frac{1}{2} & 0\\
        0 & 0 & 1 & -\frac{1}{2} & 0\\
        0 & 1 & \frac{1}{2} & -\frac{1}{2} & 0\\
    \end{array}
    \xrightarrow {\text{III-II, IV - $\frac{1}{2}$II}}
    \begin{array}({@{}cccc|c@{}})
        1 & 0 & 0 & -1 & 0\\
        0 & 0 & 1 & -\frac{1}{2} & 0\\
        0 & 0 & 0 & 0 & 0\\
        0 & 1 & 0 & -\frac{1}{4} & 0\\
    \end{array}
$$
Folglich ist Sol$(L-E_4, 0) = \langle \begin{pmatrix}1\\\frac{1}{4}\\\frac{1}{2}\\1\end{pmatrix} \rangle$.

    Der gefragte stochastische Vektor dazu ist dann $\begin{pmatrix}\frac{4}{11}\\[2pt]\frac{1}{11}\\[2pt]\frac{2}{11}\\[2pt]\frac{4}{11}\end{pmatrix}$.





\end{document}
