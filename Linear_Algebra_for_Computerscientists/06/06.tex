\documentclass[a4paper,graphics,11pt]{article}
\pagenumbering{arabic}

\usepackage[margin=1in]{geometry}
\usepackage[utf8]{inputenc}
\usepackage[T1]{fontenc}
\usepackage{lmodern}
\usepackage[ngerman]{babel}
\usepackage{amsmath, tabu}
\usepackage{amsthm}
\usepackage{amssymb}
\usepackage{complexity}
\usepackage{mathtools}
\usepackage{setspace}
\usepackage{graphicx,color,curves,epsf,float,rotating}
\usepackage{tasks}
\usepackage{delarray}
\setlength{\parindent}{0em}
\setlength{\parskip}{1em}

\newcommand{\aufgabe}[1]{\subsection*{Aufgabe #1}}
\newcommand{\up}[2]{\mathrel{\overset{\makebox[0pt]{\mbox{\normalfont\tiny #2}}}{#1}}}
\newcommand{\vect}[3]{\begin{pmatrix}#1\\#2\\#3\end{pmatrix}}
\newcommand{\vectt}[7]{\begin{pmatrix}#1\\#2\\#3\\#4\\#5\\#6\\#7\end{pmatrix}}
\newcommand{\sgn}{\text{sgn }}


\begin{document}
\noindent Gruppe \fbox{\textbf{9}}             \hfill Phil Pützstück, 377247\\
\noindent Lineare Algebra für Informatiker\\
\begin{center}
	\LARGE{\textbf{Hausaufgabe 6}}
\end{center}
\begin{center}
\rule[0.1ex]{\textwidth}{1pt}
\end{center}



\aufgabe{34}
\textbf{a)}

\begin{tabular}{c|c|c|c|c|c|c}
    Syndrom & $\vect{0}{0}{0}$ & $\vect{0}{0}{1}$ & $\vect{0}{1}{0}$ & $\vect{0}{1}{1}$ & $\vect{1}{0}{0}$ & $\vect{1}{0}{1}$ \\
    \hline
    Anführer & $\vectt{0}{0}{0}{0}{0}{0}{0}$
    & $\vectt{0}{0}{0}{0}{0}{0}{1}$
    & $\vectt{0}{0}{0}{0}{0}{1}{0}$
    & $\vectt{0}{0}{1}{0}{0}{0}{0}$,$\vectt{0}{0}{0}{1}{0}{0}{0}$
    & $\vectt{0}{0}{0}{0}{1}{0}{0}$
    & $\vectt{1}{0}{0}{0}{0}{0}{0}$,$\vectt{0}{1}{0}{0}{0}{0}{0}$\\
    \hline
    \hline
    Syndrom & $\vect{1}{1}{0}$ & $\vect{1}{1}{1}$ \\
    \hline
    Anführer
    & $\vectt{1}{0}{1}{0}{0}{0}{0}$,$\vectt{1}{0}{0}{1}{0}{0}{0}$,$\vectt{0}{1}{1}{0}{0}{0}{0}$,$\vectt{0}{1}{0}{1}{0}{0}{0}$,$\vectt{0}{0}{0}{0}{1}{1}{0}$
    & $\vectt{1}{0}{0}{0}{0}{1}{0},\vectt{0}{1}{0}{0}{0}{1}{0},\vectt{0}{0}{1}{0}{1}{0}{0},\vectt{0}{0}{0}{1}{1}{0}{0}$\\
\end{tabular}

\textbf{b)}

$x_1 = \vectt{1}{1}{0}{1}{0}{1}{1}$. Syndrom $Bx_1 = \vect{0}{0}{0}$.
Decodiere $x_1$ zu $x_1 - e_{Bx_1} = \vectt{1}{1}{0}{1}{0}{1}{1} - \vectt{0}{0}{0}{0}{0}{0}{0} = x_1 \in C$.

$x_2 = \vectt{0}{1}{1}{0}{1}{1}{1}$. Syndrom $Bx_2 = \vect{0}{0}{1}$.
Decodiere $x_2$ zu $x_2 - e_{Bx_2} = \vectt{0}{1}{1}{0}{1}{1}{1} - \vectt{0}{0}{0}{0}{0}{0}{1} = \vectt{0}{1}{1}{0}{1}{1}{0} \in C$.

$x_3 = \vectt{0}{1}{1}{1}{0}{0}{0}$. Syndrom $Bx_3 = \vect{1}{0}{1}$.
Decodiere $x_3$ zu $x_3 - e_{Bx_3} = \vectt{0}{1}{1}{1}{0}{0}{0} - \vectt{0}{1}{0}{0}{0}{0}{0} = \vectt{0}{0}{1}{1}{0}{0}{0} \in C$.

\aufgabe{35}

\textbf{Lemma 1:}
\ Sei $n \in \mathbb{N}$. Für $\pi \in S_n$ mit $\pi \neq \text{id}$ existiert ein $k \in [1,n]$ mit $\pi(k) > k$.

\textit{Beweis durch Widerspruch.}\\
Sei also $\pi \in S_n, \pi \neq \text{id}$ und $\pi(k) \leq k$ für alle $k \in [1,n]$. Wir führen Induktion über $k:$

Für $k = 1$ haben wir $\pi(k) = 1$, da $\{i \in [1,n] \mid i \leq k\} = \{1\}$, also nur eine Möglichkeit besteht.
Es gilt also $\pi(k) = k$ für $k = 1$.

Sei nun ein $k \in [1,n]$ gegeben, sodass $\forall k' \in [1,k-1] : \pi(k') = k'$. Da Permutationen Bijektionen, also
insbesondere injektiv sind, ist die Menge der möglichen Bilder für $k$ gleich \\
$\{i \in [1,n]\setminus[1,k-1] \mid i\leq k\} = \{k\}$. Folglich muss also auch $\pi(k) = k$ gelten.

Insgesamt folgt also $\forall k \in [1,n] : \pi(k) = k$, was ein Widerspruch zur Annahme $\pi \neq \text{id}$ ist.\\

Damit haben ist das zu zeigende bewiesen, d.h. es gilt
$$
    \forall \pi \in S_n\setminus\{\text{id}\} : \exists k \in [1,n] : \pi(k) > k
$$

\textbf{a)}

Sei also eine Matrix $A \in K^{n \times n}$ gegeben welche obere Dreicksform hat.
Mit Lemma 1 folgt nun:
$$
    \text{det}(A)
    = \sum_{\pi \in S_n} (\text{sgn } \pi)\prod_{j\in [1,n]} A_{\pi(j),j}
    = (\text{sgn id}) \prod_{j\in [1,n]} A_{\text{id}(j),j}
        + \sum_{\substack{\pi \in S_n\\ \pi \neq \text{id}}} (\text{sgn } \pi)
        \prod_{j \in [1,n]} A_{\pi(j),j}
$$
Da nach Lemma 1 für jede Permutation außer id also ein $k \in [1,n]$ mit $\pi(k) > k$
existiert ist das Produkt $\prod_{j\in [1,n]} A_{\pi(j), j} = 0$ für alle diese
$\pi$, da nach Konstruktion von $A$ stetes $A_{i,j} = 0$ für $i,j \in [1,n]$ mit $i > j$. Also:
$$
    \text{det}(A) = 1\cdot \prod_{j \in [1,n]} A_{j,j} + \sum_{\substack{\pi \in S_n\\ \pi \neq \text{id}}}(\text{sgn } \pi) \prod_{j \in [1,n]} 0 = \prod_{j \in [1,n]} A_{j,j}
$$

\newpage

\textbf{b)}
Ein evtl. etwas einfacherer Beweis mittels Induktion lässt sich im Skript finden.

Seien also $A \in R^{n\times n}$ mit $A = \begin{pmatrix}B & C \\ 0 & D\end{pmatrix}$
wobei $B \in R^{k\times k}, C \in R^{k \times (n-k)}, D \in R^{(n-k)\times(n-k)}$.
Dann ist:
$$
    \det(B)\cdot \det(D)
    = \left(\sum_{\pi \in S_k}(\sgn \pi) \prod_{j \in [1,k]} B_{\pi(j),j}\right)
        \cdot \left(\sum_{\pi' \in S_{n-k}}(\sgn \pi') \prod_{j \in [1,n-k]} D_{\pi'(j),j}\right)
$$$$
    = \left(\sum_{\pi \in S_k}(\sgn \pi) \prod_{j \in [1,k]} A_{\pi(j),j}\right)
        \cdot \left(\sum_{\pi' \in S_k}(\sgn \pi') \prod_{j \in [1,n-k]} A_{\pi'(j)+k,j+k}\right)
$$
Wir fassen nun alle möglichen Permutationen in $S_k$ und $S_{n-k}$ zusammen in:
$$
    X := \{\sigma \in S_n \mid \exists \pi \in S_n, \pi' \in S_{n-k} :
        (\forall j \in [1,k] : \sigma(j) = \pi(j))
        \land
        (\forall j \in [1,n-k] : \sigma(k+j) = k+\pi'(j))
        \}
$$
Sei nun $\sigma \in X$. Dann existieren $\pi \in S_k, \pi' \in S_{n-k}$ nach
Vorschrift der Menge $X$.\\
Daraus folgt $\sgn \sigma = (\sgn \pi)(\sgn \pi')$. Weiter ist dann:
$$
    \prod_{j \in [1,k]} A_{\pi(j),j} \prod_{j \in [1,n-k]} A_{\pi'(j)+k, j+k}
    = \prod_{j \in [1,n]} A_{\sigma(j), j}
$$
Das Multiplizieren der Summen entspricht den möglichen Kombinationen für
$\pi, \pi'$, da z.B. für ein $\pi \in S_k$ genau $|S_{n-k}|$ verschiedene $\sigma \in X$ existieren. Insgesamt haben wir nun:
$$
    \det(B) \cdot \det(D)
    = \sum_{\sigma \in X} (\sgn \sigma) \prod_{j \in [1,n]} A_{\sigma(j), j}
$$
Da nach Aufbau von $A$ aber stets $A_{i,j} = 0$ für $i \in [k+1,n]$ und $j \in [1,k]$,
folgt mit Lemma 1 und id $\in X$, dass:
$$
    \forall \pi \in S_n \setminus X : (\exists j \in [1,k] : (\pi(j) > k \,\Longrightarrow\, A_{\pi(j),j} = 0) \,\Longrightarrow\, \prod_{j \in [1,n]} A_{\pi(j),j} = 0)
$$$$
    \,\Longrightarrow\, \sum_{\sigma \in X}(\sgn \sigma) \prod_{j \in [1,n]} A_{\sigma(j), j}
    = \underbrace{\sum_{\pi \in S_n \setminus X} (\sgn \pi) \prod_{j \in [1,n]} A_{\pi(j),j}}_{=\ 0}
    + \sum_{\sigma \in X} (\sgn \sigma) \prod_{j \in [1,n]} A_{\sigma(j),j}
$$$$
    = \sum_{\pi \in S_n}(\sgn \pi)\prod_{j \in [1,n]} A_{\pi(j),j} = \det(A)
$$

\textbf{c)}
Nach Satz A.164 lässt sich jede Matrix über einem Körper in reduzierte Zeilenstufenform
bringen (durch Elementaroperationen, für welche wir die entsprechende Änderung der
Determinante definiert haben). Diese ist dann insbesondere eine obere Dreiecksmatrix.
Mit a) haben wir dann ein leichte Formel für die Determinante dieser reduzierten Matrix.


\end{document}
