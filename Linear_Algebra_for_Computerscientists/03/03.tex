\documentclass[a4paper,graphics,11pt]{article}
\pagenumbering{arabic}

\usepackage[margin=1in]{geometry}
\usepackage[utf8]{inputenc}
\usepackage[T1]{fontenc}
\usepackage{lmodern}
\usepackage[ngerman]{babel}
\usepackage{amsmath, tabu}
\usepackage{amsthm}
\usepackage{amssymb}
\usepackage{complexity}
\usepackage{mathtools}
\usepackage{setspace}
\usepackage{graphicx,color,curves,epsf,float,rotating}
\usepackage{tasks}
\setlength{\parindent}{0em}
\setlength{\parskip}{1em}

\newcommand{\aufgabe}[1]{\subsection*{Aufgabe #1}}
\newcommand{\up}[2]{\mathrel{\overset{\makebox[0pt]{\mbox{\normalfont\tiny #2}}}{#1}}}

\begin{document}
\noindent Gruppe \fbox{\textbf{9}}             \hfill Phil Pützstück, 377247\\
\noindent Lineare Algebra für Informatiker\\
\begin{center}
	\LARGE{\textbf{Hausaufgabe 3}}
\end{center}
\begin{center}
\rule[0.1ex]{\textwidth}{1pt}
\end{center}

\aufgabe{16}
Projektionen sind im Skript (Def. 1.84) deutlich allgemeiner als für nur 2 Untervektorräume definiert.
Wir halten uns im folgenden an die Definition des Skripts.

\textbf{a)}

Es sei ein $K$-Vektorraum $V$, ein $n \in \mathbb{N}_0$ und ein $n$-Tupel $(U_1, \cdots, U_n)$ von
$K$-Untervektorräumen von $V$ mit $\displaystyle V = \sum_{i \in [1,n]}^{\text{.}} U_i$ gegeben.

Wir überprüfen die Kriterien für Vektorraumhomomorphismen (2.2):\\
Seien $v, v' \in V$. Dann existieren $u, u' \in \bigtimes_{i \in [1,n]} U_i$ sodass
$v = \sum_{i \in [1, n]} u_i$ und $v' = \sum_{i \in [1, n]} u'_i$.
Sei nun $i \in [1,n]$. Dann gilt
$$
    \text{pr}_i^V(v + v')
    = \text{pr}_i^V(\sum_{j \in [1,n]} (u_j + u'_j))
    \up{=}{def} u_i + u'_i
    \up{=}{def} \text{pr}_i^V(\sum_{j \in [1,n]} u_j) + \text{pr}_i^V(\sum_{j \in [1,n]} u'_j)
    = \text{pr}_i^V(v) + \text{pr}_i^V(v')
$$
Sei weiter nun $a \in K$. Es gilt
$$
    \text{pr}_i^V(av)
    = \text{pr}_i^V(\sum_{j\in [1,n]} au_j)
    \up{=}{def} au_i
    \up{=}{def} a\cdot \text{pr}_i^V(\sum_{j\in[1,n]} u_j)
    = a\cdot \text{pr}_i^V(v)
$$
Damit sind die Kriterien aus (2.2) erfüllt.\\
Es folgt, dass Projektionen (Endo-)Vektorraumhomomorphismen sind.$\hfill\square$

\textbf{b)}

Sei $V$ ein $K$-Vektorraum. Sei weiter ein $n \in \mathbb{N}_0$ und ein $n$-Tupel $(U_1, \cdots, U_n)$ von $K$-Untervektorräumen von $V$ mit $V = \sum_{i \in [1,n]} U_i$ gegeben.
Dann existiert $u \in \bigtimes_{i \in [1,n]} U_i$ sodass $v = \sum_{i \in [1, n]} u_i$.\\
Sei nun $i \in [1,n]$ und $p := \text{pr}_i^V$. Es gilt
$$
    (p \circ p)(v)
    = p(p(v))
    \up{=}{def} p(u_i)
    = p (\sum_{j \in [1,n]} \delta_{j,i}\cdot u_i)
    \up{=}{def} u_i
    \up{=}{def} p(\sum_{j\in [1,n]} u_j)
    = p(v)
$$
wobei $\delta$ das Kronecker-Delta bezeichnet. Es folgt $p \circ p = p$.$\hfill\square$

\textbf{c)}

Für $v, v' \in V$ ist $\varphi(v) + \varphi(v') = \varphi(v+v') \in \text{Im}\ \varphi$.
Ferner ist $0 = \varphi(0) \in \text{Im}\ \varphi$.\\
Für $a \in K$ ist schließlich $a\varphi(v) = \varphi(av) \in \text{Im}\ \varphi$.\\
Damit ist nach (1.15)
$U_1 := \text{Im}\ \varphi$ ein $K$-Untervektorraum von $V$.

\newpage

Zuerst ist $0 = 0 - \varphi(0) \in U_2$.
Für $v, v' \in V$ ist $(v+v') \in V$ und $(v - \varphi(v)), (v' -\varphi(v')) \in U_2$.\\
Es gilt
$$
    (v - \varphi(v)) + (v' - \varphi(v'))
    = (v + v') - (\varphi(v) + \varphi(v'))
    = (v+v') - \varphi(v+v') \in U_2
$$
Ferner ist für $a \in K$ auch $av \in V$ und damit
$$
    a(v - \varphi(v)) = av - a\varphi(v) = av - \varphi(av) \in U_2
$$
Damit ist nach (1.15) $U_2$ ein $K$-Untervektorraum von $V$.

Sei nun $v \in V$. Es gilt $v = v + \varphi(v) - \varphi(v) = \varphi(v) + (v - \varphi(v)) \in (U_1 + U_2)$.
Also $V \subseteq (U_1 + U_2)$.

Sei nun $v \in (U_1 + U_2)$. Dann gilt $v = u_1 + u_2$ für $u_1 \in U_1$ und $u_2 \in U_2$.
Ferner gilt $U_1 \leq V$ und $U_2 \leq V$, also $u_1, u_2 \in V$. Es folgt direkt nach den
Vektorraumgesetzen, dass auch\\
$v = u_1 + u_2 \in V$.
Also $(U_1 + U_2) \subseteq V$. Es folgt $V = U_1 + U_2$. $\hfill\square$

\textbf{d)}
Sei $u_1 \in U_1$. Dann gilt $u_1 = \varphi(v)$ für ein $v \in V$. Ferner gilt $\varphi \circ \varphi = \varphi$ (*).
Es folgt
$$
    \varphi(u_1)
    = \varphi(\varphi(v))
    \up{=}{*} \varphi(v)
    = u_1
$$
Sei nun $u_2 \in U_2$. Dann gilt $u_1 = v - \varphi(v)$ für ein $v \in V$. Es folgt
$$
    \varphi(u_2)
    = \varphi(v - \varphi(v))
    = \varphi(v) + \varphi(-\varphi(v))
    = \varphi(v) - \varphi(v)
    = 0
$$
$\strut\hfill\square$

\aufgabe{17}
Sei $X \in V$ mit
$X = \begin{pmatrix}
        a & b\\
	    c & d
    \end{pmatrix}
$
für $a,b,c,d \in \mathbb{R}$. Es gilt
\begin{equation}
    \varphi(X)
    = AXA
    = \begin{pmatrix}
		0 & 1\\
		0 & 1
	\end{pmatrix}
    \begin{pmatrix}
		a & b\\
		c & d
	\end{pmatrix}
    \begin{pmatrix}
		0 & 1\\
		0 & 1
	\end{pmatrix}
    = \begin{pmatrix}
		c & d\\
		c & d
	\end{pmatrix}
    \begin{pmatrix}
		0 & 1\\
		0 & 1
	\end{pmatrix}
    = 
    \begin{pmatrix}
		0 & c+d\\
		0 & c+d
	\end{pmatrix}
\end{equation}
Seien nun $X, X' \in V$.
Es folgt:
$$
    \varphi(X+X')
    = A(X+X')A
    = AXA+AX'A
    = \varphi(X) + \varphi(X')
$$
Sei $X \in V$. Sei ferner $r \in \mathbb{R}$. Es gilt
$$
    r\varphi(X)
    = A(rX)A
    = r(AXA)
    = r\varphi(X)
$$
Nach (2.2) ist $\varphi$ damit linear.
\newpage
Mit (1) folgt für $a,b,c,d \in \mathbb{R}$
$$
    \text{Ker}(\varphi)
    = \{v \in V \mid \varphi(v) = 0\}
    \up{=}{(1)} \{\begin{pmatrix}
        a & b\\
        c & d
	\end{pmatrix}
    \mid c = -d\}
    = \mathbb{R}
    \begin{pmatrix}
        0 & 0\\
        1 & -1
	\end{pmatrix}
$$
Es folgt direkt $\text{Ker}(\varphi) = \langle
\begin{pmatrix}
    0 & 0\\
    1 & -1
\end{pmatrix}
\rangle$, also dass $(
\begin{pmatrix}
    0 & 0\\
    1 & -1
\end{pmatrix}
)$ eine Basis von $\text{Ker}(\varphi)$ ist, da auch
weiter $
\begin{pmatrix}
    0 & 0\\
    1 & -1
\end{pmatrix}
\neq 0
$.

Ebenso folgt mit (1) für $a,b,c,d \in \mathbb{R}$
$$
    \text{Im}(\varphi)
    = \{\varphi(v) \mid v \in V\}
    \up{=}{(1)} \{
    \begin{pmatrix}
        0 & c+d\\
        0 & c+d
    \end{pmatrix}
    \mid c,d \in \mathbb{R}\}
    =\{
    \begin{pmatrix}
        0 & x\\
        0 & x
    \end{pmatrix}
    \mid x \in \mathbb{R}\}
    = \mathbb{R}
    \begin{pmatrix}
        0 & 1\\
        0 & 1
    \end{pmatrix}
    = \mathbb{R}A
$$
Folglich gilt $\text{Im}(\varphi) = \langle A\rangle$, also dass $(A)$ eine Basis von $\text{Im}(\varphi)$ ist,
da auch weiter $A \neq 0$.

\textbf{b)}

Es ist $(v_1, v_2, v_3)$ eine Basis von $\mathbb{R}^{3\times 1}$, denn:
$$
    (1,0,0) = \frac{v_1 + v_2 - v_3}{2}
    \qquad
    (0,1,0) = \frac{v_1 - v_2 + v_3}{2}
    \qquad
    (0,0,1) = \frac{-v_1 + v_2 + v_3}{2}
$$
Damit gilt nach Proposition 2.19, dass es genau ein lineare Abbildung $\varphi$ mit
$$
    \varphi(v_1) =
\begin{pmatrix}
    2 & 0\\
    0 & 0
\end{pmatrix}
\qquad
\varphi(v_2) =
\begin{pmatrix}
    2 & 0\\
    -2 & 0
\end{pmatrix}
\qquad
\varphi(v_3) =
\begin{pmatrix}
    0 & 0\\
    0 & -2
\end{pmatrix}
$$
Für $a,b,c,x,y,z \in \mathbb{R}$ mit $a = x+y, b = x+z, c = y+z$ gilt:
$$
    \varphi(\begin{pmatrix}a\\b\\c\end{pmatrix})
    =
    \varphi(\begin{pmatrix}x+y\\x+z\\y+z\end{pmatrix})
    =
    \varphi(x\begin{pmatrix}1\\1\\0\end{pmatrix}
        +
    y\begin{pmatrix}1\\0\\1\end{pmatrix}
        +
    z\begin{pmatrix}0\\1\\1\end{pmatrix})
    =
    x\varphi(\begin{pmatrix}1\\1\\0\end{pmatrix})
        +
    y\varphi(\begin{pmatrix}1\\0\\1\end{pmatrix})
        +
    z\varphi(\begin{pmatrix}0\\1\\1\end{pmatrix})
$$$$
    =
    \begin{pmatrix}
    2x & 0\\
    0 & 0
    \end{pmatrix}
    +
    \begin{pmatrix}
    2y & 0\\
    -2y & 0
    \end{pmatrix}
    +
    \begin{pmatrix}
    0 & 0\\
    0 & -2z
    \end{pmatrix}
    =
    \begin{pmatrix}
    2(x+y) & 0\\
    -2y & -2z
    \end{pmatrix}
    =
    \begin{pmatrix}
    2a & 0\\
    b-a-c & a-b-c
    \end{pmatrix}
$$
\end{document}
