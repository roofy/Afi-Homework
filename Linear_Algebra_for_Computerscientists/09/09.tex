\documentclass[a4paper,graphics,11pt]{article}
\pagenumbering{arabic}

\usepackage[margin=1in]{geometry}
\usepackage[utf8]{inputenc}
\usepackage[T1]{fontenc}
\usepackage{lmodern}
\usepackage[ngerman]{babel}
\usepackage{amsmath, tabu}
\usepackage{amsthm}
\usepackage{amssymb}
\usepackage{complexity}
\usepackage{mathtools}
\usepackage{setspace}
\usepackage{graphicx,color,curves,epsf,float,rotating}
\usepackage{tasks}
\usepackage{delarray}
\setlength{\parindent}{0em}
\setlength{\parskip}{1em}


\newcommand{\aufgabe}[1]{\subsection*{Aufgabe #1}}
\newcommand{\up}[2]{\mathrel{\overset{\makebox[0pt]{\mbox{\normalfont\tiny #2}}}{#1}}}
\newcommand{\vect}[5]{\begin{pmatrix}#1\\#2\\#3\\#4\\#5\end{pmatrix}}

\begin{document}
\noindent Gruppe \fbox{\textbf{9}}             \hfill Phil Pützstück, 377247\\
\noindent Lineare Algebra für Informatiker\\
\begin{center}
	\LARGE{\textbf{Hausaufgabe 9}}
\end{center}
\begin{center}
\rule[0.1ex]{\textwidth}{1pt}
\end{center}



\aufgabe{52}
Die Rekursionsgleichung ist $a_{n+2} = 3a_{n+1} - 2a_n$ für $n \in \mathbb{N}_0$,
wobei $a_0 = -1$ und $a_1 = 1$.\\
Das entsprechend kodierte Polynom (siehe VL) ist dann $f = X^2-3X+2 = (X-1)(X-2)$.
$f$ hat Nullstellen $b_1 = 1, b_2 = 2$. Nach Satz 6.75 folgt nun, dass
$$
    \forall n \in \mathbb{N}_0 : a_{n+2} = -2a_n+3a_{n+1}
$$$$
    \,\Longleftrightarrow\,
    \forall n \in \mathbb{N}_0 : a_n = \frac{b_1^nb_2-b_2^nb_1}{b_2-b_1}x_0 + \frac{b_2^n-b_1^n}{b_2-b_1}x_1
    = 2^{n+1}3
$$
Dies entspricht der gesuchten geschlossenen Formel.

\aufgabe{53}

\textbf{a)}
Es gilt:
$$
    \frac{1}{2}(||v+w||^2 - ||v||^2-||w||^2)
    = \frac{1}{2}(\langle v+w, v+w \rangle - \langle v,v\rangle - \langle w,w \rangle)
$$$$
    = \frac{1}{2}(\langle v, w \rangle + \langle w,v \rangle)
    = \langle v,w \rangle
$$

\textbf{b)}
$$
    ||v+w||^2
    = \langle v+w , v+w \rangle
    = \langle v,v\rangle + 2\langle v,w\rangle + \langle w,w \rangle
$$$$
    \up{\leq}{Cauchy}\quad ||v||^2 + 2||v||||w|| + ||w||^2 = (||v|| + ||w||)^2
$$$$
    \up{\Longleftrightarrow}{Positivität}\quad ||v+w|| \leq ||v|| + ||w||
$$

\textbf{c)}
$$
    ||v|| = ||v - w + w|| \leq ||v - w|| + ||w||
    \,\Longleftrightarrow\,
    ||v|| - ||w|| \leq ||v-w||
$$
Da $v,w$ beliebig gilt also auch
$$
    | ||v|| - ||w|| | \leq ||v-w||
$$

\textbf{d)}
$$
    ||v+w||^2 + ||v-w||^2
    = \langle v,v \rangle + \langle w,w \rangle + 2\langle v,w\rangle
        + \langle v,v \rangle + \langle w,w \rangle - 2 \langle v,w \rangle
$$$$
    = 2\langle v,v \rangle + 2 \langle w,w \rangle
    = 2(||v||^2 + ||w||^2)
$$

\newpage

\textbf{e)}
Sei $v \perp w \,\Longleftrightarrow\, \langle v,w \rangle = 0$. Dann:
$$
    ||v+w|| - ||v-w||
    = \sqrt{\langle v+w, v+w\rangle} - \sqrt{\langle v-w, v-w \rangle}
$$$$
    = \sqrt{\langle v,v \rangle + \langle w,w \rangle + 2\langle v,w \rangle} -
        \sqrt{\langle v,v \rangle + \langle w,w \rangle - 2 \langle v,w \rangle}
$$$$
    = \sqrt{\langle v,v \rangle + \langle w,w \rangle} -
        \sqrt{\langle v,v \rangle + \langle w,w \rangle}
    = 0
    \,\Longleftrightarrow\, ||v+w|| = ||v-w||
$$

Sei nun $||v+w|| = ||v-w|| \,\Longleftrightarrow\, ||v+w|| - ||v-w|| = 0$.
Dann gilt (s.o.) auch
$$
   \sqrt{\langle v,v \rangle + \langle w,w \rangle + 2\langle v,w \rangle} =
        \sqrt{\langle v,v \rangle + \langle w,w \rangle - 2 \langle v,w \rangle}
$$
Da $\sqrt{\text{ }}$ injektiv und so folgt dann auch
$$
    2\langle v,w \rangle = - 2 \langle v,w \rangle
    \,\Longleftrightarrow\, \langle v,w \rangle = 0
$$

Insgesamt gilt also:
$$
    ||v+w|| = ||v-w||
    \,\Longleftrightarrow\, ||v+w||-||v-w|| = 0
    \,\Longleftrightarrow\, v \perp w
$$
\end{document}
