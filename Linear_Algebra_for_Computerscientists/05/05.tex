\documentclass[a4paper,graphics,11pt]{article}
\pagenumbering{arabic}

\usepackage[margin=1in]{geometry}
\usepackage[utf8]{inputenc}
\usepackage[T1]{fontenc}
\usepackage{lmodern}
\usepackage[ngerman]{babel}
\usepackage{amsmath, tabu}
\usepackage{amsthm}
\usepackage{amssymb}
\usepackage{complexity}
\usepackage{mathtools}
\usepackage{setspace}
\usepackage{graphicx,color,curves,epsf,float,rotating}
\usepackage{tasks}
\usepackage{delarray}
\setlength{\parindent}{0em}
\setlength{\parskip}{1em}

\newcommand{\aufgabe}[1]{\subsection*{Aufgabe #1}}
\newcommand{\up}[2]{\mathrel{\overset{\makebox[0pt]{\mbox{\normalfont\tiny #2}}}{#1}}}
\newcommand{\vect}[5]{\begin{pmatrix}#1\\#2\\#3\\#4\\#5\end{pmatrix}}

\begin{document}
\noindent Gruppe \fbox{\textbf{9}}             \hfill Phil Pützstück, 377247\\
\noindent Lineare Algebra für Informatiker\\
\begin{center}
	\LARGE{\textbf{Hausaufgabe 5}}
\end{center}
\begin{center}
\rule[0.1ex]{\textwidth}{1pt}
\end{center}



\aufgabe{28}
\textbf{a)}
Sei $v \in K^n, i \in [0,n]$. Dann ist $d(v,w) = |\{j \mid v_j - w_j \neq 0\}|$ für $w \in K^n$.\\
Insbesondere: wenn $d(v,w) = i$, dann gibt es Indizes $j_1,\cdots, j_i \in [1,n]$ sodass 
$$
    \forall k \in [1,i] : v_{j_k} - w_{j_k} \neq 0
$$
Auf deutsch; wir können $i$ verschieden Indizes aus $[1,n]$ wählen an denen sich $v$ und $w$ unterscheiden.
Offensichtlich gibt es genau $\binom{n}{i}$ Möglichkeiten diese $i$ Indizes von den möglichen $n$ zu wählen.
Weiter folgt aus $|K| = q$, dass es genau $q-1$ Elemente aus $K$ ungleich 0 gibt.\\
Wir haben also zu jedem der gewählten Indizes $(j_k)_{k \in [1,i]}$ genau $q-1$ Möglichkeiten $w_{j_k}$ einen Wert aus $K$
zuzuweisen, sodass $v_{j_k} - w_{j_k} \neq 0$ ist. Um also ein $w \in K^n$ zu finden, sodass $d(v,w) = i$, wählen
wir zuerst die $i$ Indizes an denen sich $w$ von $v$ unterscheidet und dann für jeden dieser Indizes einen Wert aus
$q-1$ Möglichkeiten.

Insgesamt haben wir nach der Produktregel der Kombinatorik damit:
$$
    |\{w \in K^n \mid d(v,w) = i\}| = \binom{n}{i}\prod_{k = 1}^{i}(q-1) = \binom{n}{i}(q-1)^i
$$
\textbf{b)}
Da $d(C) = 2e + 1$ haben wir $e \leq d(C) \leq n$, wir können also das Resultat aus a) verwenden:
$$
    |B_{e+1}(v)|
    = |\{w \in K^n \mid d(v,w) < e+1\}|
    = |\up{\displaystyle\bigcup_{i \in [0,e]}}{\Large .} \{w \in K^n \mid d(v,w) = i\}|
    = \sum_{i = 0}^{e} \binom{n}{i}(q-1)^i
$$

\textbf{c)}
Da $d(C) = 2e + 1$ und $e \in \mathbb{N}$ ist $d(C)$ ungerade und $e+1 = \frac{d(C)+1}{2}$. \\
Ebenso ist auch stets $d(w,w') \in \mathbb{N}$ für $w,w' \in K^n$. Insbesondere folgt damit:
$$
    B_{e+1}(c)
    = B_{\frac{d(C)+1}{2}}(c)
    = B_{\frac{d(C)}{2}}(c)
$$
Da durch die Parität von $d(C)$ dann
$\{x \in \mathbb{N}_0 \mid x < \frac{d(C)+1}{2}\} = \{x \in \mathbb{N}_0 \mid x < \frac{d(C)}{2}\} = [0,e]$ gilt.

Wir wissen weiter, dass für $c,c' \in C$ gilt:
$$
    B_{\frac{d(C)}{2}}(c) \cap B_{\frac{d(C)}{2}}(c') = \varnothing
    \qquad\text{sowie}\qquad
    B_{\frac{d(C)}{2}}(c) \cap C = \{c\}
$$
Aus dieser Disjunktheit folgt dann
$$
    \forall c \in C : |B_{\frac{d(C)}{2}}(c)| = \sum_{i=0}^{e} \binom{n}{i}(q-1)^i
$$

\newpage

Da also alle diese Kugeln disjunkt sind, haben wir mindestens $|C| \cdot \sum_{i=0}^{e} \binom{n}{i}(q-1)^i$
verschiedene Elemente aus $K^n$ in den Kugeln der Codes von $C$. Offensichtlich sind das kleiner oder gleichviele
wie alle Elemente von $K^n$. Ferner ist $|K| = q$ also $|K^n| = q^n$. Insgesamt folgt:
$$
    |C|\sum_{i=0}^{e} \binom{n}{i}(q-1)^i \leq q^n
    \quad\,\Longleftrightarrow\,\quad
    |C| \leq \frac{q^n}{\sum_{i=0}^{e} \binom{n}{i}(q-1)^i}
$$

\aufgabe{29}

\textbf{a)}
Wir wählen $\mathcal{B} := (\vect{1}{1}{0}{0}{0},\vect{0}{0}{0}{1}{1},\vect{1}{0}{1}{0}{0},\vect{1}{1}{1}{1}{0})$.
Da $\vect{1}{0}{1}{1}{1} = \mathcal{B}_1 + \mathcal{B}_2$, ist $\mathcal{B}$ ein EZS von $C$.\\
Weiter gilt für $a_1,a_2,a_3,a_4 \in K = \mathbb{F}_2:$
$$
    \sum_{i=1}^{4} a_i\mathcal{B}_i = \vect{a_1+a_3+a_4}{a_1+a_4}{a_3+a_4}{a_2+a_4}{a_2} = 0
    \,\Longrightarrow\, a_2 = 0
    \,\Longrightarrow\, \vect{a_1+a_3+a_4}{a_1+a_4}{a_3+a_4}{a_4}{0} = 0
    \,\Longrightarrow\, a_4 = 0
$$$$
    \,\Longrightarrow\, \vect{a_1+a_3}{a_1}{a_3}{0}{0} = 0
    \,\Longrightarrow\, a_1 = a_2 = a_3 = a_4 = 0
$$
Folglich ist $\mathcal{B}$ auch l.u. und damit eine Basis von $C$.

\textbf{b)}
Nach Proposition 3.73 gilt
$$
    \text{Sol}(H,0) = \text{Col}(A)
    \qquad\,\Longleftrightarrow\,\qquad
    \text{Sol}(A^{\text{tr}},0) = \text{Col}(H^{\text{tr}})
$$
Sei also $A \in K^{5\times 4} = \mathbb{F}_2^{5\times 4}$ mit $A_{-,i} = \mathcal{B}_i$ für $i \in [1,4]$.
Dann ist Col$(A) = C$.\\
Ferner ist damit rk $A = \text{rk } A^\text{tr} = 4$, also
$$
    \text{dim Col}(H^{\text{tr}}) = \text{dim Sol}(A^{\text{tr}}, 0) = \text{dim Ker}(\varphi_{A^{\text{tr}}}) = 5 - 
    \text{rk} \varphi_{A^{\text{tr}}} = 5 - \text{rk } A = 1
$$
Wenn also der Spaltenraum von $H^\text{tr}$ dim 1 hat, so hat der Zeilenraum von $H$ ebenfalls dim 1.\\
Die minimale Anzahl an Zeilen um dies zu erreichen ist 1.

Folglich gibt es ein $H \in K^{1\times 5} = \mathbb{F}_2^{1\times 5}$ sodass Sol$(H, 0) = C$.

\newpage 

\textbf{c)}

Wir gehen nach dem Algorithmus 3.74 vor und berechnen zuerst Sol$(A^\text{tr}, 0)$:
$$
    \begin{array}({@{}ccccc|c@{}})
        1 & 1 & 0 & 0 & 0 & 0\\
        0 & 0 & 0 & 1 & 1 & 0\\
        1 & 0 & 1 & 0 & 0 & 0\\
        1 & 1 & 1 & 1 & 0 & 0\\
    \end{array}
    \xrightarrow {\text{III + I, IV + I}}
    \begin{array}({@{}ccccc|c@{}})
        1 & 1 & 0 & 0 & 0 & 0\\
        0 & 0 & 0 & 1 & 1 & 0\\
        0 & 1 & 1 & 0 & 0 & 0\\
        0 & 0 & 1 & 1 & 0 & 0\\
    \end{array}
    \xrightarrow {\text{I + III}}
    \begin{array}({@{}ccccc|c@{}})
        1 & 0 & 1 & 0 & 0 & 0\\
        0 & 0 & 0 & 1 & 1 & 0\\
        0 & 1 & 1 & 0 & 0 & 0\\
        0 & 0 & 1 & 1 & 0 & 0\\
    \end{array}
$$$$
    \xrightarrow {\text{I + IV, III + IV}}
    \begin{array}({@{}ccccc|c@{}})
        1 & 0 & 0 & 1 & 0 & 0\\
        0 & 0 & 0 & 1 & 1 & 0\\
        0 & 1 & 0 & 1 & 0 & 0\\
        0 & 0 & 1 & 1 & 0 & 0\\
    \end{array}
    \xrightarrow {\text{I + II, III + II, IV + II}}
    \begin{array}({@{}ccccc|c@{}})
        1 & 0 & 0 & 0 & 1 & 0\\
        0 & 0 & 0 & 1 & 1 & 0\\
        0 & 1 & 0 & 0 & 1 & 0\\
        0 & 0 & 1 & 0 & 1 & 0\\
    \end{array}
$$
Es folgt Sol$(A^\text{tr}, 0) = \langle \vect{1}{1}{1}{1}{1}\rangle$.
Offensichtlich ist dann auch $(\vect{1}{1}{1}{1}{1})$ eine Basis von Sol$(A^\text{tr}, 0)$.\\
Nach dem Algorithmus ist nun $\begin{pmatrix}1 & 1& 1& 1& 1\end{pmatrix}$ die gesuchte Matrix $H$ mit Sol$(H,0) = C$.

\textbf{d)}
Für $c \in K^{5\times 1} = \mathbb{F}_2^{5\times 1}$ gilt also $Hc = 0 \,\Longleftrightarrow\,c \in C$.
Da durch $H$ nur einfach alle Einträge ausummiert werden, lässt sich sagen, dass für $c \in C$ stets
$\sum_{i=1}^{5} c_i = 0$ gilt. Da wir über $\mathbb{F}_2$ rechnen, ist dies analog dazu, dass $c$ eine gerade
Anzahl an 1 besitzt.

\textbf{e)}
Nach 4.30 haben wir $d(C) = 2$, also $\frac{d(C)}{2} = 1$. Wir können also alle 1-Fachen Fehler erkennen, da dann
das Produkt $Hc' = 1$ für ein fehlerhaftes Codewort $c'$ ist. Jedoch können wir weniger als $\frac{d(C)}{2} = 1$,
also gar keine Fehler korrigieren. Beispielsweise ist
$$
    d(\vect{1}{1}{1}{0}{0}, \vect{1}{1}{0}{0}{0}) = d(\vect{1}{1}{1}{0}{0}, \vect{1}{1}{1}{1}{0}) = 1
$$
Es gibt also keinen eindeutigen nächsten Nachbarn in $C$ für $\vect{1}{1}{1}{0}{0}$.

\end{document}
