\documentclass[a4paper,graphics,11pt]{article}
\pagenumbering{arabic}

\usepackage[margin=1in]{geometry}
\usepackage[utf8]{inputenc}
\usepackage[T1]{fontenc}
\usepackage{lmodern}
\usepackage[ngerman]{babel}
\usepackage{amsmath, tabu}
\usepackage{amsthm}
\usepackage{amssymb}
\usepackage{complexity}
\usepackage{mathtools}
\usepackage{setspace}
\usepackage{graphicx,color,curves,epsf,float,rotating}
\usepackage{tasks}
\setlength{\parindent}{0em}
\setlength{\parskip}{1em}

\newcommand{\aufgabe}[1]{\subsection*{Aufgabe #1}}
\newcommand{\up}[2]{\mathrel{\overset{\makebox[0pt]{\mbox{\normalfont\tiny #2}}}{#1}}}

\begin{document}
\noindent Gruppe \fbox{\textbf{9}}             \hfill Phil Pützstück, 377247\\
\noindent Lineare Algebra für Informatiker\\
\begin{center}
	\LARGE{\textbf{Hausaufgabe 2}}
\end{center}
\begin{center}
\rule[0.1ex]{\textwidth}{1pt}
\end{center}



\aufgabe{10}
\textbf{a)}

Sei $Y := \{(1,x) \mid x \in [0,1]_{\mathbb{Q}}\}$. Da $[0,1]_{\mathbb{Q}}$ nicht endlich ist, ist $Y$ dies ebenfalls nicht.\\
Damit ist $Y$ eine undendliche Teilmenge von $\mathbb{Q}^2$.
Ferner seien nun $v,v' \in Y$ mit $v \neq v'$ gegeben. Dann sind $v = (1,x), v' = (1,x')$ für $x,x' \in Q$.
Da $v \neq v'$ folgt sofort $x \neq x'$.\\
Seien nun $a,b \in Q$. Dann folgt
$$
    av+bv' = 0
    \,\Longleftrightarrow\, (a,ax) = (b,bx')
    \,\Longleftrightarrow\, a = b \land a(x-x') = 0
    \,\Longleftrightarrow\, a = b = 0
$$
Folglich sind $v,v'$ linear unabhängig.

\textbf{b)}

Wir führen Induktion:
Sei $n = 1$. Da $f_1(1) = 1 \neq 0$ folgt auch $f_1 \neq 0$, also $(f_1)$ l.u.\\
Sei nun $n \in \mathbb{N}$ mit $(f_i)_{i \in [1,n]}$ ist l.u, gegeben. Dann gilt nach Definition, dass
$$
    \forall a \in \mathbb{Q}^n : (\sum_{1}^{n} a_if_i)(n+1) = \sum_{1}^{n} a_if_i(n+1) = \sum_{1}^{n} a_i0 = 0
$$
Folglich kann es keine Linearkombination $f$ von $(f_i)_{i\in [1,n]}$ geben, sodass $f(n+1) = n+1$.
Da aber $f_{n+1}(n+1) = n+1 \neq 0$ ist also auch $(f_i)_{i \in [1,n+1]}$ linear unabhängig.

Nach dem Prinzip der vollst. Induktion folgt also, dass $(f_i)_{i \in [1,n]}$ für alle $n \in \mathbb{N}$ l.u. ist.
Dies ist nach 1.72 äquivalent dazu, dass $\{f_i \mid i \in \mathbb{N}\}$ l.u. ist.

\aufgabe{11}
Trivial. (Dies sei dem Leser zur Übung überlassen)




\end{document}
