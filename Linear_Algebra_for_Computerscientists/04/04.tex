\documentclass[a4paper,graphics,11pt]{article}
\pagenumbering{arabic}

\usepackage[margin=1in]{geometry}
\usepackage[utf8]{inputenc}
\usepackage[T1]{fontenc}
\usepackage{lmodern}
\usepackage[ngerman]{babel}
\usepackage{amsmath, tabu}
\usepackage{amsthm}
\usepackage{amssymb}
\usepackage{complexity}
\usepackage{mathtools}
\usepackage{setspace}
\usepackage{graphicx,color,curves,epsf,float,rotating}
\usepackage{tasks}
\usepackage{delarray}
\setlength{\parindent}{0em}
\setlength{\parskip}{1em}

\newcommand{\aufgabe}[1]{\subsection*{Aufgabe #1}}
\newcommand{\up}[2]{\mathrel{\overset{\makebox[0pt]{\mbox{\normalfont\tiny #2}}}{#1}}}

\begin{document}
\noindent Gruppe \fbox{\textbf{9}}             \hfill Phil Pützstück, 377247\\
\noindent Lineare Algebra für Informatiker\\
\begin{center}
	\LARGE{\textbf{Hausaufgabe 4}}
\end{center}
\begin{center}
\rule[0.1ex]{\textwidth}{1pt}
\end{center}



\aufgabe{22}
Sei $s = (s_1, s_2, s_3)$ mit
$s_1 = \begin{pmatrix}
    1\\
    1\\
    -1
\end{pmatrix}$,
$s_2 = \begin{pmatrix}
    0\\
    1\\
    t
\end{pmatrix}$,
$s_3 = \begin{pmatrix}
    0\\
    1\\
    1
\end{pmatrix}$.

Für $t \in \mathbb{R}$ mit $t\neq 1$ ist $s$ eine Basis von $\mathbb{R}^{3\times 1}$:
$$
    \begin{pmatrix}
        1 & 0 & 0\\
        1 & 1 & 1\\
        -1 & t & 1
    \end{pmatrix}
    \to
    \begin{pmatrix}
        1 & 0 & 0\\
        0 & 0 & 1\\
        -2 & t-1 & 1
    \end{pmatrix}
    \to
    \begin{pmatrix}
        1 & 0 & 0\\
        0 & 0 & 1\\
        -2 & t-1 & 1
    \end{pmatrix}
    \to
    \begin{pmatrix}
        1 & 0 & 0\\
        0 & 0 & 1\\
        0 & 1 & 0
    \end{pmatrix}
$$
Also insbesondere
$$
    \mathcal{B}_1 = s_1 + \frac{2}{t-1} s_2 -(1 + \frac{2}{t-1})s_3
    \qquad
    \mathcal{B}_2 = \frac{-1}{t-1} s_2 + (1+\frac{1}{t-1})s_3
$$$$
    \mathcal{B}_3 = \frac{1}{t-1} s_2 + \frac{-1}{t-1}s_3
$$
Daraus ergibt sich auch sofort die Basiswecheselmatrix von $s$ nach $\mathcal{B}$ für $t\neq 1$:
$$
    M_{s,\mathcal{B}}(\text{id}_{\mathbb{R}^{3\times 1}}) =
    \begin{pmatrix}
        1 & 0 & 0\\[5pt]
        \frac{2}{t-1} & \frac{-1}{t-1} & \frac{1}{t-1}\\[5pt]
        -(1+\frac{2}{t-1}) & 1+\frac{1}{t-1} & \frac{-1}{t-1}
    \end{pmatrix}
$$
Für $t\neq 1$ ist $s$ eine Basis und dann $\varphi_t$ eindeutig durch die Abbildungsvorschrift,
also die Bilder dar Basiselemente von $s$ bestimmt. Folglich existiert $\varphi_t$ für
alle $t \in \mathbb{R}$ mit $t \neq 1$. Weiter gilt nach Proposition 3.18, dass
$$
    M_{\mathcal{B}, \mathcal{B}}(\varphi_t)
    = M_{\mathcal{B}, s}(\text{id}_{\mathbb{R}^{3\times 1}})
        M_{s,s}(\varphi_t)
        M_{s, \mathcal{B}}(\text{id}_{\mathbb{R}^{3\times 1}})
$$
Analog gilt
$$
    M_{\mathcal{B},s}(\varphi_t)
    = M_{\mathcal{B}, s}(\text{id}_{\mathbb{R}^{3\times 1}})
        M_{s,s}(\varphi_t)
        M_{s,s}(\text{id}_{\mathbb{R}^{3\times 1}})
    =  M_{\mathcal{B}, s}(\text{id}_{\mathbb{R}^{3\times 1}})
        M_{s,s}(\varphi_t)
        E_3
    = M_{\mathcal{B}, s}(\text{id}_{\mathbb{R}^{3\times 1}})
        M_{s,s}(\varphi_t)
$$
Wir können also schreiben
$$
    M_{\mathcal{B}, \mathcal{B}}(\varphi_t)
    = M_{\mathcal{B}, s}(\varphi_t)
        M_{s, \mathcal{B}}(\text{id}_{\mathbb{R}^{3\times 1}})
$$
$M_{\mathcal{B}, s}(\varphi_t)$ ist offensichtlich analog zu Bsp. 3.19 gegeben durch die Abbildungsvorschrift:
$$
M_{\mathcal{B}, s}(\varphi_t) =
    \begin{pmatrix}
        -1 & 1 & 7\\
        2 & 2 & 2\\
        -t-4 & -1 & 2t+7
    \end{pmatrix}
$$
\newpage
Damit können wir nun $M_{\mathcal{B}, \mathcal{B}}(\varphi_t)$ ausrechnen:
$$
    M_{\mathcal{B}, \mathcal{B}}(\varphi_t)
    = M_{\mathcal{B}, s}(\varphi_t)
        M_{s, \mathcal{B}}(\text{id}_{\mathbb{R}^{3\times 1}})
$$$$
    = \begin{pmatrix}
        -1 & 1 & 7\\
        2 & 2 & 2\\
        -t-4 & -1 & 2t+7
    \end{pmatrix}
    \begin{pmatrix}
        1 & 0 & 0\\[5pt]
        \frac{2}{t-1} & \frac{-1}{t-1} & \frac{1}{t-1}\\[5pt]
        -(1+\frac{2}{t-1}) & 1+\frac{1}{t-1} & \frac{-1}{t-1}
    \end{pmatrix}
$$$$
    = \frac{1}{t-1} \begin{pmatrix}
        -8t-4 & 7t-1 & -6\\
        0 & 2(t-1) & 0\\
        -3t^2-12t-5 & 2t^2+7t+1 & -2t-8
    \end{pmatrix}
$$

\aufgabe{23}

Im folgenden steht $V(x)$ repräsentativ für Elemente aus $V$ angewandt auf $x$.
Da Elemente von $V$ nur durch ihr Bild beschrieben sind ($v(x) = \cdots$),
dieses Bild von $x$ unter $v$ jedoch nicht selbst Element von $V$ ist, wollen wir mit $v(x) \in V(x)$ ausdrücken, dass $v \in V$.

Seien $v,v' \in V$ und $a,a' \in \mathbb{R}^5$ mit $v(x) = \sum_{i=0}^{4} a_ix^i$ sowie $v'(x) = \sum_{i=0}^{4} a_i' x^i$. Es gilt
$$
    v(x)+v'(x)
    = \sum_{i=0}^{4} a_ix^i + \sum_{i=0}^{4} a_i'x^i
    = \sum_{i=0}^{4} (a_i+a_i')x^i \in V(x)
$$
Weiter sei $c \in \mathbb{R}$. Damit ist
$$
    cv(x)
    = c\sum_{i=0}^{4} a_ix^i
    = \sum_{i=0}^{4} (ca_i)x^i \in V(x)
$$
Ferner ist
$$
    0
    = \sum_{i=0}^{4} 0
    = \sum_{i=0}^{4} 0\cdot x^i \in V(x)
$$
Folglich ist $V$ ein Untervektorraum von Map$(\mathbb{R}, \mathbb{R})$.

Wir wissen aus der Analysis, dass für relle Funktionen $f,g \in \text{Map}(\mathbb{R}, \mathbb{R})$ und $c \in \mathbb{R}$ stets gilt:
\begin{equation}
    (f+g)' = f'+g' \in \text{Map}(\mathbb{R}, \mathbb{R})
    \qquad\text{sowie}\qquad
    (cf)' = cf' \in \text{Map}(\mathbb{R}, \mathbb{R})
\end{equation}
Damit ist die Ableitung ein Endomorphismus von Map$(\mathbb{R}, \mathbb{R})$. Dies ist offensichtlich analog für höhere Ableitungen.

Seien nun $f,g \in V$ sowie $c \in \mathbb{R}$ gegeben. Wir haben
$$
    \varphi(f+g)(x)
    = (f+g)''(x) + x(f+g)'(x) - (f+g)(x+1)
$$$$
    \up{=}{1} f''(x) + g''(x) + x(f'(x)+g'(x)) - (f(x+1)+g(x+1))
$$$$
    = f''(x) + xf'(x) - f(x+1) + g''(x) +xg'(x) - g(x+1)
    = \varphi(f)(x) + \varphi(g)(x)
$$
Weiter gilt
$$
    \varphi(cf)(x)
    = (cf)''(x) + x(cf)'(x) - (cf)(x+1)
$$$$
    \up{=}{1} cf''(x) + cxf'(x) - cf(x+1)
    = c(f''(x)+f'(x)-f(x+1))
    = c\varphi(f)(x)
$$
Folglich ist $\varphi$ ein Endomorhphismus auf $V$, also linear.

\newpage

\textbf{b)}

Sei $\mathcal{B} \in V^5$ mit
$$
    \mathcal{B}_0(x) = 1
    \qquad
    \mathcal{B}_1(x) = x
    \qquad
    \mathcal{B}_2(x) = x^2
    \qquad
    \mathcal{B}_3(x) = x^3
    \qquad
    \mathcal{B}_4(x) = x^4
$$
Damit haben wir für $v \in V$ mit $v(x) = \sum_{i=0}^{4} a_ix^i$, dass $v = \sum_{i=0}^{4} a_i\mathcal{B}_i$
und damit eine Linearkombination über $\mathcal{B}$ ist. Foglich ist $\mathcal{B}$ ein EZS von $V$.
Weiter gilt für $a \in \mathbb{R}^5$
$$
    \left(\sum_{i=0}^{4} a_i\mathcal{B}_i\right)(x) = 0
    \,\Longrightarrow\, a_0 + a_1x + a_2x^2 + a_3x^3 + a_4x^4 = 0
    \,\Longrightarrow\, \forall i \in [0,4] : a_i = 0
$$
Folglich ist $\mathcal{B}$ linear unabhängig und damit eine Basis von $V$.
Es gilt nach Definition der Darstellungsmatrix dass
$$
    M_{\mathcal{B}, \mathcal{B}}(\varphi) = (\kappa_\mathcal{B}(\varphi(\mathcal{B}_i)))_{i \in [0,4]}
$$
Wir haben
$$
    \varphi(\mathcal{B}_0)(x) = 0
    \qquad
    \varphi(\mathcal{B}_1)(x) = 0 + x - (x+1) = -1
$$$$
    \varphi(\mathcal{B}_2)(x) = 1 + x^2 - (x+1)^2 = -2x
    \qquad
    \varphi(\mathcal{B}_3)(x) = x + x^3 - (x+1)^3 = -3x^2-2x-1
$$$$
    \varphi(\mathcal{B}_4)(x) = x^2 + x^4 - (x+1)^4 = -4x^3-5x^2-4x-1
$$
und damit
$$
    M_{\mathcal{B}, \mathcal{B}}(\varphi)
    = -\begin{pmatrix}
        0 & 1 & 0 & 1 & 1\\
        0 & 0 & 2 & 2 & 4\\
        0 & 0 & 0 & 3 & 5\\
        0 & 0 & 0 & 0 & 4\\
        0 & 0 & 0 & 0 & 0
    \end{pmatrix}
$$

\textbf{c)}

Nach Korrolar 3.39 gilt $\text{Sol}(M_{\mathcal{B}, \mathcal{B}}(\varphi), 0) = \kappa_{\mathcal{B}}(\text{Ker}\ \varphi)$. Also:
$$
    \begin{array}({@{}ccccc|c@{}})
        0 & -1 & 0 & -1 & -1 & 0\\
        0 & 0 & -2 & -2 & -4 & 0\\
        0 & 0 & 0 & -3 & -5 & 0\\
        0 & 0 & 0 & 0 & -4 & 0\\
        0 & 0 & 0 & 0 & 0 & 0\\
    \end{array}
    \xrightarrow {\forall \cdot(-1)}
    \begin{array}({@{}ccccc|c@{}})
        0 & 1 & 0 & 1 & 1 & 0\\
        0 & 0 & 2 & 2 & 4 & 0\\
        0 & 0 & 0 & 3 & 5 & 0\\
        0 & 0 & 0 & 0 & 4 & 0\\
        0 & 0 & 0 & 0 & 0 & 0\\
    \end{array}
    \xrightarrow {\text{II}\cdot \frac{1}{2},\ \text{IV}\cdot \frac{1}{4}}
    \begin{array}({@{}ccccc|c@{}})
        0 & 1 & 0 & 1 & 1 & 0\\
        0 & 0 & 1 & 1 & 2 & 0\\
        0 & 0 & 0 & 3 & 5 & 0\\
        0 & 0 & 0 & 0 & 1 & 0\\
        0 & 0 & 0 & 0 & 0 & 0\\
    \end{array}
$$$$
    \xrightarrow {(\text{I},\ \text{II},\ \text{III})\ -\ \text{IV}(1, 2, 5)}
    \begin{array}({@{}ccccc|c@{}})
        0 & 1 & 0 & 1 & 0 & 0\\
        0 & 0 & 1 & 1 & 0 & 0\\
        0 & 0 & 0 & 3 & 0 & 0\\
        0 & 0 & 0 & 0 & 1 & 0\\
        0 & 0 & 0 & 0 & 0 & 0\\
    \end{array}
    \xrightarrow {\text{III}\cdot \frac{1}{3},\ (\text{I},\ \text{II})\ -\ \text{III}(1,1)}
    \begin{array}({@{}ccccc|c@{}})
        0 & 1 & 0 & 0 & 0 & 0\\
        0 & 0 & 1 & 0 & 0 & 0\\
        0 & 0 & 0 & 1 & 0 & 0\\
        0 & 0 & 0 & 0 & 1 & 0\\
        0 & 0 & 0 & 0 & 0 & 0\\
    \end{array}
$$
Damit haben wir durch lösen des homogenen LGS und nach Korrolar 3.39
$$
    \text{Ker}\ \varphi
    = \{\sum_{i=0}^{4} a_i\mathcal{B}_i \mid a \in \text{Sol}(M_{\mathcal{B},\mathcal{B}}(\varphi), 0)\}
    = \{\sum_{i=0}^{4} a_i\mathcal{B}_i \mid a \in \mathbb{R}
    \begin{pmatrix}
        1\\
        0\\
        0\\
        0\\
        0
    \end{pmatrix}
    \}
$$$$
    = \{a\mathcal{B}_0 \mid a \in \mathbb{R}\}
    = \langle \mathcal{B}_0 \rangle
$$
Damit ist $(\mathcal{B}_0)$ eine Basis von Ker $\varphi$.
\newpage

\textbf{d)}

Es gilt
$   \kappa_{\mathcal{B}}(g)
    = \begin{pmatrix}
        1\\
        -1\\
        0\\
        2\\
        3
    \end{pmatrix}
$.
Nach Bemerkung 3.38 ist
$\kappa_{\mathcal{B}}(\varphi^{-1}(\{g\}))
= \text{Sol}(M_{\mathcal{B}, \mathcal{B}}(\varphi), \kappa_{\mathcal{B}}(g))$\\
Jedoch gibt es keine Lösung für dieses LGS, die Letzte Zeile lässt sich nicht lösen:
$$
\text{Sol}(\begin{array}({@{}ccccc|c@{}})
        0 & -1 & 0 & -1 & -1 & 1\\
        0 & 0 & -2 & -2 & -4 & -1\\
        0 & 0 & 0 & -3 & -5 & 0\\
        0 & 0 & 0 & 0 & -4 & 2\\
        0 & 0 & 0 & 0 & 0 & 3\\
\end{array}
) \neq \varnothing
\,\Longrightarrow\, \exists a \in \mathbb{R}^5 : \sum_{i=0}^{4} a_i\cdot 0 = 0 = 3
$$
Es folgt $\varphi^{-1}(\{g\}) = \varnothing$.
\end{document}
