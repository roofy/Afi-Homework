\documentclass[12pt]{article}

\usepackage[margin=1in]{geometry} 
\usepackage{amsmath,amsthm,amssymb}
\usepackage{cite}
\usepackage{forest}

\newcommand{\up}[2]{\mathrel{\overset{\makebox[0pt]{\mbox{\normalfont\tiny #2}}}{#1}}}
\newcommand{\T}[0]{{\mathcal{T}_r^h}}
\newcommand{\pre}[0]{\text{pre}}
\newenvironment{statement2}[2]{\begin{trivlist}
\item[\hskip \labelsep {\bfseries #1}\hskip \labelsep {\bfseries #2}]}{\end{trivlist}}
\newenvironment{statement3}[3]{\begin{trivlist}
\item[\hskip \labelsep {\bfseries #1}\hskip \labelsep {\bfseries #2} {#3}\textbf{.}]}{\end{trivlist}}

\begin{document}

\title{Kraft's and McMillan's Inequalities}
\author{Phil Pützstück, 377247\\
Proseminar Informationstheorie}

\maketitle
\begin{statement2}{(1.0)}{Assumptions} \strut
    \begin{itemize}
        \item Basic Graph-theory (Trees, acyclic, directed, height, etc.).
        \item Familiarity with instantaneous Codes, as defined in \cite{ICT}.
    \end{itemize}
\end{statement2}

\begin{statement3}{(1.1)}{Definition}{(r-ary Trees from r-ary Codes)}\strut\\[2pt]
    Let $q,r \in \mathbb{N}, [0,r-1]$ be the Code-Alphabet for some $r$-ary Code $\mathcal{C}$ with word-lengths
    $l \in \mathbb{N}^q$.\\
    Set $h := \max\{l_i \mid i \in [1,q]\}$.
    Define $W := \bigcup_{i \in [0,h]} [0,r-1]^i$ to be the set of all words over $T$ of maximum length $h$. Thus
    $\mathcal{C} \subseteq W$. We define a rooted r-ary tree as a directed graph:
    $$
        V := \{v_w \mid w \in W\}\qquad E := \{(v_w, v_{w'}) \mid v_w,v_{w'} \in V \land w' = wx, x \in [0,r-1]\}
    $$
    Which means we have a vertex for each word in $W$, and an edge from $v_w$ to $v_{w'}$ iff $w$ is a prefix of $w'$
    with $|w| = |w'|-1$.
    We set $\T := (V,E)$ as the rooted $r$-ary tree of height $h$. The root $R(\T)$ is given by $v_\varepsilon$,
    since $\varepsilon \in T^0 \subseteq W$ and $\varepsilon \sqsubseteq w$ for all $w \in W$.
    We denote $V(\T) := V$ and $E(\T) := E$.
    We can easily define the height $\mathcal{H}_{\T}(v_w) := |w| - |\mathcal{W}(R(\T))|$, where
    $\mathcal{W}(v_w) = w$,
    which defines the height of a vertex $v_w \in V(\T)$ as the length of $w$ minus the length of the word at
    the root of the tree, which is usually $|\varepsilon| = 0$, but can be different.
\end{statement3}

\begin{statement3}{(1.2)}{Definition}{(Subtrees and Ordering)}\strut\\[2pt]
    Let $h,r \in \mathbb{N}, v_w, v_{w'} \in V(\T)$.
    We say $T$ is a rooted subtree of $\T$, written $T \leq \T$, iff $V(T) \subseteq V(\T), E(T) \subseteq E(\T)$
    and $T$ fullfills the standard criteria of a rooted (directed) tree.\\[5pt]
    We say $T$ is a rooted $r$-ary subtree of $\T$, written $T \leq_r \T$ iff $T \leq \T$ and $T$ is $r$-ary.\\[5pt]
    We write $v_w \leq v_{w'}$ iff $w \sqsubseteq w'$. Now let $T \leq \T$ be a rooted subtree and
    $v_w \in T \setminus \{R(T)\}$.\\
    $$
        V := \{v \in V(T) \mid v_w \leq v\} = \{v_{w'} \in V(T) \mid w \sqsubseteq w'\}
        \quad E := \{(v,v') \in V(T) \mid v,v' \in V\}
    $$
    If we have $(V,E) \leq_r \T$, meaning the Graph $(V,E)$ is a rooted $r$-ary subtree of $\T$, then we define
    $$
        \mathcal{U}_T(v_w) := \mathcal{U}_\T(v_w) = (V,E)\quad\text{and}\quad
        T \setminus v_w := T \setminus \mathcal{U}_T(v_w) = (V(T) \setminus V, E(T) \setminus E)
    $$
\end{statement3}

\newpage
\begin{statement2}{(1.3)}{Examples}\strut\\[5pt]
    $\mathcal{T}_3^2$ is given by
    \begin{center}
        \begin{forest}
            [$R(\mathcal{T}_3^2)\text{=}v_\varepsilon$
                [$v_0$
                    [$v_{00}$],
                    [$v_{01}$],
                    [$v_{02}$]
                ],
                [$v_1$
                    [$v_{10}$],
                    [$v_{11}$],
                    [$v_{12}$]
                ],
                [$v_2$
                    [$v_{20}$],
                    [$v_{21}$],
                    [$v_{22}$]
                ]
            ],
        \end{forest}\\
    \end{center}
    We have $\mathcal{H}_{\mathcal{T}_3^2}(v_\varepsilon) = 0$ and $\mathcal{H}_{\mathcal{T}_3^2}(v_{12}) = 2$.
    $v_0 \leq v_{02}$ holds, $v_0 \leq v_{10}$ does \textbf{not}.\\
    The subtree $\mathcal{U}_{\mathcal{T}_3^2}(v_1)$ and $\mathcal{T}_3^2 \setminus v_1$
    are given by:\\
    \begin{minipage}{.4\textwidth}
        \begin{center}
            \begin{forest}
                [$R(\mathcal{U}_{\mathcal{T}_3^2}(v_1))\text{=}v_1$
                    [$v_{10}$],
                    [$v_{11}$],
                    [$v_{12}$]
                ]
            \end{forest}
        \end{center}
    \end{minipage}
    \begin{minipage}{.4\textwidth}
        \begin{center}
            \begin{forest}
                [$R(\mathcal{T}_3^2 \setminus v_1)\text{=}v_\varepsilon$
                    [$v_0$
                        [$v_{00}$],
                        [$v_{01}$],
                        [$v_{02}$]
                    ],
                    [$v_2$
                        [$v_{20}$],
                        [$v_{21}$],
                        [$v_{22}$]
                    ]
                ],
            \end{forest}\\
        \end{center}
    \end{minipage}\\[10pt]
    Note $\mathcal{U}_{\T}(v_1)$ is a 3-ary rooted subtree of height $1$, but $\mathcal{T}_3^2 \setminus v_1$ is only
    a rooted subtree of height $3$, not $r$-ary for any $r \in \mathbb{N}$.
    We now have $\mathcal{H}_{\mathcal{U}_{\mathcal{T}_3^2}}(v_1) = 0,
    \mathcal{H}_{\mathcal{U}_{\mathcal{T}_3^2}}(v_{12}) = 1$, but still\\
    $\mathcal{H}_{\mathcal{T}_3^2 \setminus v_1}(v) = \mathcal{H}_{\mathcal{T}_3^2}(v)$ for $v \in V(\mathcal{T}_3^2 \setminus v_1)$.
\end{statement2}

\begin{statement3}{(1.4)}{Proposition}{(Number of nodes in rooted r-ary subtrees)}\strut\\[2pt]
    Let $h,r \in \mathbb{N}, T \leq_r \T$ be a rooted $r$-ary subtree of $\T$ with height $h' \leq h$.
    Then T has exactly $r^l$ vertices of height $l$ for $l \in [0,h']$.
    \begin{proof}
        Left as exercise for the reader.
    \end{proof}
\end{statement3}

\begin{statement3}{(1.5)}{Corollary}{(Number of leafs of $T \setminus v$)}\strut\\[2pt]
    Let $h,r \in \mathbb{N}, T \leq \T, v_w \in V(T) \setminus \{R(\T)\}$ such that
    $\mathcal{U}_T(v_w)$ is well defined, in particular $r$-ary.
    Let $L \leq r^h$ be the number of leaves of $T$. Then $T \setminus v_w$ has $L - r^{h-|w|}$
    leaves.
    \begin{proof}
        Since $\mathcal{U}_{T}(v_w)$ has height $h-\mathcal{H}_\T(v_w) = h - |w|$,
        we know $\mathcal{U}_\T(v)$ has $r^{h-|w|}$ leaves by (1.4).
        Thus $T \setminus v_w = T \setminus \mathcal{U}_T(v_w)$ has $L - r^{h-|w|}$ leaves.
    \end{proof}
\end{statement3}

\begin{statement3}{(1.6)}{Theorem}{(Kraft's Inequality)}\strut\\[2pt]
    Let $q,r \in \mathbb{N}, l \in \mathbb{N}^q$. Then there is an instantaneous $r$-ary Code $\mathcal{C}$
    with word-lengths $l$ iff
    \begin{equation}
        \sum_{k=1}^{q} \frac{1}{r^{l_k}} \leq 1
    \end{equation}

    \begin{proof}
        If $q = 1$, then we always have an instantaneous Code, and since $r \in \mathbb{N}$, (1) always holds
        as well.
        So assume WLOG that $q > 1$ and $\forall i \in [1,q-1]: 0 < l_i \leq l_{i+1}$.
        Furthermore we can assume WLOG that the Code-Alphabet of $\mathcal{C}$ is $[0,r-1]$, since any
        other Alphabet of length $r$ is in bijection to this.\\[10pt]
        We first show that (1) implies the existence of an $r$-ary prefix-Code, which by
        \cite{ICT} is instantaneous.
        Set $h := l_q$ to be the maximum length of the supposed Code-words of $\mathcal{C}$.
        Thus we should have, like in (1.1), that $\mathcal{C} \subseteq \bigcup_{i \in [0,h]} [0,r-1]^i =: W$, where
        $W$ is in bijection with $V(\T)$.
        So we construct the Code-words $w_i$ of the prefix-Code $\mathcal{C}$, with $|w_i| = l_i$ for $i \in [1,q]$
        via finite induction over $i$.\\[5pt]
        Let $i = 1$. Choose a Code-word $w_1 \in [1,r]^{l_1}$ of length $l_1$. Since $w_1 \in W$ and $l_1 > 0$ we have
        $v_{w_1} \in V(\T) \setminus \{R(\T)\}$. Define $\mathcal{T}_1 := \T \setminus v_{w_1}$. We know
        from (1.5) that $\mathcal{T}_1$ has
        $$
            r^h - r^{h - l_1} = r^h\left(1 - \sum_{k=1}^{1} \frac{1}{r^{l_k}}\right)
            \up{>}{$q > 1$} r^h\left(1 - \sum_{k=1}^{q} \frac{1}{r^{l_k}}\right) \up{\geq}{(1)} 0
        $$
        leaves. Now let $i \in [1,q-1]$ such that $\mathcal{C} := \{w_j \mid j \in [1,i]\}$ is a prefix-Code
        with $|w_j| = l_j$\\[2pt]
        for $j \in [1,i]$ and such that $\mathcal{T}_i$ is a rooted subtree of $\T$ and has $r^h(1 - \sum_{k=1}^{i}\frac{1}{r^{l_k}}) > 0$ leaves.
        Then since $l_{i+1} \leq l_q = h$ we know that there must also be at least one vertex
        $v_w \in V(\mathcal{T}_i)$ with $\mathcal{H}_{\mathcal{T}_i}(v_w) = \mathcal{H}_{\T}(v_w) = l_{i+1}
        \,\Longrightarrow\, |w| = l_{i+1}$ (since trees are connected). So set $w_{i+1} := w$. If we had $w_j \sqsubseteq w_{i+1}$ for
        some $j \in [1,i]$, then it would follow that $v_{w_j} \leq v_{w_{i+1}}$, but then we would have
        $v_{w_{i+1}} \notin V(\mathcal{T}_{j}) \subseteq V(\mathcal{T}_{i})$, a contradiction.
        Thus $\mathcal{C} := \{w_j \mid j \in [1, i+1]\}$ is still a prefix-Code.
        If $i+1 = q$ we are done, as we have constructed the desired prefix-Code. Otherwise,
        we set $\mathcal{T}_{i+1} := \mathcal{T}_i \setminus w_{i+1}$ and we get for the number of leaves:
        $$
            r^h\left(1-\sum_{k=1}^{i} \frac{1}{r^{l_k}}\right) - r^{h - l_{i+1}}
            = r^h\left(1-\sum_{k=1}^{i+1} \frac{1}{r^{l_k}}\right)
            > r^h\left(1-\sum_{k=1}^{q} \frac{1}{r^{l_k}}\right)
            \up{\geq}{(1)} 0
        $$
        Thus we constructed the desired prefix-Code $\mathcal{C}$ by finite induction.\\[10pt]
        Now we show the existence of a instantaneous $r$-ary Code $\mathcal{C}$ with word-lengths $l$ implies (1).
        We know from \cite{ICT} that $\mathcal{C}$ is a prefix-Code. Let
        $$
            L_i := \{v_w \in V(\T) \mid w_i \sqsubseteq w \land |w| = h\}
            = \{v_w \in \mathcal{U}_\T(v_{w_i}) \mid \mathcal{H}_\T(v_w) = h - |w_i|\}
        $$
        be the set of leaves in $\mathcal{U}_\T(v_{w_i})$, where $w_i \in \mathcal{C}$ with $|w_i| = l_i$ for $i \in [1,q]$. We know from (1.4) that $|L_i| = r^{h - l_i}$ 
        for $i \in [1,q]$, as we have $\mathcal{H}_{\mathcal{U}_\T(v_{w_i})}(v_w) = h - l_i$ for $v_w \in L_i$.
        Furthermore we know that for each $i\neq j \in [1,q]$ $L_i \cap L_j = \varnothing$:\\
        Assume $i,j \in [1,q]$ and WLOG $i < j$. Let $v_w \in L_i \cap L_j$. Thus we get
        $$
            v_{w_i} \leq v_w \land v_{w_j} \leq v_w \,\Longrightarrow\, w_i \sqsubseteq w \land w_j \sqsubseteq w
            \,\Longrightarrow\, w_i \sqsubseteq w_j
        $$
        which is a contradiction to the fact that $\mathcal{C}$ is a prefix-Code.
        So now, since $\T$ only has $r^h$ leafs, we have
        $$
            r^h \geq |\bigcup_{i \in [1,q]} L_i| = \sum_{i = 1}^{q} |L_i| = \sum_{i=1}^{q} r^{h-l_i}
            = r^h\sum_{i=1}^{q} \frac{1}{r^{l_i}}
            \quad\,\Longleftrightarrow\,\quad \sum_{i=1}^{q} \frac{1}{r^{l_i}} \leq 1
        $$
    \end{proof}
\end{statement3}

\newpage

\begin{statement3}{(1.7)}{Theorem}{(McMillan's Inequality)}\strut\\[2pt]
    Let $q,r \in \mathbb{N}, l \in \mathbb{N}^q$. Then there is an uniquely decodable
    $r$-ary Code $\mathcal{C}$ iff
    \begin{equation}
        \sum_{i=1}^{q} \frac{1}{r^{l_i}} \leq 1 \tag{1}
    \end{equation}

    \begin{proof}
        If we assume (1), then by Kraft's Inequality we know that $\mathcal{C}$
        is instantaneous, which by \cite{ICT} implies unique decodability.\\[10pt]
        Now assume that $\mathcal{C}$ is a unique decodable $r$-ary Code with word-lengths
        $l$.\\
        Let
        $
            K := \sum_{i=1}^{q} \frac{1}{r^{l_i}}
        $ and $n \in \mathbb{N}$.
        Then we have
        \begin{equation}
            K^n
            = \left(\sum_{i=1}^{q} \frac{1}{r^{l_i}}\right)^n
            = \sum_{i \in [1,q]^n}\prod_{k=1}^{n} \frac{1}{r^{l_{i_k}}}
            = \sum_{i \in [1,q]^n} r^{-\sum_{k=1}^{n} l_{i_k}} \tag{2}
        \end{equation}
        where the $i \in [1,q]^n$ represents $n$ choices of $q$ possible summands (with repitition).\\[10pt]
        Now there are many different $i \in [1,q]^n$ which have the same sum $\sum_{k=1}^{n} l_{i_k}$
        (consider permutations for example). Set $M := \max\{l_k\mid k \in [1,q]\}, m := \min\{l_k \mid k \in [1,q]\}$.
        Then we get $mn \leq \sum_{k=1}^{n} l_{i_k} \leq Mn$ for all $i \in [1,q]^n$ (3). We define for
        $j \in [mn,Mn], p \in [1,j]$:
        $$
            N_{j,p} := \{w_{i_1}w_{i_2}\cdots w_{i_p} \mid i \in [1,q]^n \land |w_{i_1}\cdots w_{i_n}| = j \}
        $$
        So $t \in N_{p,j}$ is a Code-sequence of length $j$, consisting of $p$ Code-words in $\mathcal{C}$.\\
        But since $\mathcal{C}$ is uniquely decodable, we know that
        $
            \forall t \in N_{j,p}: \exists!\ i \in [1,q]^n: t = w_{i_1}\cdots w_{i_n}
        $,
        meaning there is only one way to construct $t \in N_{p,k}$ from $p$ Code-words of $\mathcal{C}$.\\
        This implies that
        \begin{equation}
            |\{i \in [1,q]^n \mid \sum_{k=1}^{n} l_{i_k} = j\}|
            = |\{i \in [1,q]^n \mid \sum_{k=1}^{n} |w_{i_k}| = j\}| \tag{4}
            = |N_{j,p}|
        \end{equation}
        Furthermore, since $N_{j,p} \subseteq [0,r-1]^j$, we have $|N_{j,p}|$. Thus, from (2), (3), (4) we get
        $$
            K^n = \sum_{j = mn}^{Mn} \frac{|N_{j,n}|}{r^j} \leq \sum_{j = mn}^{Mn} 1 = (l-m)n + 1
            \,\Longrightarrow\, \frac{K^n}{n} \leq (M-m) + \frac{1}{n}
        $$
        Now $M,m,K$ are fixed, while $n$ may be arbitrarily large. From Analysis we know
        that as $n \to \infty$, the only way that $\frac{K^n}{n}$ stays bounded is if $K \leq 1$.
        Thus we get the desired result:
        $$
            \sum_{i=1}^{q} \frac{1}{r^{l_i}} = K \leq 1
        $$
    \end{proof}
\end{statement3}

\newpage
\bibliography{lit}{}
\bibliographystyle{alpha}
\end{document}
