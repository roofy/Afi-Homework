\documentclass{beamer}

\usepackage{xcolor}
\usepackage{amsmath}

\newcommand{\red}[1]{\textcolor{red}{#1}}

\begin{document}
    \begin{frame}<1>[label=defs,t]
        \frametitle{Definitionen}
        Gegeben: Eingabealphabet $\Sigma$, Eingabestring $w \in \Sigma^*$.\\[10pt]
        \pause
        Bezeichne $m \in \mathbb{N}_0 \times \mathbb{N}_0$ als Match. \hfill(startIndex, endIndex)\\
        Definiere $\mathcal{M} := \mathcal{P}(\mathbb{N}_0 \times \mathbb{N}_0)$ als Menge der Mengen
        von Matches.\\[10pt]
        \pause
        Sei nun $Parser$ := $\mathcal{M}^\mathcal{M}$.
        Für $f \in Parser$ ist also $f : \mathcal{M} \to \mathcal{M}$.\\[10pt]
        \pause
        \texttt{readChar}$_w : \Sigma \to Parser$\\
        \texttt{readChar}$_w(\sigma) =$
        \only<4>{
            $M \mapsto \{(i, j+1) \mid (i,j) \in M \land w_j = \sigma\}$
        }
        \pause
        \only<5>{
            $\displaystyle
            \underbrace{M \mapsto \{(i, j+1) \mid (i,j) \in M \land w_j = \sigma\}}_{\in Parser}
            $
        }
        \pause
        $M \mapsto \{(i, j+1) \mid (i,j) \in M \land w_j = c\}$\\[10pt]
        \texttt{concatenate}$_w : Parser \times Parser \to Parser$\\
        \pause
        \texttt{concatenate}$_w(f_1,f_2) = f_2 \circ f_1$\\[10pt]
        \pause
        \texttt{alternate}$_w : Parser \times Parser \to Parser$\\
        \pause
        \texttt{alternate}$_w(f_1,f_2) = M \mapsto f_1(M) \cup f_2(M)$\\[10pt]
        \pause
        \texttt{iterate}$_w : Parser \to Parser$\\
        \texttt{iterate}$_w(f) =$
        \only<10>{$\displaystyle M \mapsto \bigcup_{i \in \mathbb{N}_0} f^i(M)
        = M \mapsto M \cup f(M) \cup f(f(M)) \cdots$}
        \pause
        \only<11>{$\displaystyle M \mapsto \bigcup_{i \in \mathbb{N}_0} f^i(M)$}
    \end{frame}

    \againframe<2>[t]{defs}

    \begin{frame}[t]
        \frametitle{\textbf{Beispiel:} Match}
        Bezeichne $m \in \mathbb{N}_0 \times \mathbb{N}_0$ als Match. \hfill(startIndex, endIndex)\\
        Definiere $\mathcal{M} := \mathcal{P}(\mathbb{N}_0 \times \mathbb{N}_0)$ als Menge der Mengen
        von Matches.\\[5pt]
        Sei $\Sigma = \{a,b\}, w = bbaaabbb \in \Sigma^*$.\\[30pt]
        \begin{center}
            $\underset{0}{|}$
                {\Large b}
            $\underset{1}{|}$
                {\Large b}
            $\underset{2}{|}$
                {\Large a}
            $\underset{3}{|}$
                {\Large a}
            $\underset{4}{|}$
                {\Large a}
            $\underset{5}{|}$
                {\Large b}
            $\underset{6}{|}$
                {\Large b}
            $\underset{7}{|}$
                {\Large b}
            $\underset{8}{|}$\\[30pt]
        \end{center}
        \pause
        Bspw. ist $\{(0,0),(0,8),(3,7)\} \in \mathcal{M}$ eine Menge von Matches mit\\[5pt]
        $(0,0) \approx \red{||}bbaaabbb \approx \varepsilon$\\
        $(0,8) \approx \red{|bbaaabbb|} \approx w$\\
        $(3,7) \approx bba\red{|aabb|}b\approx aabb$\\
    \end{frame}

    \againframe<3>[t]{defs}

    \againframe<4>[t]{defs}

    \againframe<5>[t]{defs}

    \begin{frame}[t]
        \frametitle{\textbf{Beispiel:} \texttt{readChar}}
        \texttt{readChar}$_w : \Sigma \to Parser$\\
        \texttt{readChar}$_w(\sigma) = M \mapsto \{(i, j+1) \mid (i,j) \in M \land w_j = \sigma\}$\\[10pt]
        Sei $\Sigma = \{a,b\},\ w = bbaaabbb \in \Sigma^*,\ f_a := \texttt{readChar}_w(a)$.\\
        \begin{center}
            $\underset{0}{|}$
                {\Large b}
            $\underset{1}{|}$
                {\Large b}
            $\underset{2}{|}$
                {\Large a}
            $\underset{3}{|}$
                {\Large a}
            $\underset{4}{|}$
                {\Large a}
            $\underset{5}{|}$
                {\Large b}
            $\underset{6}{|}$
                {\Large b}
            $\underset{7}{|}$
                {\Large b}
            $\underset{8}{|}$\\[30pt]
        \end{center}
        \pause
        $$
            f_a(\{(0,0), (2,2), (2,4)\}) = \{(2,3), (2,5)\}
        $$
        \pause
        $$
            \approx f_a(\{\red{||}bbaaabbb, bb\red{||}aaabbb, bb\red{|aa|}abbb\})
            $$$$
            = \{bb\red{|a|}aabbb, bb\red{|aaa|}bbb\}
        $$
    \end{frame}

    \againframe<6>[t]{defs}

    \againframe<7>[t]{defs}

    \begin{frame}[t]
        \frametitle{\textbf{Beispiel:} \texttt{concatenate}}
        \texttt{concatenate}$_w : Parser \times Parser \to Parser$\\
        \texttt{concatenate}$_w(f_1,f_2) = f_2 \circ f_1$\\[10pt]
        Sei $\Sigma = \{a,b\},\ w = bbaaabbb \in \Sigma^*,\ f_a := \texttt{readChar}_w(a)$\\
        $f_b := \texttt{readChar}_w(b),\ f_{ab} := \texttt{concatenate}_w(f_a,f_b)$.\\
        \begin{center}
            $\underset{0}{|}$
                {\Large b}
            $\underset{1}{|}$
                {\Large b}
            $\underset{2}{|}$
                {\Large a}
            $\underset{3}{|}$
                {\Large a}
            $\underset{4}{|}$
                {\Large a}
            $\underset{5}{|}$
                {\Large b}
            $\underset{6}{|}$
                {\Large b}
            $\underset{7}{|}$
                {\Large b}
            $\underset{8}{|}$\\[30pt]
        \end{center}
        \pause
        $$
            f_{ab}(\{(0,0), (2,2), (2,4)\})
            = f_b(f_a(\{(0,0), (2,2), (2,4)\}))
        $$$$
            = f_b(\{(2,3), (2,5)\})
            = \{(2,6)\}
        $$
        \pause
        $$
            \approx f_{ab}(\{\red{||}bbaaabbb, bb\red{||}aaabbb, bb\red{|aa|}abbb\})
        $$$$
            = f_b(\{bb\red{|a|}aabbb, bb\red{|aaa|}bbb\})
            = \{bb\red{|aaab|}bb\}
        $$
    \end{frame}

    \againframe<8>[t]{defs}

    \againframe<9>[t]{defs}

    \begin{frame}[t]
        \frametitle{\textbf{Beispiel:} \texttt{alternate}}
        \texttt{alternate}$_w : Parser \times Parser \to Parser$\\
        \texttt{alternate}$_w(f_1,f_2) = M \mapsto f_1(M) \cup f_2(M)$\\[10pt]
        Sei $\Sigma = \{a,b\},\ w = bbaaabbb \in \Sigma^*,\ f_a := \texttt{readChar}_w(a)$\\
        $f_b := \texttt{readChar}_w(b),\ f_{a|b} := \texttt{alternate}_w(f_a,f_b)$.\\
        \begin{center}
            $\underset{0}{|}$
                {\Large b}
            $\underset{1}{|}$
                {\Large b}
            $\underset{2}{|}$
                {\Large a}
            $\underset{3}{|}$
                {\Large a}
            $\underset{4}{|}$
                {\Large a}
            $\underset{5}{|}$
                {\Large b}
            $\underset{6}{|}$
                {\Large b}
            $\underset{7}{|}$
                {\Large b}
            $\underset{8}{|}$\\[30pt]
        \end{center}
        \pause
        $$
            f_{ab}(\{(0,0), (2,2), (2,4)\})
            = f_a(\{\cdots\}) \cup f_b(\{\cdots\})
        $$$$
            = \{(2,3),(2,5)\} \cup \{(0,1)\}
            = \{(0,1),(2,3),(2,5)\}
        $$
        \pause
        $$
            f_{a|b}(\{\red{||}bbaaabbb, bb\red{||}aaabbb, bb\red{|aa|}abbb\}) =
        $$$$
            = \{\red{|b|}baaabbb, bb\red{|a|}aabbb, bb\red{|aaa|}bbb\}
        $$
    \end{frame}

    \againframe<10>[t]{defs}

    \begin{frame}[t]
        \frametitle{\textbf{Beispiel:} \texttt{iterate}}
        \texttt{iterate}$_w : Parser \to Parser$\\
        \texttt{iterate}$_w(f) = \displaystyle M \mapsto \bigcup_{i \in \mathbb{N}_0} f^i(M)$\\[10pt]
        Sei $\Sigma = \{a,b\},\ w = bbaaabbb \in \Sigma^*,\ f_{a^*} := \texttt{iterate}_w(f_a)$\\
        \begin{center}
            $\underset{0}{|}$
                {\Large b}
            $\underset{1}{|}$
                {\Large b}
            $\underset{2}{|}$
                {\Large a}
            $\underset{3}{|}$
                {\Large a}
            $\underset{4}{|}$
                {\Large a}
            $\underset{5}{|}$
                {\Large b}
            $\underset{6}{|}$
                {\Large b}
            $\underset{7}{|}$
                {\Large b}
            $\underset{8}{|}$\\[10pt]
        \end{center}
        \pause
        $
            f_{a^*}(\{(0,0), (2,2), (2,4)\})
            = \{\cdots\} \cup f_a(\{\cdots\}) \cup f_a(f_a(\{\cdots\}) \cup \cdots
        $\\[5pt]\pause
        $
            = \{(0,0),(2,2),(2,4)\} \cup \{(2,3),(2,5)\} \cup \{(2,4)\} \cup \{(2,5)\} \cup \varnothing \cdots
        $\\[5pt]\pause
        $
            = \{(0,0),(2,2),(2,3),(2,4),(2,5)\}
        $\\[15pt]
        \pause
        $
            f_{a^*}(\{\red{||}bbaaabb, bb\red{||}aaabbb, bb\red{|aa|}abbb\})
        $\\[5pt]
        $
            = \{\red{||}bbaaabb, bb\red{||}aaabbb, bb\red{|a|}aabbb, bb\red{|aa|}abbb, bb\red{|aaa|}bbb\}
        $
    \end{frame}

    \againframe<11>[t]{defs}

    \begin{frame}[t,label=bsp]
        \frametitle{Komplettbeispiel}
        {\small
            Sei $\Sigma = \{a,b,c,x\},\ w = xabccx$. Regex $(a\mid b)c^*$.\\[5pt]
            \pause
            $f :=$
            \texttt{concatenate$_w($}\\
            \strut\qquad\texttt{alternate$_w($readChar$_w(a)$, readChar$_w(b))$,}\\
            \strut\qquad\texttt{iterate$_w($readChar$_w(c)))$}\\[10pt]
            \pause
            Definiere Startmenge $S_w \in \mathcal{M}$ durch $S_w := \{(i,i)\mid i \in [0,|w|-1]\}$.\\[5pt]
            \pause
            \only<4>{
                \strut\\[30pt]
                \begin{center}
                    $S_w = \{(0,0),(1,1),(2,2),(3,3),(4,4),(5,5)\}$\\[5pt]
                    $\approx\{\red{||}xabccx, x\red{||}abccx, xa\red{||}bccx, xab\red{||}ccx, xabc\red{||}cx, xabcc\red{||}x\}$
                \end{center}
            }
            \pause
            \begin{center}
                $\underset{0}{|}$
                    {\Large x}
                $\underset{1}{|}$
                    {\Large a}
                $\underset{2}{|}$
                    {\Large b}
                $\underset{3}{|}$
                    {\Large c}
                $\underset{4}{|}$
                    {\Large c}
                $\underset{5}{|}$
                    {\Large x}
                $\underset{6}{|}$\\[8pt]
            \pause
            $
                f(S_w)
                = f_{c*}(f_{a|b}(S_w))
                = f_{c*}(f_a(S_w) \cup f_b(S_w))
            $\\[8pt]
            \pause
            $
                = f_{c*}(\{(1,2)\} \cup f_b(S_w))
                = f_{c*}(\{(1,2),(2,3)\})
            $\\[8pt]
            \pause
            $\displaystyle
                = \bigcup_{i \in \mathbb{N}_0} f_c^i(\{(1,2),(2,3)\})
                = \{(1,2),(2,3)\} \cup \{(2,4)\} \cup \{(2,5)\} \cup \varnothing \cdots
            $\\
            \pause
            $= \{(1,2),(2,3),(2,4),(2,5)\}$\\[8pt]
                $\approx\{x\red{|a|}bccx, xa\red{|b|}ccx, xa\red{|bc|}cx, xa\red{|bcc|}x\}$
            \end{center}
        }
    \end{frame}

    \begin{frame}[t]
        \frametitle{Rekursiv gedacht}
        {\small
            $\Sigma$ Alphabet, $\mathcal{R}_\Sigma$ als Menge der Regexe über $\Sigma$. Weiter $w \in \Sigma^*$.} \\[5pt]
        {\small
            \texttt{ext}$_w : \mathcal{R}_\Sigma \times \mathcal{M} \to \mathcal{M}$
            $$
                \texttt{ext}_w(r, M) := \begin{cases}
                    \pause\{(i, j+1) \mid (i,j) \in M \land w_j = \sigma\}
                        &, r = \sigma \in \Sigma\\
                    \pause\texttt{ext}_w(r_2, \texttt{ext}_w(r_1, M))
                        &, r = (r_1r_2), r_1,r_2 \in \mathcal{R}_\Sigma\\
                    \pause\texttt{ext}_w(r_1, M) \cup \texttt{ext}_w(r_2,M)
                        &, r = (r_1 \mid r_2),r_1,r_2 \in \mathcal{R}_\Sigma\\
                    \pause\texttt{iter}_w(t, M)
                        &, r = t^*,t \in \mathcal{R}_\Sigma
                \end{cases}
            $$$$
            \texttt{iter}_w(t, M) := \begin{cases}
                    \varnothing & \text{falls}\ M = \varnothing\\
                    M \cup \texttt{iter}_w(t, \texttt{ext}_w(t, M)) & \text{sonst}
                \end{cases}
            $$\\[5pt]
            \pause
            Aufruf wäre dann:
            $$
                \texttt{ext}_w((a\mid b)c^*, S_w)
            $$
        }
    \end{frame}

    \begin{frame}[t]
        \frametitle{Erweiterung: Priorität bei Alternation}
        0 höchste, $\infty$ niedrigste Priorität.\\[10pt]
        Match Element von $\mathbb{N}_0 \times \mathbb{N}_0 \times \mathbb{N}_0$.
        Entsprechend $\mathcal{M}$ anpassen.\\[5pt]
        $S_w := \{(i,i,0) \mid i \in [0, |w|-1]\}$\\[10pt]
        \pause
        \texttt{readChar}$_w(\sigma) = M \mapsto \{(i, j+1, p) \mid (i,j,p) \in M \land w_j = \sigma\}$\\[10pt]
        \pause
        \texttt{decrPrio}$ : \mathcal{M} \to \mathcal{M}$\\
        \texttt{decrPrio}$(M) = \{(i,j,p+1) \mid (i,j,p) \in M\}$\\[10pt]
        \pause
        \texttt{alternate}$_w(f_1,f_2) = M \mapsto f_1(M) \cup \texttt{decrPrio}(f_2(M))$\\[10pt]
        \pause
        \begin{center}
            nicht leer $>$ startIndex $>$ Priorität $>$ Länge
        \end{center}
        $$
            (0,3,0) > (1,5,0) > (1,8,1) > (1,5,1) > (0,0,0)
        $$
    \end{frame}

    \begin{frame}[t]
        \frametitle{Erweiterung: Priorität bei Alternation (cont.)}
        Unerwartetes Verhalten bei Iteration:\\
        $\Sigma = \{a,b\},\ w = ab$. $f \approx (a\mid b)^*$.
        \pause
        $$
            f(S_w) = f(\{(0,0,0),(1,1,0)\})
        $$$$
            = \{(0,0,0),(1,1,0)\} \cup f_{a|b}(\{\cdots\}) \cup f_{a|b}(f_{a|b}(\{\cdots\})) \cdots
        $$$$
            = \{(0,0,0),(1,1,0)\}
            \cup \underbrace{\{(0,1,0)\}}_{f_a(\{\cdots\})}
            \cup \underbrace{\{(1,2,1)\}}_{f_b(\{\cdots\})}
            \cup \underbrace{\{(0,2,1)\}}_{f_b(f_{a|b}(\{\cdots\}))}
        $$
        \pause
        $$
            \,\Longrightarrow\, \underbrace{(0,1,0) > (0,2,1)}_{|a|b\quad >\quad |ab|} > (1,1,0) > (1,2,1) > (0,0,0)
        $$
    \end{frame}

    \begin{frame}[t]
        \frametitle{Implementierung in Java}
        \texttt{java.util.function.Function<A,B>}\\
        \texttt{Function<A,B>.apply :\ A $\to$ B}\\
        \texttt{Function<A,B>.compose :\ Function<B,C> $\to$ Function<A,C>}\\[10pt]
        \pause
        Syntax mit Java-Lambdas, ähnlich zum Syntax hier:\\[5pt]
        \texttt{Function<String, String> f = s -> s+"!";}\\
        \texttt{System.out.println(f.apply("ok"));}\\[5pt]
        Ausgabe: \texttt{ok!}\\[10pt]
    \end{frame}

\end{document}
