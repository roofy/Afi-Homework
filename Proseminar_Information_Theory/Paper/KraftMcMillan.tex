\documentclass[12pt]{article}

\usepackage[margin=.8in]{geometry}
\usepackage{amsmath,amsthm,amssymb}
\usepackage{cite}
\usepackage{forest}

\newcommand{\up}[2]{\mathrel{\overset{\makebox[0pt]{\mbox{\normalfont\tiny #2}}}{#1}}}
\newcommand{\T}[0]{{\mathcal{T}_r^h}}
\newcommand{\he}[0]{\textit{height}}
\newcommand{\ro}[0]{\textit{root}}
\newcommand{\SUB}[0]{\textit{sub}}
\newenvironment{statement2}[2]{\begin{trivlist}
\item[\hskip \labelsep {\bfseries #1}\hskip \labelsep {\bfseries #2}]}{\end{trivlist}}
\newenvironment{statement3}[3]{\begin{trivlist}
\item[\hskip \labelsep {\bfseries #1}\hskip \labelsep {\bfseries #2} {#3}\textbf{.}]}{\end{trivlist}}

\setlength{\parindent}{0pt}
\begin{document}
\title{Kraft's and McMillan's Inequalities}
\author{Phil Pützstück, 377247\\
Proseminar Informationstheorie, Steffen van Bergerem}
\maketitle

\begin{abstract}
    We study the existence of uniquely decodable or instantaneous $r$-ary codes for some given word-lengths.
    To do this, we prove and discuss the known Kraft and McMillan Inequalities by utilising graph theory.
    The approach is based on \cite{ICT}.
\end{abstract}

\section*{Introduction}
As uniquely decodable $r$-ary codes and instantaneous codes are important concepts,
we want to know under which constraints these exist.
We specifically look at $r$-ary codes with given word-lengths, as the inequalities
we will later prove relate these concepts.
After introducing a certain rooted Tree we show the relation between it and $r$-ary codes,
which we use in the proof for Kraft's Inequality.
Following the proofs we discuss the implications of these inequalities and give an example of constructing
an instantaneous Code for parameters satisfying Kraft's Inequality.\\
For a quick introduction to the graph-terminology in this paper:\\[10pt]
Trees are acyclic, connected and undirected graphs.
A tree $T = (V,E)$ has the set of vertices $V$ and the set of edges $E$, which we denote by $V(T) := V, E(T) := E$.
We call $T$ a rooted tree, if we have a distinct vertex $v \in V(T)$,
called the root of $T$ and denoted as $\ro(T) = v$.
Note that we then have a unique path from the root of $T$ to each vertex of $T$.
In this case each vertex $v \in V(T)$ has a height, denoted $\he(v) = \he_T(v)$,
defined as the length of that unique path from $\ro(T)$ to $v$. The height of $T$ is
the maximum height of the leaves of $T$.\\[10pt]
For the rest of this paper we will only consider rooted trees, in particular $r$-ary rooted trees of some height $h$, where $r,h \in \mathbb{N}$, which
are rooted trees of finite height $h$ where each vertex of height less than $h$ has exactly $r$ children.
We introduce subtrees and an ordering of vertices:

\begin{statement3}{(1.1)}{Definition}{(Subtrees and Ordering)} Let $T, T'$ be rooted Trees.\\
    We say $T'$ is a rooted subtree of $T$, iff $T'$ is a subgraph of $T$ and write $T' \leq T$.\\
    We write $T' \leq_r T$ iff $r \in \mathbb{N}$ and $T,T'$ are both $r$-ary.\\
    Now let $v,w \in V(T)$.
    We write $v \leq w$ iff the unique path from $\ro(T)$ to $w$ visits $v$.\\[8pt]
    Let $v \in V(T) \setminus \{\ro(T)\}$. Set
    $V_v := \{u \in V(T) \mid v \leq u\}, E_v := \{(u,u') \in E(T) \mid u,u' \in V_v\}$
    and let $\SUB(v) = \SUB_T(v) := (V_v, E_v)$ be the rooted subtree of $T$ with $v$ as its root.\\[8pt]
    At last we define $T \setminus v := T \setminus \SUB(v)$ (by usual graph difference) to be the rooted subtree of $T$
    where all vertices and edges in $\SUB(v)$ (vertices $v' \in V(T)$ with $v \leq v'$ and their edges)
    are deleted from $T$. Note that $v \in V(T) \setminus \{\ro(T)\}$, so $v \neq \ro(T)$,
    as we dont want $T \setminus v$ to be empty.
\end{statement3}

\newpage

We will soon relate the height of a vertex to the length of a given word, and as we are searching for Codes
with given word-lengths, it will be useful to state some reminders about counting vertices in trees.

\begin{statement3}{(1.2)}{Reminder}{(Number of vertices of some height in rooted r-ary Trees)}\strut\\[2pt]
    Let $h,r \in \mathbb{N}$, $T$ be a rooted $r$-ary tree of height $h$.
    Then $T$ has exactly $r^{h'}$ vertices of height $h' \leq h$.
\end{statement3}

\begin{statement3}{(1.3)}{Corollary}{(Number of Leaves of $T \setminus v$)}\strut\\[2pt]
    Let $h,r \in \mathbb{N}$, $T$ be a rooted $r$-ary tree of height $h$, $T' \leq T$ and
    $v \in V(T') \setminus \{\ro(T')\}$ such that $\SUB_{T'}(v) \leq_r T$.
    If $L$ is the number of leaves of $T'$, then $T' \setminus v$ has $L - r^{h - \he_T(v)}$ leaves.

    \begin{proof}
        Since $\SUB_{T'}(v)$ is $r$-ary and has height $h - \he_T(v)$, we know $\SUB_{T'}(v)$ has
        $r^{h- \he_T(v)}$ leaves by (1.2).
        Thus $T' \setminus v = T' \setminus \SUB_{T'}(v)$ has $L - r^{h-\he_T(v)}$ leaves.
    \end{proof}
\end{statement3}

For us, intervals are over $\mathbb{N}_0$, so for $m,n \in \mathbb{N}_0$ we have $[m,n] := \{p \in \mathbb{N}_0 \mid
m \leq p \leq n\}$.\\
Now we will come to the relationship between $r$-ary codes and $r$-ary rooted trees.

\begin{statement3}{(1.4)}{Definition}{(r-ary Trees from r-ary Codes)}\strut\\[2pt]
    Let $q,r \in \mathbb{N}, A := [0,r-1]$ be the code-alphabet for some $r$-ary code $\mathcal{C}$ with word-lengths
    $\ell \in \mathbb{N}^q$. This choice of the code-alphabet can always be made since any other code-alphabet
    for $\mathcal{C}$ would stand in bijection to $A$.
    Set $h := \max\{\ell_i \mid i \in [1,q]\}$.
    Define $W := \bigcup_{i \in [0,h]} A^i$ to be the set of all words over $A$ of maximum length $h$. Thus
    $\mathcal{C} \subseteq W$. We construct a rooted $r$-ary tree $\T$ of height $h$ indexed by $W$ by setting
    $V(\T) := \{v_w \mid w \in W\}$, $\ro(\T) := v_\varepsilon$ and
    $E(\T) := \{(v_w,v_{w'}) \mid w,w' \in W, wx = w', x\in A\}$. Note that $\T$ is uniquely determined by $r,h$.
\end{statement3}

\begin{statement2}{(1.5)}{Examples.} $\mathcal{T}_3^2$ is given by
    \begin{center}
        \begin{forest}
            [$\ro(\mathcal{T}_3^2)\text{ = }v_\varepsilon$
                [$v_0$
                    [$v_{00}$],
                    [$v_{01}$],
                    [$v_{02}$]
                ],
                [$v_1$
                    [$v_{10}$],
                    [$v_{11}$],
                    [$v_{12}$]
                ],
                [$v_2$
                    [$v_{20}$],
                    [$v_{21}$],
                    [$v_{22}$]
                ]
            ],
        \end{forest}\\
    \end{center}
    We have $\he(v_\varepsilon) = 0$ and $\he(v_{12}) = 2$.
    $v_0 \leq v_{02}$ holds, $v_0 \leq v_{10}$ does \textbf{not}.\\
    The subtrees $\SUB(v_1)$ and $\mathcal{T}_3^2 \setminus v_1$ are given by:\\
    \begin{minipage}{.4\textwidth}
        \begin{center}
            \begin{forest}
                [$\ro(\SUB(v_1))\text{ = }v_1$
                    [$v_{10}$],
                    [$v_{11}$],
                    [$v_{12}$]
                ]
            \end{forest}
        \end{center}
    \end{minipage}
    \begin{minipage}{.4\textwidth}
        \begin{center}
            \begin{forest}
                [$\ro(\mathcal{T}_3^2 \setminus v_1)\text{ = }v_\varepsilon$
                    [$v_0$
                        [$v_{00}$],
                        [$v_{01}$],
                        [$v_{02}$]
                    ],
                    [$v_2$
                        [$v_{20}$],
                        [$v_{21}$],
                        [$v_{22}$]
                    ]
                ],
            \end{forest}\\
        \end{center}
    \end{minipage}\\[10pt]
    $\SUB(v_1)$ is a 3-ary rooted subtree of height $1$, but $\mathcal{T}_3^2 \setminus v_1$ is only
    a rooted subtree of height 2, not $r$-ary for any $r \in \mathbb{N}$.
    We now have $\he_{\SUB(v_1)}(v_1) = 0, \he_{\SUB(v_1)}(v_{12}) = 1$, but still
    $\he_{\mathcal{T}_3^2 \setminus v_1}(v) = \he_{\mathcal{T}_3^2}(v)$ for $v \in V(\mathcal{T}_3^2 \setminus v_1)$,
    so in particular $\he_{\mathcal{T}_3^2\setminus v_1}(v_{12}) = 2$.
\end{statement2}

\begin{statement2}{(1.6)}{Remark.} One can now see the relation between the tree $\T$ and its code-words;\\
    We have $v_w \leq v_{w'} \,\Longleftrightarrow\, w \sqsubseteq w'$
    and $\he(v_w) = |w|$ for $v_w, v_{w'} \in V(\T)$. (Proof Omitted)
\end{statement2}

\begin{statement3}{(1.7)}{Theorem}{(Kraft's Inequality)}\strut\\[2pt]
    Let $q,r \in \mathbb{N}, \ell \in \mathbb{N}^q$. Then there is an instantaneous $r$-ary code $\mathcal{C}$
    with word-lengths $\ell$ iff
    \begin{equation}
        \sum_{k=1}^{q} \frac{1}{r^{\ell_k}} \leq 1
    \end{equation}

    \begin{proof}
        If $q = 1$, meaning $\mathcal{C}$ has only one word, then $\mathcal{C}$ is an instantaneous code, and since $r \in \mathbb{N}$, (1) always holds
        as well.
        So assume w.l.o.g. that $q > 1$, $\forall i \in [1,q-1]: 0 < \ell_i \leq \ell_{i+1}$
        and that the code-alphabet of $\mathcal{C}$ is $[0,r-1]$ (we only need the alphabet to have $r$ elements).
        \\[10pt]
        We first show that (1) implies the existence of an $r$-ary prefix-code, which by
        \cite{ICT} is instantaneous.
        Set $h := \ell_q$ to be the maximum of the given word-lengths.
        Thus we have, like in (1.4), that $\mathcal{C} \subseteq \bigcup_{i \in [0,h]} [0,r-1]^i =: W$, where
        $W$ is in bijection with $V(\T)$.
        So we construct the code-words $w_i$ of the prefix-code $\mathcal{C}$, with $|w_i| = \ell_i$ for $i \in [1,q]$
        via finite induction over $i$. The idea is to remove the subtrees rooted at $v_{w_i}$ and then
        chose $v_{w_{i+1}}$ from the remaining vertices to uphold the prefix property of $\mathcal{C}$,
        since for all $j \in [1,i]$ we already removed all $v_{w}$ with $w_j \sqsubseteq w$ (see 1.6)
        before chosing $v_{w_{i+1}}$, so with $\ell_j \leq \ell_{i+1}$ we then know the prefix property still holds.\\[10pt]
        Let $i = 1$. Choose a code-word $w_1 \in [0,r-1]^{\ell_1}$ of length $\ell_1$. Since $w_1 \in W$ and $\ell_1 > 0$ we have
        $v_{w_1} \in V(\T) \setminus \{R(\T)\}$. Define $\mathcal{T}_0 := \T, \mathcal{T}_1 := \mathcal{T}_0 \setminus v_{w_1}$. We know
        from (1.3) that $\mathcal{T}_1$ has
        $$
            r^h - r^{h - \he(v_{w_1})} = r^h - r^{h - \ell_1} = r^h\left(1 - \sum_{k=1}^{1} \frac{1}{r^{\ell_k}}\right)
            > r^h\left(1 - \sum_{k=1}^{q} \frac{1}{r^{\ell_k}}\right) \up{\geq}{(1)} 0
        $$
        leaves. Now let $i \in [1,q-1]$ such that $\{w_j \mid j \in [1,i]\}$ is a prefix-code
        with $|w_j| = \ell_j$\\[2pt]
        for $j \in [1,i]$ and such that $\mathcal{T}_i$ is a rooted subtree of $\T$ and has $r^h(1 - \sum_{k=1}^{i}\frac{1}{r^{\ell_k}}) > 0$ leaves.
        Then since $\ell_{i+1} \leq \ell_q = h$ we know that there must also be at least one vertex
        $v_w \in V(\mathcal{T}_i)$ with $\he(v_w) = \ell_{i+1}$ since trees are connected and we have a leaf.
        So set $w_{i+1} := w$. If we had $w_j \sqsubseteq w_{i+1}$ for
        some $j \in [1,i]$, then also $v_{w_j} \leq v_{w_{i+1}}$ by (1.6), but then
        $v_{w_{i+1}} \notin V(\mathcal{T}_{j-1} \setminus v_{w_j}) = V(\mathcal{T}_{j}) \supseteq V(\mathcal{T}_{i})$, a contradiction as we chose $v_{w_{i+1}}$ from $\mathcal{T}_i$.
        Thus $\{w_j \mid j \in [1, i+1]\}$ is still a prefix-code.
        If $i+1 = q$ we are done, as we have constructed the desired prefix-code
        $\mathcal{C} := \{w_j \mid j \in [1,q]\}$. Otherwise,
        we set $\mathcal{T}_{i+1} := \mathcal{T}_i \setminus w_{i+1}$ and we get for the number of leaves:
        $$
            r^h\left(1-\sum_{k=1}^{i} \frac{1}{r^{\ell_k}}\right) - r^{h - \ell_{i+1}}
            = r^h\left(1-\sum_{k=1}^{i+1} \frac{1}{r^{\ell_k}}\right)
            > r^h\left(1-\sum_{k=1}^{q} \frac{1}{r^{\ell_k}}\right)
            \up{\geq}{(1)} 0
        $$
        Thus we constructed the desired prefix-code $\mathcal{C}$ by finite induction.\\[10pt]
        Now we show that the existence of an instantaneous $r$-ary code $\mathcal{C}$ with word-lengths $\ell \in \mathbb{N}^q$ implies the inequality.
        We know from \cite{ICT} that $\mathcal{C}$ is a prefix-code. Let $i \in [1,q],
        w_i \in \mathcal{C}, |w_i| = \ell_i$ and set
        $$
            L_i := \{v_w \in V(\T) \mid w_i \sqsubseteq w \land |w| = h\}
            = \{v_w \in \SUB(v_{w_i}) \mid \he_\T(v_w) = h\}
        $$
        to be the set of leaves in $\SUB(v_{w_i})$. From (1.2) and (1.6) we get that $|L_i| = r^{h - \ell_i}$
        for $i \in [1,q]$.\\
        Furthermore we know that for each $i\neq j \in [1,q]$ $L_i \cap L_j = \varnothing$, which can be quickly
        verified:\\
        Assume $i,j \in [1,q]$ and w.l.o.g. $i < j$. Let $v_w \in L_i \cap L_j$. Thus we get via (1.6):
        $$
            v_{w_i} \leq v_w\ \land\ v_{w_j} \leq v_w
            \up{\quad\Longrightarrow\quad}{(1.6)} w_i \sqsubseteq w\ \land\ w_j \sqsubseteq w
            \up{\quad\Longrightarrow\quad}{$i < j$} w_i \sqsubseteq w_j
        $$
        which is a contradiction to the fact that $\mathcal{C}$ is a prefix-code. Hence $L_i$ and $L_j$ are disjoint.\\
        So now, since $\T$ only has $r^h$ leafs, we get what we wanted to show:
        $$
            r^h \geq |\bigcup_{i \in [1,q]} L_i| = \sum_{i = 1}^{q} |L_i| = \sum_{i=1}^{q} r^{h-\ell_i}
            = r^h\sum_{i=1}^{q} \frac{1}{r^{\ell_i}}
            \quad\,\Longleftrightarrow\,\quad \sum_{i=1}^{q} \frac{1}{r^{\ell_i}} \leq 1
        $$
    \end{proof}
\end{statement3}

One could assume, that because being instantaneous implies being uniquely decodable, the constraints
for being the latter are weaker. Suprisingly, this is not the case:

\begin{statement3}{(1.8)}{Theorem}{(McMillan's Inequality)}\strut\\[2pt]
    Let $q,r \in \mathbb{N}, \ell \in \mathbb{N}^q$. Then there is an uniquely decodable
    $r$-ary code $\mathcal{C}$ iff
    \begin{equation}
        K := \sum_{i=1}^{q} \frac{1}{r^{\ell_i}} \leq 1 \tag{1}
    \end{equation}

    \begin{proof}
        If we assume (1), then by Kraft's inequality we know that $\mathcal{C}$
        is instantaneous, which by \cite{ICT} implies unique decodability.\\[10pt]
        Now assume that $\mathcal{C}$ is a uniquely decodable $r$-ary code with word-lengths
        $\ell$. We show $K \leq 1$. \\
        Let $n \in \mathbb{N}$. We have
        \begin{equation}
            K^n
            = \left(\sum_{i=1}^{q} \frac{1}{r^{\ell_i}}\right)^n
            = \sum_{i \in [1,q]^n}\prod_{k=1}^{n} \frac{1}{r^{\ell_{i_k}}}
            = \sum_{i \in [1,q]^n} r^{-\sum_{k=1}^{n} \ell_{i_k}} \tag{2}
        \end{equation}
        where the $i \in [1,q]^n$ represents $n$ choices of $q$ possible summands (with repitition).\\[10pt]
        Now there are many different $i \in [1,q]^n$ which have the same sum $\sum_{k=1}^{n} \ell_{i_k}$
        (consider permutations for example). Set $M := \max\{\ell_k\mid k \in [1,q]\}, m := \min\{\ell_k \mid k \in [1,q]\}$
        to be the min. / max. word-lengths.
        Then we get
        \begin{equation}
            \forall i \in [1,q]^n: mn \leq \sum_{k=1}^{n} \ell_{i_k} \leq Mn \tag{3}
        \end{equation}
        We define for
        $j \in [mn,Mn]$:
        $$
            N_{j} := \{w_{i_1}w_{i_2}\cdots w_{i_n} \mid i \in [1,q]^n \land |w_{i_1}\cdots w_{i_n}| = j \}
        $$
        If $N_j \neq \varnothing$, then $t \in N_{j}$ is a code-sequence of length $j$, consisting of $n$ code-words in $\mathcal{C}$.
        \\
        But since $\mathcal{C}$ is uniquely decodable, we know that
        $
            \forall t \in N_{j}: \exists!\ i \in [1,q]^n: t = w_{i_1}\cdots w_{i_n}
        $,
        meaning there is one and only one way to construct $t \in N_{j}$ from $n$ code-words of $\mathcal{C}$.
        This implies, that
        \begin{equation}
            |\{i \in [1,q]^n \mid \sum_{k=1}^{n} \ell_{i_k} = j\}|
            = |\{i \in [1,q]^n \mid \sum_{k=1}^{n} |w_{i_k}| = j\}| \tag{4}
            = |N_{j}|
        \end{equation}
        Furthermore, since $N_{j} \subseteq [0,r-1]^j$, we have $|N_{j}| \leq r^j$. Thus, from (2), (3), (4) we get
        $$
            K^n = \sum_{j = mn}^{Mn} \frac{|N_{j}|}{r^j} \leq \sum_{j = mn}^{Mn} 1 = (M-m)n + 1
            \,\Longrightarrow\, \frac{K^n}{n} \leq (M-m) + 1
        $$
        Now $m,M,K$ are fixed, while $n$ may be arbitrarily large. From Analysis we know
        that as $n \to \infty$, the only way that $\frac{K^n}{n}$ stays bounded by a constant is if $K \leq 1$.
        Thus we get the desired result:
        $$
            \sum_{i=1}^{q} \frac{1}{r^{\ell_i}} = K \leq 1
        $$
    \end{proof}
\end{statement3}

\begin{statement2}{(1.9)}{Corollary.}\strut\\[2pt]
    Let $r,q \in \mathbb{N}, \ell \in \mathbb{N}^q$. Then by the above inequalities we get
    that there exists an instantaneous $r$-ary code with word-lengths $\ell$ iff
    there exists an uniquely decodable $r$-ary code with word-lengths $\ell$.\\
\end{statement2}

So now we have necessary conditions for when instantaneous $r$-ary Codes of some
given length exist for some $r \in \mathbb{N}$.
When searching / constructing a code, one usually wants a it to be instantaneous and have
its word lengths and code-alphabet as small as possible.
These properties are related through the Inequalites we proved.
In particular it is not possible to construct an instantaneous or uniquely decodable $r$-ary Code
with arbitrarily small word-lengths for some fixed $r$, neither for an arbitrarily small
$r$, given fixed word-lengths;\\
There exists a lower bound given by these Inequalities.

\begin{statement2}{(1.10)}{Remark.}
    Note that we know that if $q,r \in \mathbb{N}, \ell \in \mathbb{N}^q$
    satisfy Kraft's Inequality, there \textbf{exists} an instantaneous $r$-ary Code. This does in no way imply
    that every code with code-words of these lengths is instantaneous. Consider for example
    $r = 2, q = 3, \ell = (1,2,3)$. Then we have $\sum_{k=1}^{q} \frac{1}{r^{\ell_k}} = \frac{7}{8} \leq 1$,
    but the $2$-ary code $\{0, 01, 011\}$ is obviously not a prefix-code $\mathcal{C}$ and thus not instantaneous.
    Similarly, by (1.9) we know that if we have some uniquely decodable code, there \textbf{exists} an
    instantaneous code with the same word lengths, not that $\mathcal{C}$ is instantaneous. For this,
    consider the code $\{0,01,11\}$, which is uniquely decodable, but not instantaneous since $0 \sqsubseteq 01$.
\end{statement2}

As the proof for Kraft's Inequality is constructive, we conclude with an example of constructing an instantaneous code
for given constraints satisfying Kraft's Inequality:

\begin{statement2}{(1.11)}{Example.}
    Let $r = 2, q = 4, \ell = (1,2,3,4)$, which satisfy the Kraft Inequality. We may chose $w_1 \in [0,r-1]^{\ell_1} = [0,1]$
    so set $w_1 := 1$. Now consider $\mathcal{T}_2^{\max \ell} \setminus v_{w_1} = \mathcal{T}_2^4 \setminus v_1$:\\
    \begin{center}
        \begin{forest}
            [$\ro(\mathcal{T}_2^4 \setminus v_1)$
                [$v_0$
                    [$v_{00}$
                        [$v_{000}$
                            [$v_{0000}$],
                            [$v_{0001}$]
                        ],
                        [$v_{001}$
                            [$v_{0010}$],
                            [$v_{0011}$]
                        ]
                    ],
                    [$v_{01}$
                        [$v_{010}$
                            [$v_{0100}$],
                            [$v_{0101}$]
                        ],
                        [$v_{011}$
                            [$v_{0110}$],
                            [$v_{0111}$]
                        ]
                    ]
                ]
            ],
        \end{forest}\\
    \end{center}
Now we chose one of the vertices at the height $\ell_2 = 2$, lets say $v_{00}$, and thus set $w_2 := 00$.
\newpage
    \begin{minipage}{.5\textwidth}
        \begin{center}
        Again consider
        $(\mathcal{T}_2^4 \setminus v_1) \setminus v_{00}$:\\[10pt]
        \begin{forest}
            [$\ro((\mathcal{T}_2^4 \setminus v_1) \setminus v_{00})$
                [$v_0$
                    [$v_{01}$
                        [$v_{010}$
                            [$v_{0100}$],
                            [$v_{0101}$]
                        ],
                        [$v_{011}$
                            [$v_{0110}$],
                            [$v_{0111}$]
                        ]
                    ]
                ]
            ]
        \end{forest}\\
        For height $l_3 = 3$ we chose $v_{011}$, $w_3:= 011$
        \end{center}
    \end{minipage}
    \begin{minipage}{.5\textwidth}
        \begin{center}
        After again removing possible ''prefix-property-breaking'' words:\\[10pt]
        \begin{forest}
            [$\ro(((\mathcal{T}_2^4 \setminus v_1) \setminus v_{00}) \setminus v_{011})$
                [$v_0$
                    [$v_{01}$
                        [$v_{010}$
                            [$v_{0100}$],
                            [$v_{0101}$]
                        ]
                    ]
                ]
            ],
        \end{forest}\\
        We have 2 choices, the leaves, left.\\
        So going with the left one we set $w_4 := 0100$.
        \end{center}
    \end{minipage}\\[20pt]
        Now we have constructed an $r$-ary prefix-code $\mathcal{C} := \{1, 00, 011, 0100\}$ with
        word-lengths $\ell$, which we know is instantaneous by \cite{ICT}.
\end{statement2}

\bibliography{lit}{}
\bibliographystyle{alpha}
\end{document}
