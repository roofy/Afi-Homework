\documentclass[a4paper,graphics,11pt]{article}
\pagenumbering{arabic}

\usepackage[margin=1in]{geometry}
\usepackage[utf8]{inputenc}
\usepackage[T1]{fontenc}
\usepackage{lmodern}
\usepackage[ngerman]{babel}
\usepackage{amsmath, tabu}
\usepackage{amsthm}
\usepackage{amssymb}
\usepackage{complexity}
\usepackage{mathtools}
\usepackage{setspace}
\usepackage{graphicx,color,curves,epsf,float,rotating}
\usepackage{tasks}
\setlength{\parindent}{0em}
\setlength{\parskip}{1em}

\newcommand{\aufgabe}[1]{\subsection*{Aufgabe #1}}
\newcommand{\up}[2]{\mathrel{\overset{\makebox[0pt]{\mbox{\normalfont\tiny #2}}}{#1}}}

\begin{document}
\noindent Gruppe \fbox{\textbf{11}}             \hfill Tobias Riedel, 379133 \\
\noindent Analysis für Informatiker             \hfill Phil Pützstück, 377247 \\
\strut\hfill Kevin Holzmann, 371116\\
\strut\hfill Gurvinderjit Singh, 369227
\begin{center}
	\LARGE{\textbf{Hausaufgabe 14}}
\end{center}
\begin{center}
\rule[0.1ex]{\textwidth}{1pt}
\end{center}

\aufgabe{1}
Es sei
$$
	A =
		\begin{pmatrix}
			a & b\\
			c & d
		\end{pmatrix}
$$

Dann ist $\det(A) = ad-bc$ nach Hinweis. Für $\det(A) \neq 0$ gilt:
$$
	\frac{1}{\det(A)} \cdot \begin{pmatrix}
		d & -b\\
		-c & a
	\end{pmatrix}
	= \begin{pmatrix}
		\dfrac{d}{ad-bc} & \dfrac{-b}{ad-bc}\\[15pt]
		\dfrac{-c}{ad-bc} & \dfrac{a}{ad-bc}
	\end{pmatrix}
	= B
$$
Dann gilt:
$$
	AB = \begin{pmatrix}
			a & b\\
			c & d
		\end{pmatrix}
	\cdot
	\begin{pmatrix}
		\dfrac{d}{ad-bc} & \dfrac{-b}{ad-bc}\\[15pt]
		\dfrac{-c}{ad-bc} & \dfrac{a}{ad-bc}
	\end{pmatrix}
$$

$$
	= \begin{pmatrix}
		\left(a\cdot\dfrac{d}{ad-bc}\right) + \left(b\cdot \dfrac{-c}{ad-bc}\right)
		& \left(a\cdot\dfrac{-b}{ad-bc}\right) + \left(b\cdot \dfrac{a}{ad-bc}\right) \\[20pt]
		\left(c\cdot\dfrac{d}{ad-bc}\right) + \left(d\cdot \dfrac{-c}{ad-bc}\right)
		& \left(c\cdot\dfrac{-b}{ad-bc}\right) + \left(d\cdot \dfrac{a}{ad-bc}\right)
	\end{pmatrix}
$$

$$
	= \begin{pmatrix}
		\dfrac{ad-bc}{ad-bc} & \dfrac{ab-ab}{ad-bc}\\[15pt]
		\dfrac{cd-cd}{ad-bc} & \dfrac{ad-bc}{ad-bc}
	\end{pmatrix}
	= \begin{pmatrix}
		1&0\\
		0 & 1
	\end{pmatrix}
	= E_2
$$

Damit gilt nach Skript, dass $B = A^{-1}$.
\newpage
\aufgabe{2}
\textbf{a)}
Es sei $x \in \mathbb{R}^n$ für ein $n \in \mathbb{N}$ gegeben. Nach Definition ist:
$$
	||x||_1 = \sum_{i=1}^{n} |x_i|
$$
Es gelten folgende 3 Eigenschaften:

\textbf{1)}
Nach Skript gilt $\forall a \in \mathbb{R}\colon \left(|a| \geq 0\right)\land\left(|a| = 0 \,\Longrightarrow\, a = 0\right)$.

Für $x = 0$, d.h. $\forall i \in [1,n]_\mathbb{N}\colon x_i = 0$ gilt also auch
$$
	||x||_1 = \sum_{i=1}^{n} |x_i| = \sum_{i=1}^{n} |0| = 0
$$

Andernfalls gilt:
$$
	\left(\exists i \in [1,n]_\mathbb{N}\colon x_i \neq 0\right) \,\Longrightarrow\, ||x||_1 = \sum_{i=1}^{n} |x_i| > 0
$$

Damit gilt in jedem Fall, dass $||x||_1 \geq 0$, sowie $\left(||x||_1 = 0\right) \,\Longrightarrow\, \left(\forall i \in [1,n]_\mathbb{N}\colon x_i = 0\right)$

\textbf{2)}
Für $c \in \mathbb{R}$ gilt:
$$
	||x\cdot c||_1
	= \sum_{i=1}^{n} |c\cdot x_i|
	= \sum_{i=1}^{n} |c|\cdot |x_i|
	= |c|\cdot \sum_{i=1}^{n} |x_i|
	= |c|\cdot ||x||_1
$$

\textbf{3)}
Sei nun ebenfalls $y \in \mathbb{R}^n$ gegeben. Es gilt:
$$
	||x+y||_1
	= \sum_{i=1}^{n} |x_i+y_i|
	\leq \sum_{i=1}^{n} \left(|x_i|+|y_i|\right)
	= \left(\sum_{i=1}^{n} |x_i|\right) + \left(\sum_{i=1}^{n} |y_i|\right)
	= ||x||_1 + ||y||_1
$$

Damit ist $||\cdot||_1$ nach Skript eine Norm.

\textbf{b)}
Es sind:
$$
	||x||_1 = |-1|+|0|+|5| = 6 \qquad ||y||_1 = |2|+|3|+|1| = 7 \qquad ||z||_1 = |\sqrt{2}|+|-1|+|0| = \sqrt{2}+1
$$
Weiter gilt
$$
	||\langle x, y\rangle\cdot z||_1
	= ||\left(-2+0+5\right) \cdot z||_1
	= |3\sqrt{2}|+|-3|
	= 3(\sqrt{2} +1)
$$
$$
	||\langle x, z\rangle\cdot y||_1
	= ||\left(-\sqrt{2}\right) \cdot y||_1
	= |-2\sqrt{2}|+|-3\sqrt{2}|+|-\sqrt{2}|
	= 7\sqrt{2}
$$
$$
	||\langle y, z\rangle\cdot x||_1
	= ||\left(2\sqrt{2}-3\right) \cdot x||_1
	= |3-2\sqrt{2}|+|5(2\sqrt{2}-3)|
	= 6(3-2\sqrt{2})
$$

\newpage
\textbf{c)}
Es seien $x,y \in \mathbb{R}^n$ für ein $n \in \mathbb{N}$ gegeben. Es gilt:
$$
	|\langle x,y\rangle|
	= \left|\sum_{i=1}^{n} x_i\cdot y_i\right|
	\leq \left|\sum_{i=1}^{n} ||x||_\infty\cdot y_i\right|
	\leq \sum_{i=1}^{n} ||x||_\infty \cdot |y_i|
	= ||x||_\infty \cdot \sum_{i=1}^{n} |y_i|
	= ||x||_\infty \cdot ||y||_1
$$
Da $\forall i \in [1,n]_\mathbb{N}\colon |x_i| \leq ||x||_\infty$ nach Definition.

\aufgabe{3}
\textbf{a)}
\strut\vspace
\strut\vspace
\strut\vspace
\strut\vspace
\strut\vspace
\strut\vspace
\strut\vspace
\strut\vspace
\strut\vspace
\strut\vspace
\strut\vspace
\strut\vspace
\strut\vspace
\strut\vspace

\textbf{b)}
Es sei $x \in \mathbb{R}^n$ für ein $n \in \mathbb{N}$ gegeben. Es notiere $a:= \max \{x_i \mid i \in [1,n]_\mathbb{N} \}$. Ferner sei $m = 1$ und $M = \sqrt{n}$.
$$
	m||x||_\infty
	= ||x||_\infty
	= \sqrt{a^2}
	\leq \sqrt{\sum_{i=1}^{n} x_i^2}
	= ||x||_2
	= \sqrt{\sum_{i=1}^{n} x_i^2}
	\leq \sqrt{\sum_{i=1}^{n} a^2}
	= \sqrt{na^2}
	= M||x||_\infty
$$

Sei nun wieder ein $x \in \mathbb{R}^n$ für ein $n \in \mathbb{N}$ gegeben und $l = 1$ sowie $L = n$. Es gilt:
$$
	l||x||_\infty
	= ||x||_\infty
	\leq \sum_{i=1}^{n} |x_i|
	= ||x||_1
	= \sum_{i=1}^{n} |x_i|
	\leq \sum_{i=1}^{n} ||x||_\infty
	= L||x||_\infty
$$
\end{document}
