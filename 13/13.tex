\documentclass[a4paper,graphics,11pt]{article}
\pagenumbering{arabic}

\usepackage[margin=1in]{geometry}
\usepackage[utf8]{inputenc}
\usepackage[T1]{fontenc}
\usepackage{lmodern}
\usepackage[ngerman]{babel}
\usepackage{amsmath, tabu}
\usepackage{amsthm}
\usepackage{amssymb}
\usepackage{complexity}
\usepackage{mathtools}
\usepackage{setspace}
\usepackage{graphicx,color,curves,epsf,float,rotating}
\usepackage{tasks}
\setlength{\parindent}{0em}
\setlength{\parskip}{1em}

\newcommand{\aufgabe}[1]{\subsection*{Aufgabe #1}}
\newcommand{\up}[2]{\mathrel{\overset{\makebox[0pt]{\mbox{\normalfont\tiny #2}}}{#1}}}

\begin{document}
\noindent Gruppe \fbox{\textbf{11}}             \hfill Tobias Riedel, 379133 \\
\noindent Analysis für Informatiker             \hfill Phil Pützstück, 377247 \\
\strut\hfill Kevin Holzmann, 371116\\
\strut\hfill Gurvinderjit Singh, 369227
\begin{center}
	\LARGE{\textbf{Hausaufgabe  13}}
\end{center}
\begin{center}
\rule[0.1ex]{\textwidth}{1pt}
\end{center}



\aufgabe{1}


\aufgabe{2}
\textbf{a)}
Da $\forall x \in [a,b] \colon m \leq f(x) \leq M$, sowie $\forall x \in [a,b]\colon g(x) \geq 0$, folgt auch:
$$
	\forall x \in [a,b] \colon mg(x)\leq f(x)g(x) \leq Mg(x)
$$
Nach Skript Satz 1.5 folgt durch betrachtung von $mg(x), Mg(x)$ und $f(x)$ als eigene Funktion:
$$
	m\int_{a}^{b} g(x)\, \mathrm{d}x
	= \int_{a}^{b} mg(x)\, \mathrm{d}x
	\leq \int_{a}^{b} f(x)g(x)\, \mathrm{d}x
	\leq \int_{a}^{b} Mg(x)\, \mathrm{d}x
	= M\int_{a}^{b} g(x)\, \mathrm{d}x
$$

\textbf{b)}
Da die Stammfunktionen von $f, g, f\cdot g$ existieren, muss also eine Zerlegungssumme folgender Form existieren:
$$
	\sum_{i=1}^{n} h_i(x_i-x_{i-1}) = \int_{a}^{b} g(x)\, \mathrm{d}x
$$
für ein $n \in \mathbb{N}, n \to \infty$, eine reelle Folge $x$ mit $a = x_0 < x_1 < ... < x_n = b$ und eine
reelle Folge $h_i \in g([x_{i-1}, x_i])$.

Ebenso muss die Zerlegugssumme für $f\cdot g$ existieren:
$$
	\sum_{i=1}^{n} k_ih_i(x_i-x_{i-1}) = \int_{a}^{b} f(x)\cdot g(x)\, \mathrm{d}x
$$
mit der zusätzlichen reellen Folgen $k_i \in f([x_{i-1}, x_i])$.
Es gilt nun:
$$
	\forall x \in [a,b]\colon f(x) \geq m \,\Longrightarrow\, \forall i \in [1,n]\colon \forall x \in f([y_{i-1}, y_i])\colon x \geq m \,\Longrightarrow\, \forall i \in [1,n]\colon k_i \geq m
$$$$
	\,\Longrightarrow\, m\int_{a}^{b} g(x)\, \mathrm{d}x
	= \sum_{i=1}^{n} mh_i(x_i-x_{i-1}) 
	\leq \sum_{i=1}^{n} k_ih_i(x_i-x_{i-1})
	= \int_{a}^{b} f(x)\cdot g(x)\, \mathrm{d}x
$$
Analog gilt:
$$
	\forall x \in [a,b]\colon f(x) \leq M \,\Longrightarrow\, \forall i \in [1,n]\colon \forall x \in f([x_{i-1}, x_i])\colon x \leq M \,\Longrightarrow\, \forall i \in [1,n]\colon k_i \leq M
$$$$
	\,\Longrightarrow\, \int_{a}^{b} f(x)\cdot g(x)\, \mathrm{d}x
	= \sum_{i=1}^{n} k_ih_i(x_i-x_{i-1})
	\leq \sum_{i=1}^{n} Mh_i(x_i-x_{i-1}) 
	= M\int_{a}^{b} g(x)\, \mathrm{d}x
$$

\newpage
\aufgabe{3}
\textbf{a)}
Die Exponentialfunktion ist differenzierbar, integrierbar und stetig auf $\mathbb{R}$.\\
$$
	\int_{0}^{\infty} 1-\frac{e^x-e^{-x}}{e^x+e^{-x}}\, \mathrm{d}x
	= \lim_{b \to \infty}\left( \int_{0}^{b} 1\, \mathrm{d}x -\int_0^b \frac{1}{z}\, \mathrm{d}z\right)
	= \lim_{b \to \infty}\left(x-\ln(|z|)\right)_0^b
$$$$
	= \lim_{b \to \infty}\left(x-\ln(e^x+e^{-x})\right)_0^b
	= \lim_{b \to \infty}\left(b-\ln(e^b+e^{-b}) - (0 -\ln(e^0+e^{-0}))\right)
$$$$
	= \lim_{b \to \infty}\left(b-\ln(e^b+e^{-b}) +\ln(2)\right)
	= \ln(2) + \lim_{b \to \infty} \ln\left(\exp\left(b-\ln(e^b+e^{-b})\right)\right)
$$$$
	= \ln(2) + \lim_{b \to \infty} \ln\left(\frac{e^b}{e^b+e^{-b}}\right)
	= \ln(2) + \lim_{b \to \infty} \ln\left(\frac{1}{1+e^{-2b}}\right)
	= \ln(2) + \ln\left(\frac{1}{1+0}\right)
$$$$
	= \ln(2) + \ln(1) = \ln(2)
$$

Da der Grenzwert $\displaystyle\lim_{b \to \infty} \int_{0}^{b} 1-\frac{e^x-e^{-x}}{e^x+e^{-x}}\, \mathrm{d}x = \ln(2)$
existiert, gilt nun auch
$$
	\int_{0}^{\infty} 1-\frac{e^x-e^{-x}}{e^x+e^{-x}}\, \mathrm{d}x = \ln(2)
$$

\textbf{b)}
Nach Skript Satz 2.13 gilt $\lim_{a \to 0} \ln(a) = -\infty$, also ist $\ln(x)$ für $x\to 0$ bestimmt divergent
gegen $-\infty$ (*). Daher gilt, dass für $\alpha \leq -1$ das gegebene uneigentliche Integral nicht existiert:\\
Wir unterteilen das Integral bei $c=1$, da $x^\alpha$ für $\alpha <0$ in $x_0=0$ nicht definiert ist, also zwei
uneigentliche Integrationsgrenzen hat:
$$
	\int_{0}^{\infty} x^\alpha\, \mathrm{d}x
	= \int_{0}^{1} x^\alpha\, \mathrm{d}x + \int_{1}^{\infty} x^\alpha\, \mathrm{d}x
$$
Wir betrachten zuerst das Integral $\int_{0}^{1} x^\alpha\, \mathrm{d}x$ für $\alpha = -1$:
$$
	\int_{0}^{1} \frac{1}{x}\, \mathrm{d}x
	= \lim_{a \to 0} \int_{a}^{1} \frac{1}{x}\, \mathrm{d}x
	= \lim_{a \to 0} \ln(1)-\ln(a)
$$
Da der Grenzwert nicht existiert (s.o.), existiert auch das uneigentliche Integral nicht für $\alpha = -1$.
Wir betrachten zunächst das Integral $\int_{0}^{1} x^\alpha\, \mathrm{d}x$:
$$
	\int_{0}^{1} x^\alpha\, \mathrm{d}x
	= \lim_{a \to 0} \int_{a}^{1} x^\alpha\, \mathrm{d}x
	= \lim_{a \to 0} \left(\frac{x^{\alpha+1}}{\alpha+1}\right)\bigg|_a^1
	= \lim_{a \to 0} \left(\frac{1^{\alpha+1}-a^{\alpha+1}}{\alpha+1}\right)
$$
Für $\alpha < -1$ ist $\alpha+1 < 0$, d.h. $\lim_{a \to 0}\limits a^{\alpha+1}$ existiert nicht und
divergiert bestimmt gegen unendlich. Also kann das gegebene uneigentliche Integral schonmal
nicht für $\alpha \leq -1$ existieren.\\
Für $0 > \alpha > -1$ gilt jedoch:
$$
	\lim_{a \to 0} \left(\frac{1^{\alpha+1}-a^{\alpha+1}}{\alpha+1}\right)
	= \frac{1^{\alpha+1}-0^{\alpha+1}}{\alpha+1}
	= \frac{1}{\alpha+1}
$$
Und damit auch
$$
	\forall 0 > \alpha > -1\colon \int_{0}^{1} x^\alpha\, \mathrm{d}x = \frac{1}{\alpha+1}
$$
\newpage

Wir überprüfen also nun ob auch das Integral $\int_{1}^{\infty} x^\alpha\, \mathrm{d}x$ für
$0 > \alpha > -1$ existiert:
$$
	\int_{1}^{\infty} x^\alpha\, \mathrm{d}x
	= \lim_{b \to \infty} \int_{1}^{b} x^\alpha\, \mathrm{d}x
	= \lim_{b \to \infty} \frac{b^{\alpha+1}-1}{\alpha+1}
$$


\end{document}
