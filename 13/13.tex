\documentclass[a4paper,graphics,11pt]{article}
\pagenumbering{arabic}

\usepackage[margin=1in]{geometry}
\usepackage[utf8]{inputenc}
\usepackage[T1]{fontenc}
\usepackage{lmodern}
\usepackage[ngerman]{babel}
\usepackage{amsmath, tabu}
\usepackage{amsthm}
\usepackage{amssymb}
\usepackage{complexity}
\usepackage{mathtools}
\usepackage{setspace}
\usepackage{graphicx,color,curves,epsf,float,rotating}
\usepackage{tasks}
\setlength{\parindent}{0em}
\setlength{\parskip}{1em}

\newcommand{\aufgabe}[1]{\subsection*{Aufgabe #1}}
\newcommand{\up}[2]{\mathrel{\overset{\makebox[0pt]{\mbox{\normalfont\tiny #2}}}{#1}}}

\begin{document}
\noindent Gruppe \fbox{\textbf{11}}             \hfill Tobias Riedel, 379133 \\
\noindent Analysis für Informatiker             \hfill Phil Pützstück, 377247 \\
\strut\hfill Kevin Holzmann, 371116\\
\strut\hfill Gurvinderjit Singh, 369227
\begin{center}
	\LARGE{\textbf{Hausaufgabe  13}}
\end{center}
\begin{center}
\rule[0.1ex]{\textwidth}{1pt}
\end{center}



\aufgabe{1}
Wir zerlegen das Intervall $[1,b]$ wie im Hinweis durch die Folge $x_i = b^{\frac{i}{n}}$. Da 
$\frac{1}{x}$ auf dem Intervall $[1,\infty)$ streng monoton fallend ist, lässt sich die Obersumme wie folgt bilden:
$$
	\int_{1}^{b} \frac{1}{x}\, \mathrm{d}x
	= \lim_{n \to \infty} \sum_{i=1}^{n} \frac{1}{x_{i-1}}(x_i-x_{i-1})
	= \lim_{n \to \infty} \sum_{i=1}^{n} b^{-\frac{i-1}{n}}\left(b^{\frac{i}{n}} - b^{\frac{i-1}{n}}\right)
$$$$
	= \lim_{n \to \infty} \sum_{i=1}^{n} b^{\frac{i-(i-1)}{n}}-b^{\frac{i-1-(i-1)}{n}}
	= \lim_{n \to \infty} \sum_{i=1}^{n}\left(b^{\frac{1}{n}}-b^0\right)
$$$$
	= \lim_{n \to \infty} n\left(b^{\frac{1}{n}}-1\right)
	= \lim_{n \to \infty} \frac{\exp\left(\frac{\ln(b)}{n}\right)-1}{\frac{1}{n}}
$$
Auf die gegebene Folge lässt sich nun der Satz von l'Hopital anwenden, da sowohl die Exponentialfunktion 
als auch $\frac{1}{n}$ stetig auf $\mathbb{R}$ sind. Weiterhin gilt nach Skript dann auch
$$
	\lim_{n \to \infty} \frac{1}{n} = 0\qquad\text{und}\qquad
	\lim_{n \to \infty} \exp\left(\frac{\ln(b)}{n}\right)-1
	= \exp\left(\lim_{n \to \infty}\frac{\ln(b)}{n}\right)
	= \exp(0) = 1
$$
Also wenden wir l'Hopital und überprüfen, ob der Grenzwert der Ableitungen existiert:
$$
	 \lim_{n \to \infty} \frac{\left[\exp\left(\frac{\ln(b)}{n}\right)-1\right]'}{\left[\frac{1}{n}\right]'}
	= \lim_{n \to \infty} \frac{-\frac{\exp\left(\frac{\ln(b)}{n}\right)\ln(b)}{n^2}}{-\frac{1}{n^2}}
$$$$
	= \lim_{n \to \infty} \exp\left(\frac{\ln(b)}{n}\right)\ln(b)
	= \exp\left(\lim_{n \to \infty} \frac{\ln(b)}{n}\right)\ln(b)
	= \exp(0)\ln(b) = \ln(b)
$$
Nach l'Hopital gilt nun auch:
$$
	\lim_{n \to \infty} \frac{\exp\left(\frac{\ln(b)}{n}\right)-1}{\frac{1}{n}} = \ln(b)
$$
Da dies also für eine Obersumme gilt, gilt es auch für das Infimum der Obersummen und aufgrund der Äquivalenz
ebenso für das Supremum der Untersummen. Insgesamt gilt also:
$$
	\int_{1}^{b} \frac{1}{x}\, \mathrm{d}x = \ln(b)
$$

\newpage

\aufgabe{2}
\textbf{a)}
Da $\forall x \in [a,b] \colon m \leq f(x) \leq M$, sowie $\forall x \in [a,b]\colon g(x) \geq 0$, folgt auch:
$$
	\forall x \in [a,b] \colon mg(x)\leq f(x)g(x) \leq Mg(x)
$$
Nach Skript Satz 1.5 folgt durch betrachtung von $mg(x), Mg(x)$ und $f(x)$ als eigene Funktion:
$$
	m\int_{a}^{b} g(x)\, \mathrm{d}x
	= \int_{a}^{b} mg(x)\, \mathrm{d}x
	\leq \int_{a}^{b} f(x)g(x)\, \mathrm{d}x
	\leq \int_{a}^{b} Mg(x)\, \mathrm{d}x
	= M\int_{a}^{b} g(x)\, \mathrm{d}x
$$

\textbf{b)}

Es sei eine Zerlegung $Z$ des Intervalls $[a,b]$ in $n \in \mathbb{N}$ Teile gegeben, welche wir jeweils
mit $Z_i$ indizieren.
Wir notieren $\Omega(\varphi, i) := \max\{\varphi(x) \mid x \in [Z_{i-1}, Z_i]\}$ und bilden nun eine Obersumme.
Es folgt durch die Gegebenheiten der Aufgabe, dass $\forall x \in [a,b]\colon m\leq f(x) \leq M \colon$
$$
	m \sum_{i=1}^{n} \Omega(g, i)(Z_i-Z_{i-1}) = \sum_{i=1}^{n} m\Omega(g,i)(Z_i-Z_{i-1})
$$$$
	\leq \sum_{i=1}^{n} \Omega(f,i)\Omega(g,i)(Z_i-Z_{i-1})
$$$$
	\leq \sum_{i=1}^{n} M\Omega(g,i)(Z_i-Z_{i-1}) = M \sum_{i=1}^{n} \Omega(g, i)(Z_i-Z_{i-1})
$$
Da dies für jede beliebige Zerlegung gilt, gilt es auch für das Infimum der Obersummen, sowie das Supremum
der Untersummen, da $f,g$ und $f\cdot g$ integrierbar sind. Es gilt also:
$$
	m\int_{a}^{b} g(x)\, \mathrm{d}x
	= m\cdot O(g)
	\leq O(f\cdot g)
	= \int_{a}^{b} f(x)g(x)\, \mathrm{d}x
	\leq M\cdot O(g)
	= M \int_{a}^{b} g(x)\, \mathrm{d}x
$$

\aufgabe{3}
\textbf{a)}
Die Exponentialfunktion ist differenzierbar, integrierbar und stetig auf $\mathbb{R}$.\\
$$
	\int_{0}^{\infty} 1-\frac{e^x-e^{-x}}{e^x+e^{-x}}\, \mathrm{d}x
	= \lim_{b \to \infty}\left( \int_{0}^{b} 1\, \mathrm{d}x -\int_0^b \frac{1}{z}\, \mathrm{d}z\right)
	= \lim_{b \to \infty}\left(x-\ln(|z|)\right)_0^b
$$$$
	= \lim_{b \to \infty}\left(x-\ln(e^x+e^{-x})\right)_0^b
	= \lim_{b \to \infty}\left(b-\ln(e^b+e^{-b}) - (0 -\ln(e^0+e^{-0}))\right)
$$$$
	= \lim_{b \to \infty}\left(b-\ln(e^b+e^{-b}) +\ln(2)\right)
	= \ln(2) + \lim_{b \to \infty} \ln\left(\exp\left(b-\ln(e^b+e^{-b})\right)\right)
$$$$
	= \ln(2) + \lim_{b \to \infty} \ln\left(\frac{e^b}{e^b+e^{-b}}\right)
	= \ln(2) + \lim_{b \to \infty} \ln\left(\frac{1}{1+e^{-2b}}\right)
	= \ln(2) + \ln\left(\frac{1}{1+0}\right)
$$$$
	= \ln(2) + \ln(1) = \ln(2)
$$

Da der Grenzwert $\displaystyle\lim_{b \to \infty} \int_{0}^{b} 1-\frac{e^x-e^{-x}}{e^x+e^{-x}}\, \mathrm{d}x = \ln(2)$
existiert, gilt nun auch
$$
	\int_{0}^{\infty} 1-\frac{e^x-e^{-x}}{e^x+e^{-x}}\, \mathrm{d}x = \ln(2)
$$

\textbf{b)}
Nach Skript Satz 2.13 gilt $\lim_{a \to 0} \ln(a) = -\infty$, also ist $\ln(x)$ für $x\to 0$ bestimmt divergent
gegen $-\infty$ (*). Daher gilt, dass für $\alpha \leq -1$ das gegebene uneigentliche Integral nicht existiert:\\
Wir unterteilen das Integral bei $c=1$, da $x^\alpha$ für $\alpha <0$ in $x_0=0$ nicht definiert ist, also zwei
uneigentliche Integrationsgrenzen hat:
$$
	\int_{0}^{\infty} x^\alpha\, \mathrm{d}x
	= \int_{0}^{1} x^\alpha\, \mathrm{d}x + \int_{1}^{\infty} x^\alpha\, \mathrm{d}x
$$
Wir betrachten zuerst das Integral $\int_{0}^{1} x^\alpha\, \mathrm{d}x$ für $\alpha = -1$:
$$
	\int_{0}^{1} \frac{1}{x}\, \mathrm{d}x
	= \lim_{a \to 0} \int_{a}^{1} \frac{1}{x}\, \mathrm{d}x
	= \lim_{a \to 0} \ln(1)-\ln(a)
$$
Da der Grenzwert nicht existiert (s.o.), existiert auch das uneigentliche Integral nicht für $\alpha = -1$.
Wir betrachten zunächst das Integral $\int_{0}^{1} x^\alpha\, \mathrm{d}x$:
$$
	\int_{0}^{1} x^\alpha\, \mathrm{d}x
	= \lim_{a \to 0} \int_{a}^{1} x^\alpha\, \mathrm{d}x
	= \lim_{a \to 0} \left(\frac{x^{\alpha+1}}{\alpha+1}\right)\bigg|_a^1
	= \lim_{a \to 0} \left(\frac{1^{\alpha+1}-a^{\alpha+1}}{\alpha+1}\right)
$$
Für $\alpha < -1$ ist $\alpha+1 < 0$, d.h. $\lim_{a \to 0}\limits a^{\alpha+1}$ existiert nicht und
divergiert bestimmt gegen unendlich. Also kann das gegebene uneigentliche Integral schonmal
nicht für $\alpha \leq -1$ existieren.\\
Für $0 > \alpha > -1$ gilt jedoch:
$$
	\lim_{a \to 0} \left(\frac{1^{\alpha+1}-a^{\alpha+1}}{\alpha+1}\right)
	= \frac{1^{\alpha+1}-0^{\alpha+1}}{\alpha+1}
	= \frac{1}{\alpha+1}
$$
Und damit auch
$$
	\forall 0 > \alpha > -1\colon \int_{0}^{1} x^\alpha\, \mathrm{d}x = \frac{1}{\alpha+1}
$$

Wir überprüfen also nun ob auch das Integral $\int_{1}^{\infty} x^\alpha\, \mathrm{d}x$ für
$0 > \alpha > -1$ existiert:
$$
	\int_{1}^{\infty} x^\alpha\, \mathrm{d}x
	= \lim_{b \to \infty} \int_{1}^{b} x^\alpha\, \mathrm{d}x
	= \lim_{b \to \infty} \frac{b^{\alpha+1}-1}{\alpha+1}
	= \lim_{b \to \infty} \frac{\exp(\alpha+1\ln(b))-1}{\alpha+1}
$$
Nach Skirpt gilt $\lim_{x \to \infty}\limits \ln(x) = \infty$ sowie
$\lim_{x \to \infty}\limits e^x = \infty$, weiterhin gilt $\alpha+1 > 0$, daher existiert der
Grenzwert nicht und es gilt:
$$
	\lim_{b \to \infty} \frac{\exp((\alpha+1)\ln(b))-1}{\alpha+1} = \infty
$$

Zusammenfassnd existiert das Integral $\int_{0}^{1} x^\alpha\, \mathrm{d}x$ nicht für
$\alpha \leq -1$, und das Integral $\int_{1}^{\infty} x^\alpha\, \mathrm{d}x$ nicht für
$0 > \alpha >-1$. Daher existiert das Integral $\int_{0}^{\infty} x^\alpha\, \mathrm{d}x$ für
kein $\alpha < 0$.

\newpage
\aufgabe{4}
\textbf{a)}
Dies ist eine lineare DGL. Es ist $a(x) = -\frac{1}{2}$ und $b(x) = \frac{3}{2}$. Dann gilt
$$
	y' = a(x)y+b(x) = -\frac{1}{2}y+\frac{3}{2} = (3-y)\frac{1}{2}
$$
Wir definieren
$$
	\varphi(x) = \exp\left(\int_{x_0}^{x} a(t)\, \mathrm{d}t\right) = \exp\left(\frac{x_0-x}{2}\right)
$$
sowie
$$
	u(x) = y_0 + \int_{x_0}^{x} \frac{b(t)}{\varphi(t)}\, \mathrm{d}t
	= y_0 + \int_{x_0}^{x} \frac{3}{2\exp\left(\frac{x_0-t}{2}\right)}\, \mathrm{d}t
$$$$
	y_0 + \left(\frac{3}{\exp(\frac{x_0-t}{2})}\right)\bigg|_{x_0}^x
	= y_0 + \frac{3}{\exp(\frac{x_0-x}{2})} -3
$$
Dann ist nach Skript
$$
	\psi(x) = \varphi(x)\cdot u(x)
	= \exp\left(\frac{x_0-x}{2}\right) \cdot\left(y_0 + \frac{3}{\exp(\frac{x_0-x}{2})} -3\right)
$$$$
	= \exp\left(\frac{x_0-x}{2}\right) \cdot\left(y_0-3\right)+3
$$
Das Vektorfeld ist nun wie folgt:

\newpage
\textbf{b)}
Diese DGL ist nicht linear, aber seperierbar.
Wir teilen wieder mit $f(x) = \frac{x}{1-x^2},\ g(y) = \frac{1-y^2}{y}$.
Wir bestimmen zuerst $F(x)$, sodass $F(x_0) = 0$:
$$
	F(x) =
	\int_{x_0}^{x} f(t)\, \mathrm{d}t
	= \int_{x_0}^{x} \frac{t}{1-t^2}\, \mathrm{d}t
	= -\frac{1}{2}\int_{x_0}^{x} \frac{-2t}{1-t^2}\, \mathrm{d}t
	= -\frac{1}{2}\ln(|1-t^2|)\big|_{x_0}^x
$$$$
	= -\frac{\ln(|1-x^2|)-\ln(|1-x_0^2|)}{2}
$$
Da nur der Definitionsbereich $(-1,1)$ relevant für $y(x)$, sowie $x_0 \in (-1,1)$ ist, gilt sowohl
$|1-x_0^2| = 1-x_0^2$ als auch $|1-x^2| = 1-x^2$ für unseren Zweck. Damit haben wir
$$
	F(x) = -\frac{\ln(1-x^2)-\ln(1-x_0^2)}{2}
$$Wir bestimmen nun $H(z)$, sodass
$H(y_0) = 0$:
$$
	H(z) = \int_{y_0}^{z} \frac{1}{g(t)}\, \mathrm{d}t
	= \int_{y_0}^{z} \frac{t}{1-t^2}\, \mathrm{d}x \up{=}{s.o.} -\frac{\ln(|1-z^2|)-\ln(|1-y_0^2|)}{2}
$$
Es ist $y_0 \in (0,1)$, also gilt $|y_0| = y_0$. Damit haben wir
$$
	H(z) = -\frac{\ln(|1-z^2|)-\ln(1-y_0^2)}{2}
$$
Nach Skript gilt nun:
$$
	H(y(x)) = F(x)
$$$$
	\,\Longleftrightarrow\, -\frac{\ln(|1-y(x)^2|) -\ln(1-y_0^2)}{2} = -\frac{\ln(1-x^2)-\ln(1-x_0^2)}{2}
$$$$
	\,\Longleftrightarrow\, \ln(1-y(x)^2) -\ln(1-y_0^2) = \ln(1-x^2)-\ln(1-x_0^2)
$$$$
	\,\Longleftrightarrow\, \ln\left(\frac{1-y(x)^2}{1-y_0^2}\right) = \ln\left(\frac{1-x^2}{1-x_0^2}\right)
$$$$
	\,\Longleftrightarrow\, \frac{1-y(x)^2}{1-y_0^2} = \frac{1-x^2}{1-x_0^2}
$$$$
	\,\Longleftrightarrow\, 1-y(x)^2 = \frac{(1-x^2)(1-y_0^2)}{1-x_0^2}
$$$$
	\,\Longleftrightarrow\, y(x) = \pm\sqrt{1-\frac{(1-x^2)(1-y_0^2)}{1-x_0^2}}
$$
Das Vektorfeld ist dann:
\end{document}
