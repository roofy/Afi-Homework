\documentclass[a4paper,graphics,11pt]{article}

\usepackage[margin=1in]{geometry}
\usepackage[utf8]{inputenc}
\usepackage[T1]{fontenc}
\usepackage{lmodern}
\usepackage[ngerman]{babel}
\usepackage{amsmath, tabu}
\usepackage{amsthm}
\usepackage{amssymb}
\usepackage{complexity}
\usepackage{mathtools}
\usepackage{setspace}
\usepackage{graphicx,color,curves,epsf,float,rotating}
\usepackage{tasks}
\usepackage{enumerate}
\setlength{\parindent}{0em}
\setlength{\parskip}{1em}

\floatname{algorithm}{Algorithmus}

\newcommand\norm[1]{\left\lVert#1\right\rVert}
\newcommand\abs[1]{\left\vert#1\right\vert}

\newcommand\aufgabe[1]{\subsection*{Aufgabe #1}}

\pagestyle{empty}
\begin{document}
\noindent Gruppe \fbox{\textbf{11}}             \hfill Phil Pützstück, 377247 \\
\noindent Analysis für Informatiker             \hfill Tobias Riedel, 379133\\

\begin{center}
	\LARGE{\textbf{Hausaufgabe 1}}
\end{center}
\begin{center}
\rule[0.1ex]{\textwidth}{1pt}
\end{center}



\aufgabe{1}
\begin{tasks}(2)
    \task $A \,\Rightarrow\,(C \land D)$
    \task $B \,\Rightarrow\,A$
    \task $D \,\Leftrightarrow\, B$
    \task $A \,\Rightarrow\,(\lnot B\,\Rightarrow\,D)$
    \task $(C \,\Rightarrow\, \lnot D)\ \land\ (D \,\Rightarrow\, \lnot B)$ 
    \task $C \,\Rightarrow\,((A \land D)\,\Rightarrow\, \lnot B)$
\end{tasks}


\aufgabe{2}
Es seien $B_{1..3}$ die Möglichkeiten für den Schatz und $A_{1..3}$ die Hinweise der jeweiligen Boxen

\begin{minipage}{0.5\linewidth}
    $B_{1}:= $ Der Schatz ist in der ersten Box \\
    $B_{2}:= $ Der Schatz ist in der zweiten Box \\
    $B_{3}:= $ Der Schatz ist in der dritten Box \\
\end{minipage}
\begin{minipage}{0.5\linewidth}
    $A_{1}:= \lnot B_{1}$ \\
    $A_{2}:= \lnot B_{2}$ \\
    $A_{3}:= B_{2}$ \\
\end{minipage}

Zu betrachten sind nun 3 Fälle, da der Schatz nur in einer Box sein kann:
\begin{center}
    \begin{tabular}{|c|c|c|c|}
        \hline
        Fall & $A_{1}$ & $A_{2}$ & $A_{3}$ \\
        \hline
        $B_{1} \land \lnot B_{2} \land \lnot B_{3}$ & 0 & 1 & 0 \\
        \hline
        $\lnot B_{1} \land B_{2} \land \lnot B_{3}$ & 1 & 0 & 1 \\
        \hline
        $\lnot B_{1} \land \lnot B_{2} \land B_{3}$ & 1 & 1 & 0 \\
        \hline
    \end{tabular}
\end{center}

Die Aufgabe beschreibt, dass immer nur einer der Hinweise Wahr ist. Somit ergibt
sich im 2. und 3. Fall ein Widerspruch, da mehr als ein Hinweis dort die Wahrheit
beschreibt.\\
Somit muss der Schatz in der ersten Box sein.


\aufgabe{3}
\textbf{(fig.1)}
    \begin{center}
        \begin{tabular}{|c|c|c|}
            \hline
            schuldig & erhängt & schuldig $\,\Rightarrow\,$ erhängt \\
            \hline
            1 & 1 & 1 \\
            \hline
            1 & 0 & 0 \\
            \hline
            0 & 1 & 1 \\
            \hline
            0 & 0 & 1 \\
            \hline
        \end{tabular}
    \end{center}

\begin{tasks}[counter-format=tsk[r])](1)
    \task Dieser Satz ist logisch korrekt. Schuldig impliziert eine Erhängung und ist
        somit auch hinreichend für eine (schuldig $\,\Rightarrow\,$ gehängt).
        Es wäre ein logischer Widerspruch, wenn er nicht gehängt würde insofern er
        schuldig ist (siehe fig.1 Zeile 2).

    \task Dieser Satz is nicht logisch korrekt. Schuldig sein ist zwar eine hinreichende,
        jedoch nicht notwendige Bedingung um erhängt zu werden. Somit können
        auch andere Ereignisse seine Erhängung verursachen.
        Es ist zu erkennen, dass erhängt werden ohne schuldig zu sein keinen
        logischen Widerspruch darstellt (siehe fig.1 Zeile 3).

    \task Dieser Satz ist nicht logisch korrekt. Schuldig sein mag eine Erhängung
        implizieren, jedoch gilt dies nicht unbedingt andersherum. Dieser Fall ist ähnlich
        zu dem in ii), es stellt keinen logischen Widerspruch dar, erhängt zu werden ohne
        schuldig zu sein. (s.o.)

    \task Dieser Satz ist logisch korrekt. Da schuldig sein hinreichend für eine Erhängung
        ist, wäre es ein logischer Widerspruch nicht erhängt zu werden insofern man
        schuldig ist (zu erkennen in fig.1 Zeile 2)
\end{tasks}


\aufgabe{4}

\begin{itemize}
    \item[] $\mathbb{N} :=$ Menge aller natürlichen Zahlen
    \item[] $x \in M :=$ x ist Teil der Menge M
    \item[] $x \notin M :=$ x ist nicht Teil der Menge M
    \item[] $\forall x :=$ ''für alle x''
    \item[] $\forall x_{1},x_{2},... \in M: L :=$ ''für alle $x_{1},x_{2},...$ aus der
        Menge $M$ gilt der Ausdruck $L$''
    \item[] $x \land y:=$ logisches und; ''es gilt x und y''
    \item[] $x\,\Rightarrow\,y:=$ logische Implikation, ''x impliziert y'', ''x ist hinreichend für y''
    \item[] $x\,\Leftrightarrow\,y:=$ logische Äquivalenz, ''x genau dann, wenn y''
    \item[] $N \subseteq M:=$ $\forall x \in N: x \in M$, ''N ist eine Teilmenge von M''
\end{itemize}



\begin{tasks}(1)
    \task $1 \in \mathbb{N}$
    \task $\forall x \in \mathbb{N}: x=x$
    \task $\forall x,y \in \mathbb{N}: (x=y) \,\Rightarrow\, (y=x)$
    \task $\forall x,y,z \in \mathbb{N}: ((x=y) \land (y=z)) \,\Rightarrow\, (x=z)$
    \task $(y \in \mathbb{N}) \land (x=y) \,\Rightarrow\, (x \in \mathbb{N})$ 
    \task $\forall x \in \mathbb{N}: (x+1) \in \mathbb{N}$
    \task $\forall x,y \in \mathbb{N}: (x=y) \Leftrightarrow (x+1=y+1)$
    \task $(y+1=1) \,\Rightarrow\, (y \notin \mathbb{N})$\qquad bzw.\qquad $\lnot\,(\exists y \in \mathbb{N}: y+1=1)$
    \task $(1 \in K) \land (\forall x \in K: (x+1) \in K) \,\Rightarrow\, \mathbb{N} \subseteq K$
\end{tasks}


\end{document}
