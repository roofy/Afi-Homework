\documentclass[a4paper,graphics,11pt]{article}
\pagenumbering{arabic}

\usepackage[margin=1in]{geometry}
\usepackage[utf8]{inputenc}
\usepackage[T1]{fontenc}
\usepackage{lmodern}
\usepackage[ngerman]{babel}
\usepackage{amsmath, tabu}
\usepackage{amsthm}
\usepackage{amssymb}
\usepackage{complexity}
\usepackage{mathtools}
\usepackage{setspace}
\usepackage{graphicx,color,curves,epsf,float,rotating}
\usepackage{tasks}
\setlength{\parindent}{0em}
\setlength{\parskip}{1em}

\newcommand{\aufgabe}[1]{\subsection*{Aufgabe #1}}
\newcommand{\up}[2]{\mathrel{\overset{\makebox[0pt]{\mbox{\normalfont\tiny #2}}}{#1}}}

\begin{document}
\noindent Gruppe \fbox{\textbf{11}}             \hfill Tobias Riedel, 379133 \\
\noindent Analysis für Informatiker             \hfill Phil Pützstück, 377247 \\
\strut\hfill Kevin Holzmann, 371116\\
\strut\hfill Gurvinderjit Singh, 369227
\begin{center}
	\LARGE{\textbf{Hausaufgabe 10}}
\end{center}
\begin{center}
\rule[0.1ex]{\textwidth}{1pt}
\end{center}



\aufgabe{1}
\textbf{a)}

Für den Grenzwert $\lim_{x \downarrow 3}\limits f(x)$ gilt stets $x > 3$ und damit
$|x-3| = x-3 > 0$, also auch:
$$
    \lim_{x \downarrow 3} f(x)
    = \lim_{x \downarrow 3} \frac{|x-3|}{x-3}
    = \lim_{x \downarrow 3} \frac{x-3}{x-3}
    = 1
$$
Für den Grenzwert $\lim_{x \uparrow 3}\limits f(x)$ gilt stets $x < 3$ und damit
$x-3 < 0 < |x-3|$, also auch:
$$
    \lim_{x \uparrow 3} f(x)
    = \lim_{x \uparrow 3} \frac{|x-3|}{x-3}
    = \lim_{x \uparrow 3} \frac{-(x-3)}{x-3}
    = -1
$$
Da nun $\lim_{x \downarrow 3}\limits f(x) \neq \lim_{x \uparrow 3}\limits f(x)$ gilt,
lässt sich schließen, dass der Grenzwert $\lim_{x \to 3}\limits f(x)$ nicht existiert und
$f(x)$ nicht stetig ist.

\textbf{b)}

Da die Polynome $x^2+5$ und $1-3x$ für $x = -2$ wohldefiniert sind, lässt sich hier der
Grenzwert durch Einsetzen individuell bestimmen. Daraus folgt auch, dass (wie bei prakitsch
jedem Polynom) der Grenzwert $\lim_{x \to -2}\limits$ existiert (1.6) und dann mit (1.10) gilt:
$$ \lim_{x \uparrow -2} x^2+5 = \lim_{x \to -2} x^2+5
    \quad\text{und}\quad
    \lim_{x \downarrow -2} 1-3x = \lim_{x \to -2} 1-3x
$$
Es folgt also:
$$
    \lim_{x \uparrow -2} x^2+5 = (-2)^2+5 = 9
    \quad\text{und}\quad
    \lim_{x \downarrow -2} 1-3x = 1-3(-2) = 7
$$
Weiterhin ist gilt wie in a) für den Grenzwert $\lim_{x \uparrow -2}\limits g(x)$, dass $x<-2$
und damit nach\\
Konstruktion $g(x) = x^2+5$. Analog dazu gilt für $\lim_{x \downarrow -2}\limits g(x)$,
dass $x>-2$ und damit nach Konstruktion $g(x) = 1-3x$. Es folgt also
$$
    \lim_{x \uparrow -2} g(x) = \lim_{x \uparrow -2} x^2+5 = 9
    \quad\text{und}\quad
    \lim_{x \downarrow -2} g(x) = \lim_{x \downarrow -2} 1-3x = 7
$$
Da nun $\lim_{x \uparrow -2}\limits g(x) \neq \lim_{x \downarrow -2}\limits g(x)$ gilt,
lässt sich schließen, dass der Grenzwert $\lim_{x \to -2}\limits g(x)$ nicht existiert und
$g(x)$ nicht stetig ist.
\newpage

\textbf{c)}
Analog zu Beispiel 1.18 lässt sich begründen, dass
$\lim_{x \uparrow 0}\limits\dfrac{1}{x^3} = -\infty$ und
$\lim_{x \downarrow 0}\limits\dfrac{1}{x^3} = \infty$ gilt:

Zu gegebenem $M > 0$ wähle $a= \dfrac{1}{\sqrt[3]{M}}$. Dann ist für $0 < x < a$:
$$
    \frac{1}{x^3} > \frac{1}{a^3} = M
$$
Also gilt $\lim_{x \downarrow 0}\limits \dfrac{1}{x^3} = \infty$. Analog dazu
wählen wir zu gegebenm $M>0$ wieder $b= -\dfrac{1}{\sqrt[3]{M}}$.\\
Dann ist für $b < x < 0$:
$$
    \frac{1}{x^3} < \frac{1}{b^3} = -M
$$
Also gilt $\lim_{x \uparrow 0}\limits \dfrac{1}{x^3} = -\infty$.\\
Weiterhin gilt für $\lim_{x \downarrow 0}\limits j(x)$ bzw. $\lim_{x \uparrow 0}\limits j(x)$ stets $x\neq0$ und damit auch
$$
    \lim_{x \downarrow 0} j(x) = \lim_{x \downarrow 0} \frac{1}{x^3} = \infty
    \qquad\text{sowie}\qquad
    \lim_{x \uparrow 0} j(x) = \lim_{x \uparrow 0} \frac{1}{x^3} = -\infty
$$
Da nun $\lim_{x \uparrow 0}\limits j(x) \neq \lim_{x \downarrow 0}\limits j(x)$ gilt,
lässt sich schließen, dass der Grenzwert $\lim_{x \to 0}\limits j(x)$ nicht existiert und
$j(x)$ nicht stetig ist.

\aufgabe{2}

\textbf{a)}\\
Es gilt für $x,y \in [0,2]$ stets $|x-y| \leq \text{max}\,[0,2] = 2$,
ebenso $|x^2-y^2| < \text{max} [0,2^2] = 4$.\\
Wir wählen also $L = 2$. Es folgt:
$$
    \forall x,y \in [0,2] \colon |f(x)-f(y)| = |x^2-y^2| \leq L \cdot |x-y| = |2x-2y| \leq 4
$$
Dies gilt, da stets $x^2 = x\cdot x \leq 2\cdot x$ für $x \in [0,2]$.
Damit ist $f$ im Intervall $[0,2]$ Lipschitz-stetig.


\textbf{b)}\\
Wäre $f$ im Intervall $[0,1]$ Lipschitz-stetig (Widerspruchsannahme), so gäbe es
ein $L \in \mathbb{R}, L \geq 0$ sodass gilt:
$$
    \forall x,y \in [0,1] \colon |f(x)-f(y)| = |\sqrt{x} -\sqrt{y} | \leq L |x-y|
$$
Sei nun $x = 2y \in [0,1]$. Wir formen die das Libschitz-Kriterium leicht um. Es müsste
gelten:
$$
    \forall y \in \left[0,\frac{1}{2}\right], x \in [0,1], x = 2y\colon
    \frac{|\sqrt{x} - \sqrt{y}|}{|x-y|}
    = \frac{\sqrt{2y} - \sqrt{y}}{2y-y}
    = \frac{\sqrt{2} - 1}{\sqrt{y}} \leq L
$$
Es lässt sich jedoch zu jedem $M > 0$ ein $a = \dfrac{1}{M^2}$ wählen, sodass für $0 < x < a$ gilt:
$$
    \frac{\sqrt{2}-1}{x} > \frac{1}{\sqrt{x}} > \frac{1}{\sqrt{a}} = M
$$
Es gilt also $\lim_{x \downarrow 0}\limits \dfrac{\sqrt{2} - 1}{\sqrt{x}} = \infty$,
damit wäre die Funktion im gegebenen Intervall für $x = 2y$ nach oben unbeschränkt.
Dies stellt einen Widerspruch dar, da sonst $L$ eine obere Schranke
für eine nach oben unbeschränkte Funktion darstellen würde. Damit haben wir ein
Gegenbeispiel gezeigt, folglich ist $f$ nicht Lipschitz-stetig.
\newpage

\textbf{c)}

Es sei eine Lipschitz-stetige Funktion $f$ mit Definitionsbereich $D$ und Lipschitzkonstante\\
$L \in \mathbb{R},\ L \geq 0$ gegeben. Es gilt also:
$$
    \forall x,y \in D\colon
    |f(x)-f(y)| \leq L|x-y|
$$
Zu gegebenem $\varepsilon > 0$ wählen wir nun $\delta = \dfrac{\varepsilon}{L}$. Es folgt:
$$
    \forall x,y \in D,\,|x-y| < \delta \colon
    |f(x)-f(y)| \leq L|x-y| < L \delta = \varepsilon
$$
Damit ist $f$ für jeden Häufungspunkt $y \in D$ stetig, also ist $f$ stetig.


\aufgabe{3}
\textbf{a)}

Es sei eine beliebige Folge $x_n$ mit $\lim_{n \to \infty}\limits x_n = 0$ gegeben.
Die Exponentialfunktion ist stets größer 0 und streng monoton steigend (III 3.21).
Weiterhin gilt:
$$
    \exp(0) = 1
    \qquad\text{und}\qquad
    \exp(-1) = \frac{1}{e} < 1
    \quad\text{also durch strenge Monotonie}\quad
    \forall x < 0 \colon \exp(x) < 1
$$
Weiterhin gilt stets $x^2 \geq 0$ und $\dfrac{-1}{x^2} < 0$ für $x\neq 0$.
Daher folgt also für $x \neq 0\colon$
$$
    0 < \left|\exp\left(\frac{-1}{x^2}\right)\right| < 1
    \quad \Longleftrightarrow\quad
    0 < \left|x\cdot\exp\left(\frac{-1}{x^2}\right)\right| < |x|
$$
Da $f(0) = 0$ gilt dann auch:
$$
    0 \leq |f(x_n)| \leq |x_n|
$$
Es gilt jedoch nach Konstruktion von $x_n\colon$
$$
    \lim_{n \to \infty}\limits x_n
    = \lim_{n \to \infty}\limits |x_n|
    = 0
    = \lim_{n \to \infty}\limits 0
$$
Somit folgt mit dem Sandwich Lemma, dass $\lim_{n \to \infty}\limits |f(x_n)| = 0$.
Weiterhin gilt nach Definition auch $f(0) = 0$, also folgt:
$$
    \lim_{n \to \infty} f(x_n) = 0 = f(0) = f\left(\lim_{n \to \infty} x_n\right)
$$
Damit ist $f$ stetig in 0.

\textbf{b)}

Da $\mathbb{R}$ dicht ist, folgt, dass 0 ein Häufingspunkt von $\mathbb{R}$ ist.\\
Zu gegebenem $\varepsilon > 0$ wählen wir $\delta = \varepsilon$.
Mit $\forall x \in \mathbb{R}\colon -1 \leq \sin(x) \leq 1$ (*) folgt:
$$
    \forall x \in \mathbb{R},\, |x-0| < \delta\colon
    |f(x) - f(0)| = |f(x)| = \left|x \sin\left(\frac{1}{x}\right)\right|
    \up{\leq}{*} |x| < \delta = \varepsilon
$$
Damit ist $f$ stetig in 0.

\newpage

\aufgabe{4}
\textbf{a)}

Sei als Abkürzung $f\colon \mathbb{R} \to \mathbb{R}\colon x \mapsto \cos(x) -x$, da
$\cos$  auf ganz $\mathbb{R}$ definiert ist.

Für $a = 0 < b = 2\pi$ gilt:
$$
    f(a) = \cos(0)-0 = 1 > \frac{1}{2} = h(0)
    \quad\text{und}\quad
    f(b) = \cos(2\pi)-2\pi = 1 -2\pi < \frac{1}{2} = h(\pi)
$$
Sei nun $m = \min\{f(x) \mid x\in[0, 2\pi]\}$ sowie
$M = \max\{f(x)\mid x\in[0,2\pi]\}$.\\[2pt]
Es folgt, dass $m \leq f(b) = 1-2\pi$ und $M \geq f(a) = 1$, also auch
$[1-2\pi, 1] \subseteq [m,M]$.\\[2pt]
Weiterhin gilt nach Zwischenwertsatz (2.9 b) nun $f([a,b]) = [m,M]$.\\[2pt]
Da aber $\frac{1}{2} \in [1-2\pi, 1] \subseteq [m,M]$ ist und $f$ stetig ist, gilt also:
$$
    \exists c \in (a,b) \colon f(c) = \frac{1}{2} = h(c)
$$
Damit gibt es mindestens eine Lösung für die Gleichung $h(x) = \cos(x)-x = f(x)$.


\textbf{b)}

Wir definieren $g\colon [0,1] \to \mathbb{R}\colon x \mapsto f(x) -f(x-1)$.
Es gilt:
$$
    g(0) = f(0)-f(-1) = f(0)-f(1)
    \qquad\text{und}\qquad
    g(1) = f(1)-f(0)
$$
Wir Unterscheiden nun 3 Fälle:

\textbf{Fall 1:} $f(0) > f(1)\colon$\\
Dann gilt $g(0) = f(0) - f(1) > 0$ und $g(1) = f(1) - f(0) <0 $. Da $f$ stetig ist gibt
es also nach Zwischenwertsatz ein $x \in (0,1)$ sodass $g(x) = f(x) - f(x-1) = 0$, also $f(x) = f(x-1)$ gilt.

\textbf{Fall 2:} $f(1) > f(0)\colon$\\
Dies ist analog zu Fall 1:
$$
    (g(0) < 0) \land (g(1) > 0)\land(\text{f stetig}) \,\Longrightarrow\, \exists x \in (0,1)\colon g(x) = 0
    \,\Longrightarrow\, f(x) = f(x-1)
$$
\textbf{Fall 3:} $f(1) = f(0)\colon$\\
Damit ist die Aussage schon bewiesen, da $1 \in [0,1]$ und dann $f(1) = f(1-1) = f(0)$ gilt.

Insgesamt gibt es stets mindestens ein $x \in [0,1]$, sodass $f(x) = f(x-1)$ gilt.
\end{document}
