\documentclass[a4paper,graphics,11pt]{article}
\pagenumbering{arabic}

\usepackage[margin=1in]{geometry}
\usepackage[utf8]{inputenc}
\usepackage[T1]{fontenc}
\usepackage{lmodern}
\usepackage[ngerman]{babel}
\usepackage{amsmath, tabu}
\usepackage{amsthm}
\usepackage{amssymb}
\usepackage{complexity}
\usepackage{mathtools}
\usepackage{setspace}
\usepackage{graphicx,color,curves,epsf,float,rotating}
\usepackage{tasks}
\setlength{\parindent}{0em}
\setlength{\parskip}{1em}

\newcommand{\aufgabe}[1]{\subsection*{Aufgabe #1}}
\newcommand{\up}[2]{\mathrel{\overset{\makebox[0pt]{\mbox{\normalfont\tiny #2}}}{#1}}}

\begin{document}
\noindent Gruppe \fbox{\textbf{11}}             \hfill Tobias Riedel, 379133 \\
\noindent Analysis für Informatiker             \hfill Phil Pützstück, 377247 \\

\begin{center}
     \LARGE{\textbf{Hausaufgabe 5}}
\end{center}
\begin{center}
\rule[0.1ex]{\textwidth}{1pt}
\end{center}



\aufgabe{1}
Seien $M_1$ und $M_2$ zwei abzählbare Mengen. Wir können ohne Einschränkung annehmen,
dass $M_1 \neq \o$ und $M_2 \neq \o$. Da $M_1$ und $M_2$ abzählbar sind, existieren
injektive Abbildungen $f_1\colon M_1 \rightarrow \mathbb{N}$ und $f_2 \colon M_2 \rightarrow \mathbb{N}$. Sei nun $C = A \up{\cup}{.}B$. Wir definieren
$$
    f_3\colon C \rightarrow \mathbb{N} \colon c\mapsto \begin{cases}
        2\cdot f_1(c) & \text{falls}\ c \in A\\
        2\cdot f_2(c)+1 & \text{falls}\ c \in B
    \end{cases}
$$
Wir unterscheiden nun für beliebig aber feste $a,b \in C$ folgende Fälle:

\textbf{Fall 1:} $a,b \in A$:\\
Da $f_1$ injektiv ist, muss $f_1(x)\neq f_1(y)$ gelten. Somit auch $2\cdot f_1(x) \neq 2\cdot f_1(y) \,\Leftrightarrow\, f_3(x)\neq f_3(y)$

\textbf{Fall 2:} $a,b \in B$:\\
Dies ist analog zu Fall 1. Es muss gelten: $2\cdot f_2(x)+1 \neq 2\cdot f_2(y)+1
\,\Leftrightarrow\, f_3(x)\neq f_3(y)$

\textbf{Fall 3:} $a \in A \land b \in B$\\
Ohne Beschränkung der Allgemeinheit ist dieser Fall analog zum Fall $a \in B \land b \in A$.
Da $f_1$ und $f_2$ beide auf $\mathbb{N}$ abbilden, folgt mit $M_1=\{x\mid\text{x ist gerade}\}$ und $M_2=\{x\mid\text{x ist ungerade}\}$:
$$
    \forall c \in C \colon 2\cdot f_1(c) \in M_1
    \quad\text{sowie}\quad
    \forall c \in C \colon 2\cdot f_2(c)+1 \in M_2
$$
Da eine natürliche Zahl nicht gerade und ungerade gleichzeitig sein kann, folgt nach\\
Konstruktion von $f_3$ :
$$
    f_3(a) \neq f_3(b)
$$
Somit gilt für jeden Fall, bzw. $\forall a\in A, b\in B\colon f_3(a) \neq f_3(b)$. Also ist $f_3$ injektiv.
Daher gibt es eine injektive Abbildung $f_3 \colon C\rightarrow \mathbb{N}$.
Nach Definition ist dann die Vereinigung zweier disjunkter abzählbarer Mengen ebenfalls abzählbar.

Da $\mathbb{Q}^\complement := \mathbb{R}\setminus \mathbb{Q}$, folgt $\mathbb{Q} \cup \mathbb{Q}^\complement = \mathbb{R}$. Somit auch $\mathbb{Q} \cap \mathbb{Q}^\complement = \o$.
Da $\mathbb{Q}$ abzählbar ist und die Vereinigung zweier disjunkter abzählbarer Mengen
wie gerade bewiesen ebenfalls abzählbar ist, muss $\mathbb{Q}^\complement$ überabzählbar sein, da sonst folgen würde, dass $\mathbb{R}$ abzählbar ist:
\begin{alignat*}{1}
    &(\mathbb{R} = \mathbb{Q} \cup \mathbb{Q}^\complement)
    \land(\mathbb{R}\ \text{überabzählbar}) \land (\mathbb{Q}\ \text{abzählbar})\\
    \,\Longrightarrow\quad&(\mathbb{Q}\ \text{nicht abzählbar} \lor \mathbb{Q}^\complement\ \text{nicht abzählbar}) \land (\mathbb{Q}\ \text{abzählbar})\\
    \,\Longrightarrow\quad&(\mathbb{Q}^\complement\ \text{nicht abzählbar})\\
\end{alignat*}

\newpage
\aufgabe{2}
\textbf{a)} $\lim_{n \to \infty}\limits \frac{\sqrt{n} + \sqrt[3]{n}}{3\sqrt{n} +1}$\\[5pt]
Nach den Limitenregeln lässt sich der Grenzwert wie folgt bestimmen:
$$
    \lim_{n \to \infty} \frac{\sqrt{n} + \sqrt[3]{n}}{3\sqrt{n} +1}
    = \lim_{n \to \infty}\limits \frac{1+\frac{\sqrt[3]{n}}{\sqrt{n}}}{3+\frac{1}{\sqrt{n}}}
    = \frac{\lim_{n \to \infty}\limits\left(1+\frac{\sqrt[3]{n}}{\sqrt{n}}\right)}{\lim_{n \to \infty}\limits\left(3+\frac{1}{\sqrt{n}}\right)}
    = \frac{\lim_{n \to \infty}\limits 1+ \lim_{n \to \infty}\limits\frac{\sqrt[3]{n}}{\sqrt{n}}}{\lim_{n \to \infty}\limits3+\lim_{n \to \infty}\limits\frac{1}{\sqrt{n}}}\\[10pt]
$$
$$
    = \frac{1+ \lim_{n \to \infty}\limits\frac{\sqrt[3]{n}}{\sqrt{n}}}{3+\lim_{n \to \infty}\limits\frac{1}{\sqrt{n}}}
    = \frac{1+ \lim_{n \to \infty}\limits\frac{\sqrt[3]{n}}{\sqrt{n}}}{3+0}
    = \frac{1+ \lim_{n \to \infty}\limits\frac{1}{\sqrt[6]{n}}}{3}
    = \frac{1+ \lim_{n \to \infty}\limits\frac{1}{n^{1/6}}}{3}
    = \frac{1+0}{3}
    = \frac{1}{3}
$$
Somit ist der Grenzwert der Folge $\dfrac{1}{3}$. Da die Folge einen Grenzwert besitzt, konvergiert sie auch gegen diesen.

\textbf{b)} $a_n = n\cdot \left(\sqrt{n^3+2} -\sqrt{n^3+1}\right)$\\[5pt]
Durch Umformungen erhält man:
$$
    n\left(\sqrt{n^3+2} - \sqrt{n^3+1}\right) = \frac{n}{\sqrt{n^3+2} + \sqrt{n^3+1}}
$$
Wir gehen nach dem Sandwich-Lemma und definieren vorerst:
$$
    (b_n)_{n \in \mathbb{N}} := \frac{1}{\sqrt{n^3+2}+\sqrt{n^3+1}}\quad \text{und}\quad
    (c_n)_{n \in \mathbb{N}} := \frac{1}{\sqrt{n}}
$$
Es folgt für alle $n\in \mathbb{N}\colon$ 
$$
    \frac{1}{\sqrt{n}} \geq \frac{n}{\sqrt{n^3+2}+\sqrt{n^3+1}}\quad\text{und}\quad
    \frac{1}{\sqrt{n^3+2} + \sqrt{n^3+1}} \leq \frac{n}{\sqrt{n^3+2} + \sqrt{n^3+1}}
$$
Also gilt stets $b_n \leq a_n \leq c_n$. Wir wissen nach 1.6, dass
$\lim_{n \to \infty}\limits c_n = 0$ gilt. Außerdem folgt analog zu Beispiel 1.11, dass
$\sqrt{n^3+2} + \sqrt{n^3+1} $ unbeschränkt nach oben ist und damit $b_n$ gegen 0
konvergiert; Zu jedem $C\in \mathbb{R}$ mit $C>0$ gibt es ein $N = \lceil C^2 \rceil \in \mathbb{N}$ sodass gilt:
$$
    \forall n \in \mathbb{N}, n\geq N \colon \sqrt{n^3+2} + \sqrt{n^3+1} \geq \sqrt{n^3} \geq \sqrt{n} \geq \sqrt{N}
    \geq \sqrt{C^2} = C
$$
Also ist $(b^{-1})_{n\in \mathbb{N}} := \sqrt{n^3+2} + \sqrt{n^3+1}$ nach oben unbeschränkt. Weiterhin lässt sich
durch Induktion zeigen, dass $(b^{-1})_n$ monoton steigt:
Sei $A(n) := \left(\forall n \in \mathbb{N}\colon (b^{-1})_{n+1} \geq (b^{-1})_n \right)$\\[5pt]
\textbf{(IA)} Es gilt $\sqrt{2^3+2} + \sqrt{2^3+1} \geq \sqrt{1^3+2} + \sqrt{1^3+1}$. Also
gilt $A(1)$.

\textbf{(IS)} Es gelte $A(n)$ für ein $n\in \mathbb{N}$. $n\mapsto n+1\colon$
$$
\sqrt{(n+2)^3+2} + \sqrt{(n+2)^3+1} \geq \sqrt{(n+1)^3+2} + \sqrt{(n+1)^3+1}
$$
Nach Prinzip der vollständigen Induktion gilt nun $\forall n \in \mathbb{N}\colon A(n)$.
Somit ist $\sqrt{n^3+2} + \sqrt{n^3+1}$ monoton steigend und unbeschränkt, und der Kehrwert,
$b_n$, geht dann für $n\mapsto \infty$ gegen 0.
Es folgt, dass $\lim_{n \to \infty}\limits b_n = \lim_{n \to \infty}\limits c_n = 0$ und
$\forall n \in \mathbb{N}\colon b_n \leq a_n \leq c_n$, also gilt nach Sandwich-Lemma nun
auch:
$$
    \lim_{n \to \infty} n\left(\sqrt{n^3+2} -\sqrt{n^3+1}\right) = 0
$$

\newpage
\textbf{c)} $\lim_{n \to \infty}\limits \left(\frac{n^2+2nz+z^2}{n^2}\right)^n$\\[5pt]
Zuerst zeigen wir eine Umformung:
$$
    \left(\frac{n^2+2nz+z^2}{n^2}\right)^n = \left(\frac{(n+z)^2}{n^2}\right)^n
    = \left(\frac{n+z}{n}\right)^{2n} = \left(1+\frac{z}{n}\right)^{2n}
$$
Somit folgt nun durch den Hinweis und die Grenzwertgesetze:
$$
    \lim_{n \to \infty} \left(1+\frac{x}{n}\right)^n = e^x\quad\Longrightarrow\quad
    \lim_{n \to \infty} \left(\frac{n^2+2nz+z^2}{n^2}\right)^n
    = \lim_{n \to \infty} \left(1+\frac{z}{n}\right)^{2n}
    = \lim_{n \to \infty} \left(\left(1+\frac{z}{n}\right)^n\right)^2
$$$$
    = \lim_{n \to \infty} \left(\left(1+\frac{z}{n}\right)^{n} \cdot 
    \left(1+\frac{z}{n}\right)^{n}\right)
    = \lim_{n \to \infty} \left(1+\frac{z}{n}\right)^{n} \cdot 
    \lim_{n \to \infty} \left(1+\frac{z}{n}\right)^{n}
    = e^z \cdot e^z = e^{2z}
$$
Es folgt, dass die Folge gegen $e^{2z}$ konvergiert, da wie gezeigt $e^{2z}$ der Grenzwert
ist und somit die Folge auch gegen diesen konvergiert.
\aufgabe{3}
\textbf{a)} Sei $A(n) := \left(\forall n \in \mathbb{N}\colon a_n \geq \frac{1}{3}\right)$\\[5pt]
\textbf{(IA)} Wir zeigen, dass die Aussage für $n=1$ gilt: $a_1 := 1 \geq \dfrac{1}{3}$. Also gilt $A(1)$.\\[5pt]
\textbf{(IS)} Es gelte nun $A(n)$ für ein $n\in \mathbb{N}$, d.h. $a_n \geq \frac{1}{3}$. Nun für $n\mapsto n+1$:
$$
    a_n \geq \frac{1}{3} \,\Longrightarrow\, a_{n+1} = \frac{4a_n}{3a_n+3} \geq \frac{\frac{4}{3}}{\frac{3}{3} +3} = \frac{1}{3}
$$
Somit gilt nach Prinzip der Induktion $\forall n \in \mathbb{N}\colon A(n)$. $a_n$ ist also nach unten beschränkt.\\
\textbf{b)}
Wir zeigen nun durch Induktion, dass $a_n$ monoton fallend ist.\\Sei $A(n):=\left(\forall n \in \mathbb{N} \colon a_n \geq a_{n+1}\right)$:\\[5pt]
\textbf{(IA)} $n=1$. Es gilt $a_1 := 1$ und $a_2 = \frac{4}{6}$. Da $1 > \frac{4}{6}$ gilt nun $A(1)$.\\[5pt]
\textbf{(IS)} Es gelte $A(n)$ für ein $n \in \mathbb{N}$. $n\mapsto n+1$:
$$
    a_{n+1} = \frac{4a_n}{3a_n+3} \up{\quad\geq\quad}{$a_n \geq \frac{1}{3}$}
    \frac{4\left(\frac{4a_n}{3a_n+3}\right)}{3\left(\frac{4a_n}{3a_n+3}\right)+3}
    = \frac{4a_{n+1}}{3a_{n+1}+3} = a_{n+2}
$$
Somit gilt nach Prinzip der vollständigen Induktion $\forall n \in \mathbb{N}\colon A(n)$.
Damit ist $a_n$ monoton fallend und beschränkt. Also konvergiert $a_n$.

\textbf{c)} 
Da $a_n$ monoton fallend ist, gilt nun $\forall n \in \mathbb{N} \colon 1\geq a_n \geq \frac{1}{3}$. Es folgt:
$$
 L = \lim_{n \to \infty}\limits a_n = \lim_{n \to \infty} a_{n+1} = \frac{4L}{3L+3}
$$
Wir lösen also nach $L$:
$$
    L = \frac{4L}{3L+3} \,\Longleftrightarrow\, 3L^2-L = 0
$$
Durch die Mitternachtsformel erhalten wir also:
$$
    \frac{-(-1)\pm\sqrt{(-1)^2-4\cdot3\cdot0}}{2\cdot 3} = \frac{1\pm1}{6}
    \,\Longrightarrow\, L_1 = 0, L_2 = \frac{2}{6} = \frac{1}{3}
$$
Da jedoch $(a_n)_{n\in \mathbb{N}} \geq \frac{1}{3}$ gilt, kann $L_1$ für unseren Fall ignoriert
werden. Somit ist
$$
    L=L_2 = \frac{1}{3} = \lim_{n \to \infty} (a_n)_{n \in \mathbb{N}}
$$
Also konvergiert $(a_n)_{n\in \mathbb{N}}$ gegen $\frac{1}{3}$.

\newpage
\aufgabe{4}
Wir zeigen zunächst die Identitäten:\\[5pt]
\begin{equation}
\begin{aligned}
   \frac{1+\sqrt{5}}{2} +1 = \frac{1 +\sqrt{5}}{2} + \frac{2}{2} = \frac{3+\sqrt{5}}{2}
   = \frac{6+2\sqrt{5}}{4} = \frac{1^2+2\sqrt{5}+5}{2^2} = \left(\frac{1+\sqrt{5}}{2}\right)^2\\
    \frac{1-\sqrt{5}}{2} +1 = \frac{1 -\sqrt{5}}{2} + \frac{2}{2} = \frac{3-\sqrt{5}}{2}
   = \frac{6-2\sqrt{5}}{4} = \frac{1^2-2\sqrt{5}+5}{2^2} = \left(\frac{1-\sqrt{5}}{2}\right)^2
\end{aligned}
\end{equation}
Nun die Induktion:\\
\textbf{(IA)} Wir zeigen dass $a_0$ hält, also für $n=0$\,:
$$
    \frac{1}{\sqrt{5}}\left(\left(\frac{1+\sqrt{5}}{2}\right)^1-\left(\frac{1-\sqrt{5}}{2}\right)^1\right)
    = \frac{1}{\sqrt{5}}\left(\frac{1+\sqrt{5} - 1 + \sqrt{5}}{2}\right)
    = \frac{1}{\sqrt{5}} \cdot \frac{2\sqrt{5}}{2} = 1
$$
Da $n_0 := 1$, gilt unsere Annahme nun für $n = 0$.\\[5pt]
\textbf{(IS)} Es gelte $A(n)$ für ein $n \in \mathbb{N}$. 
Sei $x := \dfrac{1+\sqrt{5}}{2}$ und $y:= \dfrac{\sqrt{5}-1}{2}$, $n\mapsto n+1$\,:
$$
    a_{n+1} = \frac{1}{\sqrt{5}}\left(\left(\frac{1+\sqrt{5}}{2}\right)^{n+2}-\left(\frac{1-\sqrt{5}}{2}\right)^{n+2}\right)
    = \frac{1}{\sqrt{5}} \left(x^{n+2} - y^{n+2}\right)
$$$$
    = \frac{1}{\sqrt{5}} \left(x^2x^n - y^2y^n\right)
    \up{=}{(1)} \frac{1}{\sqrt{5}} \left(\left(x+1\right)\cdot x^n - \left(y+1\right)\cdot y^n\right)
$$$$
    = \frac{1}{\sqrt{5}} (x^{n+1}+x^n - y^{n+1}-y^n)
    = \frac{1}{\sqrt{5}} (x^{n+1}-y^{n+1}+x^n-y^n)
$$$$
    = \frac{1}{\sqrt{5}} (x^{n+1}-y^{n+1}) + \frac{1}{\sqrt{5}}(+x^n-y^n)
    = a_{n} + a_{n-1}
$$
Da nach Definition $a_{n+1} := a_n + a_{n-1}$ gilt die Aussage $A(n)$ nach dem Prinzip der vollständigen Induktion nun für alle $n\in \mathbb{N}.\hfill\square$\\
Da die Fibonacci-folge eine Summe aus natürlichen Zahlen ist, ist $a_n \geq 1$. Somit:
\begin{equation}
    \frac{a_{n+2}}{a_{n+1}} = 1+\frac{a_{n}}{a_{n+1}} = 1+\frac{1}{\left(\dfrac{a_{n+1}}{a_{n}}\right)}
\end{equation}
Für den Grenzwert $L$ folgt durch $n\mapsto \infty$:
$$
    L = \lim_{n \to \infty}\frac{a_{n+1}}{a_n} = \lim_{n \to \infty}\frac{a_{n+2}}{a_{n+1}}
    \up{\,\Longrightarrow\,}{(2)} L = 1+ \frac{1}{L} \,\Longrightarrow\, L^2-L-1 = 0
$$
Die vorgegebene Grenzwert erfüllt genau diese Eigenschaft:
$$
    L^2-L-1 = \left(\frac{1+\sqrt{5}}{2}\right)^2-\frac{1+\sqrt{5}}{2}-1
    \up{=}{(1)} \left(\frac{1+\sqrt{5}}{2}+1\right) -\frac{1+\sqrt{5}}{2}-1 = 0
$$
Da zu jeder Folge immer nur ein Grenzwert existiert (Satz 1.7), gilt nun
$$\lim_{n \to \infty} \frac{a_{n+1}}{a_n} = \frac{1+\sqrt{5}}{2}$$
\end{document}
