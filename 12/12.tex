\documentclass[a4paper,graphics,11pt]{article}
\pagenumbering{arabic}

\usepackage[margin=1in]{geometry}
\usepackage[utf8]{inputenc}
\usepackage[T1]{fontenc}
\usepackage{lmodern}
\usepackage[ngerman]{babel}
\usepackage{amsmath, tabu}
\usepackage{amsthm}
\usepackage{amssymb}
\usepackage{complexity}
\usepackage{mathtools}
\usepackage{setspace}
\usepackage{graphicx,color,curves,epsf,float,rotating}
\usepackage{tasks}
\setlength{\parindent}{0em}
\setlength{\parskip}{1em}

\newcommand{\aufgabe}[1]{\subsection*{Aufgabe #1}}
\newcommand{\up}[2]{\mathrel{\overset{\makebox[0pt]{\mbox{\normalfont\tiny #2}}}{#1}}}

\begin{document}
\noindent Gruppe \fbox{\textbf{11}}             \hfill Tobias Riedel, 379133 \\
\noindent Analysis für Informatiker             \hfill Phil Pützstück, 377247 \\
\strut\hfill Kevin Holzmann, 371116\\
\strut\hfill Gurvinderjit Singh, 369227
\begin{center}
	\LARGE{\textbf{Hausaufgabe 12}}
\end{center}
\begin{center}
\rule[0.1ex]{\textwidth}{1pt}
\end{center}



\aufgabe{1}
\textbf{a)}
$$
    \lim_{h \to 0} \frac{f(x_0+h)-f(x_0)}{h}
    = \lim_{h \to 0} \frac{|h\sin(h)|}{h}
    = \left\{\begin{array}{lr}
        \lim_{h \to 0}\limits |\sin(h)| & \text{für } h \geq 0\\
        \\
        \lim_{h \to 0}\limits -|\sin(h)| & \text{für } h < 0
    \end{array}\right\} = 0
$$
Da nun der Grenzwert existiert, folgt Differenzierbarkeit von $f$ in $x_0$.

\textbf{b)}
$$
    \lim_{h \to 0} \frac{g(x_0+h)-g(x_0)}{h}
    = \lim_{h \to 0} \frac{g(h)-\sin(0)}{h}
    = \left\{\begin{array}{lr}
        \lim_{h \to 0}\limits \frac{\sin(h)}{h} = 1& \text{für } h \geq 0\\
        \\
        \lim_{h \to 0}\limits \frac{0}{h} = 0 & \text{für } h < 0
    \end{array}\right\}
$$
Da nun also der rechtsseitige Grenzwert ungleich dem linksseitigem ist, existiert der Grenzwert nicht.
Es ist also $g$ in $x_0$ nicht differenzierbar.

\textbf{c)}
$$
    \lim_{h \to 0} \frac{j(x_0+h)-j(x_0)}{h}
    = \lim_{h \to 0} \frac{j(h)-\sin^n(0)}{h}
    = \left\{\begin{array}{lr}
        \lim_{h \to 0}\limits \frac{\sin^n(h)}{h} = ?& \text{für } h \geq 0\\
        \\
        \lim_{h \to 0}\limits \frac{0}{h} = 0 & \text{für } h < 0
    \end{array}\right\}
$$


\aufgabe{2}
\textbf{a)}
Es sei $x \in \mathbb{R}$ gegeben. Es gilt:
$$
    \lim_{h \to 0} \frac{f(x+h)-f(x)}{h}
    = \lim_{h \to 0} \frac{x+h-x}{h}
    = \lim_{h \to 0} \frac{h}{h} = 1
$$

\textbf{b)}
Es sei $x \in \mathbb{R}\setminus \{0\}$ gegeben. Es gilt:
$$
    \lim_{h \to 0} \frac{g(x+h)-g(x)}{h}
    = \lim_{h \to 0} \frac{\frac{1}{x+h} - \frac{1}{x}}{h}
    = \lim_{h \to 0} \frac{x-x+h}{(x^2+hx)h}
    = \lim_{h \to 0} \frac{-1}{x^2+hx}
    = -\frac{1}{x^2}
$$
\newpage

\textbf{c)}
Es sei $n=1$. Es gilt für gegebenes $x \in \mathbb{R}\colon$
$$
    \lim_{h \to 0} \frac{k_n(x+h)-k_n(x)}{h}
    = \lim_{h \to 0} \frac{x+h-x}{h}
    = 1
$$
Also gilt die Behauptung für ein $n \in \mathbb{N}$.
Sei nun ein $n \in \mathbb{N}$ gegeben sodass die Behauptung gilt, also $k(x) = x^n$ auf ganz $\mathbb{R}$
differenzierbar ist. Es folgt für $n+1\colon$
$$
    \lim_{h \to 0} \frac{k_{n+1}(x+h)-k_{n+1}(x)}{h}
    = \lim_{h \to 0} \frac{(x+h)^{n+1}-x^{n+1}}{h}
$$$$
    = \lim_{h \to 0} \frac{\displaystyle\left(\sum_{k=0}^{n+1}\binom{n+1}{k}x^{n+1-k}h^k\right)-x^{n+1}}{h}
    = \lim_{h \to 0} \frac{\displaystyle\sum_{k=1}^{n+1}\binom{n+1}{k}x^{n+1-k}h^k}{h}
$$$$
    = \lim_{h \to 0} \sum_{k=1}^{n+1}\binom{n+1}{k}x^{n+1-k}h^{k-1}
    = \lim_{h \to 0} \binom{n+1}{1}x^{n}h^0+\sum_{k=2}^{n+1} \binom{n+1}{k}x^{n+1-k}h^{k-1}
$$$$
    = (n+1)x^n + \sum_{k=2}^{n+1} \binom{n+1}{k}x^{n+1-k}0^{k-1}
    = (n+1)x^n
$$

Der Differenzenquotient existiert dann also auch für $k_{n+1}$ und ist ein Skalar von $k_n$, welche
ebenfalls differenzierbar ist.
Nach dem Prinzip der vollständigen Induktion existiert nun also der Differenzenquotient von $k_n$ für
alle $n \in \mathbb{N}$.

\textbf{d)}
Mit der geometrischen Summe lässt sich herleiten:
$$
    (a-b)\sum_{k=0}^{n-1} a^{n-1-k}b^k
    = (a-b)a^{n-1}\sum_{k=0}^{n-1} \left(\frac{b}{a}\right)^k
    = (a-b)a^{n-1}\frac{1-\left(\frac{b}{a}\right)^n}{1-\frac{b}{a}}
$$$$
    = (a-b)a^{n-1} \frac{\left(\frac{a^n-b^n}{a^n}\right)}{\left(\frac{a-b}{a}\right)}
    = (a-b)a^{n-1} \frac{a(a^n-b^n)}{a^n(a-b)} = a^n-b^n
$$

Damit folgt nun für gegebenes $n \in \mathbb{N}$ und $x \in \mathbb{R}\setminus \{0\}\colon$
$$
    \lim_{h \to 0} \frac{j(x+h)-j(x)}{h}
    = \lim_{h \to 0} \frac{\frac{1}{(x+h)^n} - \frac{1}{x^n}}{h}
    = \lim_{h \to 0} \frac{\left(\frac{x^n-(x+h)^n}{x^n(x+h)^n}\right)}{h}
$$$$
    = \lim_{h \to 0} \frac{(x-(x+h))\sum_{k=0}^{n-1} x^{n-k-1}(x+h)^k}{hx^n(x+h)^n} 
    = \lim_{h \to 0} -\sum_{k=0}^{n-1} x^{-k-1}(x+h)^{k-n}
$$$$
    = -\sum_{k=0}^{n-1} x^{-(n+1)} = \frac{-n}{x^{n+1}}
$$

Also existiert der Differenzenquotient von $j_n(x)$ für beliebige $n \in \mathbb{N}$.

\newpage

\aufgabe{3}

\textbf{a)}
Mit partieller Integration (gekz. P) sowie der Gleichung $\sin^2(x)+\cos^2(x) = 1$ folgt:
\begin{alignat*}{2}
    \int \cos^2(x)\, \mathrm{d}x &\ \up{=}{P}\ \cos(x)\sin(x) - \int -\sin(x)\sin(x)\, \mathrm{d}x \\[2pt]
    &\ =\ \cos(x)\sin(x)+\int \sin^2(x)\, \mathrm{d}x \\[2pt]
    &\ =\ \cos(x)\sin(x)+\int 1-\cos^2(x)\, \mathrm{d}x \\[2pt]
    &\ =\ \cos(x)\sin(x)+\int 1\, \mathrm{d}x - \int \cos^2(x)\, \mathrm{d}x \\[2pt]
    &\ =\ \cos(x)\sin(x)+x - \int \cos^2(x)\, \mathrm{d}x
\end{alignat*}
Damit folgt:
\begin{alignat*}{3}
    &\int \cos^2(x)\, \mathrm{d}x &\ =\ &\cos(x)\sin(x)+x - \int \cos^2(x)\, \mathrm{d}x \\[2pt]
    \Longleftrightarrow\quad & 2\int \cos^2(x)\, \mathrm{d}x &\ =\ & \cos(x)\sin(x)+x\\[2pt]
    \Longleftrightarrow\quad & \int \cos^2(x)\, \mathrm{d}x &\ =\ & \frac{\cos(x)\sin(x)+x}{2}\\[2pt]
\end{alignat*}
Als Probe folgt mit der Produktregel und der oben genannten Gleichung:
$$
    \left[\frac{\cos(x)\sin(x)+x}{2}\right]'
    = \frac{1}{2}[\cos(x)\sin(x)]'+\frac{1}{2}[x]'
$$$$
    = \frac{1}{2}(-\sin(x)\sin(x)+\cos(x)\cos(x))+\frac{1}{2}
$$$$
    = \frac{1}{2}(1-\sin^2(x)+\cos^2(x))
$$$$
    = \cos^2(x)
$$

Daher ist also $\dfrac{\cos(x)\sin(x)+x}{2}$ eine Stammfunktion von $\cos^2(x)$.

\textbf{b)}
Nach IV 2.13 gilt $\sin(2x) = 2\sin(x)\cos(x)$. Es folgt mit der Substitutionsregel (gekz. S):
$$
    \int \frac{\sin(2x)}{\cos^2(x)}\, \mathrm{d}x
    = 2\int \frac{\sin(x)}{\cos(x)}\, \mathrm{d}x
    = -2\int \frac{-\sin(x)}{\cos(x)}\, \mathrm{d}x
    \up{=}{S} -2\ln(|\cos(x)|)
$$

Als Probe folgt mit der Kettenregel und wieder der Gleichung:
$$
    [-2\ln(|\cos(x)|)]'
    = \frac{-2[|\cos(x)|]'}{\cos(x)}
    = \frac{2\sin(x)}{\cos(x)}
    = \frac{\sin(2x)}{\cos^2(x)}
$$

Damit ist $-2\ln(\cos(x))$ eine Stammfunktion von $\dfrac{\sin(2x)}{\cos^2(x)}$.
\newpage

\textbf{c)}

$
    \displaystyle\int \frac{e^{2x}}{\sqrt{(3e^x+1)^3}}\, \mathrm{d}x
$
\quad Wir substituieren mit $u = 3e^x+1$, woraus sich auch $e^{2x} = \dfrac{(u-1)^2}{9}$ ergibt.\\[5pt]
Weiterhin gilt nun
$\frac{\mathrm{d}u}{\mathrm{d}x}
= [3e^x+1]'
= 3e^x \,\Longleftrightarrow\, \mathrm{d}x
= \dfrac{\mathrm{d}u}{3e^x}
= \dfrac{\mathrm{d}u}{3\frac{u-1}{3}}
= \dfrac{\mathrm{d}u}{u-1}
$. Wir lösen also:
$$
    \int \frac{(u-1)^2}{9} \cdot \frac{1}{\sqrt{u^3}}\cdot \frac{\mathrm{d}u}{u-1}
    = \frac{1}{9}\int \frac{u-1}{\sqrt{u^3}}\, \mathrm{d}u
$$$$
    = \frac{1}{9}\left(\int \frac{1}{\sqrt{u}}\, \mathrm{d}u - \int \frac{1}{\sqrt{u^3}}\, \mathrm{d}u\right)
    = \frac{1}{9}\left(\frac{u^{\frac{1}{2}}}{\frac{1}{2}} - \frac{u^{-\frac{1}{2}}}{-\frac{1}{2}}\right)
    = \frac{1}{9}\left(2\sqrt{u} + \frac{2}{\sqrt{u}}\right)
$$
Nun resubstituieren wir $u = 3e^x+1$ und erhalten damit:
$$
    \int \frac{e^2x}{\sqrt{(3e^x+1)^3}}\,\mathrm{d}x = \frac{1}{9}\left(2\sqrt{3e^x+1} + \frac{2}{\sqrt{3e^x+1}}\right)
$$$$
    = \frac{6e^x+4}{9\sqrt{3e^x+1}}
$$

Als Probe folgt mit der Kettenregel:
$$
    \left[\frac{6e^x+4}{9\sqrt{3e^x+1}}\right]'
    = \frac{6}{9}\left[\frac{e^x}{\sqrt{3e^x+1}}\right]' + \frac{4}{9}\left[\frac{1}{\sqrt{3e^x+1}}\right]'
$$$$
    = \frac{6}{9}\left(\frac{[e^x]'\sqrt{3e^x+1} -e^x\left[\sqrt{3e^x+1}\right]'}{\left(\sqrt{3e^x+1}\right)^2} \right)
        +\frac{4}{9}\left(-\frac{[3e^x+1]'}{2\sqrt{(3e^x+1)^3}}\right)
$$$$
    = \frac{6}{9}\left(\frac{e^x\sqrt{3e^x+1} - \dfrac{3e^{2x}}{2\sqrt{3e^x+1}}}{3e^x+1} \right)
        +\frac{4}{9}\left(-\frac{3e^x}{2\sqrt{(3e^x+1)^3}}\right)
$$$$
    = \frac{6}{9}\left(\frac{e^x}{\sqrt{3e^x+1}} - \frac{3e^{2x}}{2\sqrt{(3e^x+1)^3}}\right)
        +\frac{4}{9}\left(-\frac{3e^x}{2\sqrt{(3e^x+1)^3}}\right)
$$$$
    = \frac{6e^x}{9\sqrt{3e^x+1}} - \frac{18e^{2x}}{18\sqrt{(3e^x+1)^3}} -\frac{12e^x}{18\sqrt{(3e^x+1)^3}}
$$$$
     = \frac{2e^x}{3\sqrt{3e^x+1}} - \frac{3e^{2x}+2e^x}{3\sqrt{(3e^x+1)^3}}
$$$$
     = \frac{2e^x(3e^x+1) - 3e^{2x}-2e^x}{3\sqrt{(3e^x+1)^3}}
$$$$
    = \frac{e^{2x}}{3\sqrt{(3e^x+1)^3}}
$$

Damit ist $\dfrac{6e^x+4}{9\sqrt{3e^x+1}}$ eine Stammfunktion von $\dfrac{e^{2x}}{3\sqrt{(3e^x+1)^3}}$.

\newpage

\aufgabe{4}

\textbf{a)}

$$
    \int \frac{\sqrt{3}}{5x}+e^{4x}\, \mathrm{d}x
    = \frac{\sqrt{3}}{5}\int \frac{1}{x}\, \mathrm{d}x + \int e^{4x}\, \mathrm{d}x
    = \frac{\sqrt{3}\ln(|x|)}{5} + \frac{e^{4x}}{4} + C,\quad C \in \mathbb{R}
$$
Wir machen die Probe durch differenzieren:
$$
    \left[\frac{\sqrt{3}\ln(|x|)}{5} + \frac{e^{4x}}{4} + C\right]'
    = \frac{\sqrt{3}}{5}[\ln(|x|)]' + e^{4x}
    = \frac{\sqrt{3}}{5x} + e^{4x}
$$

\textbf{b)}

$$
    \int \frac{3}{7}x^8+\frac{25}{3} x^4-2x^3\, \mathrm{d}x
    = \frac{3}{7}\int x^3\, \mathrm{d}x + \frac{25}{3} \int x^4\, \mathrm{d}x - 2 \int x^3\, \mathrm{d}x
$$$$
    = \frac{3}{21}x^9+\frac{5}{3}x^5-\frac{1}{2}x^4 + C,\quad C \in \mathbb{R}
$$
Wir machen die Probe durch differenzieren:
$$
    \left[\frac{3}{21}x^9+\frac{5}{3}x^5-\frac{1}{2}x^4+C\right]'
    = \frac{3}{7}x^8+\frac{25}{3}x^4-2x^3
$$

\textbf{c)}
$$
    \int \cos(3x)+7x^4-\frac{4}{x}\, \mathrm{d}x
    = \int \cos(3x)\, \mathrm{d}x +7\int x^4\, \mathrm{d}x -4\int \frac{1}{x}\, \mathrm{d}x
    = \frac{1}{3}\int 3\cos(3x)\, \mathrm{d}x +\frac{7}{5}x^5-4\ln(|x|)
$$$$
    = \frac{1}{3}\sin(3x)+\frac{7}{5}x^5-4\ln(|x|) + C,\quad C \in \mathbb{R}
$$
Wir machen die Probe durch differenzieren:
$$
    \left[\frac{1}{3}\sin(3x)+\frac{7}{5}x^5-4\ln(|x|)+C\right]'
    =\cos(3x)+7x^4-\frac{4}{x}
$$




\end{document}
