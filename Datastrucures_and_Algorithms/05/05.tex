\documentclass[a4paper,graphics,11pt]{article}
\pagenumbering{arabic}

\usepackage[margin=1in]{geometry}
\usepackage[utf8]{inputenc}
\usepackage[T1]{fontenc}
\usepackage{lmodern}
\usepackage[ngerman]{babel}
\usepackage{amsmath, tabu}
\usepackage{amsthm}
\usepackage{amssymb}
\usepackage{complexity}
\usepackage{mathtools}
\usepackage{setspace}
\usepackage{graphicx,color,curves,epsf,float,rotating}
\usepackage{tasks}
\setlength{\parindent}{0em}
\setlength{\parskip}{1em}

\newcommand{\aufgabe}[1]{\subsection*{Aufgabe #1}}
\newcommand{\up}[2]{\mathrel{\overset{\makebox[0pt]{\mbox{\normalfont\tiny #2}}}{#1}}}

\usepackage{listings}
\usepackage{color}

\definecolor{dkgreen}{rgb}{0,0.6,0}
\definecolor{gray}{rgb}{0.5,0.5,0.5}
\definecolor{mauve}{rgb}{0.58,0,0.82}

\lstset{frame=tb,
    language=Java,
    aboveskip=2mm,
    belowskip=2mm,
    showstringspaces=false,
    columns=flexible,
    basicstyle={\small\ttfamily},
    numbers=left,
    numberstyle=\tiny\color{gray},
    keywordstyle=\color{blue},
    commentstyle=\color{dkgreen},
    stringstyle=\color{mauve},
    breaklines=true,
    breakatwhitespace=true,
    tabsize=4,
    literate={ä}{{\"a}}1 {Ä}{{\"A}}1 {ö}{{\"o}}1 {Ö}{{\"O}}1 {ü}{{\"u}}1 {Ü}{{\"U}}1 {ß}{{\ss}}1
}

\begin{document}
\noindent Gruppe \fbox{\textbf{14}}             \hfill Phil Pützstück, 377247\\
\noindent Datenstrukturen und Algorithmen \hfill Benedikt Gerlach, 376944\\
\strut\hfill Sebastian Hackenberg, 377550\\
\begin{center}
	\LARGE{\textbf{Hausaufgabe 5}}
\end{center}
\begin{center}
\rule[0.1ex]{\textwidth}{1pt}
\end{center}

\aufgabe{1}
\textbf{a)}

Wir zeigen, dass für $n \in \mathbb{N}$ mit $n \geq 2$ stets $T(n) \in \mathcal{O}(n)$.

Sei $n \in [2,15]_\mathbb{N}$. Dann ist
$$
    T(n) = 1 \leq n \log_2 n
$$
Also existiert ein $c = 1 > 0$ sodass $T(n) \leq c\cdot n \log_2 n$.

Sei nun ein $n \in \mathbb{N}$ gegeben, sodass für alle $n' \in \mathbb{N}$ mit $n' < n$ ein $c > 0$ existiert für das
$T(n) \leq c \cdot n\log_2 n$ ist (IV). Es folgt:
$$
    T(n) = 2\cdot T\left(\frac{n}{4}\right) + T\left(\frac{n}{2}\right) + n
    \quad\up{\leq}{IV}\quad 2 c_1\left(\frac{n}{4} \log_2 \frac{n}{4}\right) + c_2 \left(\frac{n}{2} \log_2 \frac{n}{2}\right)
    + n
$$$$
    = c_1\frac{n}{2}(\log_2 n - 2) + c_2\frac{n}{2} (\log_2 n - 1) + n
    \quad\leq\quad c\frac{n}{2} (\log_2 n - 2 + \log_2 n - 1) + n
$$$$
    = c\frac{n}{2} (2\log_2 n - 3) + n
    = cn\log_2 n - \frac{n}{2}
    \leq cn \log_2 n 
$$
wobei $c_1, c_2$ die Konstanten gemäß von (IV) sind und $c := \max \{c_1, c_2\}$.
Folglich gilt nach dem Prinzip der vollständigen Induktion,
dass es zu jedem $n \in \mathbb{N}, n \geq 2$ ein $c > 0$ gibt, sodass $T(n) \leq c\cdot n\log_2 n$ ist.
Folglich ist $T(n) \in \mathcal{O}(n\log_2n).\hfill\square$

\textbf{b)}

Sei $n \in [1,3]_{\mathbb{N}}$. Dann ist
$$
    T(n) = 1 \geq \frac{1}{3} \cdot 3^{\frac{n}{3}}
$$
Also existiert ein $c = \frac{1}{3} > 0$ sodass $T(n) \geq c\cdot 3^{\frac{n}{3}}$.

Sei nun ein $n \in \mathbb{N}$ gegeben, sodass für alle $n' \in \mathbb{N}$ mit $n' < n$ ein
$c > 0$ existiert für das $T(n) \geq c\cdot 3^{\frac{n}{3}}$ ist (IV). Es folgt:
$$
    T(n)
    = T(n-1) + T(n-2) + T(n-3)
    \quad\up{\geq}{IV}\quad c_1 3^{\frac{n-1}{3}} + c_2 3^{\frac{n-2}{3}} + c_3 3^{\frac{n-3}{3}}
$$$$
    3c\cdot 3^{\frac{n-3}{3}} = c\cdot3^\frac{3}{n}
$$
wobei $c_1, c_2, c_3$ die Konstanten gemäß von (IV) sind und $c:= \min \{c_1, c_2, c_3\}$.
Folglich gilt nach dem Prinzip der vollständigen Induktion, dass es zu jedem $n \in \mathbb{N}$ ein
$c > 0$ gibt, sodass $T(n) \geq c\cdot 3^{\frac{n}{3}}$ ist.
Foglich ist $T(n) \in \Omega(3^{\frac{n}{3}}). \hfill\square$

\newpage
\aufgabe{2}
\newpage
\aufgabe{3}
\newpage
\aufgabe{4}


\end{document}
