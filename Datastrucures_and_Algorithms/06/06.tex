\documentclass[a4paper,graphics,11pt]{article}
\pagenumbering{arabic}

\usepackage[margin=1in]{geometry}
\usepackage[utf8]{inputenc}
\usepackage[T1]{fontenc}
\usepackage{lmodern}
\usepackage[ngerman]{babel}
\usepackage{amsmath, tabu}
\usepackage{amsthm}
\usepackage{amssymb}
\usepackage{complexity}
\usepackage{mathtools}
\usepackage{setspace}
\usepackage{graphicx,color,curves,epsf,float,rotating}
\usepackage{tasks}
\setlength{\parindent}{0em}
\setlength{\parskip}{1em}

\newcommand{\aufgabe}[1]{\subsection*{Aufgabe #1}}
\newcommand{\up}[2]{\mathrel{\overset{\makebox[0pt]{\mbox{\normalfont\tiny #2}}}{#1}}}

\usepackage{listings}
\usepackage{color}

\definecolor{dkgreen}{rgb}{0,0.6,0}
\definecolor{gray}{rgb}{0.5,0.5,0.5}
\definecolor{mauve}{rgb}{0.58,0,0.82}

\usepackage{tikz}

\tikzset{
  heap/.style={
    every node/.style={circle,draw},
    level 1/.style={sibling distance=30mm},
    level 2/.style={sibling distance=10mm}
  }
}

\lstset{frame=tb,
    language=Java,
    aboveskip=2mm,
    belowskip=2mm,
    showstringspaces=false,
    columns=flexible,
    basicstyle={\small\ttfamily},
    numbers=left,
    numberstyle=\tiny\color{gray},
    keywordstyle=\color{blue},
    commentstyle=\color{dkgreen},
    stringstyle=\color{mauve},
    breaklines=true,
    breakatwhitespace=true,
    tabsize=4,
    literate={ä}{{\"a}}1 {Ä}{{\"A}}1 {ö}{{\"o}}1 {Ö}{{\"O}}1 {ü}{{\"u}}1 {Ü}{{\"U}}1 {ß}{{\ss}}1
}

\begin{document}
\noindent Gruppe \fbox{\textbf{14}}             \hfill Phil Pützstück, 377247\\
\noindent Datenstrukturen und Algorithmen \hfill Benedikt Gerlach, 376944\\
\strut\hfill Sebastian Hackenberg, 377550\\
\begin{center}
	\LARGE{\textbf{Hausaufgabe 6}}
\end{center}
\begin{center}
\rule[0.1ex]{\textwidth}{1pt}
\end{center}

\aufgabe{1}
Visuell dargestellt:
\texttt{[9,0,6,2,5,1,4,8]} $\xrightarrow{\texttt{mergesort}}$
\texttt{[9] [0] [6] [2] [5] [1] [4] [8]}

$\xrightarrow{\texttt{merge}}$ \texttt{[0,9] [2,6] [1,5] [4,8]}
$\xrightarrow{\texttt{merge}}$ \texttt{[0,2,6,9] [1,4,5,8]}
$\xrightarrow{\texttt{merge}}$ \texttt{[0,1,2,4,5,6,8,9]}

Nur das Array:\\
\texttt{[9,0,6,2,5,1,4,8]->[0,9,2,6,1,5,4,8]->[0,2,6,9,1,4,5,8]->[0,1,2,4,5,6,8,9]}

\aufgabe{2}
\textbf{a)}

\textbf{Buildheap}

\begin{minipage}{0.3\textwidth}
    \texttt{[9,2,9,6,5,1,0,7]}
    \begin{tikzpicture}[heap]
      \node {9}
        child{node{2}
            child{node{6}
                child{node{7}}}
            child{node{5}}}
        child{node{9}
            child{node{1}}
            child{node{0}}}
      ;
    \end{tikzpicture}
\end{minipage}
$\xrightarrow{\qquad\qquad\texttt{swap(6,7)}\qquad\qquad}$
\begin{minipage}{0.3\textwidth}
    \texttt{[9,2,9,7,5,1,0,6]}
    \begin{tikzpicture}[heap]
      \node {9}
        child{node{2}
            child{node{7}
                child{node{6}}}
            child{node{5}}}
        child{node{9}
            child{node{1}}
            child{node{0}}}
      ;
    \end{tikzpicture}
\end{minipage}

$\xrightarrow{\texttt{swap(2,7)}}$
\begin{minipage}{0.3\textwidth}
    \texttt{[9,7,9,2,5,1,0,6]}
    \begin{tikzpicture}[heap]
      \node {9}
        child{node{7}
            child{node{2}
                child{node{6}}}
            child{node{5}}}
        child{node{9}
            child{node{1}}
            child{node{0}}}
      ;
    \end{tikzpicture}
\end{minipage}
$\xrightarrow{\qquad\texttt{swap(2,6)}\qquad}$
\begin{minipage}{0.3\textwidth}
    \texttt{[9,7,9,6,5,1,0,2]}
    \begin{tikzpicture}[heap]
      \node {9}
        child{node{7}
            child{node{6}
                child{node{2}}}
            child{node{5}}}
        child{node{9}
            child{node{1}}
            child{node{0}}}
      ;
    \end{tikzpicture}
\end{minipage}
\newpage

\textbf{Heapsort}

\begin{minipage}{0.3\textwidth}
    \texttt{[9,7,9,6,5,1,0,2]}
    \begin{tikzpicture}[heap]
      \node {9}
        child{node{7}
            child{node{6}
                child{node{2}}}
            child{node{5}}}
        child{node{9}
            child{node{1}}
            child{node{0}}}
      ;
    \end{tikzpicture}
\end{minipage}
$\xrightarrow{\qquad\qquad\texttt{swap(9,2)}\qquad\qquad}$
\begin{minipage}{0.3\textwidth}
    \texttt{[2,7,9,6,5,1,0,9]}
    \begin{tikzpicture}[heap]
      \node {2}
        child{node{7}
            child{node{6}}
            child{node{5}}}
        child{node{9}
            child{node{1}}
            child{node{0}}}
      ;
    \end{tikzpicture}
\end{minipage}

$\xrightarrow{\texttt{swap(2,9)}}$
\begin{minipage}{0.3\textwidth}
    \texttt{[9,7,2,6,5,1,0,9]}
    \begin{tikzpicture}[heap]
      \node {9}
        child{node{7}
            child{node{6}}
            child{node{5}}}
        child{node{2}
            child{node{1}}
            child{node{0}}}
      ;
    \end{tikzpicture}
\end{minipage}
$\xrightarrow{\qquad\texttt{swap(9,0)}\qquad}$
\begin{minipage}{0.3\textwidth}
    \texttt{[0,7,2,6,5,1,9,9]}\\
    \begin{tikzpicture}[heap]
      \node {0}
        child{node{7}
            child{node{6}}
            child{node{5}}}
        child{node{2}
            child{node{1}}}
      ;
    \end{tikzpicture}
\end{minipage}

$\xrightarrow{\texttt{swap(0,7)}}$
\begin{minipage}{0.3\textwidth}
    \texttt{[7,0,2,6,5,1,9,9]}\\
    \begin{tikzpicture}[heap]
      \node {7}
        child{node{0}
            child{node{6}}
            child{node{5}}}
        child{node{2}
            child{node{1}}}
      ;
    \end{tikzpicture}
\end{minipage}
$\xrightarrow{\qquad\texttt{swap(0,6)}\qquad}$
\begin{minipage}{0.3\textwidth}
    \texttt{[7,6,2,0,5,1,9,9]}\\
    \begin{tikzpicture}[heap]
      \node {7}
        child{node{6}
            child{node{0}}
            child{node{5}}}
        child{node{2}
            child{node{1}}}
      ;
    \end{tikzpicture}
\end{minipage}

$\xrightarrow{\texttt{swap(7,1)}}$
\begin{minipage}{0.3\textwidth}
    \texttt{[1,6,2,0,5,7,9,9]}\\
    \begin{tikzpicture}[heap]
      \node {1}
        child{node{6}
            child{node{0}}
            child{node{5}}}
        child{node{2}}
      ;
    \end{tikzpicture}
\end{minipage}
$\xrightarrow{\qquad\texttt{swap(1,6)}\qquad}$
\begin{minipage}{0.3\textwidth}
    \texttt{[6,1,2,0,5,7,9,9]}\\
    \begin{tikzpicture}[heap]
      \node {6}
        child{node{1}
            child{node{0}}
            child{node{5}}}
        child{node{2}}
      ;
    \end{tikzpicture}
\end{minipage}

$\xrightarrow{\texttt{swap(1,5)}}$
\begin{minipage}{0.3\textwidth}
    \texttt{[6,5,2,0,1,7,9,9]}\\
    \begin{tikzpicture}[heap]
      \node {6}
        child{node{5}
            child{node{0}}
            child{node{1}}}
        child{node{2}}
      ;
    \end{tikzpicture}
\end{minipage}
$\xrightarrow{\qquad\texttt{swap(6,1)}\qquad}$
\begin{minipage}{0.3\textwidth}
    \texttt{[1,5,2,0,6,7,9,9]}\\
    \begin{tikzpicture}[heap]
      \node {1}
        child{node{5}
            child{node{0}}}
        child{node{2}}
      ;
    \end{tikzpicture}
\end{minipage}

\newpage

$\xrightarrow{\texttt{swap(1,5)}}$
\begin{minipage}{0.3\textwidth}
    \texttt{[5,1,2,0,6,7,9,9]}\\
    \begin{tikzpicture}[heap]
      \node {5}
        child{node{1}
            child{node{0}}}
        child{node{2}}
      ;
    \end{tikzpicture}
\end{minipage}
$\xrightarrow{\qquad\texttt{swap(5,0)}\qquad}$
\begin{minipage}{0.3\textwidth}
    \texttt{[0,1,2,5,6,7,9,9]}\\
    \begin{tikzpicture}[heap]
      \node {0}
        child{node{1}}
        child{node{2}}
      ;
    \end{tikzpicture}
\end{minipage}

$\xrightarrow{\texttt{swap(0,2)}}$
\begin{minipage}{0.3\textwidth}
    \texttt{[2,1,0,5,6,7,9,9]}\\
    \begin{tikzpicture}[heap]
      \node {2}
        child{node{1}}
        child{node{0}}
      ;
    \end{tikzpicture}
\end{minipage}
$\xrightarrow{\qquad\texttt{swap(2,0)}\qquad}$
\begin{minipage}{0.3\textwidth}
    \texttt{[0,1,2,5,6,7,9,9]}\\
    \begin{center}
        \begin{tikzpicture}[heap]
          \node {0}
            child{node{1}}
          ;
        \end{tikzpicture}
    \end{center}
\end{minipage}

$\xrightarrow{\texttt{swap(0,1)}}$
\begin{minipage}{0.3\textwidth}
    \texttt{[1,0,2,5,6,7,9,9]}\\
    \begin{center}
        \begin{tikzpicture}[heap]
          \node {1}
            child{node{0}}
          ;
        \end{tikzpicture}
    \end{center}
\end{minipage}
$\xrightarrow{\qquad\texttt{swap(1,0)}\qquad}$
\begin{minipage}{0.3\textwidth}
    \texttt{[0,1,2,5,6,7,9,9]}\\
    \begin{center}
        \begin{tikzpicture}[heap]
          \node {0}
          ;
        \end{tikzpicture}
    \end{center}
\end{minipage}

\textbf{b)}
Wir zählen analog zur Vorlesung die Anzahl der Vergleiche.

Der Best-case für \texttt{heapify} tritt ein wenn der Knoten schon die
Wurzel eines Heaps ist. In diesem Fall muss der Algorithmus nur 1 Vergleich pro Kind durchführen, da die Annahme ist,
dass alle Kinder ebenfalls schon Wurzeln von Heaps sind.
Sei $k$ die Anzahl der Kinder des Knoten, dann sind die Anzahl der benötigten Vergleiche $f_k(n) = k$, also
unabhängig von der Eingabelänge $n$. Da $k$ fest ist, haben wir $B(n) \in \Theta(f_k(n)) = \Theta(1)$.
Eingaben hierfür wären \texttt{heapify([0], 1, 0)}, \texttt{heapify([1,0], 2, 0)} und \texttt{heapify([2,1,0], 3, 0)}.

Der Best-case für \texttt{buildHeap} tritt ein, wenn das ganze Array schon ein Heap ist.
Hat der letzte innere Knoten nur ein Kind, können wir uns einen Vergleich sparen. In diesem Fall müssen wir also
$f(n) = |2\lfloor \frac{n}{2} \rfloor-1|$ Vergleiche durchführen. Denn ein Heap hat $\lfloor \frac{n}{2} \rfloor$ innere
Knoten, welche alle bis auf den letzten mit ihren beiden Kindern verglichen werden müssen.
Die Betragsstriche sind für den Fall $n=1$, in dem 0 Vergleiche benötigt werden.
Wir haben also $B(n) \in \Theta(f(n)) = \Theta(n)$.
Eine Eingabe hierfür wäre z.B. \texttt{buildHeap([5,4,3,2,1,0])}.

Der Best-case für \texttt{heapSort} ist etwas spezieller. Wenn das Eingabearray nur die gleichen Werte enthält, so
ist nach jeder \texttt{swap}-Operation mit der Wurzel und dem letzen Element der Heap immernoch ein Heap, da das letzte
Element dann eben größer-gleich allen anderen ist.
Dies spart viele Vergleichen mit \texttt{heapify}.
Wir setzen den Best-case also auf ein Eingabearray mit nur den gleichen Werten,
in dem der letzte innere Knoten wieder nur ein Kind hat.
Im \texttt{for}-Loop wird das letzte Element mit dem ersten geswapt, was in unserem Best-case keinen Unterschied macht.
Dann wird \texttt{heapify} aufgerufen und benötigt jedes mal $k$ Vergleiche wobei $k$ die Anzahl der Kinder der Wurzel ist.
Insgesamt haben wir \texttt{buildHeap} + $n$ mal (\texttt{for}-Loop) \texttt{heapify},
wobei \texttt{heapify} in den letzten beiden Durchläufen nur noch jeweil 1 und 0 Vergleiche
benötigt. Also $ f(n) = \left|2\left\lfloor \frac{n}{2} \right\rfloor -1\right| + 2n - 3$
und damit $B(n) \in \Theta(f(n)) = \Theta(n)$.
Eine Eingabe hierfür wäre z.B. \texttt{heapSort([0,0,0,0,0,0])}.



\aufgabe{3}
\texttt{[6,4,4,9,3,2,7,5]} $\xrightarrow{\texttt{pivot=5}}$
\texttt{[2,4,4,3,5,9,6,7]} $\xrightarrow{\texttt{pivot=3}}$
\texttt{[2,3,4,4,5,9,6,7]}

$\xrightarrow{\texttt{pivot=2}}$ \texttt{[2,3,4,4,5,9,6,7]}
$\xrightarrow{\texttt{pivot=4}}$ \texttt{[2,3,4,4,5,9,6,7]}
$\xrightarrow{\texttt{pivot=7}}$ \texttt{[2,3,4,4,5,6,7,9]}

$\xrightarrow{\texttt{pivot=6}}$ \texttt{[2,3,4,4,5,6,7,9]}
$\xrightarrow{\texttt{pivot=9}}$ \texttt{[2,3,4,4,5,6,7,9]}

\aufgabe{4}
\lstinputlisting{stableSort.java}
Die Idee ist, dass wir dem Schlüssel von jedem Element Information geben an welcher Stelle er im Array steht. Wir multiplizieren
also jeden Schlüssel mit der Gesamtlänge und addieren den Index des Elements im Array. So wird immernoch korrekt sortiert,
da für 2 Elemente \texttt{e1, e2} mit \texttt{e1.key < e2.key} stets auch \texttt{e1.key*n+i1 < e2.key*n+i2} gilt, da
\texttt{i1,i2} $\in [0, n-1]$.\\
Also selbst im Grenzfall \texttt{i1=n-1, i2=0} und \texttt{e2.key = e1.key + 1}, also minimalem \texttt{key}-Unterschied
und maximalem Index-Unterschied gilt immernoch
$$
    \texttt{e2.key}\cdot n + \texttt{i2} = \texttt{e1.key}\cdot n + n > \texttt{e1.key}\cdot n + n-1 = \texttt{e1.key}\cdot n +\texttt{i1}
$$
Weiter haben wir nun noch Stabilität: Für Elemente \texttt{e1,e2} mit \texttt{e1.key = e2.key}, welche
jeweils den Index \texttt{i1, i2} haben nehmen wir o.B.d.A. an, dass \texttt{i1 < i2}, also \texttt{e1} im Array
vor \texttt{e2} steht. Dann ist
$$
    \texttt{e1.key}\cdot n + \texttt{i1} < \texttt{e1.key}\cdot n + \texttt{i2} = \texttt{e2.key}\cdot n + \texttt{i2}
$$
Und \texttt{e1, e2} werden von \texttt{sort} stabil sortiert. Wegen dem Hinweis, dass wir annehmen können, dass
\texttt{int} beliebig große Zahlen speichern kann, können wir dieses Verfahren oft genug wiederholen ohne ein Problem
mit zu großen Schlüsseln zu bekommen.

Ferner gilt die Annahme, dass die Addition und Multiplikation in der \texttt{for}-Schleife konstante Laufzeit benötigen,
also die Schleife insgesamt $f(n) = 2n$ Operationen ausführt. Da die Schleife und der Sortieralgorithmus unabhängig voneinander
(nacheinander) ausgeführt werden, gilt damit
$$
    W_{\texttt{stableSort}}(n) = W_{\texttt{sort}}(n) + f(n)
    \qquad\text{sowie}\qquad
    f(n) = 2n \in \Theta(n)
$$

\aufgabe{5}
\textbf{a)}
Insertionsort??

\newpage
\textbf{b)}

Hier empfiehlt es sich, Counting-Sort zu verwenden. Der Algorithmus hat eben die gegebene Vorraussetzung, dass
nur Schlüssel zwischen 0 und einem festen $k$ existieren (hier $k=4$). Dafür kann er unter dieser Vorraussetzung
die Eingabe der Länge $n$ in $\mathcal{O}(n+k)$ sortieren. Best-, Worst-, und Average-case machen hierbei keinen Unterschied.
Dies ist deutlich schneller als alle anderen Sortierverfahren die zur Auswahl stehen.

\textbf{c)}

Für fast sortierte arrays ist Insertion-sort eine gute Wahl. Im Best-case (einem komplett sortierten Array)
ist die Laufzeit linear. Dies ist ähnlich bei Bubblesort. Jedoch ist Insertionsort nochmal besser da
in diesem Szenario alle Elemente sehr wahrscheinlich nur ein paar Plätze (oder auch keine) verschoben werden
müssen. Insertionsort muss dann nur diese paar Plätze absuchen um die richtige Stelle zu finden, während Bubblesort
das ganze Array durchläuft.

\textbf{d)}

Es macht Sinn, diese Aufgabe mit Mergesort anzugehen, da die Assistenten nach den ersten 3 Splits die 8 fast gleichgroßen
Stapel unter sich aufteilen und sofort Mergesort anwenden können. Dadurch können die 8 Assistenten parallel arbeiten,
was bei anderen Sortierverfahren nicht immer möglich ist. Sobald jeder der 8 fertig ist, mergen sie ihre sortierten Klausuren
einfach.
\end{document}
