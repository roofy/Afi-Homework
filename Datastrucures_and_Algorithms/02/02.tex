\documentclass[a4paper,graphics,11pt]{article}
\pagenumbering{arabic}

\usepackage[margin=1in]{geometry}
\usepackage[utf8]{inputenc}
\usepackage[T1]{fontenc}
\usepackage{lmodern}
\usepackage[ngerman]{babel}
\usepackage{amsmath, tabu}
\usepackage{amsthm}
\usepackage{amssymb}
\usepackage{complexity}
\usepackage{mathtools}
\usepackage{setspace}
\usepackage{graphicx,color,curves,epsf,float,rotating}
\usepackage{tasks}
\setlength{\parindent}{0em}
\setlength{\parskip}{1em}

\newcommand{\aufgabe}[1]{\subsection*{Aufgabe #1}}
\newcommand{\up}[2]{\mathrel{\overset{\makebox[0pt]{\mbox{\normalfont\tiny #2}}}{#1}}}

\usepackage{listings}
\usepackage{color}

\definecolor{dkgreen}{rgb}{0,0.6,0}
\definecolor{gray}{rgb}{0.5,0.5,0.5}
\definecolor{mauve}{rgb}{0.58,0,0.82}

\lstset{frame=tb,
    language=Java,
    aboveskip=2mm,
    belowskip=2mm,
    showstringspaces=false,
    columns=flexible,
    basicstyle={\small\ttfamily},
    numbers=left,
    numberstyle=\tiny\color{gray},
    keywordstyle=\color{blue},
    commentstyle=\color{dkgreen},
    stringstyle=\color{mauve},
    breaklines=true,
    breakatwhitespace=true,
    tabsize=4,
    literate={ä}{{\"a}}1 {Ä}{{\"A}}1 {ö}{{\"o}}1 {Ö}{{\"O}}1 {ü}{{\"u}}1 {Ü}{{\"U}}1 {ß}{{\ss}}1
}

\begin{document}
\noindent Gruppe \fbox{\textbf{14}}             \hfill Phil Pützstück, 377247\\
\noindent Datenstrukturen und Algorithmen \hfill Benedikt Gerlach, 376944\\
\strut\hfill Sebastian Hackenberg, 377550\\
\begin{center}
	\LARGE{\textbf{Hausaufgabe 2}}
\end{center}
\begin{center}
\rule[0.1ex]{\textwidth}{1pt}
\end{center}

\aufgabe{1}
\textbf{a)}

Es sei $f$ eine gegebene Funktion. Da $\forall n \in \mathbb{N}_0: 1\cdot f(n) = f(n)$
existiert eine konstante $0 < c = 1 < \infty$ und ein $n_0 = 0$ sodass gilt:
$$
	\forall n \in \mathbb{N}_0, n \geq n_0 : f(n) \leq c \cdot f(n)
$$
Damit haben wir nach Definition $f \in \mathcal{O}(f) \,\Longleftrightarrow\, f \sqsubseteq f$.
Also ist $\sqsubseteq$ reflexiv.

Ferner seien nun weiter Funktionen $g,h$ gegeben. Es gilt nach Definition:
$$
	(f \sqsubseteq g) \land (g \sqsubseteq h)
	\,\Longrightarrow\, (f \in \mathcal{O}(g)) \land (g \in \mathcal{O}(h))
$$$$
	\,\Longrightarrow\, (\exists c,n_0 : \forall n \geq n_0 : f(n) \leq c \cdot g(n)) \land 
		(\exists c',n_0' : \forall n \geq n_0 : g(n) \leq c \cdot h(n))
$$
Sei nun $n' := \max \{n_0, n_0'\}$. Dann folgt weiter:
$$
	\exists c,c' : \forall n \geq n' : f(n) \leq c \cdot g(n)\quad \land\quad g(n) \leq c' \cdot h(n)
$$$$
	\,\Longrightarrow\, \exists c,c' : \forall n \geq n' : f(n) \leq c\cdot g(n) \leq c (c'\cdot h(n))
$$$$
	\,\Longrightarrow\, \exists c,c' : \forall n \geq n' : f(n) \leq (c\cdot c')\cdot h(n))
$$$$
	\,\Longrightarrow\, f \in \mathcal{O}(h) \,\Longleftrightarrow\, f \sqsubseteq h
$$

Wir haben also gezeigt, dass für beliebige Funktionen $f,g,h$ gilt:
$$
	f \sqsubseteq g \land g \sqsubseteq h \,\Longrightarrow\, f \sqsubseteq h
$$
Damit ist $\sqsubseteq$ reflexiv und transitiv, also eine Quasiordnung $\hfill\square$

\textbf{b)}

Es gilt:
$$
	0 \sqsubseteq 4 \sqsubseteq 2^{9000} \sqsubseteq \log(n) \sqsubseteq n\cdot\log(n) \sqsubseteq n\cdot \sqrt{n} \sqsubseteq n^2 \sqsubseteq \sum_{i=0}^{n} \frac{14i^2}{1+i} \sqsubseteq n^2\cdot\log(n) \sqsubseteq \frac{n^3}{2} \sqsubseteq n^3 \sqsubseteq 2^n \sqsubseteq n! \sqsubseteq n^n
$$
\newpage

\aufgabe{2} 
\textbf{a)}

Nach Definition der Klasse $\mathcal{O}$ gilt g $\in \mathcal{O}(f)$ falls c $\cdot$ f(n) ab einer bestimmten Konstanten $n_0 \in \mathbb{N}$ eine obere Schranke von g(n) ist. Wir beweisen durch Induktion, dass die Aussage:
$$
	\frac{1}{4}n^3-7n+17 < 1\cdot(n^3)
$$
für $n_0 \ge 2$ gilt.\\



\begin{minipage}{1\linewidth}
	\textbf{Induktionsanfang}\\
	$$
		A(n) := \frac{1}{4}n^3-7n+17 < n^3  
	$$$$
		A(2) = \frac{1}{4} 2^3-7\cdot2+17 = 5 < 8 = 2^3
	$$
	Damit gilt die Annahme für n = 2.\\
	
	\textbf{Induktionsvoraussetzung}\\
	
	Es gelte $A(n)$ für ein beliebig aber festes $n \in \mathbb{N}$.\\
	
	\textbf{Induktionsschritt} $n \to n+1$
	
	$$
	\frac{1}{4}(n+1)^3-7(n+1)+17 = \frac{1}{4}(n^3+3n^2+3n+1)-7n+7+17
	$$$$
	=\frac{1}{4}n^3-7n+17+\frac{1}{4}(3n^2+3n+1)+7\ \up{<}{(IV)}\ n^3+\frac{1}{4}(3n^2+3n+1)+7
	$$$$
	< n^3+\frac{1}{4}(3n^2+3n+1)+7 \overset{\text{Lemma}}{<} n^3 +3n^2+1 = (n+1)^3
	$$
	$\hfill\square$\\
	
\end{minipage}


	\textbf{Lemma}
	
	Um die Ungleichung zu Beweisen benutzen wir vollständige Induktion.\\
	
	\textbf{Umformung}
	$$
	\frac{1}{4}(3n^2+1)+7 < 3n^2+3n+1 
	$$$$
	\Leftrightarrow 3n^2 + 3n < 12n^2+12n-24
	$$
	\textbf{Induktionsanfang}
	$$
	A(n) := 3n^2+3n < 12n^2+12n-24 , n \in \mathbb{N}, n\geq 2
	$$$$
	A(2) = 3 \cdot 2^2+3\cdot 2 = 30 < 48 = 12 \cdot 2^2+12 \cdot 2-24
	$$
	
	\newpage
	
	\textbf{Induktionsvoraussetzung}\\
	
	Es gelte $A(n)$ für ein beliebig aber festes $n \in \mathbb{N}$.\\
	
	\textbf{Induktionsschritt} $n \to n+1$
	$$
	A(n+1) = 3(n+1)^2 + 3 \cdot (n+1) = 3(n^2 + 2n + 1) + 3n + 3
	$$$$
	= 3n^2 + 3n + 6n +6 < 12n^2 + 12n - 24 + 6n + 6 \overset{\text{n > 0}}{<} 12n^2 + 12n - 24 \textbf{+} 24n + 12n
	$$$$
	=12n^2 + 24n + 12 +12n + 12 - 24 = 12 \cdot (n^2 + 2n + 1) + 12n +12 -24 
	$$$$
	= 12 \cdot (n+1)^2 + 12 \cdot (n+1) -24
	$$
	
	
	$\hfill\square$


\textbf{b)}

Da $3n^2+4 > 0$ und $ n^4 < 2n^4 $ für alle $n \geq 0$ gilt, folgt
$n^4 < 1 \cdot(2n^4+3n^2+42)$ für $n \in \mathbb{N}$. Damit existiert ein $c > 0$ sodass $g(n) < c \cdot f(n)$. Also ist $n^4 \in \mathcal{O}(2n^4+3n^2+42)$.

$\hfill\square$



\textbf{c)}

Die Aussage gilt nach der Alternativen Definition, da:
$$
\lim_{n\rightarrow\infty} \frac{\log2(n)}{n} \overset{\text{L'Hospital}}{=} \lim_{n\rightarrow\infty} \frac{(\log2(n))'}{(n)'} = \lim_{n\rightarrow\infty} \frac{\frac{1}{n\cdot \ln(2)}}{1} = 0
$$

$\hfill\square$



\textbf{d)}

Nach dem Lemma der Klasse Groß-O muss es ein $c \geq 0$ mit $c \neq \infty$ geben sodass

$$
\lim_{n\rightarrow\infty} \frac{\log n}{n^\varepsilon}, \forall \varepsilon > 0
$$

Es gilt $\lim_{n\rightarrow\infty} \log2 (n) = \infty$ und $\forall\varepsilon > 0 :
\lim_{n\rightarrow\infty}  n^\varepsilon = \infty$.

Somit ist nach L'Hospital 
$$
\lim_{n\rightarrow\infty} \frac{(\log2(n))'}{(n^\varepsilon)'} = 0
$$

Damit ist die Aussage für alle $\varepsilon>0$ wahr.

$\hfill\square$

\newpage

\textbf{e)}


Nach dem Lemma der Klasse Groß-Theta muss es ein $ 0 < c< \infty$ geben sodass

$$
\lim_{n\rightarrow\infty} \frac{a^n}{b^n} = c
$$

für die zwei beliebigen Konstanten $a,b > 1$ die Aufgabenstellung gilt.\\
Beweis durch Gegenbeispiel:\\
$$
\lim_{n\rightarrow\infty} \frac{2^n}{4^n} = \lim_{n\rightarrow\infty} \frac{1}{2n} = 0
$$

Somit ist die Aussage falsch, da $c=0$.\\

$\hfill\square$


\aufgabe{3}
\textbf{a)}

Es sei eine Funktion $g(n)$ gegeben.
Wir notieren im Sinne der Lesbarkeit
$$
	o_1 := o(g(n)) \qquad \mathcal{O}_1 := \mathcal{O}(g(n)) \qquad \Theta_1 := \Theta(g(n))
$$
Behauptung: Es gilt $o_1 = \mathcal{O}_1 \setminus \Theta_1\hfill(1)$\\
Beweis:
$$
	o_1 := \{f \mid \forall c > 0 : \exists n_0 : \forall n\geq n_0 : 0 \leq f(n) < c\cdot g(n)\}
$$$$
	= \{f \mid \forall c > 0 : \exists n_0 : \forall n\geq n_0 : 0 \leq f(n) < cg(n)\quad
$$$$
	\qquad\qquad\qquad\qquad\land \quad \lnot(\exists c_1,c_2 > 0,n_0 : \forall n\geq n_0 : c_1f(n) \leq g(n) \leq c_2f(n))\}
$$$$
	= \mathcal{O}_1 \setminus \Theta_1
$$
$\hfill\square$\\
Nun folgt direkt:
$$
	o_1 \cap \Theta_1
	\up{=}{(1)} (\mathcal{O}_1 \setminus \Theta_1) \cap \Theta_1 = \o
$$
$\hfill\square$

\textbf{b)}
Wir geben ein Gegenbeispiel.
Es seien
$$
	g(n) = 0 \qquad f(n) = n \qquad h(n) = n^2
$$
Damit gilt $f(n) \in \Omega(g(n))$, da für $c = 1 > 0$ und $n_0 = 1$ stets $c\cdot g(n) = 0 \leq f(n)$ gilt.\\
Weiter ist auch $f(n) \in \mathcal{O}(h(n))$, da für $c = 1$ und $n_0 = 1$ stets $f(n) \leq c\cdot h(n)$ gilt.

Wir nehmen nun an, es gelte $g(n) \in \Theta(h(n))$. Also auch $g(n) \in \Omega(h(n))$. Daraus folgt, dass eine Konstante
$c > 0$ existiert, sodass ab einem beliebigem Punkt stets $c \cdot h(n) = cn^2 \leq 0 = g(n)$ gilt. Dies kann offensichtlich
nur für $c = 0$ gelten, jedoch muss $c > 0$ erfüllt sein. Damit haben wir einen Widerspruch. Also kann die Aussage nicht stimmen.
$\hfill\square$

\newpage
\aufgabe{4}
\textbf{a)}

Die Best-Case Laufzeit des Algorithmus beträgt B(n)=1 für den Fall, dass die Länge das Arrays n = 0 beträgt. Im Fall $n > 0$ beträgt die Best-Case Laufzeit des Algorithmus B(n)=2n+2$\in \mathcal{O}(n)$.

\textbf{b)}

Die Worst-Case Laufzeit des Algorithmus beträgt W(n)=3n+1$\in \mathcal{O}(n)$.

\textbf{c)}

Die Average-Case Laufzeit des Algorithmus beträgt:
$$
\sum_{i=0}^{n} \frac{1}{n+1} \cdot (2 \cdot (n-i)+3i+1)
$$

\textbf{d)}\\

\lstinputlisting{02Nr4d.java}

Der Algorithmus besitzt eine bessere Average-Case Laufzeit, da er bei einem False in dem Array sofort abbricht und nicht noch das ganze restliche Array durchläuft.

\textbf{e)}\\

Es gibt keinen äquivalenten Algorithmus dessen Worst-Case Laufzeit in o(n) liegt, da er das gesamte Array durchgehen muss um zu bestätigen das kein False vorkommt. 

\end{document}
