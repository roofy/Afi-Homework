\documentclass[a4paper,graphics,11pt]{article}
\pagenumbering{arabic}

\usepackage[margin=1in]{geometry}
\usepackage[utf8]{inputenc}
\usepackage[T1]{fontenc}
\usepackage{lmodern}
\usepackage[ngerman]{babel}
\usepackage{amsmath, tabu}
\usepackage{amsthm}
\usepackage{amssymb}
\usepackage{complexity}
\usepackage{mathtools}
\usepackage{setspace}
\usepackage{graphicx,color,curves,epsf,float,rotating}
\usepackage{tasks}
\setlength{\parindent}{0em}
\setlength{\parskip}{1em}

\newcommand{\aufgabe}[1]{\subsection*{Aufgabe #1}}
\newcommand{\up}[2]{\mathrel{\overset{\makebox[0pt]{\mbox{\normalfont\tiny #2}}}{#1}}}

\usepackage{listings}
\usepackage{color}

\definecolor{dkgreen}{rgb}{0,0.6,0}
\definecolor{gray}{rgb}{0.5,0.5,0.5}
\definecolor{mauve}{rgb}{0.58,0,0.82}

\lstset{frame=tb,
    language=Java,
    aboveskip=2mm,
    belowskip=2mm,
    showstringspaces=false,
    columns=flexible,
    basicstyle={\small\ttfamily},
    numbers=left,
    numberstyle=\tiny\color{gray},
    keywordstyle=\color{blue},
    commentstyle=\color{dkgreen},
    stringstyle=\color{mauve},
    breaklines=true,
    breakatwhitespace=true,
    tabsize=4,
    literate={ä}{{\"a}}1 {Ä}{{\"A}}1 {ö}{{\"o}}1 {Ö}{{\"O}}1 {ü}{{\"u}}1 {Ü}{{\"U}}1 {ß}{{\ss}}1
}

\begin{document}
\noindent Gruppe \fbox{\textbf{14}}             \hfill Phil Pützstück, 377247\\
\noindent Datenstrukturen und Algorithmen\\
\begin{center}
	\LARGE{\textbf{Hausaufgabe 1}}
\end{center}
\begin{center}
\rule[0.1ex]{\textwidth}{1pt}
\end{center}

\aufgabe{2}
\textbf{a)}\\
Wir zeigen $A(n) = \left(\sum_{k=1}^{n} k = \frac{n(n+1)}{2}\right), n \in \mathbb{N}^{>0}$.
Es sei $n = 1$. Es gilt:
$$
	\sum_{k=1}^{n} k
	= \sum_{k=1}^{1} k
	= 1
	= \frac{2}{2}
	= \frac{1\cdot(1+1)}{2}
	= \frac{n(n+1)}{2}
$$
Damit hält die Aussage für $n = 1 \in \mathbb{N}^{>0}$.
Es sei nun ein $n \in \mathbb{N}^{>0}$ mit $A(n)$ gegeben (IV). \\
$n \to n+1$:
$$
	\sum_{k=1}^{n+1} k = (n+1) + \sum_{k=1}^{n} k
	\up{=}{IV} (n+1) + \frac{n(n+1)}{2}
	= \frac{n^2+3n+2}{2}
	= \frac{(n+1)(n+2)}{2}
$$
Also folgt damit auch $A(n+1)$. \\
Nach dem Prinzip der vollständigen Induktion gilt nun $\forall n \in \mathbb{N}^{>0}: A(n)\hfill\square$
\\

\textbf{b)}\\
Wir zeigen $A(n) = \left(\sum_{i=0}^{n} 2^i = 2^{n+1} - 1\right), n \in \mathbb{N}^{>0}$.
Es sei $n = 1$. Es gilt:
$$
	\sum_{i=0}^{n} 2^i = \sum_{i=0}^{1} 2^i = 2^0 + 2^1 = 3 = 2^2 -1 = 2^{n+1} - 1
$$
Damit hält die Aussage für $n = 1 \in \mathbb{N}^{>0}$.
Es sei nun ein $n \in \mathbb{N}^{>0}$ mit $A(n)$ gegeben (IV). \\
$n \to n+1$:
$$
	\sum_{i=0}^{n+1} 2^i = 2^{n+1} + \sum_{i=0}^{n} 2^i
	\up{=}{IV} 2^{n+1} + 2^{n+1} - 1
	= 2^{n+2} - 1
	= 2^{(n+1)+1} - 1
$$
Also folgt damit auch $A(n+1)$. \\
Nach dem Prinzip der vollständigen Induktion gilt nun $\forall n \in \mathbb{N}^{>0}: A(n)\hfill\square$
\newpage

\aufgabe{3}

\textbf{a)}
Es sei $X = \{1,2\}, r = \{(1,2)\}$. Es gilt:
$$
	\forall x \in X : (x,x) \notin r
	\quad \text{sowie}\quad
	\forall x,y,z \in X : \left((x,y) \in r \land (y, z) \in r\right) \,\Longrightarrow\, (x,z) \in r
$$
Damit ist $r$ eine irreflexive, transitive Relation auf $X$.
Offensichtlich ist $r \neq \o$, und damit nicht die leere Relation.

Ferner gilt folgendes nicht:
$$
	\forall x,y \in X : (x,y) \in r \,\Longrightarrow\, (y,x) \in r
$$
Denn wir haben für $1, 2 \in X$ und $(1,2) \in r$ jedoch $(2,1) \notin r $. Daher ist $r$ nicht symmetrisch.

Damit gibt es eine irreflexive, transitive Relation auf einer Menge $X$, sodass die ersten beiden Behauptungen
nicht stets gelten.

Wir zeigen die letzte Behauptung mit einem Beweis durch Widerspruch: \\

Wir nehmen an, es gäbe eine irreflexive, transitive Relation $r$ einer beliebigen Menge $X$ welche nicht antisymmetrisch ist. Dann folgt
$$
	\exists x,y \in X : (x,y) \in r \land (y,x) \in r
	\up{\,\Longrightarrow\,}{trans.} (x,x) \in r \,\Longrightarrow\, r\ \text{nicht irreflexiv}
$$
Also haben wir einen Widerspruch.\\
Daher muss jede irreflexive, transitive Relation $r$ auch antisymmetrisch sein.
\hfill$\square$

\newpage
\aufgabe{4}

\lstinputlisting{SecondLargest.java}
\end{document}
