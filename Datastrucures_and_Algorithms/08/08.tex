\documentclass[a4paper,graphics,11pt]{article}
\pagenumbering{arabic}

\usepackage[margin=1in]{geometry}
\usepackage[utf8]{inputenc}
\usepackage[T1]{fontenc}
\usepackage{lmodern}
\usepackage[ngerman]{babel}
\usepackage{amsmath, tabu}
\usepackage{amsthm}
\usepackage{amssymb}
\usepackage{complexity}
\usepackage{mathtools}
\usepackage{setspace}
\usepackage{graphicx,color,curves,epsf,float,rotating}
\usepackage{tasks}
\setlength{\parindent}{0em}
\setlength{\parskip}{1em}

\newcommand{\aufgabe}[1]{\subsection*{Aufgabe #1}}
\newcommand{\up}[2]{\mathrel{\overset{\makebox[0pt]{\mbox{\normalfont\tiny #2}}}{#1}}}

\usepackage{listings}
\usepackage{color}

\definecolor{dkgreen}{rgb}{0,0.6,0}
\definecolor{gray}{rgb}{0.5,0.5,0.5}
\definecolor{mauve}{rgb}{0.58,0,0.82}

\lstset{frame=tb,
    language=Java,
    aboveskip=2mm,
    belowskip=2mm,
    showstringspaces=false,
    columns=flexible,
    basicstyle={\small\ttfamily},
    numbers=left,
    numberstyle=\tiny\color{gray},
    keywordstyle=\color{blue},
    commentstyle=\color{dkgreen},
    stringstyle=\color{mauve},
    breaklines=true,
    breakatwhitespace=true,
    tabsize=4,
    literate={ä}{{\"a}}1 {Ä}{{\"A}}1 {ö}{{\"o}}1 {Ö}{{\"O}}1 {ü}{{\"u}}1 {Ü}{{\"U}}1 {ß}{{\ss}}1
}

\begin{document}
\noindent Gruppe \fbox{\textbf{14}}             \hfill Phil Pützstück, 377247\\
\noindent Datenstrukturen und Algorithmen \hfill Benedikt Gerlach, 376944\\
\strut\hfill Sebastian Hackenberg, 377550\\
\begin{center}
	\LARGE{\textbf{Hausaufgabe 8}}
\end{center}
\begin{center}
\rule[0.1ex]{\textwidth}{1pt}
\end{center}

\aufgabe{1}
Eingabearray \texttt{[3,6,2,4,5,5,0,2]}.\\
Histogramm \texttt{[1,0,2,1,1,2,1]}.\\
Positionsarray \texttt{[1,1,3,4,5,7,8]}.\\
Ergebnisarray:\\
\texttt{[0,0,0,0,0,0,0,0] -> [0,0,2,0,0,0,0,0] -> [0,0,2,0,0,0,0,0] -> [0,0,2,0,0,0,5,0]}\\
\texttt{[0,0,2,0,0,5,5,0] -> [0,0,2,0,4,5,5,0] -> [0,2,2,0,4,5,5,0] -> [0,2,2,0,4,5,5,6]}\\
\texttt{[0,2,2,3,4,5,5,6]}

\aufgabe{2}
\aufgabe{3}
Die Funktion $f : [1,100] \to [1,10], x \mapsto \lfloor \frac{10}{x} \rfloor$ ist grundsätzlich einfach zu berechnen,
wir nehmen schließlich oft an, dass Operationen wie $\div$ konstante Zeit benötigen. Weiter sind keine zu großen Zahlen im Spiel.
$f$ ist jedoch nicht surjektiv, da $f$ monoton fallend ist und $f(1) = 10, f(2) = 5$ gilt, also z.B. $9,8,7,6$ nicht getroffen
werden. Weiter ist $f$ bei weitem nicht gleichverteilt, wir haben $\forall x \in [11, 100] : f(x) = 0$, also ca 90\% der
möglichen Eingaben werden auf 0 abegebildet. Letztlich werden auch ähnliche Schlüssel auf ähnliche Bereiche verteil, insgesamt
ist die Funktion nicht sehr gut geeignet.

Die Funktion $g:\mathbb{N} \to \mathbb{Z}/101\mathbb{Z}, x \mapsto 2^x\ \text{mod}\ 101$ ist nicht so einfach zu berechnen wie
die anderen der Liste, jedoch ist eine Zweierpotenz letzendlich nur ein Left-shift, also auch nicht sehr kostspielig.
$g$ ist nicht surjektiv, da es kein $x \in \mathbb{N}$ mit $2^x\ \text{mod}\ 101 = 0 \in \mathbb{Z}/101\mathbb{Z}$ gibt.
Da wir $2^{100}\ \text{mod}\ 101 = 1$ haben,
gilt stets $2^{x+100y} \equiv 2^x\ \text{mod}\ 101$ für $x \in [0,99], z \in \mathbb{N}$. Damit haben wir (bis auf 0) eine
perfekte Gleichverteilung über $\mathbb{N}$. Hinzukommend werden aufeinanderfolgende Werte breit verteilt, beispielsweise
gilt $g(22) = 77, g(23) = 53$. Insgesamt ist $g$ eine geeignete Hashfunktion.

Die Funktion $h:[0, 100] \to [0,10], x \mapsto x\ \text{mod}\ 11$ ist sehr einfach zu berechnen. Ferner ist es surjektiv, da
z.b. $h_{[0,10]} = \text{id}_{[0,10]}$. Offensichtlich haben wir für $x \in [2,10]$ stets 9 werte in $[0,100]$ welche
unter $h$ auf $x$ abgebildet werden, für $x = 0$ oder $x=1$ gibt es 10. Das ist eine ausgeglichene Gleichverteilung über
$[0,100]$. Jedoch werden Schlüssel nicht sehr breit verteilt. Ist $x \in [0,100]$ Vielfaches von 10, so wird der Nachfolger
von $x$ auch auf den Nachfolger des Bildes von $x$ unter $h$ abgebildet. Grundsätzlich ist diese letzte Eigenschaft eine
der wichtigsten, weswegen Ich diese Funktion trotz ihrer restlichen Qualitäten nicht als gute Hashfunktion betiteln würde.

\newpage

Die Funktion $i: \mathbb{N} \to [0,50], x \mapsto \lfloor \frac{x}{2} \rfloor\ \text{mod}\ 51$ ist (analog zu $f$) wahrscheinlich
in konstanter Zeit berechnbar. Weiter ist sie surjektiv, denn
es ist $i(\{2x \mid x \in [0,50]\}) = [0,50]$. Ferner haben wir eine ausgeglichene Gleichverteilung über ganz $\mathbb{N}$,
da für $x \in [0,50]$ immer\\ $\{y \in \mathbb{N}| y=2(x+51z) \lor y=2(x+51z)+1, z \in \mathbb{N}\}$ das Urbild
von $x$ darstellt. Jedoch ist wieder analog zu $h$ eine eher schlechte Verteilung der Werte gegeben. Die Folge der Werte
über $\mathbb{N}$ folgt dem Schema $0,0,1,1,2,2,\cdots,50,50,0,0,1,1,\cdots$. Aus gleichem Grund wie für $h$ würde ich
also $i$ nicht als gute Hashfunktion bezeichnen.

\aufgabe{4}
$c = 0.01, m = 11$.\\
\texttt{[5, , , , , , , , , , ] -> [5, ,21, , , , , , , , ] -> [5, ,21,23, , , , , , , ]}
\texttt{[5,17,21,23, , , , , , , ] -> [5,17,21,23,11, , , , , , ]}\\
\texttt{[5,17,21,23,11,7, , , , , ] -> [5,17,21,23,11,7,1, , , , ]}

\aufgabe{5}
$c = 0.01, c_1 = 2, c_2 =1, m = 11$.\\
\texttt{[5, , , , , , , , , , ] -> [5, ,21, , , , , , , , ] -> [5, ,21, , ,23, , , , , ]}
\texttt{[5,17,21, , ,23, , , , ] -> [5,17,21, ,11,23, , , , , ]}\\
\texttt{[5,17,21,7,11,23, , , , , ] -> [5,17,21,7,11,23, , ,1, , ]}

\aufgabe{6}

\lstinputlisting{nr6.java}
\end{document}
