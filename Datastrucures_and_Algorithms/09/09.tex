\documentclass[a4paper,graphics,11pt]{article}
\pagenumbering{arabic}

\usepackage[margin=1in]{geometry}
\usepackage[utf8]{inputenc}
\usepackage[T1]{fontenc}
\usepackage{lmodern}
\usepackage[ngerman]{babel}
\usepackage{amsmath, tabu}
\usepackage{amsthm}
\usepackage{amssymb}
\usepackage{complexity}
\usepackage{mathtools}
\usepackage{setspace}
\usepackage{graphicx,color,curves,epsf,float,rotating}
\usepackage{tasks}
\setlength{\parindent}{0em}
\setlength{\parskip}{1em}

\newcommand{\aufgabe}[1]{\subsection*{Aufgabe #1}}
\newcommand{\up}[2]{\mathrel{\overset{\makebox[0pt]{\mbox{\normalfont\tiny #2}}}{#1}}}

\usepackage{listings}
\usepackage{color}

\usepackage{tikz}
\usetikzlibrary{arrows}

\definecolor{dkgreen}{rgb}{0,0.6,0}
\definecolor{gray}{rgb}{0.5,0.5,0.5}
\definecolor{mauve}{rgb}{0.58,0,0.82}

\lstset{frame=tb,
    language=Java,
    aboveskip=2mm,
    belowskip=2mm,
    showstringspaces=false,
    columns=flexible,
    basicstyle={\small\ttfamily},
    numbers=left,
    numberstyle=\tiny\color{gray},
    keywordstyle=\color{blue},
    commentstyle=\color{dkgreen},
    stringstyle=\color{mauve},
    breaklines=true,
    breakatwhitespace=true,
    tabsize=4,
    literate={ä}{{\"a}}1 {Ä}{{\"A}}1 {ö}{{\"o}}1 {Ö}{{\"O}}1 {ü}{{\"u}}1 {Ü}{{\"U}}1 {ß}{{\ss}}1
}

\begin{document}
\noindent Gruppe \fbox{\textbf{14}}             \hfill Phil Pützstück, 377247\\
\noindent Datenstrukturen und Algorithmen \hfill Benedikt Gerlach, 376944\\
\strut\hfill Sebastian Hackenberg, 377550\\
\begin{center}
	\LARGE{\textbf{Hausaufgabe 9}}
\end{center}
\begin{center}
\rule[0.1ex]{\textwidth}{1pt}
\end{center}

\aufgabe{1}
\textbf{a)}
Es gibt genau $4^{\binom{n}{2}}$ gerichtete Graphen mit $n$ Knoten. Für jedes Paar von den $\binom{n}{2}$ Paaren
von Knoten haben wir genau 4 Möglichkeiten: Keine Kante, Kante von A nach B, Kante von B nach A, Zwei Kanten.

\textbf{b)}
Es gibt genau $2^{\binom{n}{2}}$ ungerichtete Graphen mit $n$ Knoten. Die Idee ist wie in a). Für jedes von
den $\binom{n}{2}$ möglichen Paaren von Knoten haben wir entweder 1 Kante oder keine, also 2 Möglichkeiten
pro Paar.

\textbf{c)}
Es gibt genau $\frac{n!}{(k-1)!} = n \cdot (n-1) \cdot (n-2) \cdots (n-k)$ mögliche einfache Pfade der Länge
$k$ in einem ungerichtetem vollständigem Graphen mit $n$ Knoten. Dies liegt daran, dass wir $n$ Möglichkeiten
haben eine Startposition auszusuchen. Danach haben wir $n-1$ Möglichkeiten für den nächsten zu besuchenden
Knoten, danach $n-2$ etc. weil der Graph vollständig ist und wir nur einfache Pfade betrachten. Wenn wir
dann beim Faktor $(n-k)$ angekommen sind haben wir einen Pfad der Länge $k$.

\textbf{d)}
Da Zykel in einem ungerichtetem Graph mindestens Länge 3 haben müssen, folgt sofort, dass für Graphen mit $n < 3$
Knoten genau 0 der gesuchten Zykel existieren. Sei nun also $n \geq 3$. Dann haben wir
genau $n(n-1)(n-2)$ Möglichkeiten für einen Zykel der Länge 3 in einem vollständigem ungerichtetem Graphen
mit $n$ Knoten. Dies ist analog zu c), da wir $n$ mögliche Startknoten haben, und der letzte Knoten wieder
der Startknoten, die restlichen aber unterschiedlich sein müssen. Analog haben wir $n(n-1)(n-2)(n-3)$
Zykel der Länge 4, da wir beim vorletzten Knoten noch nicht die Möglichkeit haben zum Startknoten wiederzukehren
ohne beim letzten die Eigenschaft der paarweisen Verschiedenheit zu verletzen.
Insgesamt haben wir also $n(n-1)(n-2)+n(n-1)(n-2)(n-3)$ Zykel der Länge höchstens 4.

\textbf{e)}
\textbf{i.}
Dies ist wahr, denn für $(u,v) \in \hat{E}$ gilt $(u,v) \in E\ \lor\ (u,v) \in E'$ und dann\\
$(v,u) \in E\ \lor\ (v,u) \in E'$ also insgesamt $(v,u) \in \hat{E}$.

\textbf{ii.}
Dies ist falsch. Betrachte $V = \{1,2\}$ und $E = \{(1,2)\},\ E' = \{(2,1)\},\ \hat{E} = \{(1,2),(2,1)\}$.
Dann ist $\hat{G}$ stark zsmhängend, aber weder $G$ noch $G^T$ stark zsmhängend.

\textbf{iii.}
Dies ist wahr. Sei o.B.d.A. $G$ stark zsmhängend, dann ist offensichtlich $E \subseteq \hat{E}$, und da
die Graphen alle die gleiche Knotenmenge $V$ haben damit auch $\hat{G}$ stark zsmhängend.

\textbf{iv.}
Dies ist wahr. Sei o.B.d.A. $G$ schwach zsmhängend, d.h. wir haben\\
$\forall u,v \in V: (u,v) \in E\ \lor\ (v,u) \in E$ und damit $\forall u,v \in V: (u,v) \in E'\ \lor\ (v,u) \in E'$. Für beide Knotenmengen
folgt dann aber im entsprechenden ungerichtetem Graph mit Knotenmenge $K$, dass
$\forall u,v \in V: (u,v) \in K$. Grundlegend entsprechen umgredrehte Kanten genau den gleichen ungerichteten wie
ihr ''Urbild''.

\newpage

\textbf{f)}
Dies ist falsch. Betrachte den DAG mit Knotenmenge $V = \{a,b,c\}$ und entsprechenden Gewichten $w_a = 0, w_b = 1,
w_c = 1$.
Weiter haben wir die Kantenmenge $E = \{(b,a),(c,a)\}$. Sowohl $b$ als auch $c$ hängen von $a$ ab, jedoch
nicht untereinander. Die einzig sinvollen Pfade wären $(a,b,c)$ oder $(a,c,b)$. Je nachdem
wie man die Definition nimmt ist nun entweder keiner von beiden ein kritischer Pfad ($v_i$ muss von $v_{i-1}$ abhängig
sein) oder beide und damit nicht eindeutig. In beiden Fällen ist die Behauptung widerlegt.

\aufgabe{2}
\textbf{a)}
Die asymptotische Laufzeit beträgt $\mathcal{O}(n^2)$. \texttt{DFS} wird grundsätzlich für jeden Knoten
genau 1 mal aufgerufen. Die Methode prüft in jedem Aufruf ob vll. einer der Adjazenten Knoten noch nicht behandelt
wurde und hat dabei konstante Kosten. Wenn \texttt{DFS} rekursiv aufgerufen wird, dann wird \texttt{DFS} halt
nicht nocheinmal aus der obersten \texttt{completeDFS}-Schleife aufgerufen, welche aber trotzdem asymptotisch
in $n$ vergleiche benötigt. Früher oder später wird auf jedem Knoten \texttt{DFS} aufgerufen und eine
for-Schleife mit linear vielen vergleichen ausgeführt.

\textbf{b)}
Die vergleiche in den Schleifen fallen nun weg. Nach dem gleichen Argument wie in a), dass schlussendlich
\texttt{DFS} genau 1 mal pro Knoten aufgerufen wird und konstant viele print aufrufe hat, folgt dann
dass die asymptotische Laufzeit sich in $\mathcal{O}(n)$ befindet.

\aufgabe{3}
W für WHITE, B für BLACK, G für GRAY. Beachte wir haben keinen Knoten 0, also 0. index in arrays ignoriert.
Stack von links nach rechts ist von top zu bottom.

\textbf{Phase 1:}
Nach 1. Iteration (0. Iteration ignoriert):\\
\texttt{color = [W, B, W, W, B, B, B, B, B, B, B, B], stack = [1,5,6,7,10,9,8,11,4]}

Nach 2. Iteration:\\
\texttt{color = [W, B, B, B, B, B, B, B, B, B, B, B], stack = [2,3,1,5,6,7,10,9,8,11,4]}

\textbf{Phase 3:}
Nach 1. Aufruf:\\
\texttt{color = [W, W, B, B, W, W, W, W, W, W, W, W], stack = [3,1,5,6,7,10,9,8,11,4]}\\
\texttt{scc = [0,0,2,2,0,0,0,0,0,0,0,0]}

Nach 2. Aufruf:\\
\texttt{color = [W, B, B, B, W, B, B, B, B, B, B, B], stack = [5,6,7,10,9,8,11,4]}\\
\texttt{scc = [0,1,2,2,0,1,1,1,1,1,1,1]}

Nach 3. Aufruf:\\
\texttt{color = [W, B, B, B, B, B, B, B, B, B, B, B], stack = []}\\
\texttt{scc = [0,1,2,2,4,1,1,1,1,1,1,1]}

\textbf{b)}
\begin{center}
    \begin{tikzpicture}
        [scale=0.8, auto=left,every node/.style={circle,draw}]
        \tikzset{edge/.style = {->,> = latex'}}
        \node (n1) at (2,1) {1};
        \node (n2) at (1,3)  {2};
        \node (n3) at (3,3)  {4};

        \foreach \from/\to in {n1/n3,n2/n1}
            \draw[edge] (\from) to (\to);
    \end{tikzpicture}
\end{center}

\aufgabe{4}
\textbf{a)}

\begin{tabular}{c|*{19}{c|}c}
    \texttt{topo}(v) & 1 & 2 & 3 & 4 & 5 & 6 & 7 & 8 & 9 & 10 & 11 & 12 & 13 & 14 & 15 & 16 & 17 & 18 & 19 & 20\\
    \hline
    Knoten v & 2 & 1 & 17 & 13 & 19 & 14 & 9 & 5 & 6 & 10 & 15 & 11 & 7 & 8 & 4 & 3 & 12 & 20 & 16 & 18
\end{tabular}

\textbf{b)}
\lstinputlisting{nr4.pseudo}

\aufgabe{5}

\begin{center}
\begin{tabular}{l|*{13}{c|}c}
    Knoten & A & B & C & D & E & F & G & H & I & J & K & L & M & N\\
    \hline
    \texttt{est}    & 16 & 11 & 10 & 8 & 13 & 3 & 6 & 6 & 0 & 4 & 0 & 0 & 0 & 0\\
    \hline
    \texttt{critDep}& E & G & D & F & C & M & J & L & - & N & - & - & - & -\\
    \hline
    \texttt{eft}    & 20 & 13 & 13 & 10 & 16 & 8 & 11 & 9 & 3 & 6 & 3 & 6 & 3 & 4\\
\end{tabular}
\end{center}
Reihenfolge Schwarzfärbung: N, J, G, B, M, F, K, D, I, C, L, H, E, A\\
Gesamtdauer: 20\\
Kritischer Pfad: M, F, D, C, E, A

\newpage

\aufgabe{6}

\lstinputlisting{nr6.pseudo}

Der Algorithmus führt eine normale DFS durch und sucht dabei nach Zykeln ungerader Länge.
Wenn wir während eines DFS auf einen grauen Knoten stossen, so haben wir einen Zykel im Graphen.
Nebenbei merkt sich der Algorithmus die Parität der Länge des Pfades vom ursprünglichen Startpunkt.
Somit können wir sofort erkennen wenn wir auf einen Zykel ungerader Länge treffen und dies returnen.

\end{document}
