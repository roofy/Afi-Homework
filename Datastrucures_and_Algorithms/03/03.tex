\documentclass[a4paper,graphics,11pt]{article}
\pagenumbering{arabic}

\usepackage[margin=1in]{geometry}
\usepackage[utf8]{inputenc}
\usepackage[T1]{fontenc}
\usepackage{lmodern}
\usepackage[ngerman]{babel}
\usepackage{amsmath, tabu}
\usepackage{amsthm}
\usepackage{amssymb}
\usepackage{complexity}
\usepackage{mathtools}
\usepackage{setspace}
\usepackage{graphicx,color,curves,epsf,float,rotating}
\usepackage{tasks}
\setlength{\parindent}{0em}
\setlength{\parskip}{1em}

\newcommand{\aufgabe}[1]{\subsection*{Aufgabe #1}}
\newcommand{\up}[2]{\mathrel{\overset{\makebox[0pt]{\mbox{\normalfont\tiny #2}}}{#1}}}

\usepackage{listings}
\usepackage{color}

\usepackage{tikz}

\definecolor{dkgreen}{rgb}{0,0.6,0}
\definecolor{gray}{rgb}{0.5,0.5,0.5}
\definecolor{mauve}{rgb}{0.58,0,0.82}

\lstset{frame=tb,
    language=Java,
    aboveskip=2mm,
    belowskip=2mm,
    showstringspaces=false,
    columns=flexible,
    basicstyle={\small\ttfamily},
    numbers=left,
    numberstyle=\tiny\color{gray},
    keywordstyle=\color{blue},
    commentstyle=\color{dkgreen},
    stringstyle=\color{mauve},
    breaklines=true,
    breakatwhitespace=true,
    tabsize=4,
    literate={ä}{{\"a}}1 {Ä}{{\"A}}1 {ö}{{\"o}}1 {Ö}{{\"O}}1 {ü}{{\"u}}1 {Ü}{{\"U}}1 {ß}{{\ss}}1
}

\begin{document}
\noindent Gruppe \fbox{\textbf{14}}             \hfill Phil Pützstück, 377247\\
\noindent Datenstrukturen und Algorithmen \hfill Benedikt Gerlach, 376944\\
\strut\hfill Sebastian Hackenberg, 377550\\
\begin{center}
	\LARGE{\textbf{Hausaufgabe 3}}
\end{center}
\begin{center}
\rule[0.1ex]{\textwidth}{1pt}
\end{center}

\aufgabe{1}
\textbf{a)} \\
Es sei ein Binärbaum $\mathcal{B}$ der Höhe $h$ gegeben. Um eine maximale Anzahl an inneren Knoten zu enthalten, sollte er eine 
maximale Anzahl an Knoten enthalten, also vollständig sein.
Damit enthält $\mathcal{B}$ nach Skript $2^{h+1} - 1$ Knoten.
Die Knoten in Ebene $h$ sind nach alle Blätter von $\mathcal{B}$, alle anderen Knoten haben den maximalen out-degree
von 2. Da Blätter keine inneren Knoten sind, ergibt sich also $(2^{h+1} - 1) - 2^h = 2^h - 1$ für die Anzahl
der inneren Knoten von $\mathcal{B}.\hfill\square$

\textbf{b)}\\
Es sei ein Binärbaum $\mathcal{B}$ gegeben sodass ein Knoten $v$ einen out-degree von 1 hat und dieses Kind
ein Blatt von $\mathcal{B}$ ist.
Wir nennen diesen Nachfolger hier mal $w$. Unabhängig davon, ob $w$ ein linkes oder rechtes Kind ist,
würde die Preorder-Traversierung $(\cdots, v, w, \cdots)$ und die Postorder-Traversierung $(\cdots, w, v, \cdots)$
lauten. Durch diese fehlende Information, ob ein gegebenes ''Einzelkind'' ein linkes oder rechtes ist, lässt sich der Baum nicht vollständig von den Traversierungen rekonstruieren. Wir geben ein kleines Beispiel:

\begin{minipage}{.3\textwidth}
\strut\hfill
\end{minipage}
\begin{minipage}{.3\textwidth}
    \begin{tikzpicture}
        \node[circle,draw](z){$u$}
            child[missing]{}
            child{
                node[circle,draw]{$v$} child{node[circle,draw] {$w$}} child[missing]};
    \end{tikzpicture}
\end{minipage}
\begin{minipage}{.3\textwidth}
    \begin{tikzpicture}
        \node[circle,draw](z){$u$}
            child[missing]{}
            child{
                node[circle,draw]{$v$} child[missing] child{node[circle,draw] {$w$}}};
    \end{tikzpicture}
\end{minipage}

Beide Binärbäume haben eine Preorder-Traversierung von $(u, v, w)$ und eine Postorder-\\
Traversierung von $(w, v, u).\hfill\square$

\textbf{c)}

Es sei ein Binärbaum $\mathcal{B}$ und dessen Inorder-Traversierung $i = (i_0, \cdots, i_n)$ sowie
Mirror-Linearisierung $l = (l_0, \cdots, l_m)$ für $m,n \in \mathbb{N}$ gegeben.
Nach Definition ist $l_0$ die Wurzel von $\mathcal{B}$.\\
Dann gibt es ein $k \in [0,n]$ mit $i_k = l_0$ sodass $(i_0, \cdots, i_{k-1})$
den linken, und $(i_{k+1},\cdots, i_n)$ den rechten Teilbaum der Wurzel darstellen. Weiter ist dann $i_n$ der
rechteste Knoten von $\mathcal{B}$. Es existiert weiter ein $j \in [0, m]$ mit $l_j = i_n$, sodass $(l_1, \cdots l_j)$
die Preorder-Traversierung des rechten Teilbaumes von $\mathcal{B}$ darstellt. Dieser Teilbaum ist wie bekannt
dann eindeutig bestimmt. Dies lässt sich nun rekursiv fortführen indem wir $l_{j+1}$ als Wurzel des linken Teilbaums
betrachten und wie für $\mathcal{B}$ vorgehen, insofern $j \neq n$, also $\mathcal{B}$ überhaupt einen linken
Teilbaum besitzt. Mit dieser Methode lässt sich ein gegebener Binärbaum eindeutig aus seiner Mirror-Linearisierung
und Inorder-Traversierung rekonstruieren.

\newpage
\aufgabe{2}
Der gegebene Algorithmus ist in gewisser Weise ein Variante des bekannten Depth-First-Search.
Wir besuchen jeden Knoten nur genau einmal, denn:\\
Sobald ein Knoten $v$ besucht wird, werden alle von ihm aus erreichbaren Knoten besucht, also
$$
M := \{v' \in V \mid v' \neq v \land \exists (v_0, v_1, \cdots, v_n) : v_0 = v \land v_n = v' \land \forall i \in [1,n] : (v_{i-1}, v_i) \in E\}
$$
Da der Algorithmus auf einem azyklischem Graphen arbeitet, gilt $v \notin M$, man kann also nicht wieder zu $v$
zurückkommen, während man seine Nachfolger besucht. Sobald man dann alle Nachfolger $v' \in M$ besucht hat,
wird durch eine Member-Variable angegeben dass $v$ schon besucht wurde. Da ein Knoten beliebig viele Vorgänger
bzw. eingehende Kanten haben kann, könnte man im weiteren Verlauf des Algorithmus nochmal bei $v$ vorbeikommen.
Jedoch wird zu Beginn von \texttt{visit($v$)} überprüft, ob $v$ schon besucht wurde. Somit wird jeder Knoten eines DAG
von dem gegebenen Algorithmus genau einmal besucht.

\aufgabe{3}
\textbf{a)}

Ja, alle 5 Operationen liegen in der Komplexitätsklasse $\mathcal{O}$(1) (konstante Laufzeit, unabhängig von n).

\texttt{isEmpty()} überprüft in einem Schritt ob \texttt{first} leer ist.

\texttt{enqueueFront()} fügt ein Element als neues \texttt{first} Element ein und verkettet das ehemalige erste mit dem neuen.

\texttt{enqueueBack()} fügt ein Element als neues \texttt{last} Element ein und verkettet das ehemalige mit dem neuen.

\texttt{dequeueFront()} entfernt das derzeitige \texttt{first} Element und setzt \texttt{first} auf das ehemalige zweite der Schlange.

\texttt{dequeBack()} ist analog zu \texttt{dequeFront()}.

\textbf{b)}

Die Laufzeit der fünf Operationen auf einem unbeschränktem Array ist nicht konstant, da für den Befehl \texttt{dequeFront()} zuerst das erste Element gelöscht werden muss und dann alle anderen Elemente verschoben werden müssen um die Umsetzung mit nur einem last-Zeiger zu ermöglichen.

\textbf{c)}

Die Operationen \texttt{add()} und \texttt{contains()} liegen in $\mathcal{O}(n)$ da Sie beide das ganze Set durchgehen müssen um zu überprüfen ob ein Element enthalten ist oder nicht.\\
Wobei \texttt{contains()} noch eine extra Zeiteinheit braucht um das Element hinzuzufügen, dennoch bleibt die Operation in $\mathcal{O}(n)$.\\
Ob die Aussage korrekt ist kommt auf die Funktionsweise von \texttt{union()} an, da  die Frage ist, ob doppelte Element gelöscht werden müssen oder nicht.\\
Falls sie nicht gelöscht werden müssen bleibt auch \texttt{union()} in $\mathcal{O}(n)$. Ansonsten bräuchte die Operation $ n^2$ Schritte und läge somit in $\mathcal{O}(n^2)$.

\textbf{d)}

Fallunterscheidung:

\textbf{Fall 1:} Doppelte Elemente müssen nicht gelöscht werden:\\
Durch Implemtierung mit verketteten Listen die man einfach aneinanderhängt braucht die Operation \texttt{union()} nur eine Operation und liegt somit in $\mathcal{O}(1)$.

\textbf{Fall 2:} Doppelte Elemente müssen gelöscht werden:\\
In diesem Fall ist einen Implementierung in konstanter Laufzeit nicht möglich, da für jedes Element aus set2, set1 ganz durchlaufen werden muss.

\aufgabe{4}
\textbf{a)}

Die Speicherkomplexität ist in beiden Fällen gleich da das Array vorher schon mit k+1 Einträgen angelegt werden muss.

\textbf{b)}

B(n,k) = 2 \quad (Beide Einsen sind direkt am Anfang hintereinander).\\
W(n,k) = n  \quad (Die ganze Liste muss durchgegangen werden).

Die Anzahl an Permutationen der List ist $\frac{n!}{2}$.
Die Anzahl an Permutationen der Liste sodass die 2te 1 an Index $i \in [0, n-1]$ steht, ist
$i(n - 2)!$. Da wir von einem Laplaceraum ausgehen ist jedes Ereignis gleichwahrscheinlich. Wir müssen die Liste
immer bis zur 2ten 1 durchgehen, also $(i+1)$ Operationen durchführen. Es folgt:
$$
    A(n,k) = \frac{2}{n!}\sum_{i=1}^{n}(i+1) i(n-2)!
    = \frac{2}{n(n-1)} \sum_{i=1}^{n} i(i+1)
    = \frac{2(n+1)(n+2)}{3(n-1)} \approx \frac{2}{3}n
$$
\\

\textbf{c)}


\lstinputlisting{aufgabe4c.txt}

Dieser Algorithmus hat eine konstante Speicherkomplexität in dem Worst-Case, da er unabhängig von n immer nur zwei Werte speichern muss.
Die Asymptotische Komplexität des Algorithmus im Worst-Case ist in $\mathcal{O}(n^2)$(wenn nur die If-Abfragen als Operationen gelten).

\newpage
\textbf{d)}

\lstinputlisting{aufgabe4d1.txt}
Der Algorithmus hat eine lineare Laufzeit da er genau 2n If-Abfragen in jedem Fall außer \texttt{l.length <= 2} durchführt.
Außerdem ist der Platzbedarf $n + 3 \le n\cdot(\log n-1 )+ 4\log n$ in allen Fällen.\\[50pt]


\lstinputlisting{aufgabe4d2.txt}
Der Speicherbedarf ist in allen Fällen konstant und somit kleiner $4\log n < 5 \log n$.
 \end{document}
