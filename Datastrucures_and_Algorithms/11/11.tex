\documentclass[a4paper,graphics,11pt]{article}
\pagenumbering{arabic}

\usepackage[margin=1in]{geometry}
\usepackage[utf8]{inputenc}
\usepackage[T1]{fontenc}
\usepackage{lmodern}
\usepackage[ngerman]{babel}
\usepackage{amsmath, tabu}
\usepackage{amsthm}
\usepackage{amssymb}
\usepackage{complexity}
\usepackage{mathtools}
\usepackage{setspace}
\usepackage{graphicx,color,curves,epsf,float,rotating}
\usepackage{tasks}
\setlength{\parindent}{0em}
\setlength{\parskip}{1em}

\newcommand{\aufgabe}[1]{\subsection*{Aufgabe #1}}
\newcommand{\up}[2]{\mathrel{\overset{\makebox[0pt]{\mbox{\normalfont\tiny #2}}}{#1}}}

\usepackage{listings}
\usepackage{color}

\usepackage{tikz}
\usetikzlibrary{arrows,automata,positioning}

\definecolor{dkgreen}{rgb}{0,0.6,0}
\definecolor{gray}{rgb}{0.5,0.5,0.5}
\definecolor{mauve}{rgb}{0.58,0,0.82}

\lstset{frame=tb,
    language=Java,
    aboveskip=2mm,
    belowskip=2mm,
    showstringspaces=false,
    columns=flexible,
    basicstyle={\small\ttfamily},
    numbers=left,
    numberstyle=\tiny\color{gray},
    keywordstyle=\color{blue},
    commentstyle=\color{dkgreen},
    stringstyle=\color{mauve},
    breaklines=true,
    breakatwhitespace=true,
    tabsize=4,
    literate={ä}{{\"a}}1 {Ä}{{\"A}}1 {ö}{{\"o}}1 {Ö}{{\"O}}1 {ü}{{\"u}}1 {Ü}{{\"U}}1 {ß}{{\ss}}1
}

\begin{document}
\noindent Gruppe \fbox{\textbf{14}}             \hfill Phil Pützstück, 377247\\
\noindent Datenstrukturen und Algorithmen \hfill Benedikt Gerlach, 376944\\
\strut\hfill Sebastian Hackenberg, 377550\\
\begin{center}
	\LARGE{\textbf{Hausaufgabe 11}}
\end{center}
\begin{center}
\rule[0.1ex]{\textwidth}{1pt}
\end{center}

\aufgabe{1}
\textbf{a)}

Wert des Flusses insgesamt nach Iterationen: $1,\ 1+\phi,\ 1+2\phi,\ 2+\phi,\ 3,\ 2+2\phi\cdots$\\
Dies lässt sich nach dem Hinweis auch in Potenzen von $\phi$ schreiben, also
$$
    1,\ 1+\phi,\ 1+2\phi,\ 1+2\phi+\phi^2,\ 1+2\phi+2\phi^2,\ 1+2\phi+2\phi^2+\phi^3\cdots
$$
\textbf{b)}

Es ergibt sich ein offensichtliches Muster, wenn man den Wert des Flusses wie oben in Potenzen von $\phi$ schreibt.
Insgesamt ist der Wert des Flusses nach der $2n$-ten Iteration gegeben durch:
$$
    f_{2n} = 1+2\sum_{i=1}^{n} \phi^i
$$
Es ergibt sich folgende Konvergenz:
$$
    1+2\sum_{i=1}^{\infty} \phi^i
    \quad\up{=}{$|\phi| < 1$}\quad 1+2\frac{1}{1-\phi} = 4 + \sqrt{5} < 201
$$
Da also $201$ der eigentliche maximale Flusswert ist, konvergiert der Algorithmus hier nicht zum erwünschten maximalen
Fluss.

\textbf{c)}

Es wird behauptet, dass wenn der Algorithmus terminiert, er dann auf jeden Fall einen maximalen Fluss bestimmt.
Ferner wird jedoch nur garantiert, dass er bei rationalen Kapazitäten terminiert.
Da aber $\phi \notin \mathbb{Q}$, ist die falsche Konvergenz hier kein Problem.

\aufgabe{2}
\textbf{a)}

\begin{tikzpicture}
    [scale=0.8, auto=left,every node/.style={circle,draw}]
    \tikzset{edge/.style = {->,> = latex'}}
    \node (q1) at (0,0)  {};
    \node (q2) at (2,0)  {};
    \node (q3) at (4,0)  {};

    \foreach \from/\to in {q1/q2,q2/q3}
        \draw[edge] (\from) to (\to);
\end{tikzpicture}
\qquad\qquad
\begin{tikzpicture}
        [scale=0.8, auto=left,every node/.style={circle,draw}]
        \tikzset{edge/.style = {->,> = latex'}}
        \node (q1) at (0,0)  {};
        \node (q2) at (2,0)  {};
        \node (q3) at (4,0)  {};

        \draw[edge] (q1) to (q2);
        \draw[edge,bend left] (q2) to (q3);
        \draw[edge,bend right] (q2) to (q3);
\end{tikzpicture}
\qquad\qquad
\begin{tikzpicture}
        [scale=0.8, auto=left,every node/.style={circle,draw}]
        \tikzset{edge/.style = {->,> = latex'}}
        \node (q1) at (0,0)  {};
        \node (q2) at (2,0)  {};
        \node (q3) at (4,0)  {};
        \node (q4) at (6,0)  {};

        \draw[edge,bend left] (q1) to (q4);
        \draw[edge,bend right] (q1) to (q4);
        \draw[edge] (q1) to (q2);
        \draw[edge] (q2) to (q3);
        \draw[edge] (q3) to (q4);
\end{tikzpicture}

\newpage

\textbf{b)}

Wir können so einen Algorithmus rekursiv über die Struktur der Graphen implementieren.\\
Der Basisfall ist natürlich der Graph $s \to t$, bei welchem der maximale Fluss eben der Kapazität der Kante $(s,t)$
entspricht.
Nun können wir den rekursiven Fall betrachten:
Bei einer Serie von SP-Graphen ist der maximale Fluss eben das Minimum der maximalen Flüsse der SP-Graphen, ähnlich
wie bei den Kapazitäten der Kanten in einem augmentierenden Pfad.\\
Wenn wir 2 SP-Graphen Parallel ausführen so ist der maximale Fluss eben die Summe der maximalen Flüsse der
SP-Graphen.

Insgesamt:

MaxFlow($s,t$) = $c(s,t)$.\\
MaxFlow(Serie($s_G, t_G$)) = $\min\{$MaxFlow($s_G$), MaxFlow($t_G$)$\}$\\
MaxFlow(Parallel($s_G, t_G$)) = MaxFlow($s_G$) + MaxFlow($t_G$)

\aufgabe{3}

\begin{tabular}{c|c|c|c|c|c}
    2 & A & B & C & D & E\\
    \hline
    A & 0 & 8 & 1 & 5 & $\infty$\\
    \hline
    B & $\infty$ & 0 & $\infty$ & $\infty$ & $\infty$\\
    \hline
    C & $\infty$ & 7 & 0 & 1 & 1\\
    \hline
    D & 8 & \underline{16} & 3 & 0 & 1\\
    \hline
    E & $\infty$ & 3 & 1 & $\infty$ & 0\\
\end{tabular}
\qquad
\begin{tabular}{c|c|c|c|c|c}
    3 & A & B & C & D & E\\
    \hline
    A & 0 & 8 & 1 & 5 & $\infty$\\
    \hline
    B & $\infty$ & 0 & $\infty$ & $\infty$ & $\infty$\\
    \hline
    C & $\infty$ & 7 & 0 & 1 & 1\\
    \hline
    D & 8 & 16 & 3 & 0 & 1\\
    \hline
    E & $\infty$ & 3 & 1 & $\infty$ & 0\\
\end{tabular}
\qquad
\begin{tabular}{c|c|c|c|c|c}
    4 & A & B & C & D & E\\
    \hline
    A & 0 & 8 & 1 & \underline{2} & \underline{2}\\
    \hline
    B & $\infty$ & 0 & $\infty$ & $\infty$ & $\infty$\\
    \hline
    C & $\infty$ & 7 & 0 & 1 & 1\\
    \hline
    D & 8 & \underline{10} & 3 & 0 & 1\\
    \hline
    E & $\infty$ & 3 & 1 & \underline{2} & 0\\
\end{tabular}

\begin{tabular}{c|c|c|c|c|c}
    5 & A & B & C & D & E\\
    \hline
    A & 0 & 8 & 1 & 2 & 2\\
    \hline
    B & $\infty$ & 0 & $\infty$ & $\infty$ & $\infty$\\
    \hline
    C & \underline{9} & 7 & 0 & 1 & 1\\
    \hline
    D & 8 & 10 & 3 & 0 & 1\\
    \hline
    E & \underline{10} & 3 & 1 & 2 & 0\\
\end{tabular}
\qquad
\begin{tabular}{c|c|c|c|c|c}
    6 & A & B & C & D & E\\
    \hline
    A & 0 & \underline{5} & 1 & 2 & 2\\
    \hline
    B & $\infty$ & 0 & $\infty$ & $\infty$ & $\infty$\\
    \hline
    C & 9 & \underline{4} & 0 & 1 & 1\\
    \hline
    D & 8 & \underline{4} & \underline{2} & 0 & 1\\
    \hline
    E & 10 & 3 & 1 & 2 & 0\\
\end{tabular}

\aufgabe{4}

initiales Restnetzwerk\hfill Nächstes Flussnetzwerk mit aktuellem Fluss

\begin{tikzpicture}[node distance=3cm,every circle node/.style={circle,draw}]
    \tikzset{edge/.style = {->,> = latex'}}
    \node[circle] (s)     at (0,0)      {$s$};
    \node[circle] (b)     at (2,0)      {$B$};
    \node[circle] (e)     at (4,0)      {$E$};
    \node[circle] (c)     at (2,2)      {$C$};
    \node[circle] (a)     at (2,-2)     {$A$};
    \node[circle] (d)     at (4,-2)     {$D$};
    \node[circle] (t)     at (6,-2)     {$t$};

    \path[edge] (s) edge [above left] node {$7$} (c);
    \path[edge] (s) edge [above] node {$7$} (b);
    \path[edge] (s) edge [below left] node {$7$} (a);
    \path[edge] (c) edge [above right] node {$1$} (e);
    \path[edge] (e) edge [above right] node {$7$} (t);
    \path[edge] (b) edge [above] node {$3$} (e);
    \path[edge] (a) edge [right] node {$5$} (b);
    \path[edge] (a) edge [above] node {$3$} (d);
    \path[edge] (d) edge [above] node {$7$} (t);
    \path[edge] (d) edge [left] node {$3$} (e);
\end{tikzpicture}
\hfill
\begin{tikzpicture}[node distance=3cm,every circle node/.style={circle,draw}]
    \tikzset{edge/.style = {->,> = latex'}}
    \node[circle] (s)     at (0,0)      {$s$};
    \node[circle] (b)     at (2,0)      {$B$};
    \node[circle] (e)     at (4,0)      {$E$};
    \node[circle] (c)     at (2,2)      {$C$};
    \node[circle] (a)     at (2,-2)     {$A$};
    \node[circle] (d)     at (4,-2)     {$D$};
    \node[circle] (t)     at (6,-2)     {$t$};

    \path[edge] (s) edge [above left] node {$0/7$} (c);
    \path[edge] (s) edge [above] node {$0/7$} (b);
    \path[edge] (s) edge [below left] node {$3/7$} (a);
    \path[edge] (c) edge [above right] node {$0/1$} (e);
    \path[edge] (e) edge [above right] node {$0/7$} (t);
    \path[edge] (b) edge [above] node {$0/3$} (e);
    \path[edge] (a) edge [right] node {$0/5$} (b);
    \path[edge] (a) edge [above] node {$3/3$} (d);
    \path[edge] (d) edge [above] node {$3/7$} (t);
    \path[edge] (d) edge [left] node {$0/3$} (e);
\end{tikzpicture}

\newpage

Restnetzwerk\hfill Nächstes Flussnetzwerk mit aktuellem Fluss

\begin{tikzpicture}[node distance=3cm,every circle node/.style={circle,draw}]
    \tikzset{edge/.style = {->,> = latex'}}
    \node[circle] (s)     at (0,0)      {$s$};
    \node[circle] (b)     at (2,0)      {$B$};
    \node[circle] (e)     at (4,0)      {$E$};
    \node[circle] (c)     at (2,2)      {$C$};
    \node[circle] (a)     at (2,-2)     {$A$};
    \node[circle] (d)     at (4,-2)     {$D$};
    \node[circle] (t)     at (6,-2)     {$t$};

    \path[edge] (s) edge [above left] node {$7$} (c);
    \path[edge] (s) edge [above] node {$7$} (b);
    \path[edge] (s) edge [below left] node {$4$} (a);
    \path[edge] (a) [bend left=45] edge [below left] node {$3$} (s);
    \path[edge] (c) edge [above right] node {$1$} (e);
    \path[edge] (e) edge [above right] node {$7$} (t);
    \path[edge] (b) edge [above] node {$3$} (e);
    \path[edge] (a) edge [right] node {$5$} (b);
    \path[edge] (d) edge [above] node {$3$} (a);
    \path[edge] (d) edge [below] node {$4$} (t);
    \path[edge] (t) [bend left=45] edge [below] node {$3$} (d);
    \path[edge] (d) edge [left] node {$3$} (e);
\end{tikzpicture}
\hfill
\begin{tikzpicture}[node distance=3cm,every circle node/.style={circle,draw}]
    \tikzset{edge/.style = {->,> = latex'}}
    \node[circle] (s)     at (0,0)      {$s$};
    \node[circle] (b)     at (2,0)      {$B$};
    \node[circle] (e)     at (4,0)      {$E$};
    \node[circle] (c)     at (2,2)      {$C$};
    \node[circle] (a)     at (2,-2)     {$A$};
    \node[circle] (d)     at (4,-2)     {$D$};
    \node[circle] (t)     at (6,-2)     {$t$};

    \path[edge] (s) edge [above left] node {$0/7$} (c);
    \path[edge] (s) edge [above] node {$3/7$} (b);
    \path[edge] (s) edge [below left] node {$3/7$} (a);
    \path[edge] (c) edge [above right] node {$0/1$} (e);
    \path[edge] (e) edge [above right] node {$3/7$} (t);
    \path[edge] (b) edge [above] node {$3/3$} (e);
    \path[edge] (a) edge [right] node {$0/5$} (b);
    \path[edge] (a) edge [above] node {$3/3$} (d);
    \path[edge] (d) edge [above] node {$3/7$} (t);
    \path[edge] (d) edge [left] node {$0/3$} (e);
\end{tikzpicture}

Restnetzwerk\hfill Nächstes Flussnetzwerk mit aktuellem Fluss

\begin{tikzpicture}[node distance=3cm,every circle node/.style={circle,draw}]
    \tikzset{edge/.style = {->,> = latex'}}
    \node[circle] (s)     at (0,0)      {$s$};
    \node[circle] (b)     at (2,0)      {$B$};
    \node[circle] (e)     at (4,0)      {$E$};
    \node[circle] (c)     at (2,2)      {$C$};
    \node[circle] (a)     at (2,-2)     {$A$};
    \node[circle] (d)     at (4,-2)     {$D$};
    \node[circle] (t)     at (6,-2)     {$t$};

    \path[edge] (s) edge [above left] node {$7$} (c);
    \path[edge] (b) [bend right=20] edge [above,pos=0.3] node {$3$} (s);
    \path[edge] (s) edge [below] node {$4$} (b);
    \path[edge] (s) edge [below left] node {$4$} (a);
    \path[edge] (a) [bend left=45] edge [below left] node {$3$} (s);
    \path[edge] (c) edge [above right] node {$1$} (e);
    \path[edge] (e) edge [below left] node {$4$} (t);
    \path[edge] (t) [bend right=45] edge [above right] node {$3$} (e);
    \path[edge] (e) edge [above] node {$3$} (b);
    \path[edge] (a) edge [right] node {$5$} (b);
    \path[edge] (d) edge [above] node {$3$} (a);
    \path[edge] (d) edge [below] node {$4$} (t);
    \path[edge] (t) [bend left=45] edge [below] node {$3$} (d);
    \path[edge] (d) edge [left] node {$3$} (e);
\end{tikzpicture}
\hfill
\begin{tikzpicture}[node distance=3cm,every circle node/.style={circle,draw}]
    \tikzset{edge/.style = {->,> = latex'}}
    \node[circle] (s)     at (0,0)      {$s$};
    \node[circle] (b)     at (2,0)      {$B$};
    \node[circle] (e)     at (4,0)      {$E$};
    \node[circle] (c)     at (2,2)      {$C$};
    \node[circle] (a)     at (2,-2)     {$A$};
    \node[circle] (d)     at (4,-2)     {$D$};
    \node[circle] (t)     at (6,-2)     {$t$};

    \path[edge] (s) edge [above left] node {$1/7$} (c);
    \path[edge] (s) edge [above] node {$3/7$} (b);
    \path[edge] (s) edge [below left] node {$3/7$} (a);
    \path[edge] (c) edge [above right] node {$1/1$} (e);
    \path[edge] (e) edge [above right] node {$4/7$} (t);
    \path[edge] (b) edge [above] node {$3/3$} (e);
    \path[edge] (a) edge [right] node {$0/5$} (b);
    \path[edge] (a) edge [above] node {$3/3$} (d);
    \path[edge] (d) edge [above] node {$3/7$} (t);
    \path[edge] (d) edge [left] node {$0/3$} (e);
\end{tikzpicture}

Restnetzwerk

\begin{tikzpicture}[node distance=3cm,every circle node/.style={circle,draw}]
    \tikzset{edge/.style = {->,> = latex'}}
    \node[circle] (s)     at (0,0)      {$s$};
    \node[circle] (b)     at (2,0)      {$B$};
    \node[circle] (e)     at (4,0)      {$E$};
    \node[circle] (c)     at (2,2)      {$C$};
    \node[circle] (a)     at (2,-2)     {$A$};
    \node[circle] (d)     at (4,-2)     {$D$};
    \node[circle] (t)     at (6,-2)     {$t$};

    \path[edge] (s) edge [below right,pos=0.9] node {$6$} (c);
    \path[edge] (c) [bend right=45] edge [above left] node {$1$} (s);
    \path[edge] (b) [bend right=20] edge [above,pos=0.3] node {$3$} (s);
    \path[edge] (s) edge [below] node {$4$} (b);
    \path[edge] (s) edge [below left] node {$4$} (a);
    \path[edge] (a) [bend left=45] edge [below left] node {$3$} (s);
    \path[edge] (e) edge [above right] node {$1$} (c);
    \path[edge] (e) edge [below left] node {$3$} (t);
    \path[edge] (t) [bend right=45] edge [above right] node {$4$} (e);
    \path[edge] (e) edge [above] node {$3$} (b);
    \path[edge] (a) edge [right] node {$5$} (b);
    \path[edge] (d) edge [above] node {$3$} (a);
    \path[edge] (d) edge [below] node {$4$} (t);
    \path[edge] (t) [bend left=45] edge [below] node {$3$} (d);
    \path[edge] (d) edge [left] node {$3$} (e);
\end{tikzpicture}

Der maximale Fluss hat den Wert 7.

\newpage
\aufgabe{5}

\end{document}
