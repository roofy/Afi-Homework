\documentclass[a4paper,graphics,11pt]{article}
\pagenumbering{arabic}

\usepackage[margin=1in]{geometry}
\usepackage[utf8]{inputenc}
\usepackage[T1]{fontenc}
\usepackage{lmodern}
\usepackage[ngerman]{babel}
\usepackage{amsmath, tabu}
\usepackage{amsthm}
\usepackage{amssymb}
\usepackage{complexity}
\usepackage{mathtools}
\usepackage{setspace}
\usepackage{graphicx,color,curves,epsf,float,rotating}
\usepackage{tasks}
\setlength{\parindent}{0em}
\setlength{\parskip}{1em}

\newcommand{\aufgabe}[1]{\subsection*{Aufgabe #1}}
\newcommand{\up}[2]{\mathrel{\overset{\makebox[0pt]{\mbox{\normalfont\tiny #2}}}{#1}}}

\usepackage{listings}
\usepackage{color}

\definecolor{dkgreen}{rgb}{0,0.6,0}
\definecolor{gray}{rgb}{0.5,0.5,0.5}
\definecolor{mauve}{rgb}{0.58,0,0.82}

\lstset{frame=tb,
    language=Java,
    aboveskip=2mm,
    belowskip=2mm,
    showstringspaces=false,
    columns=flexible,
    basicstyle={\small\ttfamily},
    numbers=left,
    numberstyle=\tiny\color{gray},
    keywordstyle=\color{blue},
    commentstyle=\color{dkgreen},
    stringstyle=\color{mauve},
    breaklines=true,
    breakatwhitespace=true,
    tabsize=4,
    literate={ä}{{\"a}}1 {Ä}{{\"A}}1 {ö}{{\"o}}1 {Ö}{{\"O}}1 {ü}{{\"u}}1 {Ü}{{\"U}}1 {ß}{{\ss}}1
}

\begin{document}
\noindent Gruppe \fbox{\textbf{14}}             \hfill Phil Pützstück, 377247\\
\noindent Datenstrukturen und Algorithmen \hfill Benedikt Gerlach, 376944\\
\strut\hfill Sebastian Hackenberg, 377550\\
\begin{center}
	\LARGE{\textbf{Hausaufgabe 4}}
\end{center}
\begin{center}
\rule[0.1ex]{\textwidth}{1pt}
\end{center}

\aufgabe{1}
\newpage
\aufgabe{2}
\newpage
\aufgabe{3}
\textbf{a)}

Wir nehmen an, dass $\succeq$ eine Halbordnung ist. (vll noch beweisen?)

Sei $M \subseteq \mathbb{T}$. Da die reellen Zahlen totalgeordent sind, existiert ein
$m \in M$ und ein $n \in \mathbb{N}$ sodass
$$
    \forall m' \in M : \forall n' \in \mathbb{N} : m'(n') \leq m(n)
$$
Es folgt, dass $\forall m' \in M : m \succeq m' \,\Longrightarrow\, m' \succeq m$.
Damit ist $m$ eine obere Schranke von $M$.\\
Sei nun $m' \in \mathbb{T}$. Sei ferner eine weitere obere Schranke $m' \in \mathbb{T}$ von $M$ gegeben.
Es muss also gelten, dass
\begin{equation}
    \forall n' \in \mathbb{N} : m(n) \leq m'(n')
\end{equation}
denn sonst wäre $m'$ keine obere Schranke von $M$ im Sinne von $\succeq$, da $m \in M$.
Daraus folgt aber eben genau $m \succeq m'$, also ist $m'$ im Sinne von $\succeq$ keine 
kleinere obere Schranke von $M$ als $m$.\\
Da $m' \in \mathbb{T}$ eine beliebige obere Schranke war, folgt daraus, dass $m$ die kleinste obere Schranke von $M$ ist.

Wir gehen analog für die größte unter Schranke von $M$ vor:\\
Durch die Totalordnung der reellen Zahlen ist die Existenz eine $m \in M$ und $n \in \mathbb{N}$
gegeben, sodass
$$
    \forall m' \in M : \forall n' \in \mathbb{N} : m(n) \leq m'(n')
$$
Es folgt, dass $\forall m' \in M : m' \succeq m \,\Longrightarrow\, m \succeq m'$. Damit ist $m$ schonmal eine untere Schranke von $M$.\\
Sei nun $m' \in \mathbb{T}$ Sei ferner eine weitere untere Schranke $m' \in \mathbb{T}$ von $M$ gegeben.
Es muss also gelten, dass
$$
    \forall n' \in \mathbb{N} : m'(n') \leq m(n)
$$
denn sonst wäre $m'$ keine untere Schranke von $M$ im Sinne von $\succeq$, da $m \in M$.
Daraus folgt aber eben genau $m' \succeq m$, also ist $m'$ im Sinne von $\succeq$ keine größere
Schranke von $M$ als $m$.\\
Da $m' \in \mathbb{T}$ eine beliebige untere Schranke war, folgt daraus,
dass $m$ die größte obere Schranke von $M$ ist.

Insgesamt existieren für jede beliebige Teilmenge $M \subseteq \mathbb{T}$ eine kleinste obere und größte
untere Schranke im Sinne von $\succeq$. Folglich ist $(\mathbb{T}, \succeq)$ ein vollständiger Verband.

\textbf{b)}
\newpage
\aufgabe{4}

\end{document}
