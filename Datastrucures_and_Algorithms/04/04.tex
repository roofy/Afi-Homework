\documentclass[a4paper,graphics,11pt]{article}
\pagenumbering{arabic}
\usepackage{forest}

\usepackage[margin=0.7in]{geometry}
\usepackage[utf8]{inputenc}
\usepackage[T1]{fontenc}
\usepackage{lmodern}
\usepackage[ngerman]{babel}
\usepackage{amsmath, tabu}
\usepackage{amsthm}
\usepackage{amssymb}
\usepackage{complexity}
\usepackage{mathtools}
\usepackage{setspace}
\usepackage{graphicx,color,curves,epsf,float,rotating}
\usepackage{tasks}
\setlength{\parindent}{0em}
\setlength{\parskip}{1em}

\newcommand{\aufgabe}[1]{\subsection*{Aufgabe #1}}
\newcommand{\up}[2]{\mathrel{\overset{\makebox[0pt]{\mbox{\normalfont\tiny #2}}}{#1}}}

\usepackage{listings}
\usepackage{color}

\definecolor{dkgreen}{rgb}{0,0.6,0}
\definecolor{gray}{rgb}{0.5,0.5,0.5}
\definecolor{mauve}{rgb}{0.58,0,0.82}

\lstset{frame=tb,
    language=Java,
    aboveskip=2mm,
    belowskip=2mm,
    showstringspaces=false,
    columns=flexible,
    basicstyle={\small\ttfamily},
    numbers=left,
    numberstyle=\tiny\color{gray},
    keywordstyle=\color{blue},
    commentstyle=\color{dkgreen},
    stringstyle=\color{mauve},
    breaklines=true,
    breakatwhitespace=true,
    tabsize=4,
    literate={ä}{{\"a}}1 {Ä}{{\"A}}1 {ö}{{\"o}}1 {Ö}{{\"O}}1 {ü}{{\"u}}1 {Ü}{{\"U}}1 {ß}{{\ss}}1
}

\begin{document}
\noindent Gruppe \fbox{\textbf{14}}             \hfill Phil Pützstück, 377247\\
\noindent Datenstrukturen und Algorithmen \hfill Benedikt Gerlach, 376944\\
\strut\hfill Sebastian Hackenberg, 377550\\
\begin{center}
	\LARGE{\textbf{Hausaufgabe 4}}
\end{center}
\begin{center}
\rule[0.1ex]{\textwidth}{1pt}
\end{center}

\aufgabe{1}
Der Suchbereich wird in jedem Schleifendurchlauf durch verschieben der \texttt{left}, \texttt{right} sowie
\texttt{lmid} und \texttt{rmid} Zeiger (im folgenden als $l, r, lm, rm$) dreigeteilt. Im nächsten Durchlauf ist die Größe des Arrays
$$
    lm - l = \left\lceil \frac{2l + r}{3} \right\rceil - l
    = \left\lceil \frac{r - l}{3} \right\rceil
    \qquad \text{oder}\qquad
    r - rm = r - \left\lfloor \frac{l + 2r}{3} \right\rfloor
    = \left\lfloor \frac{r - l}{3} \right\rfloor
$$
oder aber auch
$rm - lm = \left\lfloor \frac{l + 2r}{3} \right\rfloor - \left\lceil \frac{2l + r}{3} \right\rceil
= \left\lfloor\frac{r - l}{3} \right\rfloor$.\\
Also ist die neue Größe im schlimmsten Fall
$\left\lceil \frac{r - l}{3} \right\rceil := \left\lceil\frac{n-1}{3}\right\rceil$.\\
Die maximale Anzahl der Schleifendurchläufe lässt sich also durch folgende Rekursionsgleichung beschreiben
$$
    S(n) = \begin{cases}
        1 & n = 1\\
        1 + S(\lceil\frac{n-1}{3}\rceil) & \text{sonst}
    \end{cases}
$$
Wir führen Induktion über $n \in \mathbb{N}$ mit der Hypothese, dass $S(n) = 1 + \lfloor\log_3 n\rfloor$.\\
Sei also $n = 1$. Dann ist
$$
    S(n) = 1 = 1 + \lfloor \log_3 1\rfloor
$$
Sei nun $n > 1$. Dann
$$
    S(n)
    = 1 + S\left(\left\lceil\frac{n-1}{3}\right\rceil\right)
    = 1 + \left\lfloor\log_3\left\lceil\frac{n-1}{3}\right\rceil\right\rfloor + 1
    = 1 + \lfloor \log_3 n\rfloor
$$
Nach dem Prinzip der vollständigen Induktion gilt nun $S(n) = 1 + \lfloor\log_3 n\rfloor \in \mathcal{O}(\log n).\hfill\square$
\aufgabe{2}
\textbf{a)}
Der Worst-case tritt ein, wenn das gesuchte Element $K$ entweder sofort am Anfang oder am Ende des
Arrays liegt. In diesem Fall muss der jeweils rechte oder linke ''Index-pointer'' genau $n-1$
mal verschoben werden, wobei genau $n$ mal die Schleifenbedingung überprüft wird.
Es ist also $W(n) = n\in \mathcal{O}(n)$.

\textbf{b)}
Der Best-case tritt ein, wenn das gesuchte Element $K$ genau in der Mitte des arrays liegt, also
bei $\lfloor\frac{n}{2}\rfloor$. In diesem Fall werden beide
''Index-pointer'' genau $\lfloor\frac{n}{2}\rfloor$ mal bewegt, wobei die Schleifenbedingung
dann $\lfloor\frac{n}{2}\rfloor + 1$ mal überprüft wird.
Es ist also $B(n) = \lfloor\frac{n}{2}\rfloor + 1 \in \mathcal{O}(n)$

\textbf{c)} 
Wir nehmen an $n$ gerade. Die Wkeit dass das gesuchte Element in Index $i$ vorkommt ist $\frac{1}{n}$.
Die benötigten Vergleiche für $K$ bei Index $0\leq i \leq n$
sind symmetrisch mit globalem Minimum um $i = \frac{n}{2}$ aufgebaut, wodurch die doppelte Summe des
Intervalls der Indexe $[\frac{n}{2}, n-1]$ genügt:
$$
    A(n)
    = \frac{1}{n} \sum_{i=0}^{n-1}\left( \frac{n}{2} + \left|\frac{n}{2} -i\right|\right)
    = \frac{2}{n} \sum_{i=\frac{n}{2}}^{n-1} i
    = \frac{2}{n} \cdot \frac{n(3n-2)}{8}
    = \frac{3n-2}{4} \in \mathcal{O}(n)
$$

\newpage
\aufgabe{3}
\textbf{a)}
Wir zeigen zuerst; $\preceq$ ist eine Halbordnung auf $\mathbb{T}$.
Für $t \in \mathbb{T}$ gilt
$$
    \forall n \in \mathbb{N} : t(n) \leq t(n) \,\Longrightarrow\, t \preceq t
$$
Für weitere $r,s \in \mathbb{T}$ gilt ferner
$$
    t \preceq r \land r \preceq s
    \,\Longrightarrow\,
    \forall n \in \mathbb{N} : t(n) \leq r(n) \land r(n) \leq s(n)
    \,\Longrightarrow\, \forall n \in \mathbb{N} : t(n) \leq r(n) \leq s(n)
$$$$
    \,\Longrightarrow\, \forall n \in \mathbb{N} : t(n) \leq s(n)
    \,\Longrightarrow\, t \preceq s
$$
Ferner gilt noch
$$
    t \preceq s \land s \preceq t
    \,\Longrightarrow\, \forall n \in\mathbb{N} : t(n) \leq s(n) \land s(n) \leq t(n)
    \,\Longrightarrow\, \forall n \in\mathbb{N} : s(n) = t(n)
    \,\Longrightarrow\, s = t
$$
Insgesamt ist $\preceq$ auf $\mathbb{T}$ reflexiv, transitiv und antisymmetrisch, also eine Halbordnung.
 
Sei nun $M \subseteq \mathbb{T}$. Wir definieren die Menge der Bilder der Abbildunges aus $M$ als
$B := \bigcup_{m \in M} \text{Im}(m)$.\\
Wir unterscheiden 3 Fälle:\\
\textbf{Fall 1:} $\infty \notin B$.\\
Damit folgt nach Definition $B \subseteq \mathbb{R}_{\geq 0}$.
Da die rellen Zahlen totalgeordnet sind, exisitert $s := \text{max}\ B$ sowie $0 \leq r := \text{min}\ B$.
Damit existiert sowohl eine obere als auch eine untere Schranke für $B$.
Einer Charakterisierung der rellen Zahlen (siehe z.B. B. Stamm, Analysis für Informatiker, WS17/18 RWTH)
zufolge, existiert zu jeder beschränkten Teilmenge sowohl Supremum (kleinste obere Schranke) als auch Infimum
(größte untere Schranke).

Sei nun also $S := \text{sup}\ B$ sowie $R := \text{inf}\ B$. Dann exisiteren nach Definition von $\mathbb{T}$
Funktionen $f,g$ mit $f(n) = S$ und $g(n) = R$ für alle $n \in \mathbb{N}$. Nach Konstruktion folgt dann
$$
    \forall m \in M : \forall n \in \mathbb{N} : R = g(n) \leq m(n) \leq f(n) = S
$$
Nach Definition von $\preceq$ gilt dann auch
$$
    \forall m \in M : (g \preceq m) \land (m \preceq f)
$$
Dadurch sind $g$ und $f$ respektiv untere und obere Schranken von $M$ bzgl. $\preceq$. Da jedoch nach Konstruktion
von $g$ und $f$ mindestens ein $m \in M, n \in \mathbb{N}$ mit $m(n) = f(n) = S := \text{sup}\ B$ sowie
ein $m \in M, n \in \mathbb{N}$ mit $m(n) = g(n) = R := \text{inf}\ B$, kann es keine kleiner obere
Schranke als $f$ bzw größere unter Schranke als $g$ von $M$ bzgl. $\preceq$ geben. Folglich sind
$f$ und $g$ auch eben Supremum und Infimum von $M$ bzgl. $\preceq$.\\
Damit ist also $(\mathbb{T}, \preceq)$ ein vollständiger Verband.

\textbf{Fall 2:} $\infty \in B \land \exists b \in B : b \in \mathbb{R}_{\geq 0}$.\\
Da nun mindestens ein $m \in M, n \in \mathbb{N}$ existiert mit $m(n) \in \mathbb{R}_{\geq 0}$, also $m(n) < \infty$
können wir die größte untere Schranke analog zu Fall 1 finden. Offensichtlich muss die obere Schranke
dann die Funktion $f(n) = \infty$ für alle $n \in \mathbb{N}$ sein, damit folgende Bedingung erfüllt ist:
$$
    \forall m \in M : \forall n \in \mathbb{N} : m(n) \leq f(n)
    \,\Longleftrightarrow\, \forall m \in M : m \preceq f
$$
Weiter gilt $\infty \in B$, also $\exists m \in M, n \in \mathbb{N} : m(n) = \infty$, daher kann können wir keine
kleinere obere Schranke als $f$ wählen. Folglich lässt sich wieder eine kleinste obere Schranke von $M$
bzgl. $\preceq$ finden.

\textbf{Fall 3:} $B = \{\infty\}$.\\
In diesem Fall gilt offensichtlich, dass $f(n) = \infty$ sowohl Supremum als auch Infimum von $M$ bzgl. $\preceq$
sein muss.

Insgesamt lässt sich in allen Fällen eine kleinste obere sowie ein größte untere Schranke für jede Teilmenge
$M \subseteq \mathbb{T}$ finden. Folglich ist $(\mathbb{T}, \preceq)$ ein vollständiger Verband. $\hfill\square$

\textbf{b)}\\
Seien $t, t' \in \mathbb{T}$ mit $t \preceq t'$ gegeben. Es folgt für $n = 0$
$$
    (\Phi(t))(0) = 1 \leq 1 = (\Phi(t'))(0)
$$
sowie für $n \in \mathbb{N}$
$$
    \forall n \in \mathbb{N}: (\Phi(t))(n)
    = 2t\left(\bigg\lfloor\frac{n}{2}\bigg\rfloor\right)+n
    \quad\up{\leq}{$t \preceq t'$}\quad
    2t'\left(\bigg\lfloor\frac{n}{2}\bigg\rfloor\right)+n
    = (\Phi(t'))(n)
$$
Also gilt
$$
    \forall n \in \mathbb{N}_0 : (\Phi(t))(n) \leq (\Phi(t'))(n)
$$
Es folgt $\Phi(t) \preceq \Phi(t')$. Damit ist $\Phi$ nach Definition monoton bzgl. $\preceq$.

\textbf{c)}\\
Die Idee ist es, zu zeigen, dass für $f \in \mathbb{X}$ stets $\Psi(f(n)) \sqsubseteq f(n)$ gilt.
Denn da $\Psi$ monoton bzgl. $\sqsubseteq$ ist, und $(\mathbb{X}, \sqsubseteq)$ ein vollständiger
Verband ist folgt dann nach dem Prinzip der Fixpunktinduktion, dass $\Psi$ einen Fixpunkt $p$
mit $p \sqsubseteq f$ besitzt. Dann zeigt man, dass $T(n)$ eben dieser Fixpunkt $p$ ist, und
wenn die gegebene Halbordnung $\sqsubseteq$ etwa das Wachstum von Funktionen in einer
sinnvollen Weise vergleicht (wie z.B. $\preceq$ in den vorherigen Teilaufgaben), so
folgt dann aus $T(n) \sqsubseteq f(n)$, dass $T(n) \in \mathcal{O}(f(n))$ liegt.

\textbf{d)}\\
Sei $\Phi, \preceq, \mathbb{T}$ wie zuvor gegeben. Dann sei $2n \log n \in \mathbb{T}$.
Es gilt für alle $n \in \mathbb{N}$:
$$
    \Phi(2n\log n)
    \leq 2\left(2\frac{n}{2} \log \frac{n}{2}\right) + n
    = 2n\log\frac{n}{2} + n
    = 2n\log_2 n - 2n \log_2 2 + n
$$$$
    = 2n \log_2 n - n
    \leq 2n \log_2 n
$$
Folglich gilt auch $\Phi(2n \log n)\preceq 2n \log n$. Da $\preceq$ eine Halbordnung und
$\Phi$ monoton bzgl. dieser ist, folgt nach dem Satz der Fixpunktinduktion, dass $\Phi$ einen
Fixpunkt $p$ mit $p \preceq 2n\log n$ hat. Offensichtlich folgt nach Definition von $T(n)$,
dass $\Phi(T(n)) = T(N)$, also eben $T(n)$ dieser Fixpunkt $p$ ist.
Insgesamt gilt also $T(n) \in \mathcal{O}(2n\log n)$, da nach Definition von $\preceq$ gilt,
dass
$$
\forall n \in \mathbb{N} : T(n) \leq 2n\log n
$$
Damit existiert eine Konstante $c = 1 > 0$ und ein $n_0 \in \mathbb{N}$ mit $n_0 = 1$ sodass
die Bedingung für $T(n) \in \mathcal{O}(2n\log n)$ erfüllt ist.

\textbf{e)}\\
Wir führen Induktion über alle $n \in \mathbb{N}$.
Sei $n = 1$. Es gilt:
$$
    T(n) = 1 \leq 1^2 = n^2
$$
Also existiert ein $c = 1 > 0$ sodass $T(n) \leq c n^2$ ist.
Sei nun $n = 2$. Es gilt:
$$
    T(n)
    = 2 T\left(\bigg\lfloor \frac{n}{2}\bigg\rfloor\right) + n
    = 2 T(1) + 2
    = 4
    \leq n^2
$$
Also existiert wieder ein $c = 1 > 0$ sodass $T(n) \leq c n^2$ ist.

Sei nun ein beliebig aber festes $n \in \mathbb{N}$ gegeben, sodass $T(\lfloor \frac{n}{2} \rfloor) \leq cn^2$ für ein $c > 0$ (IV).
Es folgt:
$$
    T(n) = 2 T\left(\left\lfloor \frac{n}{2} \right\rfloor\right)+n
    \up{\leq}{IV} 2 \left(\frac{n}{2}\right)^2+ n
    = \frac{n^2}{2} + n
    \leq n^2 + n^2 = 2n^2
$$
Daher existiert ein $c = 2 > 0$ sodass $T(n) \leq cn^2$ gilt.

Insgesamt lässt sich nach Prinzip der vollständigen Induktion zu jedem $n \in \mathbb{N}$ ein $c > 0$
finden, sodass $T(n) \leq cn^2$ ist. Es folgt $T(n) \in \mathcal{O}(n^2).\hfill\square$

\textbf{f)}

Wir gehen analog zu d) vor. Also seien wieder $\Phi, \preceq, \mathbb{T}, T(n)$ wie zuvor gegeben.\\
Dann sei $n^2 \in \mathbb{T}$. Für $n = 1$ gilt:
$$
    \Phi(n^2) = 1 \leq n^2
$$
Sei nun $n \in \mathbb{N}, n\geq 2$. Es gilt für ein $a \in \mathbb{N}$ mit $n = 2+a$:
\begin{equation}
    \frac{n^2}{2} + n = \frac{(2+a)^2}{2} + 2+a = 4 + 2a +\frac{a^2}{2} \leq (2+a)^2 = n^2
\end{equation}
Ferner gilt:
$$
    \Phi(n^2) = 2 \left(\left\lfloor\frac{n}{2}\right\rfloor\right)^2 + n
    = \frac{n^2}{2} + n
    \ \up{\leq}{(1)}\ \frac{n^2}{2} + \frac{n^2}{2}
    = n^2
$$

Folglich gilt auch $\Phi(n^2)\preceq n^2$.
Analog zu d) folgt mit dem Satz der Fixpunktinduktion, dass\\
$T(n) \in \mathcal{O}(n^2)$.

\textbf{g)}

Die Methode der Fixpunktinduktion ist in dieser Aufgabe leicht gefallen, da wir schon
einen vollständigen Verband, sowie eine bzgl. der Halbordnung monotone Funktion auf der Menge
des Verbands gegeben hatten. Wäre dies nicht gegeben, so denke ich würde die Methode der Substitution
einfacher und schneller gehen.

\aufgabe{4}
\textbf{a)}

In allen Brüchen wird implizit abgerundet. Im Sinne der Lesbarkeit kennzeichnen wir dies im folgenden nicht
explizit.
\begin{center}
    \begin{forest} l sep+=2ex
        [$T(n)$ \\\hline
            $\mathbf{n}$\\
            ,align=c,draw
            [$T(\frac{n}{16})$ \\\hline
                $\mathbf{\frac{n}{16}}$\\
                ,align=c,draw
                [...]
                [...]
                [...]
                [...]
            ]
            [$T(\frac{n}{16})$ \\\hline
                $\mathbf{\frac{n}{16}}$\\
                ,align=c,draw
                [$T(\frac{n}{256})$ \\\hline
                    $\mathbf{\frac{n}{256}}$\\
                    ,align=c,draw
                    [\vdots
                        [...\ $T(1)$\ ...]
                    ]
                    [\vdots
                        [...\ $T(1)$\ ...]
                    ]
                    [\vdots
                        [...\ $T(1)$\ ...]
                    ]
                    [\vdots
                        [...\ $T(1)$\ ...]
                    ]
                ]
                [...]
                [...]
                [...]
            ]
            [$T(\frac{n}{16})$ \\\hline
                $\mathbf{\frac{n}{16}}$\\
                ,align=c,draw
                [...]
                [...]
                [...]
                [$T(\frac{n}{256})$ \\\hline
                    $\mathbf{\frac{n}{256}}$\\
                    ,align=c,draw
                    [\vdots
                        [...\ $T(1)$\ ...]
                    ]
                    [\vdots
                        [...\ $T(1)$\ ...]
                    ]
                    [\vdots
                        [...\ $T(1)$\ ...]
                    ]
                    [\vdots
                        [...\ $T(1)$\ ...]
                    ]
                ]
            ]
            [$T(\frac{n}{16})$ \\\hline
                $\mathbf{\frac{n}{16}}$\\
                ,align=c,draw
                [...]
                [...]
                [...]
                [...]
            ]
        ]
    \end{forest}
\end{center}

Der Baum hat $\log_{16}n$ Ebenen und entsprechend $4^{\log_{16}n} = n^{\log_{16}4} = \sqrt{n}$  Blätter.
Es folgt
$$
    T(n) = \sum_{i=0}^{\log_{16}n - 1} \left(\frac{4}{16}\right)^i n\ + \sqrt{n}\ 
    \approx\ \frac{4}{3}n + \sqrt{n} \in \mathcal{O}(n)
$$
\newpage
\textbf{b)}

Wir benutzen das Mastertheorem. Es ist $b = 4 \geq 1$, $c = 16 > 1$ und $f(n) = n$.
Weiter ist dann $E := \log_cb = \log_{16} 4 = 0.5$, also $n^E = \sqrt{n}$.
Der 3. Fall des Mastertheorems trifft hier zu, denn wir haben für $\varepsilon = 0.5 > 0$, dass
$$
    f(n) = n \in \Omega(n^{E+\varepsilon}) = \Omega(n)
$$
Weiter gilt
$$
    b\cdot f\left(\frac{n}{c}\right)
    = 4\cdot f\left(\frac{n}{16}\right)
    = \frac{n}{4}
    \leq \frac{n}{2}
    = \frac{1}{2} \cdot f(n)
$$
Also existiert ein $d = \frac{1}{2} < 1$ sodass $b\cdot f(\frac{n}{c}) \leq d\cdot f(n)$ sogar für
alle $n \in \mathbb{N}$ gilt. Nach dem Mastertheorem folgt nun, dass
$$
    T(n) \in \Theta(f(n)) = \Theta(n)
$$
Unsere Vermutung aus a) stimmt also und wurde sogar noch von unten eingegrenzt.
\end{document}
