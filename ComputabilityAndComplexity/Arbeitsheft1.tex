\documentclass[a4paper,graphics,11pt]{article}
\pagenumbering{arabic}

\usepackage[margin=1in]{geometry}
\usepackage[utf8]{inputenc}
\usepackage[T1]{fontenc}
\usepackage{lmodern}
\usepackage[ngerman]{babel}
\usepackage{amsmath, tabu}
\usepackage{amsthm}
\usepackage{amssymb}
\usepackage{complexity}
\usepackage{mathtools}
\usepackage{setspace}
\usepackage{graphicx,color,curves,epsf,float,rotating}
\usepackage{tasks}
\usepackage{delarray}
\setlength{\parindent}{0em}
\setlength{\parskip}{1em}

\newcommand{\up}[2]{\mathrel{\overset{\makebox[0pt]{\mbox{\normalfont\tiny #2}}}{#1}}}
\newcommand{\vect}[5]{\begin{pmatrix}#1\\#2\\#3\\#4\\#5\end{pmatrix}}
\newcommand{\eps}[0]{\varepsilon}
\newcommand{\godel}[1]{\langle #1 \rangle}
\newcommand{\Iff}[0]{\,\Longleftrightarrow\,}

\begin{document}

\begin{center}
    \LARGE \textbf{Lösungsvorschlag Arbeitsheft 1}
\end{center}

\section{Der Rice Trick}

\textbf{a)}

Zuerst baut man eine TM $M'$ aus $M$ und $M_2$, welche bei Eingabe $x \in \Sigma^*$ mithilfe der
Universellen TM dann $M$ bei Eingabe $\eps$ simuliert und darauf, falls dieser Vorgang terminiert, die TM $M_2$
bei Eingabe $x$ simuliert und dessen Ausgabe übernimmt. Da hier $\godel{M}, \godel{M_2}$ beim Bau von $M'$
schon feststehen, kann man diese als Konstanten in $\godel{M'}$ speichern. Dann ist $M'$ letztendlich nur
die Universelle TM, mit einem Unterprogramm, welches nach der ersten Simulation alle Bänder löscht und
die Simulation von $M_2$ auf $x$ vorbereitet.

Die TM $M''$ sei nun als 2-Band-TM aufgefasst, wobei man auf Band 1 eben $M_1$ auf der Eingabe simuliert,
und auf Band2 eben $M'$ auf der Eingabe parallel simuliert. Diese Parallelität kann mit einer Art
Produktkonstruktion der DFA's von $M_1$ und $M'$ geschehen, welche dann auf dem Zustandsraum $Q_{M_1} \times Q_{M'}$
arbeitet und eine entsprechend angepasste Übergangsfunktion besitzt.

Schließlich können wir $M^+$ als Simulation von $M''$ ansehen, wobei wir zwischen jedem Simulationsschritt
die Akzeptanz von $M_1$ und $M'$ überprüfen.

\strut

\textbf{b)}

Durch $\godel{M} \in H_\eps$ wird $M'$ stets terminieren. Wenn also die Eingabe $x \in \Sigma^*$ nicht in $L_1$
ist, so wird trotzdem nach endlicher Zeit noch $x \in L_2$ geprüft. Es gilt also
$$
    \godel{M} \in H_\eps \,\Longrightarrow\, L(M^+) = L_1 \cup L_2
$$

\textbf{c)}

Durch $\godel{M} \notin H_\eps$ wird $M'$ niemals dazu kommen, $x \in L_2$ für die Eingabe $x \in \Sigma^*$ zu
überprüfen. Es folgt
$$
    \godel{M} \notin H_\eps \,\Longrightarrow\, L(M^+) = L_1
$$

\strut

\textbf{d)}
Aus den beiden obigen Fällen folgt mit $L_1 = \varnothing$ gut und $L_2$ schlecht sofort, dass
$$
    \godel{M} \in H_\eps \,\Longrightarrow\, L(M^+) = L_1 \cup L_2 = L_2 \,\Longrightarrow\, \godel{M^+} \notin L_\mathcal{E}
$$
sowie dass
$$
    \godel{M} \notin H_\eps \,\Longrightarrow\, L(M^+) = L_1 = \varnothing \,\Longrightarrow\, \godel{M^+} \in L_\mathcal{E}
$$
Folglich akzeptiert $T(\mathcal{E})$ die Gödelnummer $\godel{M^+}$ \textbf{genau dann}, wenn $\godel{M} \notin H_\eps$.

\newpage

\textbf{e)}

Gäbe es eine solche TM $T(\mathcal{E})$, so könnte man mit dieser als Unterprogramm $H_\eps$ entscheiden, indem
man zu den festen $\godel{M_1},\godel{M_2}$ mit den beschriebenen Eigenschaften und gegebener Eingabe
$\godel{M}$ die TM $\godel{M^+}$ konstruiert und das Akzeptanzverhalten von $T(\mathcal{E})$ auf $\godel{M^+}$
invertiert.

\strut

\textbf{f)}

Was wir von den Sprachen $L_1,L_2$ benötigen, damit die Argumentation so bestehen kann, ist, dass
genau eine der Sprachen $L_1$ und $L_1 \cup L_2$ gut ist. Wenn also $L_1$ schlecht ist, so benötigen
wir nur eine gute Sprache $L_2$. Wenn wir nun $M^+$ zu diesen so wie zuvor konstruieren haben wir
analog zu $d)$, dass
$$
    \godel{M} \in H_\eps \,\Longleftrightarrow\, \godel{M^+} \in L_\mathcal{E}
$$
also dass wir wie in e) beschrieben $H_\eps$ entscheiden können (nur diesmal ohne das Akzeptanzverhalten von $T(\mathcal{E})$ zu invertieren).

\strut

\textbf{g)}

Dies ist analog zu d), da wenn $\godel{M} \notin H_\eps$, die TM $M'$ aus der Konstruktion von $M^+$ (siehe a))
niemals halten wird, also $M^+$ genau $L_1$ entscheidet. Damit $\godel{M^+} \in L_\mathcal{E}$, $T(\mathcal{E})$
akzeptiert $\godel{M^+}$.

\strut

\textbf{h)}

Ebenfalls analog zu d) und g), da wenn $\godel{M} \in H_\eps$ dann $M^+$ genau $L_1 \cup L_2 = L_2$ entscheidet,
also $\godel{M^+} \notin L_\mathcal{E}$ und $T(\mathcal{E})$ akzeptiert $\godel{M^+}$ nicht.

\strut

\textbf{i)}

Aus g) und h) folgt, dass für eine feste TM $A$ mit $\godel{A} \in L_\mathcal{E}$ nun
$$
    f : \Sigma^* \to \Sigma^*, w \mapsto \begin{cases}
        \godel{M^+} &, w = \godel{M} \text{ für eine TM $M$}\\
        \godel{A} &, w \text{ keine Gödelnummer}
    \end{cases}
$$
eine (berechenbare!) Reduktion $\overline{H_\eps} \leq L_\mathcal{E}$ darstellt. Denn wenn $w \in \Sigma^*$ keine Gödelnummer ist,
so ist schonmal $w \in \overline{H_\eps}$ und $f(w) = \godel{A} \in L_\mathcal{E}$. Ist $w = \godel{M}$ für
eine TM $M$, so ist nach g) und h) nun
$$
    f(w) = \godel{M^+} \in L_\mathcal{E} \iff w \in \overline{H_\eps}
$$
Damit haben wir also eine korrekte Reduktion $\overline{H_\eps} \leq L_\mathcal{E}$. Der Widerspruch ergibt sich,
durch die Annahme, dass $L(\mathcal{E})$ rekursiv aufzählbar ist. Denn dann wäre auch $\overline{H_\eps}$
rekursiv aufzählbar, und da nach VL schon $H_\eps$ rekursiv aufzählbar ist, wäre dann $H_\eps$ entscheidbar.

\newpage

\textbf{j)}
Die 8 nicht-rekursiv-aufzählbaren Mengen, für die das Werkzeug benutzbar ist:
\begin{enumerate}
    \item $\{\godel{M} \mid L(M) = \varnothing\}$ mit $\varnothing = L_1 \subseteq L_2 = \Sigma^*$
    \item $\{\godel{M} \mid \eps \notin L(M)\}$ mit $\varnothing = L_1 \subseteq L_2 = \Sigma^*$
    \item $\{\godel{M} \mid L(M) \text{ regulär}\}$ mit
        $\varnothing = L_1 \subseteq L_2 = \{0^n1^n \mid n \in \mathbb{N}\}$ kontextfrei also
        rek. aufzählbar
    \item $\{\godel{M} \mid L(M) \text{ nicht regulär}\}$ mit $\{0^n1^n \mid n \in \mathbb{N}\} = L_1 \subseteq L_2 = \Sigma^*$
    \item $\{\godel{M} \mid L(M) \text{ rekursiv}\}$ mit $\varnothing = L_1 \subseteq L_2 = H_\eps$
    \item $\{\godel{M} \mid L(M) \text{ nicht rekursiv}\}$ mit $H_\eps = L_1 \subseteq L_2 = \Sigma^*$
    \item $\{\godel{M} \mid |L(M)| = 1\}$ mit $\{0\} = L_1 \subseteq L_2 = \{0,1\}$
    \item $\{\godel{M} \mid |L(M)| \leq 3\}$ mit $\varnothing = L_1 \subseteq L_2 = \{0,1,00,11\}$
\end{enumerate}

\strut

\textbf{k)}

Das ist analog zu d), f), g) und h). Mit $\godel{M} \in H_\eps$ folgt $L(\godel{M^+}) = L_1 \cup L_2 = L_2$,
also $\godel{M^+} \in L_\mathcal{E}$ da $L_2$ nun gut ist. Ebenso ist mit $\godel{M} \notin H_\eps$ dann
$L(\godel{M^+}) = L_1$, also $\godel{M^+} \notin L_\mathcal{E}$, da $L_1$ hier schlecht.
Damit folgt die Behauptung.

\strut

\textbf{l)}

Mit analoger Argumentation zu i) erhält man eine Reduktion $H_\eps \leq L_\mathcal{E}$. Da wir bereits
aus der VL wissen, dass $H_\eps$ rekursiv aufzählbar ist, gibt es hier keinen Widerspruch.

\strut

\textbf{m)}

Wir zeigen die rekursive Aufzählbarkeit von $L := \{\godel{M} \mid L(M) \neq \varnothing\}$.

Wie im Beweis dass semi-entscheidbare Sprachen rekursiv aufzählbar sind (VL 6) können wir
zu einer Eingabe nach einem Syntaxcheck in ''Runden'' arbeiten; Da die Eingabe nun in der Form $\godel{M}$ ist,
können wir in der $i$-ten Runde $M$ auf den ersten $i$ Worten der kanonischen Aufzählung von $\{0,1\}^*$
für jeweils $i$ Schritte simulieren. Dies führen wir für jedes $i \in \mathbb{N}$ durch und akzeptieren
sobald eines der Worte von $M$ akzeptiert wird.

Wenn nun $L(M) \neq \varnothing$, so existieren $w \in \{0,1\}^*$ und $j,k \in \mathbb{N}$ sodass
$w = w_j$ und $w$ von $M$ in $k$ Schritten akzeptiert wird. Damit wird $w$ von $M$ in der $i = \max(j,k)$-ten
Runde akzeptiert und wir akzeptieren $\godel{M}$.

Andererseits wird es kein Wort geben welches von $M$ akzeptiert wird, sodass wir Berechnung für ewig
weiterläuft, also $\godel{M}$ auch nicht akzeptiert wird.

Damit ist also $L$ rekursiv aufzählbar. Die gesuchten Sprachen sind bspw. $L_1 = \varnothing, L_2 = \{0\}$.

\newpage

\section{Ein weiterer Rice Trick}

\textbf{a)}

Ähnlich wie in der a) vom letzten Kapitel baut man eine Art Produktkonstruktion welche auf 2 Bändern
parallel arbeitet. Dabei wird auf Band 1 eine Universelle TM, welche $M_4$ auf der Eingabe $x$ simuliert,
ausgeführt und auf Band 2 eine modifizierte Universelle TM, welche $M$ für $|x|$ Schritte auf $\eps$ simuliert,
ausgeführt. Da wir nicht frühzeitig abbrechen müssen, können wir hier akzeptieren, sobald beide ''Unterprogramme''
akzeptiert haben (wobei die 2. Berechnung eben akzeptiert, wenn der Endzustand von $M$ nicht erreicht wird).

\strut

\textbf{b)}

Im Fall $\godel{M} \notin H_\eps$ wird die zweite Berechnung nie den Endzustand von $M$ erreichen, sodass
wir nur die Akzeptanz der ersten Berechnung, welche $x \in L_4$ überprüft, benötigen, um zu akzeptieren.
Es gilt also
$$
    \godel{M} \notin H_\eps \,\Longrightarrow\, L(M^{++}) = L_4
$$

\textbf{c)}

Im Fall $\godel{M} \in H_\eps$ wird $M$ auf $\eps$ in $k \in \mathbb{N}$ Schritten halten.
Folglich haben wir für Eingaben $x \in \Sigma^*$ mit $|x| < k$ das Szenario b) erhalten,
und für die restlichen Eingaben $x$ mit $|x| \geq k$ wird $M^{++}$ verwerfen. Es folgt
$$
    \godel{M} \in H_\eps \,\Longrightarrow\, L(M^{++}) = L_4 \cap \bigcup_{i=0}^{k-1} \Sigma^i
    = \{x \in L_4 : |x| < k\}
$$
wobei $k = \min\{n \in \mathbb{N} \mid M \text{ hält auf $\eps$ in $n$ Schritten}\}$.
Da $\Sigma$ stets endlich ist kann es nur endlich viele Wörter mit höchstens Länge $k$ geben, sodass
$L(M^{++})$ eine endliche Teilmenge von $L_4$ darstellt und damit nach dem gegebenen Szenario nicht gut ist.

\strut

\textbf{d)}

Dies folgt sofort aus b):
$$
    \godel{M} \notin H_\eps \,\Longrightarrow\, L(M^{++}) = L_4 \,\Longrightarrow\, \godel{M^{++}} \in L_\mathcal{E}
$$
Also akzeptiert $T(\mathcal{E})$ auch $\godel{M^{++}}$.

\strut

\textbf{e)}

Analog zu d) folgt dies aus c):
$$
    \godel{M} \in H_\eps \,\Longrightarrow\, L(M^{++}) \text{ endliche Teilmenge von $L_4$}
    \,\Longrightarrow\, \godel{M^{++}} \notin L_\mathcal{E}
$$
Also wird $\godel{M^{++}}$ nicht von $T(\mathcal{E})$ akzeptiert.

\newpage

\textbf{f)}

Wie in Aufgabe i) des letzten Kapitels bekommt man nun eine Reduktion $\overline{H_\eps} \leq L_\mathcal{E}$,
woraus mit der Annahme, dass $L_\mathcal{E}$ rekursiv aufzählbar ist, die Entscheidbarkeit von $H_\eps$
folgt. Widerspruch.

\strut

\textbf{g)}

Die nicht-rekursiv-aufzählbaren Mengen, für die das Werkzeug benutzbar ist:

\begin{itemize}
    \item $\{\godel{M} \mid L(M) = \{0,1\}^*\}$
    \item $\{\godel{M} \mid L(M) \text{ enthält alle Worte in $\{0,1\}^*$ mit gerader Länge}\}$
    \item $\{\godel{M} \mid L(M) \text{ ist nicht regulär}\}$ da endliche Mengen stets regulär.
    \item $\{\godel{M} \mid L(M) \text{ ist nicht rekursiv}\}$ da endliche Mengen stets rekursiv.
    \item $\{\godel{M} \mid |L(M)| = \infty\}$
\end{itemize}

\strut

\textbf{h)}

Übrig auf der Liste sind
\begin{enumerate}
    \item $\{\godel{M} \mid L(M) \neq \varnothing\}$ 
    \item $\{\godel{M} \mid \eps \in L(M)\}$
    \item $\{\godel{M} \mid 11101 \in L(M)\}$
    \item $\{\godel{M} \mid |L(M)| \geq 3\}$
\end{enumerate}

Die erste Menge wurde im letzten Kapitel, Aufgabe m) als rekursiv aufzählbar bewiesen.

Mengen 2 und 3 Lassen sich trivialerweise semi-entscheiden, indem wir einfach nach einem Syntaxcheck
die gegebene TM auf $\eps$ bzw. 11101 simulieren und die Ausgabe übernehmen.

Menge 4 lässt sich analog zu 1 entscheiden, nur dass wir erst akzeptieren, sobald mindestens 3 Wörter
akzeptiert wurden.

Damit sind alle übrig-gebliebenen Mengen rekursiv-aufzählbar.

\newpage

\section{Unentscheidbarkeit für context-freie Grammatiken}

Wir nehmen im folgenden an, dass die Definition der Grammatiken $S_i$ fehlerhaft sind, und eigentlich die folgenden
Produktionsregelen gemeint sind:
$$
    S_1 \to d_1[S_1]x_1 \mid d_2[S_1]x_2 \mid d_3[S_1]x_3 \mid \cdots \mid d_k[S_1]x_k
$$$$
    S_2 \to d_1[S_2]y_1 \mid d_2[S_2]y_2 \mid d_3[S_2]y_3 \mid \cdots \mid d_k[S_2]y_k
$$
wobei hier die standard EBNF-Schreibweise verwendet wird, dass $X \to \alpha[\beta]\gamma$ als optionales
$\beta$, also $X \to \alpha\beta\gamma \mid \alpha\gamma$ zu verstehen ist.

\textbf{a)}

Halt nen DPDA schreiben, ich kehre nicht.

\strut

\textbf{b)}

Deterministisch-kontextfreie Sprachen sind unter Komplementbildung abgeschlossen. Folglich sind
$\overline{L(G_1)}$ und $\overline{L(G_2)}$ deterministisch-kontextfreie Sprachen und es gibt einen Algorithmus
der Grammatiken $G'_1$ und $G'_2$ berechnet, sodass $L(G'_i) = \overline{L(G_i)}$ für $i=1,2$.

\strut

\textbf{c)}

Kontextfreie Sprachen sind unter Vereinigung abgeschlossen, und deterministisch-kontextfreie Sprachen sind eine
echte Unterklasse der kontextfreien Sprachen. Folglich sind
$$
    L_3 := L(G_1) \cup L(G'_2)
    \qquad\text{und}\qquad
    L_4 := L(G_1') \cup L(G_2)
$$
beides kontextfreie Sprachen. Damit existieren kontextfreie Grammatiken $G_i$ mit $L(G_i) = L_i$ für $i=3,4$,
welche durch einen Algorithmus berechnet werden können. (Bspw neues Startsymbol und Auswahl zwischen
Startsymbolen der beiden Grammatiken; $S_{new} \to S_{G_1} \mid S_{G'_2}$)

\strut

\textbf{d)}

Angenommen die gegebene PCP-Instanz hat einen Lösung $i_1,\cdots,i_n \in [1,k]_\mathbb{N}$.\\
Dann haben wir $x_{i_1}\cdots x_{i_n} = y_{i_1} \cdots y_{i_n}$. Folglich kann man aus $S_1$ und $S_2$
das selbe Wort
$$
    S_j
    \vdash d_{i_1}S_jx_{i_1}
    \vdash d_{i_1}d_{i_2}S_jx_{i_2}x_{i_1}\quad
    \vdash^*\quad d_{i_1}\cdots d_{i_k}S_jx_{i_k}\cdots x_{i_1}
    \quad=\quad d_{i_1}\cdots d_{i_k}S_j y_{i_k}\cdots y_{i_1}
$$
ableiten, wobei $j=1,2$. Damit ist $L(G_1) \cap L(G_2) \neq \varnothing$.

Sei nun $L(G_1) \cap L(G_2) \neq \varnothing$. Dann existiert ein Wort $w \in L(G_1) \cap L(G_2)$.
Per Definition von $G_1,G_2$ ist dann
$w = d_{i_1}\cdots d_{i_n}x_{i_n}\cdots x_{i_1} = d_{i_1}\cdots d_{i_n}y_{i_n}\cdots y_{i_1}$ für
$i_1,\cdots,i_n \in [1,k]_\mathbb{N}$. Damit ist dann $i_1,\cdots,i_n$ eine Lösung der PCP-Instanz.

Folglich ist es unentscheidbar, ob zwei gegebene kontextfreie Sprachen leeren Schnitt haben, da
man sonst das PCP entscheiden könnte (Konstruktionen der Grammatiken sind berechenbar).

\newpage

\textbf{e)}

Angenommen es gilt $L(G_1) \cap L(G_2) \neq \varnothing$. Das ist nach d) äquivalent dazu, dass
zu $w \in L(G_1) \cap L(G_2)$ eine Lösung $i = i_1\cdots,i_n$ der gegebenen PCP-Instanz existiert.
Insbesondere ist aber dann auch
$$
    i^j
    := \underbrace{i,i,\cdots,i}_{j \text{ mal}}
    := \underbrace{i_1,\cdots,i_n,\quad\cdots\quad,i_1,\cdots,i_n}_{j \text{ mal } i_1,\cdots,i_n}
$$
eine Lösung für jedes $j \in \mathbb{N}$. Folglich haben PCP-Instanzen unendlich viele Lösungen
sobald sie eine Lösung haben. Ferner gibt es zu jeder dieser Lösungen genau 1 Wort in $L(G_1) \cap L(G_2)$:
$$
    i^j
    \quad\text{korrespondiert zu}\quad
    D_i^jX_i^j \in L(G_1) \cap L(G_2)
$$
wobei $D_i^j := (d_{i_1}d_{i_2}\cdots d_{i_n})^j
:= \underbrace{d_{i_1},\cdots,d_{i_n},\quad\cdots\quad,d_{i_1},\cdots,d_{i_n}}_{j \text{ mal } d_{i_1},\cdots,d_{i_n}}$
und analoges für\\
$X_i^j := (x_{i_n},x_{i_{n-1}},\cdots,x_{i_1})^j = (y_{i_n},y_{i_{n-1}},\cdots,y_{i_1})^j$ gilt.
Also ist
$$
    L(G_1) \cap L(G_2) \neq \varnothing \iff |L(G_1) \cap L(G_2)| = \infty
$$
Damit ist es unentscheidbar, ob zwei gegebene kontextfreie Grammatiken unendlich viele gemeinsame Worte erzeugen,
da wir sonst das Schnittproblem aus d) entscheiden könnten.

\strut

\textbf{f)}

Dies ist simple Mengenlehre; für Mengen $A,B$ gilt stets:
$$
    \varnothing
    = A \cap B
    = A \cap \overline{\overline{B}}
    = A \setminus \overline{B}
    \quad\iff\quad
    A \subseteq \overline{B}
$$
Damit folgt aus d), dass das Inklusionsproblem für kontextfreie Sprachen unentscheidbar ist.

\strut

\textbf{g)}

Wieder simple Mengenlehre; Auch aus f) folgt für Mengen $A,B$, dass
$$
    A \cup \overline{B} = \overline{B}
    \quad\iff\quad
    A \subseteq \overline{B}
    \quad\iff\quad
    A \cap B = \varnothing
$$
Damit folgt aus d), dass das Äquivalenzproblem für kontextfreie Sprachen unentscheidbar ist.

\strut

\textbf{h)}

Wieder simple Mengenlehre; für Mengen $A,B \subseteq \Omega$ gilt stets:
$$
    \Omega = \overline{A} \cup \overline{B}
    \qquad\iff\qquad
    \overline{\Omega} = \overline{\overline{A} \cup \overline{B}} = A \cap B
$$
Da in unserem Beispiel $\Omega = \Sigma^*$ und daher $\overline{\Omega} = \varnothing$ folgt die Behauptung
wieder aus d).

\newpage

\textbf{i)}

Eine Solche Grammatik könnte wie folgt berechnet konstruiert werden:
$$
    S_5 \to S' \mid S''
    \qquad
    S' \to S_{\Sigma^*}\$S_{L_0}
    \qquad
    S'' \to S_{L_4}\$S_{\Sigma^*}
$$
wobei $S_{\Sigma^*}, S_{L_0}, S_{L_4}$ die Startsymbole der kontextfreien Grammatiken für $\Sigma^*, L_0$ und $L_4$
sind. Da sich $G_4$ mit $L(G_4) = L_4$ nach a),b),c) aus der gegebenen PCP-Instanz berechnen lässt,
ist also auch $G_5$ eine berechenbare kontextfreie Grammatik.

\strut

\textbf{j)}

Da $L_0 \subseteq \Sigma^*$ folgt sofort
$$
    L_4 = L(G_4) = \Sigma^*
    \quad\Longrightarrow\quad L(G_5) = \Sigma^*\$L_0 \cup L_4\$\Sigma^* = \Sigma^*\$L_0 \cup \Sigma^*\$\Sigma^*
    = \Sigma^*\$\Sigma^*
$$
wobei letzteres trivialerweise regulär ist, da es schon als regulärer Ausdruck gegeben ist.

Sei nun $\Sigma^* \setminus L_4 \neq \varnothing$. Dann existiert ein $p \in \Sigma^* \setminus L_4$.
Angenommen $L_5$ ist regulär. Dann gibt es einen DFA $D$, welcher $L_5$ entscheidet.
Hat man nun ein Wort $w \in \Sigma^*$ gegeben, so können wir das Wort $p\$w$ dem DFA $D$ übergeben
und damit entscheiden, ob $w \in L_0$. Folglich gäbe es einen DFA welcher $L_0$ entscheidet, was
die Regularität von $L_0$ zeigen würde, Widerspruch.

\strut

\textbf{k)}

Aus h) und j) folgt nun die Unentscheidbarkeit des Regularitätsproblems für kontextfreie Grammatiken;
Könnten wir die Regularität von $L_5$ entscheiden, so könnte man (durch invertieren des Ergebnisses)
entscheiden, ob $L_4 = \Sigma^*$, was nach h) unentscheidbar ist.



\newpage



\section{Das zehnte Hilbert'sche Problem}

\textbf{a)}

Siehe HA 7.1. Man benutzt zu einer Instanz $p \in \mathbb{Z}[x_1,\cdots,x_k]$ dann
$$
    f(p(x_1,\cdots,x_k))
    := p'(x_1,x'_1,\cdots,x_k,x'_k)
    := p(x_1 - x'_1, \cdots, x_k - x'_k)
$$
Da $\forall z \in \mathbb{Z}: \exists n,m \in \mathbb{N} : z = n-m$ ist $f$ eine funktionierende Reduktion.

\strut

\textbf{b)}

Zu einem gegebenen Polynom $p \in \mathbb{Z}[x_1,\cdots,x_n]$ konstruieren wir das Polynom
$$
    q = \prod_{w \in \{0,1\}^n} p_w
    \qquad\text{mit}\qquad
    p_w(x_1,\cdots,x_n) := p(x_1+w_1, x_2+w_2, \cdots, x_n+w_n)
$$
Ist nun $a \in \mathbb{N}^n$ mit $p(a) = 0$, so gilt $p_w(b) = 0$ wobei $w_i := \begin{cases}0 &, a_i\text{ gerade}\\
1 &, a_i \text{ ungerade}\end{cases}$ und $b_i := a_i - w_i$. Offensichtlich sind alle $b_i \in \mathbb{N}$
gerade, und $b$ eine Nullstelle von $q$.

Ist hingegen $b \in \mathbb{N}^n$ mit $q(b) = 0$ und $b_i$ gerade, so existiert ein $w \in \{0,1\}^n$ mit $p_w(b) = 0$.
Definiere dann $a_i := b_i + w_i \in \mathbb{N}$, dann gilt $p(a) = 0$. Folglich gilt
$$
    \godel{p} \in \text{Dioph}(\mathbb{N}) \iff \godel{q} \in \text{Dioph}(\mathbb{N}_g)
$$
sodass wir (mit noch einem Syntaxcheck) eine Reduktion Dioph($\mathbb{N}$) $\leq$ Dioph($\mathbb{N}_g$) haben.
Da Dioph($\mathbb{N}$) unentscheidbar, folgt dies nun auch für Dioph($\mathbb{N}_g$).

\strut

\textbf{c)}

Zu einem gegebenen Polynom $p \in \mathbb{Z}[x_1,\cdots,x_n]$ konstruieren wir das Polynom
$$
    q(x_1,\cdots,x_n) := p(x_1 - 1,\cdots,x_n-1)
$$
Ist nun $a \in \mathbb{N}^n$ mit $a_i$ gerade und $p(a) = 0$, so ist $q(b) = 0$ für $b_i := a_i+1 \in \mathbb{N}$ ungerade.

Analog ist zu $b \in \mathbb{N}^n$ mit $b_i$ ungerade und $q(b) = 0$ dann $p(a) = 0$ für $a_i := b_i - 1 \in \mathbb{N}$ gerade.
Folglich gilt
$$
    \godel{p} \in \text{Dioph}(\mathbb{N}_g) \iff \godel{q} \in \text{Dioph}(\mathbb{N}_u)
$$
sodass wir (mit noch einem Syntaxcheck) eine Reduktion Dioph($\mathbb{N}_g$) $\leq$ Dioph($\mathbb{N}_u$) haben.
Da Dioph($\mathbb{N}_g$) unentscheidbar, folgt dies nun auch für Dioph($\mathbb{N}_u$).

\newpage

\textbf{d)}

Sei also $f : \Sigma^* \to \Sigma^*$ eine Abbildung, welche Müll auf Müll abbildet. Zu einem korrekt-kodiertem
Polynom $p \in \mathbb{Z}[x_1,\cdots,x_k]$ definieren wir
$$
    f(p(x_1,\cdots,x_k)) := p'(x_{1,1},x_{1,2},x_{1,3},x_{1,4},\cdots,x_{k,1},x_{k,2},x_{k,3},x_{k,4})
$$
wobei
$$
    p'(x_{1,1},x_{1,2},x_{1,3},x_{1,4},\cdots,x_{k,1},x_{k,2},x_{k,3},x_{k,4})
    := p(\sum_{i=1}^{4} x_{1,i}^2, \cdots ,\sum_{i=1}^{4} x_{k,i}^2)
$$
Offensichtlich ist $p'$ ebenfalls ein Polynom und $f$ ist berechenbar. Falls $(a_1,\cdots,a_k) \in \mathbb{N}^k$
eine Nullstelle von $p$ ist, so gilt nach Lagrange, dass $\forall a_i: \exists b_{1,1},b_{1,2},b_{1,3},b_{1,4} \in \mathbb{N}: \sum_{i=1}^{4} b_{1,i}^i = a_i$.
Damit ist dann $(b_{1,1},b_{1,2},b_{1,3},b_{1,4},\cdots,b_{k,1},b_{k,2},b_{k,3},b_{k,4}) \in \mathbb{Z}^{4k}$
eine Nullstelle von $p'$.

Für die Rückrichtung sei nun
$(b_{1,1},b_{1,2},b_{1,3},b_{1,4},\cdots,b_{k,1},b_{k,2},b_{k,3},b_{k,4}) \in \mathbb{Z}^{4k}$
eine Nullstelle von $p'$. Dann ist zu $a_i := \sum_{j=1}^{4} b_{i,j}^2 \in \mathbb{N}$ für $i\in[1,k]_{\mathbb{N}}$ nun
$(a_1,\cdots,a_k) \in \mathbb{N}^k$ eine Nullstelle von $p$.

\strut

\textbf{e)}

Zu $q_1,\cdots,q_k \in \mathbb{Z}[x_1,\cdots,x_n]$ gilt
$$
    \forall i \in [1,k]_{\mathbb{N}}: q_i(x) = 0
    \qquad\iff\qquad
    \underbrace{\sum_{i=1}^{k} q_i(x)^2}_{\in \mathbb{Z}[x_1,\cdots,x_n]} = 0
$$

\textbf{f)}

Wir gehen systematisch vor und starten mit dem gegebenen Gleichungssystem $p(x) = 0$.
\begin{enumerate}
    \item Solange es Gleichungen $g(x) = a$ mit $g(x) = q(x) + r(x)$ mit
        $\deg(q) > 2, 0 \leq \deg(r) \leq 2$ gibt, ersetze die Gleichung $g(x) = a$ durch
        $$
            q(x) = b \qquad r(x) = c \qquad b+c = a
        $$
    \item Solange es Gleichungen $g(x) = a$ mit $g(x) = q(x)\cdot r(x)$ mit
        $\deg(q) \geq 2, \deg(r) = 1$ gibt, ersetze die Gleichung $g(x) = a$ durch
        $$
            q(x) = b \qquad r(x) = c \qquad bc = a
        $$
    \item Ersetze alle Gleichungen der Form $g(x) = a$ durch $g(x) - a = 0$.
\end{enumerate}
Beispiel: $p \in \mathbb{Z}[x,y,z]$ mit $p(x,y,z) = 4x^2y-yz^2+1$. Man erhält nach ausführen der Schritte:

\begin{minipage}{.3\textwidth}
$4x^2 - e = 0$\\
$y-f = 0$\\
$ef -a = 0$\\
$z^2 - g = 0$\\
$-y -h = 0$\\
$gh - c = 0$\\
$1 - d = 0$\\
$c+d - b = 0$\\
$a+b = 0$
\end{minipage}
\begin{minipage}{.7\textwidth}
    Da dies alles Äquivalenzumformungen waren, stimmen die Lösungsmengen
    der ursprünglichen Gleichung und des Gleichungssystems überein.
    Im Beispiel haben wir unter anderem:
    $$
        x = 0\qquad
        y= 1\qquad
        z = 1
    $$
    Bzw im Gleichungssystem
    $$
        x = a = b = e = 0 \qquad
        y = z = d = f = g = 1\qquad
        c = h = -1
    $$
\end{minipage}

\textbf{g)}

Sei $p \in \mathbb{Z}[x_1,\cdots,x_k]$ also ein gegebenes Polynom.

\textbf{Fall 1:} $p(x)^2$ hat nur positive Koeffizienten. Es gilt
$$
    \exists a \in \mathbb{Z}^k : p(a) = 0 \iff p(a)^2 = 0 \iff p(a)+1 \not> 1
$$
da ja stets $p(x)^2 \geq 0$. In diesem Fall hätten wir also 2 Polynome mit nur positiven koeffizienten,
$p_2(x) := p(x)^2 + 1, p_1(x) := 1$ gefunden.

\textbf{Fall 2:} $p(x)^2$ lässt sich schreiben als $p(x)^2 = p_2(x) - p_1(x)$ für
$p_2,p_1$ polynome mit positiven ganzzahligen Koeffizienten. Da stets $p_2(x) \geq p_1(x)$ gilt nun
$$
    \exists a \in \mathbb{Z}^k: p(a) = 0
    \iff
    p(a)^2 = 0
    \iff
    p_2(a) \not> p_1(a)
$$

Offensichtlich sind $p_1,p_2$ aus $p$ berechenbar. Damit könnten wir also entscheiden ob $p$ eine ganzzahlige
Nullstelle hat; falls $\forall x \in \mathbb{Z}^k : p_2(x) > p_1(x)$ so ist $\godel{p} \notin$ Dioph,
andernfalls ist $\godel{p} \in$ Dioph. Folglich kann das gegebene Problem nicht entscheidbar sein.

Beispiel für univariate: $p,p_1,p_2 \in \mathbb{Z}[x]$ mit $p(x) = -x^2+2x+3$ also $p(x)^2 = x^4-4x^3-2x^2+12x+9$.
Dann sind $p_2(x) = x^4+12x+9$ und $p_1(x) = 4x^3+2x^2$. Und wir haben
$$
    \{x \in \mathbb{Z} \mid p_2(x) \not> p_1(x)\} = \{-1,3\} = \{x \in \mathbb{Z} \mid p(x) = 0\}
$$

\end{document}
