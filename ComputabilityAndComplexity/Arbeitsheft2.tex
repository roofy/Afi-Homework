\documentclass[a4paper,graphics,11pt]{article}
\pagenumbering{arabic}

\usepackage[margin=1in]{geometry}
\usepackage[utf8]{inputenc}
\usepackage[T1]{fontenc}
\usepackage{lmodern}
\usepackage[ngerman]{babel}
\usepackage{amsmath, tabu}
\usepackage{amsthm}
\usepackage{amssymb}
\usepackage{complexity}
\usepackage{mathtools}
\usepackage{setspace}
\usepackage{graphicx,color,curves,epsf,float,rotating}
\usepackage{tasks}
\usepackage{delarray}
\usepackage{tikz}

\setlength{\parindent}{0em}
\setlength{\parskip}{1em}

\newcommand{\up}[2]{\mathrel{\overset{\makebox[0pt]{\mbox{\normalfont\tiny #2}}}{#1}}}
\newcommand{\vect}[5]{\begin{pmatrix}#1\\#2\\#3\\#4\\#5\end{pmatrix}}
\newcommand{\eps}[0]{\varepsilon}
\newcommand{\godel}[1]{\langle #1 \rangle}
\newcommand{\Iff}[0]{\,\Longleftrightarrow\,}

\begin{document}

\begin{center}
    \LARGE \textbf{Lösungsvorschlag Arbeitsheft 2}
\end{center}

\section{Kern-Mengen in gerichteten Graphen}

Eine Kernmenge $K \subseteq V$ eines gerichteten Graphen $G = (V,E)$ erfüllt folgende Eigenschaften:
\begin{equation}
    K^2 \cap E = \varnothing
    \qquad\text{bzw.}\qquad
    \forall k \in K: Succ(k) \cap K = \varnothing
\end{equation}
\begin{equation}
    \forall v \in V \setminus K: Succ(v) \cap K \neq \varnothing
\end{equation}
Dabei ist $Succ(v) := \{w \in V: (v,w) \in E\}$ die Menge Nachfolger eines Knotens.


\textbf{a)}

Sei also $n \in \mathbb{N}$, $V = [1,n], E = \{(i,j) \in V^2 \mid i \neq j\}$ und $G = (V,E)$ gerichtet.\\
Setze $K = \{1\}$. Es gilt:
$$
    (1,1) \notin E \,\Longrightarrow\, (1)
$$
$$
    (V\setminus K) \times \{1\} \subset E \,\Longrightarrow\, (2)
$$
Folglich erfüllt $K$ beide Eigenschaften einer Kernmenge.

\strut

\textbf{b)}

Im folgenden Schreiben wir $[a]_n$ anstatt $k \text{ (mod $n$)}$ für Lesbarkeit.

Sei $C_n := ([1,n], \{(i, [i+1]_n) \mid i \in [1,n]\})$ der gerichtete Kreis mit $n$ Knoten.

Sei nun $n = 2\ell \geq 3$ für $\ell \in \mathbb{N}$. Setze $K := 2[1,\ell] = \{2,4,6,\cdots,2\ell\} \subset V_{C_n}$. Dann gilt:
$$
    \forall k \in K: Succ(k) = \{(k, [k+1]_n)\} \not\subset K^2 \,\Longrightarrow\, (1)
$$
$$
    \forall v \in V_{C_n}\setminus K: Succ(v) = \{(v, v+1)\} \in V_{C_n} \times K \,\Longrightarrow\, (2)
$$
Damit ist $K$ eine Kernmenge von $C_n$.

Sei nun $n = 2\ell+1 \geq 3$ für $\ell \in \mathbb{N}$. Angenommen $K \subseteq V_{C_n}$ wäre eine Kernmenge.

Offensichtlich muss dann $K \neq \varnothing$. Sei nun $v \in V_{C_n}$. Dann:
$$
    v \in K \,\Longrightarrow\, (1) \land Succ(v) = \{[v+1]_n\} \,\Longrightarrow\, [v+1]_n \notin K
$$
$$
    v \notin K \,\Longrightarrow\, (2) \land Succ(v) = \{[v+1]_n\} \,\Longrightarrow\, [v+1]_n \in K
$$
Sei nun $k \in K$. Wenn man obige Resultate endlich oft iteriert anwendet, so erhält man
$$
    k \in K
    \,\Longrightarrow\, [k+1]_n \notin K
    \,\Longrightarrow\, [k+2]_n \in K
    \,\Longrightarrow\, \cdots
    \,\Longrightarrow\, [k+2\ell]_n \in K
$$
Jedoch ist $[k+2\ell]_n = [k-1]_n$. Ferner gilt $Succ([k-1]_n) = \{k\}$, damit
ist aber (1) verletzt, und $K$ kann keine Kernmenge sein.

Insgesamt haben genau die gerichteten Kreise mit einer geraden Anzahl an Knoten, $C_{2\ell}, \ell \in \mathbb{N}$
eine Kernmenge.

\newpage

\textbf{c)}

Wir beweisen dies durch Angabe eines Algorithmus und dessen Korrektheitsbeweis.

Der Algorithmus wird die Knoten des Baumes in $V$ färben mit $c: V \to \{0,1,2\}$, wobei $0$ für nicht gefärbt steht.
$c(v) = 1$ soll dabei $v \in K$ bedeuten, und $c(v) = 2$ dann $v \notin K$, wobei $K$ die gesuchte Kernmenge ist.

\begin{enumerate}
    \item Solange $c^{-1}(0) \neq \varnothing$, es also ungefärbte Knoten gibt:
        \begin{enumerate}
            \item Setze $c(k) := 1$ für alle $k \in \{v \in c^{-1}(0) \mid Succ(v) = \varnothing\}$.\\
                Knoten ohne Nachfolger müssen in $K$ liegen.\\

            \item Setze $c(v) := 2$ für alle $v \in c^{-1}(0)$ mit $Succ(v) \cap c^{-1}(1) \neq \varnothing$.\\
                Knoten mit Nachfolgern in $K$ können wegen (1) nicht in $K$ sein.\\

            \item Setze $c(k) := 1$ für $k \in \{v \in c^{-1}(0) \mid \forall w \in Succ(v): c(w) = 2\}$.\\
                Knoten die nur Nachfolger haben, welche (schon festgelegt) nicht in $K$ liegen, dürfen in $K$ liegen.
        \end{enumerate}
\end{enumerate}

Wir zeigen dass zu jeder Zeit $c^{-1}(1)$ eine Kernmenge von $c^{-1}(\{1,2\})$ bildet.
Ferner sagen wir, dass $\varnothing$ eine korrekte Kernmenge von $\varnothing$ ist.

Sei nun $c^{-1}(1)$ eine Korrekte Kernmenge von $c^{-1}(\{1,2\})$.\\
Durch anwenden von (a) werden beide Eigenschaften (1) und (2)
einer Kernmenge erhalten, denn:
\begin{itemize}
    \item Für (1): Sei $k \in V$ einer der gerade gefärbten Knoten, also $c(k) = 1$ und $Succ(k) = \varnothing$.
        Wegen letzterem, gilt schonmal $Succ(k) \cap K = \varnothing$, also kann $k$ nicht (1) verletzen.

        Betrachte nun $v \in \{w \in c^{-1}(\{1,2\}) \mid k \in Succ(w)\}$ insofern nichtleer. Wäre $c(v) = 1$, so müsste
        $v$ durch die Schritte $(a)$ oder $(c)$ des Algorithmus gefärbt worden sein.
        Wegen $Succ(v) \neq \varnothing$ kann es nicht durch $(a)$ gewesen sein. Da $k$ bis vor kurzem
        ungefärbt und nun mit 1, also insbesondere nie $c(k) = 2$ gegolten hat, kann dies auch nicht
        durch Schritt $(c)$ passiert sein.
        Folglich muss $c(v) = 2$, und damit ist Bedingung (1) nicht verletzt.

    \item Für (2) spielt das 1-färben (hinzufügen von Knoten zur Kernmenge) keine Rolle.
\end{itemize}
Folglich ist nach anwenden von (a) $c^{-1}(1)$ immernoch eine korrekte Kernmenge von $c^{-1}(\{1,2\})$.

\newpage

Wir untersuchen nun Schritt (b). Sie wieder $c^{-1}(1)$ eine korrekte Kernmenge von $c^{-1}(\{1,2\})$.
Es werden durch anwenden von (b) wieder die Bedingungen erhalten:
\begin{itemize}
    \item Für (1) spielt das 2-färben (hinzufügen zu $V \setminus K$) keine Rolle.

    \item Für (2): Da jeder der gerade 2-gefärbten Knoten per Vorraussetzung von (b) einen Nachfolger in $c^{-1}(1)$ hat,
        bleibt die Bedingung (2) für $c^{-1}(\{1,2\})$ erhalten.
\end{itemize}
Folglich ist nach anwenden von (b) $c^{-1}(1)$ immernoch eine korrekte Kernmenge von $c^{-1}(\{1,2\})$.

\strut\\

Wir untersuchen nun Schritt (c). Sie wieder $c^{-1}(1)$ eine korrekte Kernmenge von $c^{-1}(\{1,2\})$.
Es werden durch anwenden von (c) wieder die Bedingungen erhalten:
\begin{itemize}
    \item Für (1): Sei $k \in V$ einer der gerade gefärbten Knoten, also $c(k) = 1$ und $\forall w \in Succ(k): c(w) = 2$.
        Wegen letzterem kann $k$ nicht selbst die Bedingung (1) verletzen.

        Betrachte nun $v \in \{w \in c^{-1}(\{1,2\}) \mid k \in Succ(w)\}$ insofern nichtleer. Wäre $c(v) = 1$, so müsste
        $v$ durch die Schritte $(a)$ oder $(c)$ des Algorithmus gefärbt worden sein.
        Wegen $Succ(v) \neq \varnothing$ kann es nicht durch $(a)$ gewesen sein. Da $k$ bis vor kurzem
        ungefärbt und nun mit 1, also insbesondere nie $c(k) = 2$ gegolten hat, kann dies auch nicht
        durch Schritt $(c)$ passiert sein.
        Folglich muss $c(v) = 2$, und damit ist Bedingung (1) nicht verletzt.

    \item Für (2) spielt das 1-färben (hinzufügen von Knoten zur Kernmenge) keine Rolle.

\end{itemize}
Folglich ist nach anwenden von (c) $c^{-1}(1)$ immernoch eine korrekte Kernmenge von $c^{-1}(\{1,2\})$.

\strut\\

Wir zeigen nun dass in jeder Iteration mindestens ein Knoten gefärbt werden kann:

Es ist klar, dass jeder (endliche) orientierte Baum mindestens einen Knoten besitzt, welcher keine Nachfolger hat.
Andererseits hätte man einen Pfad, in welchem jeder Knoten einen Nachfolger hat. Da Bäume per Definition aber
azyklisch sind, hätte man damit per trivialer Induktion einen unendlich langen Pfad in dem orientiertem Baum,
was grundsätzlich nicht möglich ist.

Folglich können wir zu beginn mindestens einen Knoten 1-färben (zur Kernmenge hinzufügen).

Sei $c^{-1}(0) \neq \varnothing$, da wir sonst fertig sind, und $Succ(v) \neq \varnothing$ für $v \in c^{-1}(0)$,
da Knoten ohne Nachfolger beim ersten (a) gefärbt werden.

Dann gibt es ein $v \in c^{-1}(0)$ mit $Succ(v) \subseteq c^{-1}(\{1,2\})$, also einen ungefärbten Knoten, welcher
nur gefärbte Nachfolger hat, da sonst jeder Knoten einen ungefärbten Nachfolger hat und wir damit einen Zykel oder einen
unendlichen Pfad in $c^{-1}(0)$ hätten.

Falls nun $Succ(v) \subseteq c^{-1}(2)$, so lässt sich (c) anwenden, andernfalls (b).

Insgesamt lässt sich mindestens ein Knoten in jeder Iteration färben. Da wir von endlichen Eingabebäumen ausgegangen sind,
wird der Algorithmus also nach endlich vielen Schritten terminieren. Dann ist aber auch $V = c^{-1}(\{1,2\})$
und wir haben die Kernmenge $c^{-1}(1)$ von $V$.

\newpage

\textbf{d)}

Das Zertifikat ist einfach eine Kodierung der Kernmenge selbst. Dies ist offensichtlich kürzer als die Eingabe da
$K \subseteq V$. Ferner lässt sich dies in polynomieller Zeit verifizieren, indem man für alle Knoten
alle Nachfolger betrachtet und je nach Knoten die Bedingung (1) oder (2) überprüft. Dies geht in quadratischer Zeit.

\textbf{e)}
Nein, dies würde Bedingung (1) widersprechen.

\textbf{f)}
Nein, dies würde Bedingung (2) verletzen, da $A(x_i)$ und $A(\overline{x_i})$ nur einen ausgehenden Pfeil zum jeweils
anderen haben. Wenn beide nicht in der Kernmenge wären, hätten sie keinen Pfeil zu einem Knoten in der Kernmenge.

\textbf{g)} k.

\textbf{h)}
Bei $m$ Klauseln werden $3m$ der $B_j(c_i)$-Knoten eingeführt. Für $n$ Variablen werden $2n$ Knoten eingeführt (1 pro Literal).
Folglich ist $|V_{G(\Phi)}| = 3m+2n \leq 3(m+n)$ also polynomiell in $m+n$.

\textbf{i)}
Wir haben $2n$ ausgehende Pfeile der Literal-Knoten (1 pro Literal). Jeder der $B_j(c_i)$-Knoten hat pro Literal in Klausel $c_i$
genau 1 ausgehenden Pfeil zum entsprechendem Literal, also $3$ Pfeile pro $B_j(c_i)$-Knoten, von denen es $3m$ gibt, was
$9m$ ausgehende Pfeile der $B_j(c_i)$-Knoten macht. Insgesamt haben wir $9m + 2n$ Pfeile, was polynomiell in $m+n$ beschränkt ist.

\textbf{j)}
Die Anzahl der Kanten / Pfeile in einem Graphen $G = (V,E)$ ist beschränkt durch $|V|^2$.
Wenn also $|V|$ schon polynomiell beschränkt in der Eingabe ist, so ist dies auch $|V|^2$.

\textbf{k)}
Da jeder Literal-Knoten genau einen ausgehenden Pfeil hat, folgt mit e) und f) schon, dass die Bedingung (1) einer Kernmenge
erfüllt ist, da diese nur durch die Knoten der Kernmenge (also hier die Literalknoten) verletzt werden kann.

Der jeweils andere Literalknoten, welcher nicht in der Kernmenge liegt, verletzt die (2) Bedingung nicht, da sein
einziger ausgehender Pfeil eben auf sein entsprechendes gegenüber in der Kernmenge zeigt.

Alle Klauselknoten haben Pfeile zu den Literalen, welche die Klausel wahrmachen (und damit in der Kernmenge liegen). Insgesamt ist also Bedingung (2) auch
im Graphen erfüllt. Daher bildet die gegebene Menge der Literalknoten eine Kernmenge für $G(\Phi)$.

\textbf{l)}
Eine Wahrheitsbelegung weißt jeder Variable $x$ einen Wert zu. Dann hat genau eines der Literale $x$ und $\overline{x}$
den Wahrheitswert 1, sodass per Definition der Kernmenge genau einer der Literalknoten $A(x), A(\overline{x})$ in dieser liegt.

\textbf{m)}
Die 3 Klauselknoten $B_1(c),B_2(c),B_3(c)$ einer Klausel $c$ bilden einen $C_3$. Wären mindestens $2$ dieser in der Kernmenge,
also $B_i, B_j$ mit $i \neq j$, so hätten wir wegen $B_1 \rightarrow B_2 \rightarrow B_3 \rightarrow B_1$ eine Verletzung
der Bedingung (1) einer Kernmenge. Folglich darf höchstens einer der 3 Klauselknoten in $K$ liegen.

\textbf{n)}
Sei also $c = (\ell_1 \lor \ell_2 \lor \ell_3)$ eine der Klausel. Angenommen $A(\ell_i) \notin K$ für $i=1,2,3$.
Nach $m)$ existiert ein $j \in \{1,2,3\}$ sodass $B_j(c) \notin K$ keinen Pfeil auf einen Klauselknoten in $K$ hat.
Da $B_j(c)$ aber sonst nur ausgehende Pfeile zu $A(\ell_i)$ für $i=1,2,3$ hat, wäre dann Bedingung (2) der Kernmenge
verletzt. Folglich $\exists j \in \{1,2,3\}: A(\ell_j) \in K$.

\textbf{o)}
Die Wahrheitsbelegung $\varphi: X \to \{0,1\}$ ist $\varphi(x) := \begin{cases}1 &, A(x) \in K\quad (\up{\iff}{e} A(\overline{x}) \notin K)\\ 0 &, \text{sonst}\\\end{cases}$
\newpage

\section{Drei-Färbbarkeit von Graphen}

Im folgenden betrachten wir ungerichtete Graphen $G = (V,E)$.
Sei für $v \in V$ dann
$$
    Adj(v) := \{w \in V \mid (v,w) \in E \quad\lor\quad (w,v) \in E\}
$$

\textbf{a)}

Wir betrachten $C_n$ für $n \geq 3$. Wir schreiben wieder $[k]_n$ für $k$ (mod n).\\
Entsprechend ist $V_{C_n} = [0,n-1]$.

\textbf{Fall 1:} $n = 2\ell$ für $\ell \in \mathbb{N}$:\\
Hier lässt sich $C_n$ sogar 2-färben, indem man $c(1+2k) := 1$ für $k \in [0,\ell-1]$ setzt, also alle ungeraden
Knoten in $V_{C_n}$ mit 1 färbt. Alle anderen färbt man mit 2, also $c(2k) := 2$ für $k \in [0,\ell-1]$.
Dann folgt für $k \in V_{C_n}$ sofort, dass $c([k-1]_n) \neq c(k) \neq c([k+1]_n)$, und da $Adj(k) = \{[k-1]_n, [k+1]_n\}$, damit $c$ eine 2-Färbung
von $C_n$ ist.

\textbf{Fall 2:} $n = 2\ell+1$ für $\ell \in \mathbb{N}$:\\
Wir setzen $c(0) := 3$. Dann analog zum oberen Fall färben wir die restlichen ungeraden Knoten mit 1, die geraden mit 2, also
$$
    c(1+2k) := 1\quad\text{für $k \in [0,\ell-1]$}
    \qquad\text{sowie}\qquad
    c(2k) := 2\quad\text{für $k \in [1,\ell-1]$}
$$
Dann ist für $c(n) \neq c(0) \neq c(1)$ und $Adj(0) = \{n,1\}$. Ferner gilt für $k \in [1,n]$, dass $c([k-1]_n) \neq c(k) \neq c([k+1]_n)$ und $Adj(k) = \{[k-1]_n, [k+1]_n\}$, also Insgesamt $c$ eine 3-Färbung von $C_n$ ist.

Folglich lässt sich $C_n$ in jedem Fall 3-Färben.


\textbf{b)}

Wir betrachten einen endlichen Baum $T = (V,E)$ mit maximaler Tiefe $d \in \mathbb{N}$.
Sei $D(v) \in [0,d]$ die Ebene bzw Tiefe eines Knotens $v \in V$. $T$ lässt sich 2-färben, indem man
$$
    \forall v \in \{w \in V \mid D(v) \text{ gerade}\}: c(v) := 1
    \quad\text{und}\quad
    \forall v \in \{w \in V \mid D(v) \text{ ungerade}\}: c(v) := 2
$$
setzt. Dank der Struktur des Baumes gilt
$$
    \forall v \in V: \forall w \in Adj(v): [D(w)]_2 \neq [D(v)]_2
$$
also dass alle zu $v$ adjazenten Knoten auf ebenen mit anderer Parität als $v$ liegen.

Seien nun $u,v \in V$. Wir fügen die Kante $(u,v)$ zu $E$ hinzu. Falls $c(u) \neq c(v)$, so muss nichts geändert werden.
Falls jedoch $c(u) = c(v)$ so setzen wir $c(u) := 3$, und benutzen unsere dritte Farbe.
Dann ist $c$ eine korrekte 3-Färbung des veränderten $T$.

\newpage

\textbf{c)}

\begin{center}
    \begin{tikzpicture}
        \node[shape=circle,draw=black,fill=red] (A) at (0,0)     {};
        \node[shape=circle,draw=black,fill=blue] (B) at (0,-1)    {};
        \node[shape=circle,draw=black,fill=red] (C) at (2,-1.5)  {};
        \node[shape=circle,draw=black,fill=red] (D) at (-2,-1.5)   {};
        \node[shape=circle,draw=black,fill=blue] (E) at (-3,-2.5)   {};
        \node[shape=circle,draw=black,fill=green] (F) at (3,-2.5)  {};
        \node[shape=circle,draw=black,fill=blue] (G) at (-1.3,1)  {};
        \node[shape=circle,draw=black,fill=blue] (H) at (1.3,1)  {};
        \node[shape=circle,draw=black,fill=green] (I) at (2.5,.8)  {};
        \node[shape=circle,draw=black,fill=green] (J) at (-2.5,.8)  {};
        \node[shape=circle,draw=black,fill=red] (K) at (-.7,2.5)  {};
        \node[shape=circle,draw=black,fill=green] (L) at (.7,2.5)  {};
        \node[shape=circle,draw=black,fill=blue] (M) at (-3.3,2.5)  {};
        \node[shape=circle,draw=black,fill=red] (N) at (-4.5,3)  {};
        \node[shape=circle,draw=black,fill=blue] (O) at (3.3,2.5)  {};
        \node[shape=circle,draw=black,fill=red] (P) at (4.5,3)  {};
        \node[shape=circle,draw=black,fill=green] (Q) at (0,6)  {};
        \node[shape=circle,draw=black,fill=blue] (R) at (0,5)  {};
        \node[shape=circle,draw=black,fill=green] (S) at (-1.5,3.3)  {};
        \node[shape=circle,draw=black,fill=red] (T) at (1.5,3.3)  {};

        \path [-] (A) edge node[left] {} (B);
        \path [-] (A) edge node[left] {} (G);
        \path [-] (A) edge node[left] {} (H);
        \path [-] (F) edge node[left] {} (E);
        \path [-] (D) edge node[left] {} (E);
        \path [-] (D) edge node[left] {} (B);
        \path [-] (B) edge node[left] {} (C);
        \path [-] (C) edge node[left] {} (F);
        \path [-] (G) edge node[left] {} (J);
        \path [-] (H) edge node[left] {} (I);
        \path [-] (J) edge node[left] {} (D);
        \path [-] (I) edge node[left] {} (C);
        \path [-] (K) edge node[left] {} (G);
        \path [-] (K) edge node[left] {} (L);
        \path [-] (L) edge node[left] {} (H);
        \path [-] (N) edge node[left] {} (M);
        \path [-] (M) edge node[left] {} (J);
        \path [-] (N) edge node[left] {} (E);
        \path [-] (O) edge node[left] {} (P);
        \path [-] (O) edge node[left] {} (I);
        \path [-] (P) edge node[left] {} (F);
        \path [-] (N) edge node[left] {} (Q);
        \path [-] (Q) edge node[left] {} (P);
        \path [-] (Q) edge node[left] {} (R);
        \path [-] (R) edge node[left] {} (S);
        \path [-] (R) edge node[left] {} (T);
        \path [-] (S) edge node[left] {} (K);
        \path [-] (S) edge node[left] {} (M);
        \path [-] (T) edge node[left] {} (L);
        \path [-] (T) edge node[left] {} (O);
    \end{tikzpicture}
\end{center}

\textbf{d)}
Das Zertfikat ist einfach die 3-partition der Knoten hintereinandergeschrieben, welches offensichtlich als
permutation der Knoteneingabe der Länge her polynomiell in der Eingabelänge beschränkt ist.
Zuerst überprüft man das Zertifikat auf Vollständigkeit, dass auch wirklich jeder Knoten genau 1 mal vorkommt.
Dann betrachtet man der Reihe nach die Knoten einer Farbe und überprüft für jeden, dass sie nicht adjazent zu einem
der Knoten mit der selben Farbe sind. Dies dauert maximal $\mathcal{O}(|V|^2)$ Operationen.

\textbf{e)}

\begin{center}
    \begin{tikzpicture}
        \node[shape=circle,draw=black] (A) at (-4,0)     {$y_2$};
        \node[shape=circle,draw=black] (B) at (-2,0)    {$y_1$};
        \node[shape=circle,draw=black] (C) at (-3,2)   {$y_3$};
        \node[shape=circle,draw=black] (D) at (2,0)  {$y_4$};
        \node[shape=circle,draw=black] (E) at (4,0)   {$y_5$};
        \node[shape=circle,draw=black] (F) at (3,2)  {$y_6$};
        \node[shape=circle,draw=black] (G) at (4,-1.5)   {$\gamma$};
        \node[shape=circle,draw=black] (H) at (-5.5,0)   {$\alpha$};
        \node[shape=circle,draw=black] (I) at (-4.5,2)   {$\beta$};

        \path [-] (A) edge node[left] {} (B);
        \path [-] (B) edge node[left] {} (C);
        \path [-] (C) edge node[left] {} (A);
        \path [-] (B) edge node[left] {} (D);
        \path [-] (D) edge node[left] {} (E);
        \path [-] (E) edge node[left] {} (F);
        \path [-] (F) edge node[left] {} (D);
        \path [-] (E) edge node[left] {} (G);
        \path [-] (H) edge node[left] {} (A);
        \path [-] (I) edge node[left] {} (C);
    \end{tikzpicture}
\end{center}

\textbf{f)}

Sei also $c(\alpha) = c(\beta) = c(\gamma) = 1$. Dann muss $c(y_1) = 1$, da $\{c(y_2), c(y_3)\} = \{2,3\}$.
Folglich ist $c(y_4) \in \{2,3\}$, wodurch mit $c(\gamma) = 1$ dann auch $\{c(y_4), c(y_5)\} = \{2,3\}$ folgt.
Also bleibt nur die Wahl $c(y_6) = 1$.

\newpage

\textbf{g)}
Betrachte einfach alle Fälle:


\begin{minipage}{.5\textwidth}
\begin{tabular}{*8{c|}c}
    $\alpha$ & $\beta$ & $\gamma$ & $y_1$ & $y_2$ & $y_3$ & $y_4$ & $y_5$ & $y_6$\\
    \hline
    1 & 1 & 1 & 1 & 2 & 3 & 2 & 3 & 1\\
    \hline
    1 & 1 & 2 & 1 & 2 & 3 & 2 & 3 & 1\\
    \hline
    1 & 1 & 3 & 1 & 2 & 3 & 3 & 2 & 1\\
    \hline
    1 & 2 & 1 & 3 & 2 & 1 & 2 & 3 & 1\\
    \hline
    1 & 2 & 2 & 3 & 2 & 1 & 2 & 3 & 1\\
    \hline
    1 & 2 & 3 & 3 & 2 & 1 & 3 & 2 & 1\\
    \hline
    1 & 3 & 1 & 2 & 3 & 1 & 2 & 3 & 1\\
    \hline
    1 & 3 & 2 & 2 & 3 & 1 & 2 & 3 & 1\\
    \hline
    1 & 3 & 3 & 2 & 3 & 1 & 3 & 2 & 1\\
    \hline
    2 & 1 & 1 & 3 & 1 & 2 & 2 & 3 & 1\\
\end{tabular}
\end{minipage}
\begin{minipage}{.5\textwidth}
\begin{tabular}{*8{c|}c}
    $\alpha$ & $\beta$ & $\gamma$ & $y_1$ & $y_2$ & $y_3$ & $y_4$ & $y_5$ & $y_6$\\
    \hline
    2 & 1 & 2 & 3 & 1 & 2 & 2 & 3 & 1\\
    \hline
    2 & 1 & 3 & 2 & 1 & 3 & 3 & 2 & 1\\
    \hline
    2 & 2 & 1 & 2 & 1 & 3 & 3 & 2 & 1\\
    \hline
    2 & 3 & 1 & 1 & 3 & 2 & 2 & 3 & 1\\
    \hline
    3 & 1 & 1 & 2 & 1 & 3 & 3 & 2 & 1\\
    \hline
    3 & 1 & 2 & 2 & 1 & 3 & 2 & 3 & 1\\
    \hline
    3 & 1 & 3 & 2 & 1 & 3 & 3 & 2 & 1\\
    \hline
    3 & 2 & 1 & 1 & 2 & 3 & 2 & 3 & 1\\
    \hline
    3 & 3 & 1 & 3 & 1 & 2 & 2 & 3 & 1\\
\end{tabular}
\end{minipage}


\textbf{h)}
Offensichtlich und auch schon öfters benutzt.

\textbf{i)}
Folgt aus h), da $A(x_i), A(\overline{x_i}), DUMMY$ ein Dreieck bilden.

\textbf{j)} k.

\textbf{k)}
Es gibt $2n$ Variablenknoten sowie $m$ Gadgets mit jeweils 6 Knoten ($\alpha,\beta,\gamma$ werden verschmolzen).
Dazu haben wir die 3 Knoten WAHR,FALSCH,DUMMY.
Insgesamt gibt es also $2n+6m+3 \in \mathcal{O}(m+n)$ Knoten.

\textbf{l)}
Es gibt pro Variable 3 Kanten; zwischen den entsprechenden Literalknoten und von beiden zu DUMMY. Ein Gadget hat 10 Kanten
an sich. Ferner kommen pro Gadget 2 Kanten hinzu; von $y_6$ zu FALSCH und DUMMY.
Damit haben wir insgesamt $3n+12m \in \mathcal{O}(m+n)$ Kanten.

\textbf{m)}
Sei also $c = (\ell_1 \lor \ell_2 \lor \ell_3)$ eine Klausel aus $\Phi$ und $\varphi$ die erfüllende Variablenbelegung.\\
Dann ex. $i \in \{1,2,3\}: \varphi(\ell_i) = 1$. Da dann $A(\ell_i)$ mit einem der $\alpha,\beta,\gamma$ Knoten aus $G_9(c)$
verschmolzen ist, folgt mit g), dass es eine 3-Färbung gibt in der $y_6$ die Farbe WAHR hat.
Da verschiedene Gadgets abgesehen von den Literalknoten nicht miteinander verbunden sind, haben wir damit
eine korrekte 3-Färbung von $G(\Phi)$.

\textbf{n)}
Dies folgt aus h), da die Knoten $y_6$, DUMMY, FALSCH ein Dreieck bilden.

\textbf{o)}
Da die Knoten $\alpha,\beta,\gamma$ eines Gadgets $G_9(c)$ mit den Literalknoten $A(\ell_1),A(\ell_2),A(\ell_3)$ verschmolzen
sind, können diese nicht die Farbe DUMMY annehmen. Wenn also keiner der 3 WAHR ist, so müssen alle 3 FALSCH sein.
Nach f) müsste dann aber $y_6$ ebenfalls FALSCH sein, Widerspruch. Folglich muss mindestens einer der 3 Literalknoten WAHR sein.

\textbf{p)}
Wenn also $G(\Phi)$ eine 3-Färbung hat, so ist $G_9(c)$ für jede Klausel $c$ der Eingabeformel korrekt gefärbt.
Nach o) ist dann mindestens einer der 3 Knoten $\alpha,\beta,\gamma$, welche mit den entsprechenden Literalknoten der Klausel $c$
verschmolzen sind, WAHR gefärbt. Die Wahrheitsbelegung $\varphi : X \to \{0,1\}$ ist dann
$$
    \varphi(x) := \begin{cases}
        1 &, A(x) \text{ hat Farbe WAHR}\\
        0 &,\text{sonst}
    \end{cases}
$$
Dann ist nämlich pro Klausel mindestens eines der vorkommenden Literale erfüllt und damit $\Phi$ erfüllt.

\newpage

\textbf{q)}
Wir gehen von endlichen Graphen aus.
Man kann sich pro Zusammenhangskomponente des Graphen einen beliebigen ''Startknoten'' $v \in V$ wählen.
Wir setzen o.B.d.A $c(v) := 0$. Sei nun $k \in V$ gefärbt. Dann setzen wir $c(\ell) := [c(k)+1]_2$ für alle noch nicht
gefärbten Knoten $\ell \in Adj(k)$.
Nach endlich vielen Schritten werden alle Knoten gefärbt sein. Nun kann man in polynomieller Zeit testen, ob der Graph
korrekt gefärbt ist.
Da das vertauschen von Farben semantisch keine Rolle spielt, gibt es praktisch nur diese eine Möglichkeit den Graphen zu 2-färben.
Damit lässt sich also in polynomieller Zeit entscheiden, ob ein endlicher Graph 2-färbbar ist.

\newpage

\section{Drei-Färbbarktei von planaren Graphen}


\textbf{a)}
Mit $n$ Knoten hat der Vollständige Graph $K_n$ dann $\binom{n}{2} = \frac{n(n-1)}{2}$ Knoten.
Für $n > 5$ ist dann
$$
    0 < (n-3)(n-4) = n^2-7n+12
    \quad\Longrightarrow\quad |E| = \frac{n(n-1)}{2} > 3n-6
$$
was dem Eulerschen Polyedersatz widerspricht, also $K_n$ nicht planar sein kann.
Die Fälle $n=3,4$ sind wie folgt:\\

\begin{minipage}{.5\textwidth}
\begin{center}
    \begin{tikzpicture}
        \node[shape=circle,draw=black] (A) at (-1,0)     {};
        \node[shape=circle,draw=black] (B) at (1,0)    {};
        \node[shape=circle,draw=black] (C) at (0,1.5)   {};

        \path [-] (A) edge node[left] {} (B);
        \path [-] (B) edge node[left] {} (C);
        \path [-] (C) edge node[left] {} (A);
    \end{tikzpicture}
\end{center}
\end{minipage}
\begin{minipage}{.5\textwidth}
\begin{center}
    \begin{tikzpicture}
        \node[shape=circle,draw=black] (A) at (-1,0)     {};
        \node[shape=circle,draw=black] (B) at (1,0)    {};
        \node[shape=circle,draw=black] (C) at (1,1)   {};
        \node[shape=circle,draw=black] (D) at (-1,1)   {};

        \path [-] (A) edge node[left] {} (B);
        \path [-] (B) edge node[left] {} (C);
        \path [-] (C) edge node[left] {} (A);
        \path [-] (A) edge node[left] {} (D);
        \path [-] (D) edge node[left] {} (C);
        \path [-] (D) edge [out looseness = 3, bend right = 120] node[left] {} (B);
    \end{tikzpicture}
\end{center}
\end{minipage}

\textbf{b)}

Man kann einen Baum mit maximalem fan-out von $\Delta := \max\{|Adj(v)|-1 : v \in V\} \in \mathbb{N}$
in die Euklidische Ebene ''quetschen'', indem man die Kinder eines Knoten $v$ in der $n$-ten Ebene des Baumes
auf ein Liniensegment der Länge $\frac{1}{\Delta^n}$ eine Längeneinheit unter $v$ gleichverteilt.
Der Einfachheit halber nehmen wir $\Delta = 2k+1 \geq 3$ für ein $k\in \mathbb{N}$ an, da Plätze für nicht vorhandene
Kinder kein Problem darstellen.

Wir geben der Wurzel $r \in V$ die Koordinaten $P(r) := (0,0)$.
Die Kinder eines Knotens $v \in V$ mit Koordinaten $P(v) = (x,n) \in \mathbb{Q}\times \mathbb{N}$ werden dann auf
die Positionen
$$
    \mathcal{P}_v := [x-\frac{1}{\Delta^n}, x+\frac{1}{\Delta^n}, \frac{2}{\Delta^n(\Delta - 1)}] \times \{n+1\}
    = \{(x-\frac{1}{\Delta^n}, n+1), \cdots, (x,n+1), \cdots, (x+\frac{1}{\Delta^n}, n+1)\}
$$
verteilt, wobei $[a,b,c] := \{a+nc \mid n \in \mathbb{N}, a+nc \leq b\}$ die lineare Gleichverteilung zwischen $a$ und $b$ mit
Distanz $c$ darstellt.
Die Kanten bleiben geraden von Knoten zu Knoten.

Seien $v,w \in V$ mit $P(v) = (x,n+1), P(w) = (y,n+1)$. Sei o.B.d.A. $x < y$. Dann gilt
$$
    |(y-\frac{1}{\Delta^{n+2}}) - (x+\frac{1}{\Delta^{n+2}})| = y-x \geq \frac{2}{\Delta^{n+1}(\Delta -1)} > 0
$$
sodass die ''Kinderbereiche'' $\mathcal{P}_v, \mathcal{P}_w$ zweier Knoten der selben Ebene sich niemals überschneiden bzw. überlappen.
Damit sind Bäume also planar. Man könnte dies auch mit dem Satz von Kuratowski zeigen, was aber weniger anschaulich ist.

\textbf{c)}

Wir geben ein Gegenbeispiel. Folgender Graph hat 6 Knoten und 9 Kanten, also $|E| = 9 \leq 12 = 3|V|-6$, ist jedoch ein
bekanntes Beispiel für nicht-planare Graphen:

\begin{center}
    \begin{tikzpicture}
        \node[shape=circle,draw=black] (A) at (0,1)     {};
        \node[shape=circle,draw=black] (B) at (0,-1)    {};
        \node[shape=circle,draw=black] (F) at (-2,1)   {};
        \node[shape=circle,draw=black] (C) at (-2,-1)   {};
        \node[shape=circle,draw=black] (D) at (2,1)   {};
        \node[shape=circle,draw=black] (E) at (2,-1)   {};

        \path [-] (A) edge node[left] {} (B);
        \path [-] (A) edge node[left] {} (C);
        \path [-] (A) edge node[left] {} (E);
        \path [-] (D) edge node[left] {} (B);
        \path [-] (D) edge node[left] {} (C);
        \path [-] (D) edge node[left] {} (E);
        \path [-] (F) edge node[left] {} (B);
        \path [-] (F) edge node[left] {} (C);
        \path [-] (F) edge node[left] {} (E);
    \end{tikzpicture}
\end{center}

\newpage

\textbf{d)}
Auf offensichtliche weise gilt PLANAR-3-COLORING $\leq_p$ 3-COLORING $\in \textsf{NP}$ und damit
auch PLANAR-3-COLORING $\in \textsf{NP}$.


\begin{minipage}{.5\textwidth}
\textbf{e)}
\begin{center}
    \begin{tikzpicture}[scale=.5]
        \node[shape=circle,draw=black,fill=green] (A) at (2,0)     {};
        \node[shape=circle,draw=black,fill=red] (B) at (0,0)     {};
        \node[shape=circle,draw=black,fill=green] (C) at (-2,0)    {};

        \node[shape=circle,draw=black,fill=green] (D) at (2,-2)    {};
        \node[shape=circle,draw=black,fill=blue] (E) at (0,-2)    {};
        \node[shape=circle,draw=black,fill=blue] (F) at (-2,-2)   {};

        \node[shape=circle,draw=black,fill=blue] (G) at (2,2)     {};
        \node[shape=circle,draw=black,fill=blue] (H) at (0,2)     {};
        \node[shape=circle,draw=black,fill=green] (I) at (-2,2)    {};

        \node[shape=circle,draw=black,fill=red] (J) at (0,5)     {};
        \node[shape=circle,draw=black,fill=red] (K) at (-5,0)    {};
        \node[shape=circle,draw=black,fill=red] (L) at (5,0)     {};
        \node[shape=circle,draw=black,fill=red] (M) at (0,-5)    {};


        \path [-] (B) edge node[left] {} (A);
        \path [-] (B) edge node[left] {} (C);
        \path [-] (C) edge node[left] {} (K);
        \path [-] (A) edge node[left] {} (L);
        \path [-] (B) edge node[left] {} (H);
        \path [-] (B) edge node[left] {} (E);
        \path [-] (H) edge node[left] {} (A);
        \path [-] (A) edge node[left] {} (E);
        \path [-] (E) edge node[left] {} (C);
        \path [-] (C) edge node[left] {} (H);
        \path [-] (H) edge node[left] {} (I);
        \path [-] (A) edge node[left] {} (G);
        \path [-] (E) edge node[left] {} (D);
        \path [-] (H) edge node[left] {} (J);
        \path [-] (E) edge node[left] {} (M);
        \path [-] (C) edge node[left] {} (F);
        \path [-] (K) edge node[left] {} (I);
        \path [-] (I) edge node[left] {} (J);
        \path [-] (J) edge node[left] {} (G);
        \path [-] (G) edge node[left] {} (L);
        \path [-] (L) edge node[left] {} (D);
        \path [-] (D) edge node[left] {} (M);
        \path [-] (M) edge node[left] {} (F);
        \path [-] (F) edge node[left] {} (K);
    \end{tikzpicture}
\end{center}
\end{minipage}
\begin{minipage}{.5\textwidth}
\textbf{f)}
\begin{center}
    \begin{tikzpicture}[scale=.5]
        \node[shape=circle,draw=black,fill=red] (A) at (2,0)     {};
        \node[shape=circle,draw=black,fill=green] (B) at (0,0)     {};
        \node[shape=circle,draw=black,fill=red] (C) at (-2,0)    {};

        \node[shape=circle,draw=black,fill=green] (D) at (2,-2)    {};
        \node[shape=circle,draw=black,fill=blue] (E) at (0,-2)    {};
        \node[shape=circle,draw=black,fill=green] (F) at (-2,-2)   {};

        \node[shape=circle,draw=black,fill=green] (G) at (2,2)     {};
        \node[shape=circle,draw=black,fill=blue] (H) at (0,2)     {};
        \node[shape=circle,draw=black,fill=green] (I) at (-2,2)    {};

        \node[shape=circle,draw=black,fill=red] (J) at (0,5)     {};
        \node[shape=circle,draw=black,fill=blue] (K) at (-5,0)    {};
        \node[shape=circle,draw=black,fill=blue] (L) at (5,0)     {};
        \node[shape=circle,draw=black,fill=red] (M) at (0,-5)    {};


        \path [-] (B) edge node[left] {} (A);
        \path [-] (B) edge node[left] {} (C);
        \path [-] (C) edge node[left] {} (K);
        \path [-] (A) edge node[left] {} (L);
        \path [-] (B) edge node[left] {} (H);
        \path [-] (B) edge node[left] {} (E);
        \path [-] (H) edge node[left] {} (A);
        \path [-] (A) edge node[left] {} (E);
        \path [-] (E) edge node[left] {} (C);
        \path [-] (C) edge node[left] {} (H);
        \path [-] (H) edge node[left] {} (I);
        \path [-] (A) edge node[left] {} (G);
        \path [-] (E) edge node[left] {} (D);
        \path [-] (H) edge node[left] {} (J);
        \path [-] (E) edge node[left] {} (M);
        \path [-] (C) edge node[left] {} (F);
        \path [-] (K) edge node[left] {} (I);
        \path [-] (I) edge node[left] {} (J);
        \path [-] (J) edge node[left] {} (G);
        \path [-] (G) edge node[left] {} (L);
        \path [-] (L) edge node[left] {} (D);
        \path [-] (D) edge node[left] {} (M);
        \path [-] (M) edge node[left] {} (F);
        \path [-] (F) edge node[left] {} (K);
    \end{tikzpicture}
\end{center}
\end{minipage}

\textbf{g)}
Sei die Benennung wie folgt:
\begin{center}
    \begin{tikzpicture}[scale=.5]
        \node[shape=circle,draw=black] (A) at (2,0)     {a};
        \node[shape=circle,draw=black] (B) at (0,0)     {b};
        \node[shape=circle,draw=black] (C) at (-2,0)    {c};

        \node[shape=circle,draw=black] (D) at (2,-2)    {d};
        \node[shape=circle,draw=black] (E) at (0,-2)    {e};
        \node[shape=circle,draw=black] (F) at (-2,-2)   {f};

        \node[shape=circle,draw=black] (G) at (2,2)     {g};
        \node[shape=circle,draw=black] (H) at (0,2)     {h};
        \node[shape=circle,draw=black] (I) at (-2,2)    {i};

        \node[shape=circle,draw=black] (J) at (0,5)     {u};
        \node[shape=circle,draw=black] (L) at (5,0)     {v'};
        \node[shape=circle,draw=black] (O) at (0,-5)    {u'};
        \node[shape=circle,draw=black] (P) at (-5,0)    {v};


        \path [-] (B) edge node[left] {} (A);
        \path [-] (B) edge node[left] {} (C);
        \path [-] (C) edge node[left] {} (P);
        \path [-] (A) edge node[left] {} (L);
        \path [-] (B) edge node[left] {} (H);
        \path [-] (B) edge node[left] {} (E);
        \path [-] (H) edge node[left] {} (A);
        \path [-] (A) edge node[left] {} (E);
        \path [-] (E) edge node[left] {} (C);
        \path [-] (C) edge node[left] {} (H);
        \path [-] (H) edge node[left] {} (I);
        \path [-] (A) edge node[left] {} (G);
        \path [-] (E) edge node[left] {} (D);
        \path [-] (H) edge node[left] {} (J);
        \path [-] (E) edge node[left] {} (O);
        \path [-] (C) edge node[left] {} (F);
        \path [-] (P) edge node[left] {} (I);
        \path [-] (I) edge node[left] {} (J);
        \path [-] (J) edge node[left] {} (G);
        \path [-] (G) edge node[left] {} (L);
        \path [-] (L) edge node[left] {} (D);
        \path [-] (D) edge node[left] {} (O);
        \path [-] (O) edge node[left] {} (F);
        \path [-] (F) edge node[left] {} (P);
    \end{tikzpicture}
\end{center}

Sei also eine 3-Färbung $c$ gegeben. Wir sagen o.B.d.A. $c(b) = 1, c(a)=c(c)=2, c(e)=c(h) = 3$.
Angenommen $c(u) \neq c(u')$. Wieder o.B.d.A. $c(u) = 1, c(u') = 2$. Es muss $c(d) = 1, c(g) = 3$.
Nun haben wir also $c(g) = 3, c(a) = 2, c(d) = 1$ somit kann $c(v')$ nicht korrekt sein.

Folglich müssen $u,u'$ in jeder 3-Färbung die selbe Farbe haben. Das selbe gilt für $v,v'$, da der Graph
mit vertauschten $u$'s und $v$'s der gleiche wäre.

\textbf{h)}
Offensichtlich kann man die Strecken so wählen, dass 2 Strecken höchstens 1 Schnittpunkt haben.
Schneidet sich eine Strecke mir mehreren anderen im selben Punkt, so können die Schnittpunkte um
vielfache eines geeigneten $\eps > 0$ verschoben werden sodass die Schnittpunkte disjunkt sind.
Man hat prakitsch $|\mathbb{R}|$ mögliche Schnittpunkte für jede Kante, was reichen sollte.

\textbf{i)}
Seien also $S,S'$ zwei Streckenzüge mit Anfangs und Endknoten $S=(a,b), S'=(x,y)$ und einem Schnittpunkt $p \in \mathbb{R}^2$. Es lässt sich ein $\eps > 0$
finden sodass $K_\eps(p)$ keine weiteren Schnittpunkte enthält. Dieser Kreis wird nun ''ausgestanzt'' und wir fügen
das Kreuzungsgadget hinein, wobei wir $u$ mit $a$ und $v$ mit $x$ verschmelzen, und die Kanten $(u',b),(v',y)$
hinzufügen, sodass sich keine neuen Schnittpunkte bilden:

\begin{minipage}{.5\textwidth}
\begin{center}
    \begin{tikzpicture}[scale=.45]
        \node[shape=circle,draw=black] (N) at (0,7)     {b};
        \node[shape=circle,draw=black] (O) at (0,-7)    {a};
        \node[shape=circle,draw=black] (P) at (-7,0)    {x};
        \node[shape=circle,draw=black] (Q) at (7,0)     {y};

        \path [-] (P) edge node[left] {} (Q);
        \path [-] (N) edge node[left] {} (O);
    \end{tikzpicture}
\end{center}
\end{minipage}
$\longrightarrow$
\begin{minipage}{.5\textwidth}
\begin{center}
    \begin{tikzpicture}[scale=.45]
        \node[shape=circle,draw=black] (A) at (2,0)     {};
        \node[shape=circle,draw=black] (B) at (0,0)     {};
        \node[shape=circle,draw=black] (C) at (-2,0)    {};

        \node[shape=circle,draw=black] (D) at (2,-2)    {};
        \node[shape=circle,draw=black] (E) at (0,-2)    {};
        \node[shape=circle,draw=black] (F) at (-2,-2)   {};

        \node[shape=circle,draw=black] (G) at (2,2)     {};
        \node[shape=circle,draw=black] (H) at (0,2)     {};
        \node[shape=circle,draw=black] (I) at (-2,2)    {};

        \node[shape=circle,draw=black] (J) at (0,5)     {u'};
        \node[shape=circle,draw=black] (L) at (5,0)     {v'};

        \node[shape=circle,draw=black] (N) at (0,7)     {b};
        \node[shape=circle,draw=black] (O) at (0,-7)    {a+u};
        \node[shape=circle,draw=black] (P) at (-7,0)    {x+v};
        \node[shape=circle,draw=black] (Q) at (7,0)    {y};

        \path [-] (B) edge node[left] {} (A);
        \path [-] (B) edge node[left] {} (C);
        \path [-] (C) edge node[left] {} (P);
        \path [-] (A) edge node[left] {} (L);
        \path [-] (B) edge node[left] {} (H);
        \path [-] (B) edge node[left] {} (E);
        \path [-] (H) edge node[left] {} (A);
        \path [-] (A) edge node[left] {} (E);
        \path [-] (E) edge node[left] {} (C);
        \path [-] (C) edge node[left] {} (H);
        \path [-] (H) edge node[left] {} (I);
        \path [-] (A) edge node[left] {} (G);
        \path [-] (E) edge node[left] {} (D);
        \path [-] (H) edge node[left] {} (J);
        \path [-] (E) edge node[left] {} (O);
        \path [-] (C) edge node[left] {} (F);
        \path [-] (P) edge node[left] {} (I);
        \path [-] (I) edge node[left] {} (J);
        \path [-] (J) edge node[left] {} (G);
        \path [-] (G) edge node[left] {} (L);
        \path [-] (L) edge node[left] {} (D);
        \path [-] (D) edge node[left] {} (O);
        \path [-] (O) edge node[left] {} (F);
        \path [-] (F) edge node[left] {} (P);

        \path [-] (N) edge node[left] {} (J);
        \path [-] (Q) edge node[left] {} (L);
    \end{tikzpicture}
\end{center}
\end{minipage}
Denn da $u,u'$ und $v,v'$ stets bei einer korrekten 3-Färbung die selbe Farbe haben, wird damit gesichert, dass
$a,b$ und $x,y$ jeweils verschiedene Farben haben. Falls jedoch mehrere Kreuzungsgadgets auf einem Streckenzug von
Knoten $x$ zu Knoten $y$ sind, wird das $v$ des ersten Gadgets mit $x$ verschmolzen, und das $v'$ des letzten
gadgets mit $y$ verbunden, und alle $v,v'$ dazwischen je paarweise miteinander verschmolzen, also
das $v'$ des $i$-ten Gadgets mit dem $v$ des $i+1$-ten:

\begin{center}
    \begin{tikzpicture}[scale=.45]
        \node[shape=circle,draw=black] (A) at (2,0)     {};
        \node[shape=circle,draw=black] (B) at (0,0)     {};
        \node[shape=circle,draw=black] (C) at (-2,0)    {};

        \node[shape=circle,draw=black] (D) at (2,-2)    {};
        \node[shape=circle,draw=black] (E) at (0,-2)    {};
        \node[shape=circle,draw=black] (F) at (-2,-2)   {};

        \node[shape=circle,draw=black] (G) at (2,2)     {};
        \node[shape=circle,draw=black] (H) at (0,2)     {};
        \node[shape=circle,draw=black] (I) at (-2,2)    {};

        \node[shape=circle,draw=black] (J) at (0,5)     {};
        \node[shape=circle,draw=black] (L) at (5,0)     {};

        \node[shape=circle,draw=black] (N) at (0,7)     {};
        \node[shape=circle,draw=black] (O) at (0,-7)    {};
        \node[shape=circle,draw=black] (P) at (-7,0)    {x+v};

        \node[shape=circle,draw=black] (A2) at (12,0)     {};
        \node[shape=circle,draw=black] (B2) at (10,0)     {};
        \node[shape=circle,draw=black] (C2) at (8,0)    {};

        \node[shape=circle,draw=black] (D2) at (12,-2)    {};
        \node[shape=circle,draw=black] (E2) at (10,-2)    {};
        \node[shape=circle,draw=black] (F2) at (8,-2)   {};

        \node[shape=circle,draw=black] (G2) at (12,2)     {};
        \node[shape=circle,draw=black] (H2) at (10,2)     {};
        \node[shape=circle,draw=black] (I2) at (8,2)    {};

        \node[shape=circle,draw=black] (J2) at (10,5)     {};
        \node[shape=circle,draw=black] (L2) at (15,0)     {};

        \node[shape=circle,draw=black] (N2) at (10,7)     {};
        \node[shape=circle,draw=black] (O2) at (10,-7)    {};

        \node[shape=circle,draw=black] (A3) at (22,0)     {};
        \node[shape=circle,draw=black] (B3) at (20,0)     {};
        \node[shape=circle,draw=black] (C3) at (18,0)    {};

        \node[shape=circle,draw=black] (D3) at (22,-2)    {};
        \node[shape=circle,draw=black] (E3) at (20,-2)    {};
        \node[shape=circle,draw=black] (F3) at (18,-2)   {};

        \node[shape=circle,draw=black] (G3) at (22,2)     {};
        \node[shape=circle,draw=black] (H3) at (20,2)     {};
        \node[shape=circle,draw=black] (I3) at (18,2)    {};

        \node[shape=circle,draw=black] (J3) at (20,5)     {};
        \node[shape=circle,draw=black] (L3) at (25,0)     {v'};

        \node[shape=circle,draw=black] (N3) at (20,7)     {};
        \node[shape=circle,draw=black] (O3) at (20,-7)    {};
        \node[shape=circle,draw=black] (Q3) at (27,0)    {y};



        \path [-] (B) edge node[left] {} (A);
        \path [-] (B) edge node[left] {} (C);
        \path [-] (C) edge node[left] {} (P);
        \path [-] (A) edge node[left] {} (L);
        \path [-] (B) edge node[left] {} (H);
        \path [-] (B) edge node[left] {} (E);
        \path [-] (H) edge node[left] {} (A);
        \path [-] (A) edge node[left] {} (E);
        \path [-] (E) edge node[left] {} (C);
        \path [-] (C) edge node[left] {} (H);
        \path [-] (H) edge node[left] {} (I);
        \path [-] (A) edge node[left] {} (G);
        \path [-] (E) edge node[left] {} (D);
        \path [-] (H) edge node[left] {} (J);
        \path [-] (E) edge node[left] {} (O);
        \path [-] (C) edge node[left] {} (F);
        \path [-] (P) edge node[left] {} (I);
        \path [-] (I) edge node[left] {} (J);
        \path [-] (J) edge node[left] {} (G);
        \path [-] (G) edge node[left] {} (L);
        \path [-] (L) edge node[left] {} (D);
        \path [-] (D) edge node[left] {} (O);
        \path [-] (O) edge node[left] {} (F);
        \path [-] (F) edge node[left] {} (P);
        \path [-] (N) edge node[left] {} (J);

        \path [-] (B2) edge node[left] {} (A2);
        \path [-] (B2) edge node[left] {} (C2);
        \path [-] (C2) edge node[left] {} (L);
        \path [-] (A2) edge node[left] {} (L2);
        \path [-] (B2) edge node[left] {} (H2);
        \path [-] (B2) edge node[left] {} (E2);
        \path [-] (H2) edge node[left] {} (A2);
        \path [-] (A2) edge node[left] {} (E2);
        \path [-] (E2) edge node[left] {} (C2);
        \path [-] (C2) edge node[left] {} (H2);
        \path [-] (H2) edge node[left] {} (I2);
        \path [-] (A2) edge node[left] {} (G2);
        \path [-] (E2) edge node[left] {} (D2);
        \path [-] (H2) edge node[left] {} (J2);
        \path [-] (E2) edge node[left] {} (O2);
        \path [-] (C2) edge node[left] {} (F2);
        \path [-] (L) edge node[left] {} (I2);
        \path [-] (I2) edge node[left] {} (J2);
        \path [-] (J2) edge node[left] {} (G2);
        \path [-] (G2) edge node[left] {} (L2);
        \path [-] (L2) edge node[left] {} (D2);
        \path [-] (D2) edge node[left] {} (O2);
        \path [-] (O2) edge node[left] {} (F2);
        \path [-] (F2) edge node[left] {} (L);
        \path [-] (N2) edge node[left] {} (J2);

        \path [-] (B3) edge node[left] {} (A3);
        \path [-] (B3) edge node[left] {} (C3);
        \path [-] (C3) edge node[left] {} (L2);
        \path [-] (A3) edge node[left] {} (L3);
        \path [-] (B3) edge node[left] {} (H3);
        \path [-] (B3) edge node[left] {} (E3);
        \path [-] (H3) edge node[left] {} (A3);
        \path [-] (A3) edge node[left] {} (E3);
        \path [-] (E3) edge node[left] {} (C3);
        \path [-] (C3) edge node[left] {} (H3);
        \path [-] (H3) edge node[left] {} (I3);
        \path [-] (A3) edge node[left] {} (G3);
        \path [-] (E3) edge node[left] {} (D3);
        \path [-] (H3) edge node[left] {} (J3);
        \path [-] (E3) edge node[left] {} (O3);
        \path [-] (C3) edge node[left] {} (F3);
        \path [-] (L2) edge node[left] {} (I3);
        \path [-] (I3) edge node[left] {} (J3);
        \path [-] (J3) edge node[left] {} (G3);
        \path [-] (G3) edge node[left] {} (L3);
        \path [-] (L3) edge node[left] {} (D3);
        \path [-] (D3) edge node[left] {} (O3);
        \path [-] (O3) edge node[left] {} (F3);
        \path [-] (F3) edge node[left] {} (L2);
        \path [-] (N3) edge node[left] {} (J3);

        \path [-] (L3) edge node[left] {} (Q3);
    \end{tikzpicture}
\end{center}
Dann müssen $x+v$ und $v'$ immernoch die selbe Farbe haben bei einer korrekten 3-Färbung, und die oben diskutierten Eigenschaften
bleiben erhalten.

\textbf{j)}
Der resultierende Graph ist planar, da sich in dem ursprünglichen Graphen in jedem Punkt höchstens 2 Streckenzüge
geschnitten haben, und die Schnittpunkte nun durch planare Kreuzungsgadgets ersetzt wurden. Damit schneiden sich
keine 2 Streckenzüge des resultierenden Graphen.

\textbf{k)}
Die Größe eines Kreuzungsgadgets ist konstant. Ferner können sich die Strecken des (sinnvollen) Eingabegraphen in höchstens $|E|^2$
Punkten schneiden (jede Kante jeder andere höchstens 1 mal) sodass wir höchstens $|E|^2$ der Kreuzungsgadgets
hinzufügen müssen. Die Schnittpunkte selbst lassen sich auch in polynomieller Zeit,
im Notfall mit numerischen Methoden bis auf Maschinengenauigkeit (wo wir dann das Gadget ''einstanzen'' können), finden.

\textbf{l)}
Angenommen der Eingabegraph hat eine 3-Färbung. Die Knoten im planaren Graphen, welche aus dem Eingabegraphen stammen
können ihre Farbe übernehmen. In einem Kreuzungsgadget übernehmen die 4 Knoten $u,u',v,v'$ die Farbe des gegenüberliegende Knoten,
insofern dieser gerade gefärbt wurde. Mit den Variablennamen der obigen Skizze: Damit hat der Knoten $x+v$ die Farbe von
$x$ im Eingabegraphen, und alle verschmolzenen Knoten auf dem Weg durch die 3 Gadgets übernehmen diese Farbe, inklusive
dem letzten dieser Knoten, $v'$. Dies stellt kein Problem für die Kante $(v',y)$ dar, da ja $(x,y)$ eine Kante im Eingabegraphen
war und damit $c(v') = c(x) \neq c(y)$. Also lässt sich nach e),f) die 3-Färbung auf die restlichen Knoten der Kreuzungsgadgets
erweitern.

Nun angenommen wir haben eine 3-Färbung des resultierenden planaren Graphen $G'$. Da $V \subseteq V'$, also alle Knoten
des Eingabegraphen in $V'$ vorkommen, können wir einfach die Farben übernehmen. Für $x,y \in V$ mit $(x,y) \in E$ haben wir dann:

\textbf{Fall 1:} $x$ und $y$ sind durch eine Kante in $G'$ verbunden, ohne Gadget (also $(x,y) \in E'$).\\
Dann folgt aber schon sofort $c(x) \neq c(y)$, da $c$ ja eine korrekte 3-Färbung von $G'$ ist.

\textbf{Fall 2:} $x$ und $y$ haben Kreuzungsgadget(s) zwischen sich (also $(x,y) \notin E$).\\
Dann haben wir einen Fall wie in der Skizze oben, und es folgt dass $y$ zu einem Knoten $v'$ adjazent ist,
der per Kontruktion die selbe Farbe wie $x$ haben muss. Folglich $c(y) \neq c(v') = c(x)$, da $c$ eine Korrekte
3-Färbung von $G'$ ist.

Damit folgt in beiden Fällen aber, dass $c(x) \neq c(y)$. Also haben wir eine korrekte 3-Färbung des Eingabegraphen $G$
durch Einschränkung von c auf $V$.


\end{document}
