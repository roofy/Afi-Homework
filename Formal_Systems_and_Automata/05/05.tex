\documentclass[a4paper,graphics,11pt]{article}
\pagenumbering{arabic}

\usepackage[margin=1in]{geometry}
\usepackage[utf8]{inputenc}
\usepackage[T1]{fontenc}
\usepackage{lmodern}
\usepackage[ngerman]{babel}
\usepackage{amsmath, tabu}
\usepackage{amsthm}
\usepackage{amssymb}
\usepackage{complexity}
\usepackage{mathtools}
\usepackage{setspace}
\usepackage{graphicx,color,curves,epsf,float,rotating}
\usepackage{tasks}
\setlength{\parindent}{0em}
\setlength{\parskip}{1em}
\usepackage{tikz}
\usetikzlibrary{automata, arrows}

\newcommand{\aufgabe}[1]{\subsection*{Aufgabe #1}}
\newcommand{\up}[2]{\mathrel{\overset{\makebox[0pt]{\mbox{\normalfont\tiny #2}}}{#1}}}

\begin{document}
\noindent Gruppe \fbox{\textbf{3}}             \hfill Phil Pützstück, 377247\\
\noindent Formale Systeme und Automaten \hfill Benedikt Gerlach, 376944\\
\strut\hfill Sebastian Hackenberg, 377550\\
\begin{center}
	\LARGE{\textbf{Hausaufgabe 5}}
\end{center}
\begin{center}
\rule[0.1ex]{\textwidth}{1pt}
\end{center}

\aufgabe{5}
\textbf{Wir berechnen:} $r_Q(q_0, q_0)$. Mit $x = q_1$ erhalten wir:
$$
    r_Q(q_0, q_0) =
    r_{\{q_0\}}(q_0, q_0)
    + r_{\{q_0\}}(q_0, q_1)r_{\{q_0\}}(q_1, q_1)^*r_{\{q_0\}}(q_1, q_0)
$$
\textbf{Wir berechnen:} $r_{\{q_0\}}(q_0, q_0)$. Mit $x = q_0$ erhalten wir:
\begin{align*}
    r_{\{q_0\}}(q_0, q_0) &=
    r_{\varnothing}(q_0, q_0)
    + r_{\varnothing}(q_0, q_0)r_{\varnothing}(q_0, q_0)^*r_{\varnothing}(q_0, q_0)\\
    &= (a+\varepsilon) + (a+\varepsilon)(a+\varepsilon)^*(a+\varepsilon)\\
    &= a^*
\end{align*}
\textbf{Wir berechnen:} $r_{\{q_0\}}(q_0, q_1)$. Mit $x = q_0$ erhalten wir:
\begin{align*}
    r_{\{q_0\}}(q_0, q_1)
    &= r_{\varnothing}(q_0, q_1)
    + r_{\varnothing}(q_0, q_0)r_{\varnothing}(q_0, q_0)^*r_{\varnothing}(q_0, q_1)\\
    &= (b+c) + (a+\varepsilon)(a+\varepsilon)^*(b+c)\\
    &= (b+c) + a^*(b+c)\\
    &= a^*(b+c)
\end{align*}
\textbf{Wir berechnen:} $r_{\{q_0\}}(q_1, q_1)$. Mit $x = q_0$ erhalten wir:
\begin{align*}
    r_{\{q_0\}}(q_1, q_1) &=
    r_{\varnothing}(q_1, q_1)
    + r_{\varnothing}(q_1, q_0)r_{\varnothing}(q_0, q_0)^*r_{\varnothing}(q_0, q_1)\\
    &= \varepsilon + a(a+\varepsilon)^*(b+c)\\
    &= \varepsilon + aa^*(b+c)
\end{align*}
\textbf{Wir berechnen:} $r_{\{q_0\}}(q_1, q_0)$. Mit $x = q_0$ erhalten wir:
\begin{align*}
    r_{\{q_0\}}(q_1, q_0) &=
    r_{\varnothing}(q_1, q_0)
    + r_{\varnothing}(q_1, q_0)r_{\varnothing}(q_0, q_0)^*r_{\varnothing}(q_0, q_0)\\
    &= a + a(a + \varepsilon)^*(a + \varepsilon)\\
    &= a + aa^*\\
    &= aa^*
\end{align*}
Durch Rückeinsetzen erhalten wir nun:
\begin{align*}
    r_Q(q_0, q_0) &=
    r_{\{q_0\}}(q_0, q_0)
    + r_{\{q_0\}}(q_0, q_1)r_{\{q_0\}}(q_1, q_1)^*r_{\{q_0\}}(q_1, q_0)\\
    &= a^* + a^*(b+c)(\varepsilon + aa^*(b+c))^*aa^*\\
    &= a^* + a^*(b+c)(aa^*(b+c))^*aa^*
\end{align*}
\newpage
\aufgabe{6}
\textbf{a)}
Angenommen, $L_1$ ist regulär. Wir wählen $n$ zu $L_1$ gemäß Pumping-Lemma und betrachten
das Wort $w = a^nb^nc^{2n} \in L_1$. Das Pumping-Lemma liefert Zerlegung
$$
    w = xyz
    \quad \text{mit}\quad
    |xy| \leq n
    \quad \text{und}\quad
    y\neq \varepsilon
    \quad \text{sowie}\quad
    xz = xy^0z \in L_1
$$
Wegen $|xy| \leq n$ und $y\neq \varepsilon$ gilt $x = a^j$ mit $j \geq 0$ und $y = a^k$ mit $k > 0$.\\[3pt]
Jedoch:
$$
    xz = a^{n-k}b^nc^{2n} \notin L_1
    \quad\text{weil}\quad
    k > 0 \,\Longrightarrow\, n-k + n \neq 2n
$$
Dies führt also zu einem Widerspruch. Folglich ist $L_1$ nicht regulär.

\textbf{b)}
Angenommen, $L_2$ ist regulär. Wir wählen $n$ zu $L_2$ gemäß Pumping-Lemma und betrachten
das Wort $w = b^na^{n+1} \in L_2$. Das Pumping-Lemma liefert Zerlegung
$$
    w = xyz
    \quad \text{mit}\quad
    |xy| \leq n
    \quad \text{und}\quad
    y\neq \varepsilon
    \quad \text{sowie}\quad
    xy^3z \in L_2
$$
Wegen $|xy| \leq n$ und $y \neq \varepsilon$ gilt $x = b^j$ mit $j \geq 0$ und $y = b^k$ mit $k > 0$.\\[3pt]
Jedoch:
$$
    xy^3z = a^{n+2k}b^{n+1} \notin L_2
    \quad\text{weil}\quad
    k > 0 \,\Longrightarrow\, 2k \geq 2 \,\Longrightarrow\, n+2k \not< n+1
$$
Dies führt also zu einem Widerspruch. Folglich ist $L_2$ nicht regulär.

\aufgabe{7}
Zuerst bilden teilen wir die Zustände in Endzustände und nicht-Endzustände:
$$
    \mathcal{B}_1 := \{q_0, q_1, q_5\}
    \qquad\qquad
    \mathcal{B}_2 := \{q_2, q_3, q_4\}
$$
Wir verfeinern $\mathcal{B}_2$ bzgl der $b$-Transition und $\mathcal{B}_1$:
$$
    \mathcal{B}_1 := \{q_0, q_1, q_5\}
    \qquad\qquad
    \mathcal{B}_3 := \{q_2\}
    \qquad\qquad
    \mathcal{B}_4 := \{q_3, q_4\}
$$
Wir verfeinern $\mathcal{B}_1$ bzgl. der $a$-Transition und $\mathcal{B}_3$:
$$
    \mathcal{B}_5 := \{q_1\}
    \qquad\qquad
    \mathcal{B}_6 := \{q_0, q_5\}
    \qquad\qquad
    \mathcal{B}_3 := \{q_2\}
    \qquad\qquad
    \mathcal{B}_4 := \{q_3, q_4\}
$$
Es lässt sich nun keine Zustandsmenge noch weiter verfeinern.
Der minimale DFA ist dann:
\begin{center}
    \begin{tikzpicture}[>=stealth',shorten >=1pt,auto,node distance=4cm]
        \node[state, initial, accepting]    (q6)                    {$\mathcal{B}_6$};
        \node[state]                        (q4)    [right of=q6]   {$\mathcal{B}_4$};
        \node[state,accepting]              (q5)    [below of=q6]   {$\mathcal{B}_5$};
        \node[state]                        (q3)    [right of=q5]   {$\mathcal{B}_3$};

        \path[->] (q6)  edge [bend left]    node {$a$} (q4);
        \path[->] (q6)  edge                node {$b$} (q5);

        \path[->] (q5)  edge                node {$a$} (q3);
        \path[->] (q5)  edge [bend left]    node {$b$} (q6);

        \path[->] (q4)  edge                node {$a$} (q6);
        \path[->] (q4)  edge [loop right]   node {$b$} (q4);

        \path[->] (q3)  edge [bend left]    node {$a$} (q5);
        \path[->] (q3)  edge                node {$b$} (q6);
    \end{tikzpicture}
\end{center}

\newpage
\aufgabe{8}

\textbf{a)}

Der Satz von Nerode besagt: $index(L) < \infty \Rightarrow$ L ist regulär.\\
Somit folgt: L ist nicht regulär $\Rightarrow index(L) = \infty$\\
Wir zeigen das L nicht regulär und somit der Index von L unendlich ist.

\textbf{Beweis durch Widerspruch}\\
Angenommen L ist Regulär.\\
Wähle n zu L gemäß Pumping Lemma und betracht w = $b^n aa $ $b^n$.\\
Puming Lemma liefert Zerlegung.\\
w = xyz mit |xy| $\leq$ n und y $\neq$ $\varepsilon$ und xz $\in$ L.\\
Wegen |xy| $\leq$ und y $\neq$ $\varepsilon$ gilt x = $b^j$ mit j $\geq$ 0 und y = $b^k$ mit k > 0.\\
Aber xz = $b^n-kaa$ $b^n$ $\not\in$ L, weil n-k > n.\\
\textbf{Widerspruch!}\\
Somit ist der Index von L unendlich.\\

\textbf{AUFBAU DER KLASSEN ?}

\textbf{b)}

\end{document}
