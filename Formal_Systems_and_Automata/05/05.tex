\documentclass[a4paper,graphics,11pt]{article}
\pagenumbering{arabic}

\usepackage[margin=1in]{geometry}
\usepackage[utf8]{inputenc}
\usepackage[T1]{fontenc}
\usepackage{lmodern}
\usepackage[ngerman]{babel}
\usepackage{amsmath, tabu}
\usepackage{amsthm}
\usepackage{amssymb}
\usepackage{complexity}
\usepackage{mathtools}
\usepackage{setspace}
\usepackage{graphicx,color,curves,epsf,float,rotating}
\usepackage{tasks}
\setlength{\parindent}{0em}
\setlength{\parskip}{1em}
\usepackage{tikz}
\usetikzlibrary{automata, arrows}

\newcommand{\aufgabe}[1]{\subsection*{Aufgabe #1}}
\newcommand{\up}[2]{\mathrel{\overset{\makebox[0pt]{\mbox{\normalfont\tiny #2}}}{#1}}}

\begin{document}
\noindent Gruppe \fbox{\textbf{3}}             \hfill Phil Pützstück, 377247\\
\noindent Formale Systeme und Automaten \hfill Benedikt Gerlach, 376944\\
\strut\hfill Sebastian Hackenberg, 377550\\
\begin{center}
	\LARGE{\textbf{Hausaufgabe 5}}
\end{center}
\begin{center}
\rule[0.1ex]{\textwidth}{1pt}
\end{center}

\aufgabe{5}
\textbf{Wir berechnen:} $r_Q(q_0, q_0)$. Mit $x = q_1$ erhalten wir:
$$
    r_Q(q_0, q_0) =
    r_{\{q_0\}}(q_0, q_0)
    + r_{\{q_0\}}(q_0, q_1)r_{\{q_0\}}(q_1, q_1)^*r_{\{q_0\}}(q_1, q_0)
$$
\textbf{Wir berechnen:} $r_{\{q_0\}}(q_0, q_0)$. Mit $x = q_0$ erhalten wir:
\begin{align*}
    r_{\{q_0\}}(q_0, q_0) &=
    r_{\varnothing}(q_0, q_0)
    + r_{\varnothing}(q_0, q_0)r_{\varnothing}(q_0, q_0)^*r_{\varnothing}(q_0, q_0)\\
    &= (a+\varepsilon) + (a+\varepsilon)(a+\varepsilon)^*(a+\varepsilon)\\
    &= a^*
\end{align*}
\textbf{Wir berechnen:} $r_{\{q_0\}}(q_0, q_1)$. Mit $x = q_0$ erhalten wir:
\begin{align*}
    r_{\{q_0\}}(q_0, q_1)
    &= r_{\varnothing}(q_0, q_1)
    + r_{\varnothing}(q_0, q_0)r_{\varnothing}(q_0, q_0)^*r_{\varnothing}(q_0, q_1)\\
    &= (b+c) + (a+\varepsilon)(a+\varepsilon)^*(b+c)\\
    &= (b+c) + a^*(b+c)\\
    &= a^*(b+c)
\end{align*}
\textbf{Wir berechnen:} $r_{\{q_0\}}(q_1, q_1)$. Mit $x = q_0$ erhalten wir:
\begin{align*}
    r_{\{q_0\}}(q_1, q_1) &=
    r_{\varnothing}(q_1, q_1)
    + r_{\varnothing}(q_1, q_0)r_{\varnothing}(q_0, q_0)^*r_{\varnothing}(q_0, q_1)\\
    &= \varepsilon + a(a+\varepsilon)^*(b+c)\\
    &= \varepsilon + aa^*(b+c)
\end{align*}
\textbf{Wir berechnen:} $r_{\{q_0\}}(q_1, q_0)$. Mit $x = q_0$ erhalten wir:
\begin{align*}
    r_{\{q_0\}}(q_1, q_0) &=
    r_{\varnothing}(q_1, q_0)
    + r_{\varnothing}(q_1, q_0)r_{\varnothing}(q_0, q_0)^*r_{\varnothing}(q_0, q_0)\\
    &= a + a(a + \varepsilon)^*(a + \varepsilon)\\
    &= a + aa^*\\
    &= aa^*
\end{align*}
Durch Rückeinsetzen erhalten wir nun:
\begin{align*}
    r_Q(q_0, q_0) &=
    r_{\{q_0\}}(q_0, q_0)
    + r_{\{q_0\}}(q_0, q_1)r_{\{q_0\}}(q_1, q_1)^*r_{\{q_0\}}(q_1, q_0)\\
    &= a^* + a^*(b+c)(\varepsilon + aa^*(b+c))^*aa^*\\
    &= a^* + a^*(b+c)(aa^*(b+c))^*aa^*
\end{align*}
\newpage
\aufgabe{6}
\textbf{a)}
Angenommen, $L_1$ ist regulär. Wir wählen $n$ zu $L_1$ gemäß Pumping-Lemma und betrachten
das Wort $w = a^nb^nc^{2n} \in L_1$. Das Pumping-Lemma liefert Zerlegung
$$
    w = xyz
    \quad \text{mit}\quad
    |xy| \leq n
    \quad \text{und}\quad
    y\neq \varepsilon
    \quad \text{sowie}\quad
    xz = xy^0z \in L_1
$$
Wegen $|xy| \leq n$ und $y\neq \varepsilon$ gilt $x = a^j$ mit $j \geq 0$ und $y = a^k$ mit $k > 0$.\\[3pt]
Jedoch:
$$
    xz = a^{n-k}b^nc^{2n} \notin L_1
    \quad\text{weil}\quad
    k > 0 \,\Longrightarrow\, n-k + n \neq 2n
$$
Dies führt also zu einem Widerspruch. Folglich ist $L_1$ nicht regulär.

\textbf{b)}
Angenommen, $L_2$ ist regulär. Wir wählen $n$ zu $L_2$ gemäß Pumping-Lemma und betrachten
das Wort $w = b^na^{n+1} \in L_2$. Das Pumping-Lemma liefert Zerlegung
$$
    w = xyz
    \quad \text{mit}\quad
    |xy| \leq n
    \quad \text{und}\quad
    y\neq \varepsilon
    \quad \text{sowie}\quad
    xy^3z \in L_2
$$
Wegen $|xy| \leq n$ und $y \neq \varepsilon$ gilt $x = b^j$ mit $j \geq 0$ und $y = b^k$ mit $k > 0$.\\[3pt]
Jedoch:
$$
    xy^3z = a^{n+2k}b^{n+1} \notin L_2
    \quad\text{weil}\quad
    k > 0 \,\Longrightarrow\, 2k \geq 2 \,\Longrightarrow\, n+2k \not< n+1
$$
Dies führt also zu einem Widerspruch. Folglich ist $L_2$ nicht regulär.

\aufgabe{7}
Zuerst bilden teilen wir die Zustände in Endzustände und nicht-Endzustände:
$$
    \mathcal{B}_1 := \{q_0, q_1, q_5\}
    \qquad\qquad
    \mathcal{B}_2 := \{q_2, q_3, q_4\}
$$
Wir verfeinern $\mathcal{B}_2$ bzgl der $b$-Transition und $\mathcal{B}_1$:
$$
    \mathcal{B}_1 := \{q_0, q_1, q_5\}
    \qquad\qquad
    \mathcal{B}_3 := \{q_2\}
    \qquad\qquad
    \mathcal{B}_4 := \{q_3, q_4\}
$$
Wir verfeinern $\mathcal{B}_1$ bzgl. der $a$-Transition und $\mathcal{B}_3$:
$$
    \mathcal{B}_5 := \{q_1\}
    \qquad\qquad
    \mathcal{B}_6 := \{q_0, q_5\}
    \qquad\qquad
    \mathcal{B}_3 := \{q_2\}
    \qquad\qquad
    \mathcal{B}_4 := \{q_3, q_4\}
$$
Es lässt sich nun keine Zustandsmenge noch weiter verfeinern.
Der minimale DFA ist dann:
\begin{center}
    \begin{tikzpicture}[>=stealth',shorten >=1pt,auto,node distance=4cm]
        \node[state, initial, accepting]    (q6)                    {$\mathcal{B}_6$};
        \node[state]                        (q4)    [right of=q6]   {$\mathcal{B}_4$};
        \node[state,accepting]              (q5)    [below of=q6]   {$\mathcal{B}_5$};
        \node[state]                        (q3)    [right of=q5]   {$\mathcal{B}_3$};

        \path[->] (q6)  edge [bend left]    node {$a$} (q4);
        \path[->] (q6)  edge                node {$b$} (q5);

        \path[->] (q5)  edge                node {$a$} (q3);
        \path[->] (q5)  edge [bend left]    node {$b$} (q6);

        \path[->] (q4)  edge                node {$a$} (q6);
        \path[->] (q4)  edge [loop right]   node {$b$} (q4);

        \path[->] (q3)  edge [bend left]    node {$a$} (q5);
        \path[->] (q3)  edge                node {$b$} (q6);
    \end{tikzpicture}
\end{center}

\newpage
\aufgabe{8}

\textbf{a)}

Seien $w,v \in \Sigma^*$ mit $w \neq v$ gegeben. Es gelte o.B.d.A. $|w| \geq |v|$. Wir unterscheiden 2 Fälle:

\textbf{Fall 1:} es ist $v \not\sqsubset w$, also $v$ kein echtes Präfix von $w$.\\[5pt]
Dann gilt offensichtlich $ww^\mathcal{R} \in L$. Wäre nun $vw^\mathcal{R} \in L$, so folgt auch
$$
    vw^\mathcal{R} = (vw^\mathcal{R})^\mathcal{R} = wv^\mathcal{R} \,\Longrightarrow\, v \sqsubset w
$$
Dies stellt einen Widerspruch zur Annahme dar. Also gibt es ein $u = w^\mathcal{R} \in \Sigma^*$ sodass
$wu \in L$ aber $vu \notin L$. Folglich gilt $w \not\sim_L v \,\Longrightarrow\, w/_L \neq v/_L$.

\textbf{Fall 2:} es ist $v \sqsubset w$, also $v$ ein echtes Präfix von $w$, also auch $|v| < |w|$.\\[5pt]
Dann sei ein $x \in \Sigma$ gegeben sodass $vx \not\sqsubset w$, also $vx$ kein echtes Präfix mehr von $w$ ist.
Dann gilt offensichtlich $wxw^\mathcal{R} \in L$. Wäre nun $vxw^\mathcal{R} \in L$, so folgt auch
$$
    vxw^\mathcal{R} = (vxw^\mathcal{R})^\mathcal{R} = wxv^\mathcal{R} \,\Longrightarrow\, vx \sqsubset w
$$
Dies stellt einen Widerspruch zur Annahme dar. Also gibt es ein $u = xw^\mathcal{R} \in \Sigma^*$ sodass
$wu \in L$ aber $vu \notin L$. Folglich gilt $w \not\sim_L v \,\Longrightarrow\, w/_L \neq v/_L$.

Insgesamt gilt für verschiedene Worte $w,v \in \Sigma^*$ stets $w/_L \neq v/_L$.
Dementsprechend gilt\\
$\forall w \in \Sigma^* : w/_L = \{w\}$, also auch index$(L) = \infty$.
Die trennenden Wörter sind zu je zwei Wörtern $w,v \in \Sigma^*$ wie oben beschrieben je nach Fall zu finden,
also $w^R$ oder $xw^R$ für ein $x \in \Sigma$.

\textbf{b)} Es gilt:

$\varepsilon \not\sim_K a$, denn $\varepsilon b = b \in K$ jedoch $ab \notin K$. Analog gilt
$\varepsilon a = a \in K$ jedoch $ba \notin K$, also $\varepsilon \not\sim_K b$.
Weiter haben wir $a \not\sim_K b$ durch $aa \in K, ba \notin K$.

Dann haben wir noch $ab$ sowie $ba$ und es gilt:\\
$\varepsilon \not\sim_K ab$ durch $\varepsilon\varepsilon = \varepsilon \in K, ab\varepsilon = ab \notin K$.
$\varepsilon \not\sim_K ba$ durch $\varepsilon\varepsilon = \varepsilon \in K, ba\varepsilon = ba \notin K$.\\
$a \not\sim_K ab$ durch $a\varepsilon = a \in K, ab\varepsilon = ab \notin K$.
$a \not\sim_K ba$ durch $a\varepsilon = a \in K, ba\varepsilon = ba \notin K$.\\
$b \not\sim_K ab$ durch $b\varepsilon = b \in K, ab\varepsilon = ab \notin K$.
$b \not\sim_K ba$ durch $b\varepsilon = b \in K, ba\varepsilon = ba \notin K$.\\
$ab \not\sim_K ba$ durch $aba = aba \in K, baa = ba \notin K$.

Sei nun $w \in \Sigma^*$ mit $w \notin \{\varepsilon, a, b, ab, ba\}$, sonst ist die Angehörigkeit zu einer
der Äquivalenzklassen offensichtlich.

Wir unterscheiden $3$ Fälle:

\textbf{Fall 1:} $|w|_{ab} = |w|_{ba}$. Dann ist $w = vc$ für ein $v \in \Sigma^*$ und $c \in \{a,b\}$.\\
Für $x \in \Sigma^*$ gilt stets
$$
    |wx|_{ab} - |wx|_{ba} = |w|_{ab} + |cx|_{ab} - |w|_{ba} - |cx|_{ba} = |cx|_{ab} - |cx|_{ba}
$$
Da jedes Infix $ab$ bzw. $ba$ aus $wx$ entweder in $vc$, oder in $cx$ vorkommt (da $|ab| = |ba| = 2$).\\
Es folgt $wx \in K \,\Longleftrightarrow\, cx \in K$ und damit $w \in c/_K (= a/_K\ \text{oder}\ b/_K)$.

\newpage

\textbf{Fall 2:} $|w|_{ab} > |w|_{ba}$.

Wenn $w = cvc$ für ein $c \in \Sigma$ und $v \in \Sigma^*$ wäre, so hätten
wir zu jedem Infix $ab$ aus $w$ genau ein Infix $ba$ in $w$, also $|w|_{ab} = |w|_{ba}$. Weiter würde
im Fall $w = bva$ für ein $v \in \Sigma^*$ stets $|w|_{ba} > |w|_{ab}$ gelten.\\
Also ist $w = avb$ für $v \in \Sigma^*$.

Für $x \in \Sigma^*$ gilt stets
$$
    |wx|_{ab} - |wx|_{ba} = |avbx|_{ab} - |avbx|_{ba} = |avb|_{ab} + |bx|_{ab} - |avb|_{ba} - |bx|_{ba}
$$$$
    = 1 + |x|_{ab} - |bx|_{ba}
    = |abx|_{ab} - |abx|_{ba}
$$
Analog zu Fall 1, da jedes Infix $ab$ bzw. $ba$ aus $avbx$ entweder in $avb$ oder $bx$ vorkommen muss (da
$|ab| = |ba| = 2$). Es folgt $wx \in K \,\Longleftrightarrow\, abx \in K$ und damit $w \in ab/_K$.

\textbf{Fall 3:} $|w|_{ba} > |w|_{ab}$.

Dies ist analog zu Fall 2. Es muss $w = bva$ für ein $v \in \Sigma^*$. Dann gilt für $x \in \Sigma^*$ stets
$$
    |wx|_{ba} - |wx|_{ab} = |bax|_{ba} - |bax|_{ab}
$$
Damit folgt $wx \in K \,\Longleftrightarrow\, ba \in K$ also $w \in ba/_K$.

Insgesamt gilt für alle $w \in \Sigma^*$, dass $w$ in eine der genannten Äquivalenzklassen gehört.
Da auch alle dieser verschieden sind, haben wir index$(K) = 5$.



\end{document}
