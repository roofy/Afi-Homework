\documentclass[a4paper,graphics,11pt]{article}
\pagenumbering{arabic}

\usepackage[margin=1in]{geometry}
\usepackage[utf8]{inputenc}
\usepackage[T1]{fontenc}
\usepackage{lmodern}
\usepackage[ngerman]{babel}
\usepackage{amsmath, tabu}
\usepackage{amsthm}
\usepackage{amssymb}
\usepackage{complexity}
\usepackage{mathtools}
\usepackage{setspace}
\usepackage{graphicx,color,curves,epsf,float,rotating}
\usepackage{tasks}
\usepackage{tikz}
\usetikzlibrary{automata, arrows}

\setlength{\parindent}{0em}
\setlength{\parskip}{1em}

\newcommand{\aufgabe}[1]{\subsection*{Aufgabe #1}}
\newcommand{\up}[2]{\mathrel{\overset{\makebox[0pt]{\mbox{\normalfont\tiny #2}}}{#1}}}

\begin{document}
\noindent Gruppe \fbox{\textbf{3}}             \hfill Phil Pützstück, 377247\\
\noindent Formale Systeme und Automaten \hfill Benedikt Gerlach, 376944\\
\strut\hfill Sebastian Hackenberg, 377550\\
\begin{center}
	\LARGE{\textbf{Hausaufgabe 2}}
\end{center}
\begin{center}
\rule[0.1ex]{\textwidth}{1pt}
\end{center}



\aufgabe{6}

\begin{tikzpicture}[>=stealth',shorten >=1pt,auto,node distance=4cm]
	\node[state, initial, accepting]	(p0q0)						{$p_0,q_0$};
	\node[state, accepting]				(p1q0) 	[right of=p0q0]		{$p_1,q_0$};
	\node[state]						(p2q0) 	[right of=p1q0]		{$p_2,q_0$};
	\node[state]						(p0q1) 	[below of=p0q0]		{$p_0,q_1$};
	\node[state]						(p1q1) 	[right of=p0q1]		{$p_1,q_1$};
	\node[state]						(p2q1) 	[right of=p1q1]		{$p_2,q_1$};
	
	\path[->] (p0q0) edge					node {b} (p1q0);
	\path[->] (p0q0) edge					node {a} (p0q1);
	
	\path[->] (p1q0) edge [loop above]		node {b} (p1q0);
	\path[->] (p1q0) edge [pos=0.1]			node {a} (p2q1);
	
	\path[->] (p2q0) edge [loop above]		node {b} (p2q0);
	\path[->] (p2q0) edge [bend left]		node {a} (p2q1);
	
	\path[->] (p0q1) edge [bend left]		node {a} (p0q0);
	\path[->] (p0q1) edge					node {b} (p1q1);
	
	\path[->] (p1q1) edge [pos=0.1]			node {a} (p2q0);
	\path[->] (p1q1) edge [loop below]		node {b} (p1q1);
	
	\path[->] (p2q1) edge [loop below]		node {b} (p2q1);
	\path[->] (p2q1) edge 					node {a} (p2q0);

\end{tikzpicture}

Die Endzustände wurden so gewählt damit die akzeptierenden Läufe von $L(\mathcal{A})$ enthalten sind und die akzeptierenden Läufe von $L(\mathcal{B})$ nicht.

\aufgabe{7}
Sei $\mathcal{A} = (Q, \Sigma, \delta, q_0, F)$ ein DFA mit Sprache $L(\mathcal{A})$.
Wir definieren den $\varepsilon$-NFA $\mathcal{B} := (Q', \Sigma', \Delta, q_{-1}, F)$,
wobei $Q' := Q \cup \{q_{-1}\}$, $\Sigma' := \Sigma \cup \{\varepsilon\}$ und
$$
	\Delta := \{(q, a, q') \mid \delta(q, a) = q'\}
		\cup \{(q_{-1}, \varepsilon, f) \mid f \in F\}
$$
für $q,q' \in Q$ und $a \in \Sigma$.

Wir zeigen zuerst $L_{\text{suff}}(\mathcal{A}) \subseteq L(\mathcal{B})$.\\[2pt]
Sei $w \in L_{\text{suff}}(\mathcal{A})$ gegeben. Seien ferner $m,n \in \mathbb{N}$.\\
Dann existiert ein $u \in L(\mathcal{A})$ sodass $uw \in L(\mathcal{A})$.
Durch $u \in L(\mathcal{A})$ existiert ein Lauf von $\mathcal{A}$ über $u$;
$(r_0, r_1 \cdots r_m)$ sodass $r_0 = q_0$ und $r_n \in F$.
Ferner gibt es durch $uw \in L(\mathcal{A})$ eine Zustandsfolge
$(x_0, \sigma_1, x_1, \sigma_2 \cdots \sigma_n, x_n)$ sodass $x_0 = r_m$, $x_n \in F$ und
$(\sigma_1, \sigma_2, \cdots \sigma_n) = w$. Nun können wir den Lauf $(q_{-1}, \varepsilon, x_0, \sigma_1, x_1, \sigma_2, x_2, \cdots \sigma_n, x_n)$ in $\mathcal{B}$ angeben. Da beide
Automaten die selben Endzustände haben und $q_{-1}$ der Startzustand von $\mathcal{B}$ ist, folgt daraus $w \in L(\mathcal{B})$.
\newpage

Wir zeigen nun $L(\mathcal{B}) \subseteq L_{\text{suff}}(\mathcal{A})$.\\[2pt]
Sei $w \in L(\mathcal{B})$ gegeben. Sei ferner $n \in \mathbb{N}$.\\
Dann existiert ein Lauf $r := (q_{-1}, \varepsilon, r_0, \sigma_1 \cdots, \sigma_n, r_n)$
in $\mathcal{B}$ mit $(\sigma_1 \cdots \sigma_n) = w$ und $r_0, r_n \in F$.
Da $F$ eben die Endzustände von $\mathcal{A}$ sind, gibt es auch ein Wort $u \in L(\mathcal{A})$ sodass der Lauf von $\mathcal{A}$ über $u$ eben an $r_1 \in F$ endet. Nun lässt sich der Lauf
von $w$ ohne die $\varepsilon$-Transition $(r_0, \sigma_1, \cdots \sigma_n, r_n)$ einwandfrei
an den von $u$ in dem DFA $\mathcal{A}$ anhängen um zu einem Weitern Endzustand von $\mathcal{A}$ zu kommen. Es folgt $uw \in L(\mathcal{A})$ und damit $w \in L_{\text{suff}}(\mathcal{A})$.

Insgesamt gilt also $L(\mathcal{B}) = L_{\text{suff}}(\mathcal{A})$. Aus der Vorlesung ist
bekannt, dass alle Sprachen, welche $\varepsilon$-NFA-erkennbar sind auch DFA-erkennbar sind.
$\hfill\square$

\aufgabe{8}
\textbf{a)}\\

\begin{tikzpicture}[>=stealth',shorten >=1pt,auto,node distance=2cm]
	\node[state, initial]					(q0)				{$q_0$};
	\node[state]						(q1) 	[above of=q0]	{$q_1$};
	\node[state]						(q2) 	[right of=q0]		{$q_2$};
	\node[state]						(q3) 	[right of=q2]		{$q_3$};
	\node[state, accepting]				(q4) 	[right of=q3]		{$q_4$};
	\node[state]						(q5) 	[right of=q4]		{$q_5$};
	

	\path[->] (q0) edge	[bend left]		node {a,b,c} (q1);
	\path[->] (q0) edge					node {c} (q2);
	
	\path[->] (q1) edge					node {a,b,c} (q0);
	
	\path[->] (q2) edge	 				node {c} (q3);
	
	\path[->] (q3) edge					node {c} (q4);

	\path[->] (q4) edge					node {a,b,c} (q5);

	\path[->] (q5) edge	[bend left]			node {a,b,c} (q4);

\end{tikzpicture}

\textbf{b)}

Alle läufe von $\mathcal{A}$ auf dem Wort bbba:\\
($q_0,b,q_0,b,q_0,b,q_1,a,q_2$)\\
($q_0,b,q_1,b,q_0,b,q_1,a,q_2$)\\

\textbf{c)}
Das Wort cbbbca wird akzeptiert, wenn die Präfixe die Erreichbarkeitsmenge\\
$E(\mathcal{A},w):=\{q_0,q_1\}$  haben.

Wenn jedoch für  die Erreichbarkeitsmenge der Präfixe des Wortes cbbbca $q_2 \in E(\mathcal{A},w)$ gilt\\,wird  das Wort nicht akzeptiert.



\newpage
\aufgabe{9}
Wir zeigen ($w \in L(\mathcal{A}) \,\Longrightarrow\, ac$ kommt nicht als Infix in $w$ vor) für Wörter $w$ der Länge
$n \in \mathbb{N}$.

Sei $n = 0$. Es folgt sofort $w = \varepsilon$. Weiter gilt $\varepsilon \in L(\mathcal{A})$. Ferner kann in $w$ nicht $ac$ als Infix vorkommen.
Damit hält die Aussage für $n = 0$.

Sei nun ein beliebig aber festes $n \in \mathbb{N}$ gegeben sodass
für jedes Wort $w'$ mit $|w'| = n$ aus $w' \in L(\mathcal{A})$ folgt, dass $ac$ nicht als Infix in $w'$ vorkommt (IV).

Sei dann $w = w'x$ mit $w \in L(\mathcal{A})$ für $w' \in L(\mathcal{A})$ mit $|w'| = n$ und $x \in \Sigma_\mathcal{A}$ gegeben.\\
Dann ist $|w| = n+1$. Wir unterscheiden 2 Fälle:

\textbf{Fall 1:} Der Lauf von $\mathcal{A}$ auf $w' = (w'_0, \cdots, w'_n)$ endet in $q_a \in F_{\mathcal{A}}$.\\
Dann gilt nach IV dass $ac$ nicht in $w'$ als Infix vorkommt. Jedoch gilt $w'_n = a$. \\
Es folgt aber $x \neq c$, da sonst durch $\delta_\mathcal{A}(q_a, c) = q_{ac}$ und $q_{ac} \notin F_{\mathcal{A}}$ direkt $w'x = w \notin L(\mathcal{A})$ folgen würde. Folglich kann auch in $w$ nicht das Infix $ac$ vorkommen.

\textbf{Fall 2:} Der Lauf von $\mathcal{A}$ auf $w' = (w'_0, \cdots, w'_n)$ endet in $q_{\varepsilon} \in F_{\mathcal{A}}$.\\
Dann gilt nach IV dass $ac$ nicht in $w'$ als Infix vorkommt. Weiter ist $w'_n \neq a$. Damit folgt
für $x \in \Sigma_\mathcal{A}$, dass $w = w'x$ ebenfalls nicht $ac$ als Infix enthält.


Insgesamt folgt die Behauptung in allen Fällen für $n+1$. Nach dem Prinzip der vollständigen Induktion
gilt nun für alle Wörter $w \in \Sigma_\mathcal{A}^*$, dass
$$
    w \in L(\mathcal{A}) \,\Longrightarrow\, ac\ \text{kommt nicht als Infix in}\ w\ \text{vor}
$$

Wir zeigen ($ac$ kommt nicht als Infix in $w$ vor $\,\Longrightarrow\, w \in L(\mathcal{A})$) für Wörter $w$ der
Länge $n \in \mathbb{N}$.

Sei $n = 0$. Für jedes Wort $w$ mit $|w| = n$ folgt dann $w = \varepsilon$. Offensichtlich kann $ac$ nicht als
Infix in $w$ vorkommen. Weiter ist der Lauf von $\mathcal{A}$ über $\varepsilon$ akzeptierend, es folgt also
$w \in L(\mathcal{A})$.

Sei nun ein beliebig aber festes $n \in \mathbb{N}$ gegeben sodass für jedes Wort $w'$ mit $|w'| = n$ in welchem
nicht $ac$ als Infix vorkommt folgt, dass $w' \in L(\mathcal{A})$ (IV).

Sei dann $w = w'x$ mit sodass $ac$ nicht als Infix in $w, w'$ vorkommt mit $x \in \Sigma_\mathcal{A}$ gegeben.\\
Dann ist $|w| = n+1$. Wir unterscheiden 2 Fälle:

\textbf{Fall 1:} $w'_n = a$.\\
Da $ac$ nicht in $w = w'x$ als Infix vorkommt folgt sofort $x \neq c$. Durch $w' \in L(\mathcal{A})$ (IV) und
$w'_n = a$ folgt, dass der Lauf von $\mathcal{A}$ über $w'$ in $q_a$ endet. Da dann das nächste eingelesene Symbol,
$x \neq c$ ist, folgt, dass $w = w'x \in L(\mathcal{A})$.

\textbf{Fall 2:} $w'_n \neq a$.\\
Da $w' \in L(\mathcal{A})$ nach IV endet der Lauf von $\mathcal{A}$ über $w'$ in $q_\varepsilon$. Es gilt
für jedes $x \in \Sigma_\mathcal{A}$ nun immernoch, dass $ac$ nicht als Infix in $w = w'x$ vorkommt. Weiter gilt
$\forall x \in \Sigma_\mathcal{A} : \delta_\mathcal{A}(q_\varepsilon, x) \in F_\mathcal{A}$.
Folglich ist also $w \in L(\mathcal{A})$.

Insgesamt folgt die Behauptung in allen Fällen für $n+1$. Nach dem Prinzip der vollständigen Indultion gilt
nun für alle Wörter $w \in \Sigma_\mathcal{A}^*$, dass
$$
    ac\ \text{kommt nicht als Infix in}\ w\ \text{vor} \,\Longrightarrow\, w \in L(\mathcal{A})
$$
$\hfill\square$
\end{document}
