\documentclass[a4paper,graphics,11pt]{article}
\pagenumbering{arabic}

\usepackage[margin=1in]{geometry}
\usepackage[utf8]{inputenc}
\usepackage[T1]{fontenc}
\usepackage{lmodern}
\usepackage[ngerman]{babel}
\usepackage{amsmath, tabu}
\usepackage{amsthm}
\usepackage{amssymb}
\usepackage{complexity}
\usepackage{mathtools}
\usepackage{setspace}
\usepackage{graphicx,color,curves,epsf,float,rotating}
\usepackage{tasks}
\usepackage{tikz}
\usetikzlibrary{automata, arrows}

\setlength{\parindent}{0em}
\setlength{\parskip}{1em}

\newcommand{\aufgabe}[1]{\subsection*{Aufgabe #1}}
\newcommand{\up}[2]{\mathrel{\overset{\makebox[0pt]{\mbox{\normalfont\tiny #2}}}{#1}}}

\begin{document}
\noindent Gruppe \fbox{\textbf{3}}             \hfill Phil Pützstück, 377247\\
\noindent Formale Systeme und Automaten \hfill Benedikt Gerlach, 376944\\
\strut\hfill Sebastian Hackenberg, 377550\\
\begin{center}
	\LARGE{\textbf{Hausaufgabe 2}}
\end{center}
\begin{center}
\rule[0.1ex]{\textwidth}{1pt}
\end{center}



\aufgabe{6}

\begin{tikzpicture}[>=stealth',shorten >=1pt,auto,node distance=4cm]
	\node[state, initial, accepting]	(p0q0)						{$p_0,q_0$};
	\node[state, accepting]				(p1q0) 	[right of=p0q0]		{$p_1,q_0$};
	\node[state]						(p2q0) 	[right of=p1q0]		{$p_2,q_0$};
	\node[state]						(p0q1) 	[below of=p0q0]		{$p_0,q_1$};
	\node[state]						(p1q1) 	[right of=p0q1]		{$p_1,q_1$};
	\node[state]						(p2q1) 	[right of=p1q1]		{$p_2,q_1$};
	
	\path[->] (p0q0) edge					node {b} (p1q0);
	\path[->] (p0q0) edge					node {a} (p0q1);
	
	\path[->] (p1q0) edge [loop above]		node {b} (p1q0);
	\path[->] (p1q0) edge					node {a} (p2q1);
	
	\path[->] (p2q0) edge [loop above]		node {b} (p2q0);
	\path[->] (p2q0) edge [bend left]		node {a} (p2q1);
	
	\path[->] (p0q1) edge [bend left]		node {a} (p0q0);
	\path[->] (p0q1) edge					node {b} (p1q1);
	
	\path[->] (p1q1) edge 					node {a} (p2q0);
	\path[->] (p1q1) edge [loop below]		node {b} (p1q1);
	
	\path[->] (p2q1) edge [loop below]		node {b} (p2q1);
	\path[->] (p2q1) edge 					node {a} (p2q0);

\end{tikzpicture}

\end{document}
