\documentclass[a4paper,graphics,11pt]{article}
\pagenumbering{arabic}

\usepackage[margin=1in]{geometry}
\usepackage[utf8]{inputenc}
\usepackage[T1]{fontenc}
\usepackage{lmodern}
\usepackage[ngerman]{babel}
\usepackage{amsmath, tabu}
\usepackage{amsthm}
\usepackage{amssymb}
\usepackage{complexity}
\usepackage{mathtools}
\usepackage{setspace}
\usepackage{graphicx,color,curves,epsf,float,rotating}
\usepackage{tasks}
\usepackage{tikz}
\usetikzlibrary{automata, arrows}

\setlength{\parindent}{0em}
\setlength{\parskip}{1em}

\newcommand{\aufgabe}[1]{\subsection*{Aufgabe #1}}
\newcommand{\up}[2]{\mathrel{\overset{\makebox[0pt]{\mbox{\normalfont\tiny #2}}}{#1}}}

\begin{document}
\noindent Gruppe \fbox{\textbf{3}}             \hfill Phil Pützstück, 377247\\
\noindent Formale Systeme und Automaten \hfill Benedikt Gerlach, 376944\\
\strut\hfill Sebastian Hackenberg, 377550\\
\begin{center}
	\LARGE{\textbf{Hausaufgabe 2}}
\end{center}
\begin{center}
\rule[0.1ex]{\textwidth}{1pt}
\end{center}



\aufgabe{6}

\begin{tikzpicture}[>=stealth',shorten >=1pt,auto,node distance=4cm]
	\node[state, initial, accepting]	(p0q0)						{$p_0,q_0$};
	\node[state, accepting]				(p1q0) 	[right of=p0q0]		{$p_1,q_0$};
	\node[state]						(p2q0) 	[right of=p1q0]		{$p_2,q_0$};
	\node[state]						(p0q1) 	[below of=p0q0]		{$p_0,q_1$};
	\node[state]						(p1q1) 	[right of=p0q1]		{$p_1,q_1$};
	\node[state]						(p2q1) 	[right of=p1q1]		{$p_2,q_1$};
	
	\path[->] (p0q0) edge					node {b} (p1q0);
	\path[->] (p0q0) edge					node {a} (p0q1);
	
	\path[->] (p1q0) edge [loop above]		node {b} (p1q0);
	\path[->] (p1q0) edge [pos=0.2]			node {a} (p2q1);
	
	\path[->] (p2q0) edge [loop above]		node {b} (p2q0);
	\path[->] (p2q0) edge [bend left]		node {a} (p2q1);
	
	\path[->] (p0q1) edge [bend left]		node {a} (p0q0);
	\path[->] (p0q1) edge					node {b} (p1q1);
	
	\path[->] (p1q1) edge [pos=0.2]			node {a} (p2q0);
	\path[->] (p1q1) edge [loop below]		node {b} (p1q1);
	
	\path[->] (p2q1) edge [loop below]		node {b} (p2q1);
	\path[->] (p2q1) edge 					node {a} (p2q0);

\end{tikzpicture}

Die Endzustände wurden so gewählt damit die akzeptierenden Läufe von $L(\mathcal{A})$ enthalten sind und die akzeptierenden Läufe von $L(\mathcal{B})$ nicht.

\aufgabe{7}
Sei $\mathcal{A} = (Q, \Sigma, \delta, q_0, F)$ ein DFA mit Sprache $L(\mathcal{A})$.
Wir definieren den $\varepsilon$-NFA $\mathcal{B} := (Q', \Sigma', \Delta, q_{-1}, F)$,
wobei $Q' := Q \cup \{q_{-1}\}$, $\Sigma' := \Sigma \cup \{\varepsilon\}$ und
$$
	\Delta := \{(q, a, q') \mid \delta(q, a) = q'\}
		\cup \{(q_{-1}, \varepsilon, f) \mid f \in F\}
$$
für $q,q' \in Q$ und $a \in \Sigma$.

Wir zeigen zuerst $L_{\text{suff}}(\mathcal{A}) \subseteq L(\mathcal{B})$.\\[2pt]
Sei $w \in L_{\text{suff}}(\mathcal{A})$ gegeben. Seien ferner $m,n \in \mathbb{N}$.\\
Dann existiert ein $u \in L(\mathcal{A})$ sodass $uw \in L(\mathcal{A})$.
Durch $u \in L(\mathcal{A})$ existiert ein Lauf von $\mathcal{A}$ über $u$;
$(r_0, r_1 \cdots r_m)$ sodass $r_0 = q_0$ und $r_n \in F$.
Ferner gibt es durch $uw \in L(\mathcal{A})$ eine Zustandsfolge
$(x_0, \sigma_1, x_1, \sigma_2 \cdots \sigma_n, x_n)$ sodass $x_0 = r_m$, $x_n \in F$ und
$(\sigma_1, \sigma_2, \cdots \sigma_n) = w$. Nun können wir den Lauf $(q_{-1}, \varepsilon, x_0, \sigma_1, x_1, \sigma_2, x_2, \cdots \sigma_n, x_n)$ in $\mathcal{B}$ angeben. Da beide
Automaten die selben Endzustände haben und $q_{-1}$ der Startzustand von $\mathcal{B}$ ist, folgt daraus $w \in L(\mathcal{B})$.
\newpage

Wir zeigen nun $L(\mathcal{B}) \subseteq L_{\text{suff}}(\mathcal{A})$.\\[2pt]
Sei $w \in L(\mathcal{B})$ gegeben. Sei ferner $n \in \mathbb{N}$.\\
Dann existiert ein Lauf $r := (r_0, \sigma_1, r_1, \sigma_2 \cdots, \sigma_n, r_n)$
in $\mathcal{B}$ mit $r_0 = q_{-1}$, $\sigma_1 = \varepsilon$, $(\sigma_2 \cdots \sigma_n) = w$ und $r_1, r_n \in F$.
Da $F$ eben die Endzustände von $\mathcal{A}$ sind, gibt es auch ein Wort $u \in L(\mathcal{A})$ sodass der Lauf von $\mathcal{A}$ über $u$ eben an $r_1 \in F$ endet. Nun lässt sich der Lauf
von $w$ ohne die $\varepsilon$-Transition $(r_1, \sigma_2, \cdots \sigma_n, r_n)$ einwandfrei
an den von $u$ in dem DFA $\mathcal{A}$ anhängen um zu einem Weitern Endzustand von $\mathcal{A}$ zu kommen. Es folgt $uw \in L(\mathcal{A})$ und damit $w \in L_{\text{suff}}(\mathcal{A})$.

Insgesamt gilt also $L(\mathcal{B}) = L_{\text{suff}}(\mathcal{A})$. Aus der Vorlesung ist
bekannt, dass alle Sprachen, welche $\varepsilon$-NFA-erkennbar sind auch DFA-erkennbar sind.
$\hfill\square$

\aufgabe{8}
\textbf{a)}\\

\begin{tikzpicture}[>=stealth',shorten >=1pt,auto,node distance=4cm]
	\node[state, initial, accepting]	(q0)						{$q_0$};
	\node[state]						(q1) 	[right of=q0]		{$q_1$};
	\node[state]						(q2) 	[below of=q0]		{$q_2$};
	\node[state]						(q3) 	[right of=q2]		{$q_3$};

	\path[->] (q0) edge	[bend left]		node {u} (q1);
	\path[->] (q0) edge					node {v} (q2);
	
	\path[->] (q1) edge					node {u} (q0);
	\path[->] (q1) edge	[bend left]		node {v} (q3);
	
	\path[->] (q2) edge	[bend left]		node {v} (q0);
	\path[->] (q2) edge	 				node {u} (q3);
	
	\path[->] (q3) edge					node {v} (q1);
	\path[->] (q3) edge	[bend left]		node {v} (q2);

\end{tikzpicture}



\textbf{b)}

Alle läufe von $\mathcal{A}$ auf dem Wort bbba:\\
($q_0,b,q_0,b,q_0,b,q_1,a,q_2$)\\
($q_0,b,q_1,b,q_0,b,q_1,a,q_2$)\\

\textbf{c)}

$E(\mathcal{A},w):=\{q_0,q_1\}$

Das Wort \textit{cbbca} wird akzeptiert.\\

\end{document}
