\documentclass[a4paper,graphics,11pt]{article}
\pagenumbering{arabic}

\usepackage[margin=1in]{geometry}
\usepackage[utf8]{inputenc}
\usepackage[T1]{fontenc}
\usepackage{lmodern}
\usepackage[ngerman]{babel}
\usepackage{amsmath, tabu}
\usepackage{amsthm}
\usepackage{amssymb}
\usepackage{complexity}
\usepackage{mathtools}
\usepackage{setspace}
\usepackage{graphicx,color,curves,epsf,float,rotating}
\usepackage{tasks}
\setlength{\parindent}{0em}
\setlength{\parskip}{1em}

\newcommand{\aufgabe}[1]{\subsection*{Aufgabe #1}}
\newcommand{\up}[2]{\mathrel{\overset{\makebox[0pt]{\mbox{\normalfont\tiny #2}}}{#1}}}

\begin{document}
\noindent Gruppe \fbox{\textbf{3}}             \hfill Phil Pützstück, 377247\\
\noindent Formale Systeme und Automaten \hfill Benedikt Gerlach, 376944\\
\strut\hfill Sebastian Hackenberg, 377550\\
\begin{center}
	\LARGE{\textbf{Hausaufgabe 1}}
\end{center}
\begin{center}
\rule[0.1ex]{\textwidth}{1pt}
\end{center}

\aufgabe{6}
\textbf{a)}\\[2pt]
Diese Aussage ist wahr. Seien $K, L$ zwei beliebige Sprachen. Sei ferner $w \in (K \cap L)^*$
mit $|w| = n, n \in \mathbb{N}_0$. Es gilt:
\begin{alignat*}{3}
	w \in (K \cap L)^*
	&\,\Longrightarrow\,&& \forall i \in [1,n]_{\mathbb{N}_0} : w_i \in (K \cap L)\\[1pt]
	&\,\Longrightarrow\,&& \forall i \in [1,n]_{\mathbb{N}_0} : (w_i \in K) \land (w_i \in L),\\[1pt]
	&\,\Longrightarrow\,&& \forall i \in [1,n]_{\mathbb{N}_0} : (w_i \in K) \lor (w_i \in L),\\[1pt]
	&\,\Longrightarrow\,&& \forall i \in [1,n]_{\mathbb{N}_0} : w_i \in (K \cup L) \\[1pt]
	&\,\Longrightarrow\,&& w \in (K \cup L)^*
\end{alignat*}
Es folgt $(K \cap L)^* \subseteq (K \cup L)^*\hfill\square$\\

\textbf{b)}\\[2pt]
Diese Aussage ist falsch. Wir geben ein Gegenbeispiel:\\[2pt]
Sei $K = \{a\}, L = \{b\}$. Es ist $w = ab \in (K \cup L)^*$. Jedoch ist $K \cap L =\o$ und
damit $(K \cap L)^* = \{\varepsilon\}$. Also ist $w \notin (K \cap L)*$.
Es folgt $(K \cup L)^* \nsubseteq (K \cap L)^*$.\\

\textbf{c)}\\[2pt]
Diese Aussage ist wahr. Seien $K, L$ zwei beliebige Sprachen mit $K \subseteq L$. \\
Sei ferner $w \in K^*$ mit $|w| = n, n \in \mathbb{N}_0$ gegeben. Es gilt:
$$
	w \in K^*
	\,\Longrightarrow\, \forall i \in [1,n]_{\mathbb{N}_0} : w_i \in K
	\up{\ \Longrightarrow\ }{$K \subseteq L} \forall i \in [1,n]_{\mathbb{N}_0} : w_i \in L
	\,\Longrightarrow\, w \in L^*
$$
Es folgt $K \subseteq L \,\Longrightarrow\, K^* \subseteq L^*\hfill\square$\\

\textbf{d)}\\[2pt]
Diese Aussage ist falsch. Wir geben ein Gegenbeispiel:\\[2pt]
Sei $K = \{aa\}, L = \{a\}$. Dann gilt $K^* \subseteq L^*$, da $(aa)^n = a^{2n}$.
Also gibt es zu jedem $w \in K^*$ mit $|w| = n, n \in \mathbb{N}_0$ ein $w' \in L*$ mit
$|w| = 2n$. Jedoch gilt $aa \neq a$ und damit $K \nsubseteq L$.

\newpage

\aufgabe{7}

\end{document}
