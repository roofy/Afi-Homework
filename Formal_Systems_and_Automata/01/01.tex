\documentclass[a4paper,graphics,11pt]{article}
\pagenumbering{arabic}

\usepackage[margin=1in]{geometry}
\usepackage[utf8]{inputenc}
\usepackage[T1]{fontenc}
\usepackage{lmodern}
\usepackage[ngerman]{babel}
\usepackage{amsmath, tabu}
\usepackage{amsthm}
\usepackage{amssymb}
\usepackage{complexity}
\usepackage{mathtools}
\usepackage{setspace}
\usepackage{graphicx,color,curves,epsf,float,rotating}
\usepackage{tasks}
\usepackage{tikz}
\usetikzlibrary{automata, arrows}

\setlength{\parindent}{0em}
\setlength{\parskip}{1em}

\newcommand{\aufgabe}[1]{\subsection*{Aufgabe #1}}
\newcommand{\up}[2]{\mathrel{\overset{\makebox[0pt]{\mbox{\normalfont\tiny #2}}}{#1}}}

\begin{document}
\noindent Gruppe \fbox{\textbf{3}}             \hfill Phil Pützstück, 377247\\
\noindent Formale Systeme und Automaten \hfill Benedikt Gerlach, 376944\\
\strut\hfill Sebastian Hackenberg, 377550\\
\begin{center}
	\LARGE{\textbf{Hausaufgabe 1}}
\end{center}
\begin{center}
\rule[0.1ex]{\textwidth}{1pt}
\end{center}

\aufgabe{6}
\textbf{a)}\\[2pt]
Diese Aussage ist wahr. Seien $K, L$ zwei beliebige Sprachen. Sei ferner $w \in (K \cap L)^*$
mit $|w| = n, n \in \mathbb{N}_0$. Es gilt:
\begin{alignat*}{3}
	w \in (K \cap L)^*
	&\,\Longrightarrow\,&& \forall i \in [1,n]_{\mathbb{N}_0} : w_i \in (K \cap L)\\[1pt]
	&\,\Longrightarrow\,&& \forall i \in [1,n]_{\mathbb{N}_0} : (w_i \in K) \land (w_i \in L),\\[1pt]
	&\,\Longrightarrow\,&& \forall i \in [1,n]_{\mathbb{N}_0} : (w_i \in K) \lor (w_i \in L),\\[1pt]
	&\,\Longrightarrow\,&& \forall i \in [1,n]_{\mathbb{N}_0} : w_i \in (K \cup L) \\[1pt]
	&\,\Longrightarrow\,&& w \in (K \cup L)^*
\end{alignat*}
Es folgt $(K \cap L)^* \subseteq (K \cup L)^*\hfill\square$\\

\textbf{b)}\\[2pt]
Diese Aussage ist falsch. Wir geben ein Gegenbeispiel:\\[2pt]
Sei $K = \{a\}, L = \{b\}$. Es ist $w = ab \in (K \cup L)^*$. Jedoch ist $K \cap L =\o$ und
damit $(K \cap L)^* = \{\varepsilon\}$. Also ist $w \notin (K \cap L)*$.
Es folgt $(K \cup L)^* \nsubseteq (K \cap L)^*$.\\

\textbf{c)}\\[2pt]
Diese Aussage ist wahr. Seien $K, L$ zwei beliebige Sprachen mit $K \subseteq L$. \\
Sei ferner $w \in K^*$ mit $|w| = n, n \in \mathbb{N}_0$ gegeben. Es gilt:
$$
	w \in K^*
	\,\Longrightarrow\, \forall i \in [1,n]_{\mathbb{N}_0} : w_i \in K
	\up{\ \Longrightarrow\ }{$K \subseteq L$} \forall i \in [1,n]_{\mathbb{N}_0} : w_i \in L
	\,\Longrightarrow\, w \in L^*
$$
Es folgt $K \subseteq L \,\Longrightarrow\, K^* \subseteq L^*\hfill\square$\\

\textbf{d)}\\[2pt]
Diese Aussage ist falsch. Wir geben ein Gegenbeispiel:\\[2pt]
Sei $K = \{aa\}, L = \{a\}$. Dann gilt $K^* \subseteq L^*$, da $(aa)^n = a^{2n}$.
Also gibt es zu jedem $w \in K^*$ mit $|w| = n, n \in \mathbb{N}_0$ ein $w' \in L*$ mit
$|w| = 2n$. Jedoch gilt $aa \neq a$ und damit $K \nsubseteq L$.

\newpage

\aufgabe{7}

\begin{tikzpicture}[>=stealth',shorten >=1pt,auto,node distance=4cm]
	\node[state, initial]	(q0)				{$q_0$};
	\node[state, accepting] (q1) [right of=q0]	{$q_1$};
	\node[state, accepting] (q2) [right of=q1]	{$q_2$};
	\node[state]			(q3) [below of=q2]	{$q_3$};

	\path[->] (q0) edge					node {a}	(q1);
	\path[->] (q0) edge [loop above]	node {b,c}	(q0);
	\path[->] (q1) edge					node {a}	(q2);
	\path[->] (q1) edge [loop above]	node {b,c}	(q1);
	\path[->] (q2) edge					node {a}	(q3);
	\path[->] (q2) edge [loop above]	node {b,c}	(q2);
	\path[->] (q3) edge [above]			node {a}	(q1);
	\path[->] (q3) edge [loop below]	node {b,c}	(q2);
\end{tikzpicture}

\textbf{b)}

\begin{tikzpicture}[>=stealth',shorten >=1pt,auto,node distance=4cm]
	\node[state, initial, accepting]	(q0)				{$q_0$};
	\node[state, accepting]				(q1) [right of=q0]	{$q_1$};
	\node[state, accepting]				(q2) [right of=q1]	{$q_2$};
	\node[state]						(q3) [right of=q2]	{$q_3$};

	\path[->] (q0) edge	[loop above]	node {a}	(q0);
	\path[->] (q0) edge					node {b}	(q1);

	\path[->] (q1) edge					node {b}	(q2);
	\path[->] (q1) edge	[bend left]	node {a}	(q0);

	\path[->] (q2) edge	[loop above]	node {b}	(q2);
	\path[->] (q2) edge					node {a}	(q3);

	\path[->] (q3) edge	[loop above]	node {a,b}	(q2);
\end{tikzpicture}

\textbf{c)}

\begin{tikzpicture}[>=stealth',shorten >=1pt,auto,node distance=4cm]
	\node[state, initial]	(q0)						{$q_0$};
	\node[state]			(q1) [above right of=q0]	{$q_1$};
	\node[state, accepting]	(q2) [right of=q1]			{$q_2$};
	\node[state]			(q3) [below right of=q0]	{$q_3$};
	\node[state]			(q4) [right of=q3]			{$q_4$};
	\node[state]			(q5) [above right of=q4]	{$q_5$};

	\path[->] (q0) edge [loop above]	node {b} (q0);
	\path[->] (q0) edge					node {a} (q1);
	\path[->] (q0) edge					node {c} (q3);

	\path[->] (q1) edge [loop above]	node {a} (q1);
	\path[->] (q1) edge	[bend left]		node {b} (q0);
	\path[->] (q1) edge					node {c} (q2);

	\path[->] (q2) edge [loop above]	node {a,b,c} (q2);

	\path[->] (q3) edge [loop above]	node {c} (q3);
	\path[->] (q3) edge	[bend left]		node {b} (q0);
	\path[->] (q3) edge					node {a} (q4);

	\path[->] (q4) edge [loop below]	node {a} (q3);
	\path[->] (q4) edge					node {c} (q2);
	\path[->] (q4) edge					node {b} (q5);

	\path[->] (q5) edge [loop above]	node {b,c} (q5);
	\path[->] (q5) edge	[bend left]		node {a} (q4);
\end{tikzpicture}

\newpage

\aufgabe{8}
Wir betrachten zur Lösung des Problems den Rest der Zahlen bei Ganzzahldivision mit 5.
Wir starten mit einem Rest von 0. Aus der Vorlesung ist bekannt, dass wenn wir an eine Binärzahl w
eine 0 oder 1 konkatenieren, gilt: $wN = 2w+N$, wobei N eine 0 oder 1 ist.

\begin{tikzpicture}[>=stealth',shorten >=1pt,auto,node distance=4cm]
	\node[state, initial, accepting]	(q0)			{$q_0$};
	\node[state]			(q1) [above right of=q0]	{$q_1$};
	\node[state]			(q2) [below right of=q1]	{$q_2$};
	\node[state]			(q3) [below of=q2]			{$q_3$};
	\node[state]			(q4) [below of=q0]			{$q_4$};

	\path[->] (q0) edge [loop above]	node {0} (q0);
	\path[->] (q0) edge					node {1} (q1);

	\path[->] (q1) edge					node {0} (q2);
	\path[->] (q1) edge	[pos=0.8]		node {1} (q3);

	\path[->] (q2) edge					node {0} (q4);
	\path[->] (q2) edge	[pos=0.8]		node {1} (q0);

	\path[->] (q3) edge	[bend left, pos=0.55]	node {0} (q1);
	\path[->] (q3) edge	[bend right]	node {1} (q2);

	\path[->] (q4) edge [loop above]	node {1} (q4);
	\path[->] (q4) edge					node {0} (q3);
\end{tikzpicture}

\aufgabe{9}

\textbf{a)}
Der Automat $\mathcal{A}_1$ erkennt alle Sprachen über dem gegebenem Alphabet, in denen keine Worte vorkommen,
welche das Symbol a 2 mal hintereinander enthalten.

\textbf{b)}
Der Automat $\mathcal{A}_2$ erkennt alle Sprachen über dem gegebenem Alphabet, in denen nur Worte vorkommen,
welche sowohl als Präfix als auch als Suffix ein a haben und dazwischen nur eine ungerade Anzahl des Symbols b
enthalten.

\end{document}
