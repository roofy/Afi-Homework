\documentclass[a4paper,graphics,11pt]{article}
\pagenumbering{arabic}

\usepackage[margin=1in]{geometry}
\usepackage[utf8]{inputenc}
\usepackage[T1]{fontenc}
\usepackage{lmodern}
\usepackage[ngerman]{babel}
\usepackage{amsmath, tabu}
\usepackage{amsthm}
\usepackage{amssymb}
\usepackage{complexity}
\usepackage{mathtools}
\usepackage{setspace}
\usepackage{graphicx,color,curves,epsf,float,rotating}
\usepackage{tasks}
\setlength{\parindent}{0em}
\setlength{\parskip}{1em}
\usepackage{tikz}
\usetikzlibrary{automata, arrows}

\newcommand{\aufgabe}[1]{\subsection*{Aufgabe #1}}
\newcommand{\up}[2]{\mathrel{\overset{\makebox[0pt]{\mbox{\normalfont\tiny #2}}}{#1}}}

\begin{document}
\noindent Gruppe \fbox{\textbf{3}}             \hfill Phil Pützstück, 377247\\
\noindent Formale Systeme und Automaten \hfill Benedikt Gerlach, 376944\\
\strut\hfill Sebastian Hackenberg, 377550\\
\begin{center}
	\LARGE{\textbf{Hausaufgabe 10}}
\end{center}
\begin{center}
\rule[0.1ex]{\textwidth}{1pt}
\end{center}



\aufgabe{5}
Sei ein PDA $\mathcal{A} := (Q, \Sigma, \Gamma, \Delta, q_0, Z_0)$, der mit
leerem Stapel akzeptiert gegeben.\\
Wir zeigen, dass der PDA $\mathcal{A}' := (Q', \Sigma, \Gamma, \Delta', q_0', Z_0', F)$,
die gleiche Sprache akzeptiert,
wobei
$$
    Q' := Q \cup \{q_0',f\}\qquad
    F := \{f\}\qquad
    \Delta' := \Delta \cup \{(q, \varepsilon, Z_0', f, \varepsilon) \mid q \in Q'\} \cup \{(q_0', \varepsilon, Z_0', q_0, Z_0Z_0')\}
$$
Wir zeigen zuerst $L(\mathcal{A}) \subseteq L(\mathcal{A}')$.\\
Sei $w \in L(\mathcal{A})$, d.h. es gibt einen Lauf $(\kappa_0, \cdots, \kappa_n)$
mit
$$
    \forall i \in [0,n] : \kappa_i = (q_i, \gamma_i, v_i),\ q_i \in Q, \gamma_i \in \Gamma^*, w \sqsubseteq v_i
$$
Dann ist weiter $\kappa_n = (q_n, \varepsilon, \varepsilon)$, also auch $\gamma_n = v_n = \varepsilon$.

Wir definieren nun das Tupel $\kappa' := ((q'_0, Z_0', w), \kappa'_0, \cdots, \kappa'_n, (f, \varepsilon, \varepsilon))$ mit
$$
    \forall i \in [0,n] : \kappa'_i := (q_i, \gamma_iZ_0',v_i)
$$
Dann ist $\kappa'$ ein Lauf über $w$ auf $\mathcal{A}'$: Wir beginnen im neuen Startzustand $q'_0$ mit neuem
Stapelstartsymbol $Z_0'$. Durch $(q_0', \varepsilon, Z_0', q_0, Z_0Z_0')$ kommen wir dann zu $\kappa'_0$,
was für $\mathcal{A}$ gleich zu $\kappa_0$ wirkt, da nun $Z_0$ oben auf dem Stapel liegt.
Von da an kann der gegebenen PDA $\mathcal{A}$ wie zuvor laufen bis er schließlich zur Konfiguration
$\kappa'_n$ kommt. Hier hatte er im originellen Lauf nun mit einem leerem Stapel akzeptiert ($\kappa_n$), doch
hier ist jetzt $\kappa_n' = (q_n, Z_0', \varepsilon)$. Da $(q_n, \varepsilon, Z_0', f, \varepsilon) \in \Delta'$
können wir dann noch zum Endzustand $f$ übergehen und akzeptieren in der Konfiguration $(f,\varepsilon,\varepsilon)$.
Damit ist $w \in L(\mathcal{A}')$.

Wir zeigen nun $L(\mathcal{A}') \subseteq L(\mathcal{A})$.\\
Sei also $w \in L(\mathcal{A}')$. Wir gehen komplett analog vor:\\
Sei $\kappa' := ((q'_0, Z_0', w), \kappa'_0, \cdots, \kappa'_n, (f, \varepsilon, \varepsilon))$
der Lauf von $\mathcal{A}'$ über $w$. Dabei gelte
$$
    \forall i \in [0,n] : \kappa_i' = (q_i, \gamma_iZ_0', v_i),\ q_i \in Q, \gamma_i \in \Gamma^*, w \sqsubseteq v_i
$$
Wir haben nun nach Konstruktion von $\Delta'$, dass $\kappa_0' = (q_0, Z_0Z_0', w)$ und
$\kappa_n' = (q_n, Z_0', \varepsilon)$, also $\gamma_0 = Z_0$ und $\gamma_n = \varepsilon$.
Damit das Tupel $\kappa := ((q_i, \gamma_i, v_i))_{i \in [0,n]}$ ein Lauf von $\mathcal{A}$ über $w$,
also $w \in L(\mathcal{A})$.

Insgesamt gilt also $L(\mathcal{A}) = L(\mathcal{A}')$.


\newpage

\aufgabe{6}
\textbf{a)}

Der gesuchte PDA akzeptiert mit leerem Stapel:
\begin{center}
 \begin{tikzpicture}[>=stealth',shorten >=1pt,auto,node distance=4cm]
        \node[state, initial]   (q0)                        {$q_0$};
        \path[->] (q0)  edge [loop right]               node 
        {\Large$\substack{
            \varepsilon, S, ZR\\[2pt]
            \varepsilon, Z, EX\\[2pt]
            \varepsilon, R, KX\\[2pt]
            \varepsilon, X, XX\\[5pt]
            0, Z, \varepsilon\\[2pt]
            0, X, \varepsilon\\[2pt]
            1, X, \varepsilon\\[2pt]
            1, E, \varepsilon\\[2pt]
            \texttt{,}, K, \varepsilon
        }$} (q0);
    \end{tikzpicture}
\end{center}

\textbf{b)}

Linksableitung in $\mathcal{G}:$
$$
    S \to ZR \to EXR \to 1XR \to 10R \to 10KX \to 10,X \to 10,XX
$$$$
    \to 10,1X \to 10,1XX \to 10,10X \to 10,101
$$
Lauf auf $\mathcal{A}_\mathcal{G}:$
$$
    (q_0, S, 10,101) \to
    (q_0, ZR, 10,101) \to
    (q_0, EXR, 10,101) \to
    (q_0, XR, 0,101)
$$$$
    \to (q_0, R, ,101)
    \to (q_0, KX, ,101)
    \to (q_0, X, 101)
    \to (q_0, XX, 101)
    \to (q_0, X, 01)
$$$$
    \to (q_0, XX, 01)
    \to (q_0, X, 1)
    \to (q_0, \varepsilon, \varepsilon)
$$

\aufgabe{7}

Die (vereinfachte) gesuchte Grammatik hat folgende Produktionsregeln (mit Startregel $S$):
$$
    S \to [qZq] \mid [qZr]
$$$$
    [qZq] \to \varepsilon
    \qquad
    [qXq] \to a
    \qquad
    [rXq] \to a
$$$$
    [qZr] \to b[rXq][qZq][qZr]
\qquad\qquad
    [qZq] \to b[rXq][qZq][qZq]
$$$$
    [rXr] \to b[rXr][rXr]
\qquad\qquad
    [rXq] \to b[rXq][qXq]
    \mid
    b[rXr][rXq]
$$

\newpage

\aufgabe{8}
Startstapelsymbol $Z_0$. $B^{-}$ bedeutet es fehlt momentan ein $b$, $B$ bedeutet
wir haben momentan eins zu viel.

\textbf{a)}
\begin{center}
 \begin{tikzpicture}[>=stealth',shorten >=1pt,auto,node distance=5cm]
        \node[state, initial]   (q0)                        {$q_0$};
        \node[state,accepting]   (q1)  [below left of=q0]                      {$q_1$};
        \node[state] (q2)  [below right of=q0]                      {$q_2$};
        \path[->] (q0)  edge [loop above]               node 
        {\Large$\substack{
            a, Z_0,B^{-}B^{-}Z_0 \\[2pt]
            b, Z_0,BZ_0 \\[5pt]
            a, B^{-},B^{-}B^{-}B^{-} \\[2pt]
            b, B^{-},\varepsilon \\[5pt]
            b, B, BB \\[2pt]
        }$} (q0);
        \path[->] (q0) edge node {$\$,Z_0, \varepsilon$} (q1);
        \path[->] (q0) edge node {$a,B,\varepsilon$} (q2);

        \path[->] (q2) edge [bend left] node {
            \Large$\substack{
                \varepsilon, B, \varepsilon\\[2pt]
                \varepsilon, Z_0, B^{-}Z_0\\[2pt]
            }$
        } (q0);
    \end{tikzpicture}
\end{center}

\textbf{b)}
\begin{center}
 \begin{tikzpicture}[>=stealth',shorten >=1pt,auto,node distance=5cm]
        \node[state, initial, accepting]   (q0)                     {$0$};
        \node[state] (q1)  [below left of=q0]                       {$+$};
        \node[state] (q2)  [below of=q1]                            {$q$};
        \node[state] (q3)  [below right of=q0]                      {$-$};
        \node[state] (q4)  [below of=q3]                            {$r$};

        \path[->] (q0) edge node [sloped,above] {$b,Z_0, BZ_0$} (q1);
        \path[->] (q0) edge node {$a,Z_0, B^-B^-Z_0$} (q3);

        \path[->] (q1) edge [loop left] node {$b,B,BB$} (q1);

        \path[->] (q1) edge node {$a,B, \varepsilon$} (q2);

        \path[->] (q2) edge node {$\Large\substack{
            \varepsilon,B, \varepsilon\\
            \varepsilon, Z_0, B^-Z_0
        }$} (q4);

        \path[->] (q3) edge [loop right] node {$a, B^-, B^-B^-B^-$} (q3);
        \path[->] (q3) edge node {$b, B^-, \varepsilon$} (q4);

        \path[->] (q4) edge [bend left=20] node [sloped,above] {$\varepsilon, B, B$} (q1);
        \path[->] (q4) edge [bend left=40] node [pos=0.6,left] {$\varepsilon, Z_0, Z_0$} (q0);
        \path[->] (q4) edge [bend left=40] node [pos=0.8,left] {$\varepsilon, B^-, B^-$} (q3);
    \end{tikzpicture}
\end{center}

\newpage

\aufgabe{9}
\textbf{a)}

Sei also $\mathcal{A}$ ein DPDA und $v,w \in L(\mathcal{A})$ mit $v\neq w$. Wir nehmen an, dass $v \sqsubset w$.
Da $v \in L(\mathcal{A})$ und $\mathcal{A}$ ein DPDA existiert ein Lauf $(\kappa_0, \cdots, \kappa_n)$ über
$v$ auf $\mathcal{A}$, sodass $\kappa_n = (q, \varepsilon, \varepsilon)$. Da nun $w \in L(\mathcal{A}),\ v \sqsubset w$
und $\mathcal{A}$ deterministisch ist haben wir auch einen Lauf $(\kappa_0, \cdots, \kappa_n, \cdots, \kappa_m)$
über $w$, d.h. während des Laufes von $w$ haben wir einen leeren Stapel. Dann muss es aber eine
Transition mit leerem Stapel geben, welche es nach Definition der DPDA's nicht gibt. Damit haben wir einen Widerspruch.
Folglich gilt $v \not\sqsubset w$, unter anderem auch da der Automat sich sonst in $\kappa_n$ zwischen akzeptieren
und weitermachen entscheiden müsste, was eben genau eine nicht-deterministische Eigenschaft ist.\\


\textbf{b)}

Nach a) haben wir damit, dass DPDA's mit finalen Zuständen mächtiger sind als diese, welche mit
leerem Stapel akzeptieren. Beispielsweise können wir mit den letzteren nur Sprachen $L$ mit
$\forall w,v \in L: w \not\sqsubset v$ erkennen, welche nicht die reguläre Sprache $L(a^*)$
beinhaltet. Andererseits wissen wir, dass jede reguläre Sprache DPDA-erkennbar ist, damit folgt die Behauptung.



\end{document}

