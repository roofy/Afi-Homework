\documentclass[a4paper,graphics,11pt]{article}
\pagenumbering{arabic}

\usepackage[margin=1in]{geometry}
\usepackage[utf8]{inputenc}
\usepackage[T1]{fontenc}
\usepackage{lmodern}
\usepackage[ngerman]{babel}
\usepackage{amsmath, tabu}
\usepackage{amsthm}
\usepackage{amssymb}
\usepackage{complexity}
\usepackage{mathtools}
\usepackage{setspace}
\usepackage{graphicx,color,curves,epsf,float,rotating}
\usepackage{tasks}
\setlength{\parindent}{0em}
\setlength{\parskip}{1em}
\usepackage{tikz}
\usetikzlibrary{automata, arrows}

\newcommand{\aufgabe}[1]{\subsection*{Aufgabe #1}}
\newcommand{\up}[2]{\mathrel{\overset{\makebox[0pt]{\mbox{\normalfont\tiny #2}}}{#1}}}

\begin{document}
\noindent Gruppe \fbox{\textbf{3}}             \hfill Phil Pützstück, 377247\\
\noindent Formale Systeme und Automaten \hfill Benedikt Gerlach, 376944\\
\strut\hfill Sebastian Hackenberg, 377550\\
\begin{center}
	\LARGE{\textbf{Hausaufgabe 10}}
\end{center}
\begin{center}
\rule[0.1ex]{\textwidth}{1pt}
\end{center}



\aufgabe{5}
Sei ein PDA $\mathcal{A} := (Q, \Sigma, \Gamma, \Delta, q_0, Z_0)$, der mit
leerem Stapel akzeptiert gegeben.\\
Wir zeigen, dass der PDA $\mathcal{A}' := (Q', \Sigma, \Gamma, \Delta', q_0, Z_0, F)$,
die gleiche Sprache akzeptiert,
wobei
$$
    Q' := Q \cup \{f\}\qquad
    F' := \{f\}\qquad
    \Delta' := \Delta \cup \{(q, \varepsilon, \varepsilon, f, \varepsilon) \mid q \in Q\}
$$
Wir zeigen zuerst $L(\mathcal{A}) \subseteq L(\mathcal{A}')$.\\
Sei also $w \in L(\mathcal{A})$, d.h. es gibt einen Lauf $(\kappa_0, \cdots, \kappa_n)$
über $w$ auf $\mathcal{A}$ mit $\kappa_n = (q, \varepsilon, \varepsilon)$ für ein
$q \in Q$.
Folglich gibt es einen Lauf $(\kappa_0, \cdots, \kappa_n, \kappa_{n+1})$ über $w$
auf $\mathcal{A}'$, sodass $\kappa_{n+1} = (f, \varepsilon, \varepsilon)$,
da wir die gleichen Transitionen wie in $\mathcal{A}$ verwenden können und schließlich
von $q$ aus die Transition $(q, \varepsilon, \varepsilon, f, \varepsilon) \in \Delta'$
wodurch wir im Zustand $f$ enden und den beschriebenen Lauf bekommen.
Folglich gilt $w \in L(\mathcal{A}')$.

Wir zeigen nun $L(\mathcal{A}') \subseteq L(\mathcal{A})$.\\
Sei also, $w \in L(\mathcal{A}')$, d.h. es gibt einen Lauf $(\kappa_0, \cdots, \kappa_n, \kappa_{n+1})$ über $w$ auf $\mathcal{A}'$ mit $\kappa_{n+1} = (q, \gamma, \varepsilon)$ mit $q \in F$
und $\gamma \in \Gamma^*$. Da $F = \{f\}$ folgt $q = f$.
Weiter ist für $q \in Q, \sigma \in \Sigma, Z \in \Gamma, \gamma \in \Gamma^*$
$$
    (q, \sigma, Z, f, \gamma) \in \Delta' \,\Longrightarrow\, \sigma = Z = \gamma = \varepsilon
$$
nach Konstruktion von $\Delta'$. Folglich muss $\kappa_n = (q, \varepsilon, \varepsilon)$ und $\kappa_{n+1} = (f, \varepsilon, \varepsilon)$.
Da wir ausser den Transitionen zu $f$ aber keine anderen Hinzugefügt haben,
folgt, dass der Lauf $(\kappa_0, \cdots, \kappa_n)$ über $w$ auch auf
$\mathcal{A}$ existiert. Da dieser aber durch $\kappa_n = (q, \varepsilon, \varepsilon)$
beendet ist, also mit leerem Stapel aufhört und $\mathcal{A}$ eben genau dann
akzeptiert, folgt also auch $w \in L(\mathcal{A})$.

Insgesmat haben wir $L(\mathcal{A}) = L(\mathcal{A}')$.

\newpage

\aufgabe{6}
\textbf{a)}
Der gesuchte PDA akzeptiert mit leerem Stapel:
\begin{center}
 \begin{tikzpicture}[>=stealth',shorten >=1pt,auto,node distance=4cm]
        \node[state, initial]   (q0)                        {$q_0$};
        \path[->] (q0)  edge [loop right]               node 
        {\Large$\substack{
            \varepsilon, S, ZR\\[2pt]
            \varepsilon, Z, EX\\[2pt]
            \varepsilon, R, KX\\[2pt]
            \varepsilon, X, XX\\[5pt]
            0, Z, \varepsilon\\[2pt]
            0, X, \varepsilon\\[2pt]
            1, X, \varepsilon\\[2pt]
            1, E, \varepsilon\\[2pt]
            \texttt{,}, K, \varepsilon
        }$} (q0);
    \end{tikzpicture}
\end{center}

\textbf{b)}
Linksableitung in $\mathcal{G}:$
$$
    S \to ZR \to EXR \to 1XR \to 10R \to 10KX \to 10,X \to 10,XX
$$$$
    \to 10,1X \to 10,1XX \to 10,10X \to 10,101
$$
Lauf auf $\mathcal{A}_\mathcal{G}:$
$$
    (q_0, S, 10,101) \to
    (q_0, ZR, 10,101) \to
    (q_0, EXR, 10,101) \to
    (q_0, XR, 0,101)
$$$$
    \to (q_0, R, ,101)
    \to (q_0, KX, ,101)
    \to (q_0, X, 101)
    \to (q_0, XX, 101)
    \to (q_0, X, 01)
$$$$
    \to (q_0, XX, 01)
    \to (q_0, X, 1)
    \to (q_0, \varepsilon, \varepsilon)
$$

\newpage

\aufgabe{7}




\end{document}
