\documentclass[a4paper,graphics,11pt]{article}
\pagenumbering{arabic}

\usepackage[margin=1in]{geometry}
\usepackage[utf8]{inputenc}
\usepackage[T1]{fontenc}
\usepackage{lmodern}
\usepackage[ngerman]{babel}
\usepackage{amsmath, tabu}
\usepackage{amsthm}
\usepackage{amssymb}
\usepackage{complexity}
\usepackage{mathtools}
\usepackage{setspace}
\usepackage{graphicx,color,curves,epsf,float,rotating}
\usepackage{tasks}
\setlength{\parindent}{0em}
\setlength{\parskip}{1em}
\usepackage{tikz}
\usetikzlibrary{automata, arrows}

\newcommand{\aufgabe}[1]{\subsection*{Aufgabe #1}}
\newcommand{\up}[2]{\mathrel{\overset{\makebox[0pt]{\mbox{\normalfont\tiny #2}}}{#1}}}

\begin{document}
\noindent Gruppe \fbox{\textbf{3}}             \hfill Phil Pützstück, 377247\\
\noindent Formale Systeme und Automaten \hfill Benedikt Gerlach, 376944\\
\strut\hfill Sebastian Hackenberg, 377550\\
\begin{center}
	\LARGE{\textbf{Hausaufgabe 6}}
\end{center}
\begin{center}
\rule[0.1ex]{\textwidth}{1pt}
\end{center}

\aufgabe{5}
\textbf{a)}
Wir konstruieren zuerst einen DFA $\mathcal{A}$ mit
$L(\mathcal{A}) = \{a,b\}^* \setminus (\{a^n \mid n \in \mathbb{N}\} \cup \{b^n \mid n\in \mathbb{N}\})$,
welcher also nur alle Wörter über $\{a,b\}$, welche sowohl $b$ als auch $a$ enthalten, erkennt.\\
Dieser könnte z.B. wie folgt aussehen:
\begin{center}
    \begin{tikzpicture}[>=stealth',shorten >=1pt,auto,node distance=3cm]
        \node[state, initial]   (q0)                        {$q_0$};
        \node[state]            (q1)    [above right of=q0] {$q_1$};
        \node[state]            (q2)    [below right of=q0] {$q_2$};
        \node[state,accepting]  (q3)    [right of=q1]       {$q_3$};
        \node[state,accepting]  (q4)    [right of=q2]       {$q_4$};

        \path[->] (q0)  edge                node {$a$} (q1);
        \path[->] (q0)  edge                node {$b$} (q2);

        \path[->] (q1)  edge [loop below]   node {$a$} (q1);
        \path[->] (q2)  edge [loop above]   node {$b$} (q5);

        \path[->] (q1)  edge                node {$b$} (q3);

        \path[->] (q2)  edge                node {$a$} (q4);

        \path[->] (q3)  edge [loop below]   node {$a,b$} (q3);
        \path[->] (q4)  edge [loop above]   node {$a,b$} (q4);
    \end{tikzpicture}
\end{center}
Dann wandeln wir den gegebenen NFA zu einem DFA $\mathcal{B}$ um und berechnen das Inklusionsproblem
$L(\mathcal{B}) \subseteq L(\mathcal{A})$.

\textbf{b)}
Wir konstruieren den NFA $\mathcal{A}$ mit $L(\mathcal{A}) = \{va \mid v \in \Sigma^*\}$. Der also
alle Wörter über $\Sigma^*$ erkennt, welche mit $a$ enden. Dieser könnte z.B. wie folgt aussehen:
\begin{center}
    \begin{tikzpicture}[>=stealth',shorten >=1pt,auto,node distance=3cm]
        \node[state, initial]   (q0)                    {$q_0$};
        \node[state, accepting] (q1)    [right of=q0]   {$q_1$};

        \path[->] (q0)  edge [loop above]   node {$a,b$} (q0);
        \path[->] (q0)  edge                node {$a$} (q1);
    \end{tikzpicture}
\end{center}
Da FA's nach Satz 2.40 unter dem Schnitt abgeschlossen sind, können wir den NFA $\mathcal{A} \cap \mathcal{B}$
berechnen, wobei $\mathcal{B}$ der gegebene NFA ist. Dieser erkennt nun alle Wörter aus $\mathcal{B}$
welche mit einem $a$ aufhören. Nun können wir das Unendlichkeitsproblem auf dem NFA
$\mathcal{A} \cap \mathcal{B}$ lösen. Wenn dies also unendlich ist, so kann es keine obere Schranke
geben, ist er jedoch endlich so gibt es ein $k = \max\{|w| \mid w \in L(\mathcal{A}\cap \mathcal{B})\}$
welches eine obere Schranke der Länge aller Wörter aus $\mathcal{B}$, welche mit $a$ aufhören, darstellt.
Es ist $K := \{|w| \mid w \in L(\mathcal{A}\cap \mathcal{B})\} \subseteq \mathbb{N}$ und damit bzgl. $\leq$
totalgeordnet, was bei endlichkeit von $K$ die Existenz eines Maximums aus $K$ impliziert.

\newpage
\aufgabe{6}
\textbf{a)}
\begin{center}
    \begin{tikzpicture}[>=stealth',shorten >=1pt,auto,node distance=4cm]
        \node[state, initial]   (q0)                        {$q_0$};
        \node[state]            (q1)    [right of=q0] {$q_1$};
        \node[state]            (q2)    [right of=q1] {$q_2$};

        \path[->] (q0)  edge [loop above]   node {$b/\varepsilon$} (q0);
        \path[->] (q0)  edge [bend left]   node {$a/\varepsilon$} (q1);

        \path[->] (q1)  edge [loop above]   node {$a/\varepsilon$} (q1);
        \path[->] (q1)  edge [bend left]   node {$b/\varepsilon$} (q2);

        \path[->] (q2)  edge    node {$a/1$} (q1);
        \path[->] (q2)  edge [bend left]   node {$b/\varepsilon$} (q0);
    \end{tikzpicture}
\end{center}
\textbf{b)}

Seien
$\mathcal{M}_1 = (Q_1, \Sigma_1, \Sigma_2, \delta_1, q_{01}, \lambda_1),
\mathcal{M}_2 = (Q_2, \Sigma_2, \Sigma_3, \delta_2, q_{02}, \lambda_2)$
die Mealy-Automaten, welche jeweils die Funktionen $G_1, G_2$ berechnen.
Die Idee ist, beide Automaton gleichzeitig laufen zu lassen, wir machen also eine Produktkonstruktion.
Wir übergeben die Ausgabe von $\mathcal{M}_1$ an die Transitionsfunktion von $\mathcal{M_2}$, wechseln
intern in beiden Automaton entsprechend die Zustände und geben nur das berechnete Symbol von $\mathcal{M}_2$
bei der Transition aus.

Der Mealy-Automat $\mathcal{M}$ mit $G_{\mathcal{M}} = G_2 \circ G_1$ ist also wie folgt definiert:
$$
    \mathcal{M} := (Q_1 \times Q_2, \Sigma_1, \Sigma_3, \delta, (q_{01}, q_{02}), \lambda)
$$
wobei für $q_1 \in Q_1, q_2 \in Q_2, a \in \Sigma_1$ gilt:
$$
    \delta((q_1,q_2), a) = (\delta_1(q_1, a), \delta_2(q_2, \lambda_1(q_1, a)))
    \quad\text{und}\quad
    \lambda((q_1,q_2), a) = \lambda_2(q_2, \lambda_1(q_1, a))
$$

\aufgabe{7}

\textbf{a)}

Wir haben $\mathcal{G} = (\{S,A\}, \{a,b,c\}, P, S)$ wobei $P$ folgende Regeln hat:
$$
    S \to aSa \mid Ac
    \qquad\qquad\qquad
    A \to bAc \mid \varepsilon
$$
Wir machen zuerst die $n$ $a$'s am Anfang und ende des Wortes, können dann zu einem anderen ''Zustand''
$A$ übergehen wobei wir ein $c$ so platzieren, dass es rechts der Mitte des Wortes vorkommt.
Dann können wir in $A$ die $m$ $b$'s und $c$'s wie gewünscht erschaffen.\\

\textbf{b)}

Wir haben $\mathcal{G} = (\{S, A\}, \{a,b,c\}, P, S)$ wobei $P$ die folgenden Regeln hat:
$$
    S \to aSc \mid A
    \qquad\qquad\qquad
    A \to bbAc \mid \varepsilon
$$
Wir erschaffen zuerst die $n$ $a$'s am Anfang des Wortes bzw. $n$ $c$'s am Ende des Wortes.
Dann können wir zum ''Zustand'' $A$ übergehen und dort die weitern $m$ $bb$'s und $c$'s
entsprechend erschaffen.

\newpage

\textbf{c)}

Wir haben $\mathcal{G} = (\{S\}, \{a,b,c\}, P, S)$ wobei $P$ die folgenden Regeln hat:
$$
    S \to bS \mid Sb \mid aSc \mid \varepsilon
$$
Wir erschaffen analog zu a) und b) die Teilworte $u$ und $v$ von $w$ jeweils von der Mitte von $w$ aus.
Also gilt für eine Satzform $xSy$ stets später $x \in (a+b)^*$, $y \in (b+c)^*$ und $|x|_a = |y|_c$.
Wir können durch beliebiges einschieben von $b$'s die gestalt von $x$ und $y$ um die Mitte von
$w$ beliebig formen um dann $a$ sowie $c$ gleichzeitig hinzuzufügen, ohne Worte aus der angegebenen
Sprachen nicht abzudecken.\\

\textbf{d)}

Wir haben $\mathcal{G} = (\{S\}, \{a,b,c\}, P, S)$ wobei $P$ die folgenden Regeln hat:
$$
    S \to cS \mid bS \mid bSaS \mid \varepsilon
$$
Wir können also stets beliebig viel $c$'s und $b$'s überall einfügen. Um ein $a$ in das Wort
zu schreiben müssen wir jedoch irgendwo vorher mindestens ein $b$ schreiben. Dies entspricht
eben den Anforderungen.
\end{document}
