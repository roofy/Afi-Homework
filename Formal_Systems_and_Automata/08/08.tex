\documentclass[a4paper,graphics,11pt]{article}
\pagenumbering{arabic}

\usepackage[margin=1in]{geometry}
\usepackage[utf8]{inputenc}
\usepackage[T1]{fontenc}
\usepackage{lmodern}
\usepackage[ngerman]{babel}
\usepackage{amsmath, tabu}
\usepackage{amsthm}
\usepackage{amssymb}
\usepackage{complexity}
\usepackage{mathtools}
\usepackage{setspace}
\usepackage{graphicx,color,curves,epsf,float,rotating}
\usepackage{tasks}
\setlength{\parindent}{0em}
\setlength{\parskip}{1em}
\usepackage{tikz}
\usetikzlibrary{automata, arrows}

\newcommand{\aufgabe}[1]{\subsection*{Aufgabe #1}}
\newcommand{\up}[2]{\mathrel{\overset{\makebox[0pt]{\mbox{\normalfont\tiny #2}}}{#1}}}

\begin{document}
\noindent Gruppe \fbox{\textbf{3}}             \hfill Phil Pützstück, 377247\\
\noindent Formale Systeme und Automaten \hfill Benedikt Gerlach, 376944\\
\strut\hfill Sebastian Hackenberg, 377550\\
\begin{center}
	\LARGE{\textbf{Hausaufgabe 8}}
\end{center}
\begin{center}
\rule[0.1ex]{\textwidth}{1pt}
\end{center}

\aufgabe{5}
\textbf{a)}
$
\langle_S
    a
    \langle_A
        a
        \langle_S
            \langle_B
                \langle_S
                    \langle_A
                    \rangle_A
                    b
                \rangle_S
                a
                b
                \rangle_S
            \rangle_B
        b
        \rangle_S
    \rangle_A
\rangle_S
$
und
$
\langle_S
    \langle_A
        a
        \langle_S
            \langle_B
                a
                b
                \langle_S
                    a
                    \langle_A
                    \rangle_A
                \rangle_S
            \rangle_B
        \rangle_S
        b
    \rangle_A
    b
\rangle_S
$

\textbf{b)}
Sei also eine kontextfreie Grammatik $\mathcal{G} := \{N, T, S, P\}$ gegeben.\\
Wir konstruieren
$\mathcal{G}' := (N, T \cup \{\langle_X, \rangle_X \mid X \in N\}, S, P')$.
Für jede Produktionsregel $p \in P$ mit $p = X\to \alpha$,
wobei $X \in N$ und $\alpha$ eine Satzform ist, haben wir eine Produktionsregel
$p' = X \to \langle_X \alpha \rangle_X$ in $P'$. Damit wird wie in der Konstruktion
der Baumkodierung nach jedem Ableitungsschritt bzw. um jede Ebene im Ableitungsbaum
die entsprechende Klammerung hinzugefügt.

\aufgabe{6}
\textbf{a)}
$\mathcal{G} := (\{Q_0, Q_1, Q_3\}, \{a,b\}, P, Q_0\}$ mit $P$ gegeben durch
$$
    Q_0 \to aQ_1 \mid bQ_3
    \qquad
    Q_1 \to bQ_3 \mid b
    \qquad
    Q_3 \to aQ_1 \mid a
$$

\textbf{b)}
$\mathcal{G} := (\{Q_0, Q_1, Q_2\}, \{a,b\}, P, Q_0\}$ mit $P$ gegeben durch
$$
    Q_0 \to bQ_0 \mid aQ_1 \mid a
    \qquad
    Q_1 \to bQ_1 \mid aQ_2 \mid b \mid a
    \qquad
    Q_2 \to bQ_2 \mid aQ_0 \mid b
$$

\aufgabe{7}
Wir wissen aus Tutoraufgabe 1, dass die kontextfreien Grammatiken unter
Spiegelbild-bildung abgeschlossen sind.
Für eine gegebene linkslineare Grammatik $\mathcal{G}$ können wir also die
Grammatik $\mathcal{G}^{\mathcal{R}}$ konstruieren. Offensichtlich ist diese
dann rechtslinear, denn:\\
Für jede Produktionsregel $p$ der Form $N \to a$, mit $N$ als Nichtterminal und
$a$ als Terminal haben wir auch die Produktionsregel
$N \to a$ in $\mathcal{G}^{\mathcal{R}}$.\\
Für jede Produktionsregel $p$ der Form $N \to Ba$ mit $N,B$ als Nichtterminal und
$a$ als Terminal haben wir die Produktionsregel
$N \to aB$ in $\mathcal{G}^{\mathcal{R}}$. Da also $\mathcal{G}$ nur Produktionsregeln
dieser Form enthalten kann, kann $\mathcal{G}^\mathcal{R}$ nur Produktionsregeln
enthalten, welche rechtslinear sind.\\
Damit ist also $L(\mathcal{G}^\mathcal{R})$ regulär, und damit auch
$L(\mathcal{G}^\mathcal{R})^\mathcal{R} = L(\mathcal{G})$ regulär, da
die regulären Sprachen unter Spiegelbild-bildung ebenfalls abgeschlossen sind.

Zu einer gegebenen regulären Sprache $L$ existiert eine rechtslineare Grammatik $\mathcal{G}$ sodass \\
$L = L(\mathcal{G}) \setminus \{\varepsilon\}$. Da für jede Sprache $K$ stets
$(K^\mathcal{R})^\mathcal{R} = K$, also die Spiegelbild-Operation ihr eigenes Inverses
ist, folgt nach oben stehendem Argument analog, dass dann $\mathcal{G}^\mathcal{R}$
linkslinear ist.

Insgesamt erzeugen die linkslinearen Grammatiken genau die regulären Sprachen.

\newpage
\aufgabe{8}
\textbf{a)}

Angenommen, $L_1$ sei kontextfrei.\\
Sei $n \geq 1$ gemäß Pumping-Lemma gegeben.
Wir betrachten $z = a^nb^nc^{n^2} \in L_1$.\\
Das Pumping-Lemma liefert die
Zerlegung $z = uvwxy$ mit $vx \neq \varepsilon$ und $|vwx| \leq n$.

\textbf{Fall 1:} $vwx = \sigma^n$ für $\sigma \in \{a,b\}$.\\
Es folgt nach Pumping-Lemma, dass auch $z_2 = uv^2wx^2y \in L_1$, jedoch haben wir dann\\
$|z_2|_c < |z_2|_a \cdot |z_2|_b$, da $vx \neq \varepsilon$ und alle $c's$ in $y$ liegen. Dann ist jedoch $z_2 \notin L_1$. Widerspruch!

\textbf{Fall 2:} $vwx$ liegt ganz im Präfix $a^nb^n$ (und Fall 1 trifft nicht zu).
Es folgt nach Pumping-Lemma, dass auch $z_0 = uwy \in L_1$, jedoch haben wir dann
$|z_0|_c > |z_0|_a \cdot |z_0|_b$, da $vx \neq \varepsilon$ und alle $c's$ in $y$ liegen. Dann ist aber $z_0 \notin L_1$. Widerspruch!

\textbf{Fall 3:} $vwx$ liegt ganz im Suffix $b^nc^{n^2}$ (aber $vwx \neq b^n$).
Es folgt nach dem Pumping-Lemma, dass auch $z_0 = uwy \in L_1$. Es sind
alle $a's$ von $z_0$ in $u$. Wir haben also insgesamt die Gleichung
$$
    |z_0|_c = |z_0|_a \cdot |z_0|_b
    \,\Longleftrightarrow\, n^2-j = n\cdot(n-k)
$$
für $j,k \in \mathbb{N}$ mit $j+k = |vwx| = n$. Andererseits folgt damit
$
    n = n-k+\frac{j}{n} \,\Longrightarrow\, j=nk
$.\\
Setzen wir dies nun in $j+k = n$ ein, folgt $k = \frac{n}{n+1}$. Da jedoch
$n,k \in \mathbb{N}$ haben wir hiermit einen Widerspruch, da für $n > 0$ stets
$\frac{n}{n+1} \notin \mathbb{N}$. Damit haben wir also $|z_0|_c \neq |z_0|_a \cdot |z_0|_b$ und $uwy \notin L_1$. Insgesamt ist also in jedem Fall $L_1$ nicht kontextfrei.\\


\textbf{b)}

Angenommen $L_2$ sei kontextfrei.\\
Sei $n\geq 1$ gemäß Pumping-Lemma, gegeben.
Wir betrachten $z = a^nb^{n+1}c^{n+2} \in L_2$.\\
Das Pumping-Lemma liefert die Zerlegung $z = uvwxy$ mit $vx \neq \varepsilon$ und
$|vwx| \leq n$.

\textbf{Fall 1:} $vwx$ liegt ganz im Präfix $a^nb^{n+1}$.\\
Es folgt nach Pumping-Lemma, dass auch $z_2 = uv^2wx^2y \in L_2$, jedoch haben wir
dann mindestens ein $a$ oder $b$ mehr als in $z$. Wenn wir mindestens ein $b$ mehr
haben, also $|z_2|_b > |z|_b$, dann folgt $|z_2|_b \not< |z_2|_c$, da alle
$c's$ von $z_2$ in $y$ liegen. Wenn wir kein $b$ mehr haben, so müssen wir durch
$vx \neq \varepsilon$ ein $a$ mehr haben, also $|z_2|_a > |z|_a$ und damit $|z_2|_a \not< |z_2|_b$. In beiden Fällen ist dann jedoch $z_2 \notin L_2$. Widerspruch!

\textbf{Fall 2:} $vwx$ liegt ganz im Suffix $b^{n+1}c^{n+2}$.
Es folgt nach Pumping-Lemma, dass auch $z_0 = uwy \in L_2$, jedoch haben wir dann
mindestens ein $b$ oder $c$ weniger als in $z$. Wenn also $|z_0|_b < |z|_b$, dann
folgt $|z_0|_a \not< |z_0|_b$, da alle $a's$ von $z_0$ in $u$ sind.
Wenn wir $|z_0|_c < |z|_c$, dann folgt $|z_0|_b \not< |z_0|_c$.
In beiden Fällen ist dann jedoch $z_0 \notin L_2$. Widerspruch!

\end{document}
