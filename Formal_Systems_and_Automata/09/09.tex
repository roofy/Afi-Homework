\documentclass[a4paper,graphics,11pt]{article}
\pagenumbering{arabic}

\usepackage[margin=1in]{geometry}
\usepackage[utf8]{inputenc}
\usepackage[T1]{fontenc}
\usepackage{lmodern}
\usepackage[ngerman]{babel}
\usepackage{amsmath, tabu}
\usepackage{amsthm}
\usepackage{amssymb}
\usepackage{complexity}
\usepackage{mathtools}
\usepackage{setspace}
\usepackage{graphicx,color,curves,epsf,float,rotating}
\usepackage{tasks}
\setlength{\parindent}{0em}
\setlength{\parskip}{1em}
\usepackage{tikz}
\usetikzlibrary{automata, arrows}

\newcommand{\aufgabe}[1]{\subsection*{Aufgabe #1}}
\newcommand{\up}[2]{\mathrel{\overset{\makebox[0pt]{\mbox{\normalfont\tiny #2}}}{#1}}}

\begin{document}
\noindent Gruppe \fbox{\textbf{3}}             \hfill Phil Pützstück, 377247\\
\noindent Formale Systeme und Automaten \hfill Benedikt Gerlach, 376944\\
\strut\hfill Sebastian Hackenberg, 377550\\
\begin{center}
	\LARGE{\textbf{Hausaufgabe 9}}
\end{center}
\begin{center}
\rule[0.1ex]{\textwidth}{1pt}
\end{center}



\aufgabe{6}

 \begin{tikzpicture}[>=stealth',shorten >=1pt,auto,node distance=4cm]
        \node[state, initial]   (q0)                        {$q_0$};
        \node[state]            (q1)    [above right of=q0] {$q_1$};
        \node[state]            (q2)    [ right of=q1] {$q_2$};
        \node[state]            (q3)    [below right of=q2] {$q_3$};
        \node[state]            (q4)    [ below  of=q3] {$q_4$};
        \node[state]            (q5)    [ left of=q4] {$q_5$};
        \node[state,accepting]            (q6)    [left  of=q5] {$q_6$};
       

        \path[->] (q0)  edge                node {$a,Z_0,Z_0$} (q1);
        
        \path[->] (q1)  edge                node {$b,Z_0,ZZ_0\atop{ b,Z,ZZ}$} (q2);
        \path[->] (q1)  edge [loop above]                node {$a,Z_0,Z_0\atop  a,Z,Z$} (q1);
        \path[->] (q1)  edge [bend right]             node {$c,Z,Z \atop c,Z_0,Z_0$} (q3);
        
        \path[->] (q2)  edge                node {$c,Z,Z$} (q3);
        \path[->] (q2)  edge [loop above]              node {$b,Z,Z$} (q2);
        \path[->] (q2)  edge [bend left]                node {$a,Z,Z$} (q1);
        
        \path[->] (q3)  edge                node {$a,Z,Z\atop a,Z_0,Z_0$} (q4);
        
        
        \path[->] (q4)  edge    [loop right]            node {$a,Z,Z\atop a,Z_0,Z_0$} (q4);
        \path[->] (q4)  edge                node {$b,Z,\varepsilon$} (q5);
        \path[->] (q4)  edge  [bend left]              node {$\varepsilon,Z_0,\varepsilon$} (q6);
        
        \path[->] (q5)  edge                node {$\varepsilon,Z_0,\varepsilon$} (q6);
        \path[->] (q5)  edge [bend left]               node {$a,Z,Z\atop a,Z_0,Z_0$} (q4);
         \path[->] (q5)  edge [loop above]               node {$b,Z,Z\atop b,Z_0,Z_0$} (q5);
        
         
       
    \end{tikzpicture}
    
Der Automat fügt ein Z in den Kellerspeicher für jedes vollständige Infix $ab$ in $u$ ein und entfernt ein Z aus dem Kellerspeicher für jedes vollständige $ab$ in v.

Akzeptierender Lauf auf $acaa$:\\
$(q_0,Z_0,acaa)\rightarrow(q_1,Z_0,caa)\rightarrow(q_3,Z_0,aa)\rightarrow(q_4,Z_0,a)\rightarrow(q_4,Z_0,\varepsilon)\rightarrow(q_6,\varepsilon,\varepsilon)$

Akzeptierender Lauf auf $abaacabba$:\\
$(q_0,Z_0,abaacabba)\rightarrow(q_1,Z_0,baacabba)\rightarrow(q_2,ZZ_0,aacabba)\rightarrow(q_1,ZZ_0,acabba)\rightarrow(q_1,ZZ_0,cabba)$

$\rightarrow(q_3,ZZ_0,abba)\rightarrow(q_4,ZZ_0,bba)\rightarrow(q_5,Z_0,ba)\rightarrow(q_5,Z_0,a)\rightarrow(q_4,Z_0,\varepsilon)\rightarrow(q_6,\varepsilon,\varepsilon)$



\newpage

\aufgabe{7}
Die Idee ist, eine Produktkonstruktion durchzuführen. Da wir nur einen PDA besitzen haben wir nicht das Problem,
zwei Speicher gleichzeitig verwalten zu müssen. Wir lassen also ganz normal beide Automaton laufen
und übernehmen den Speicher des PDA für den Speicher des Produkt-PDA's.

Sei also $\mathcal{A} = (Q_\mathcal{A}, \Sigma, \Gamma, \Delta, q_{0_\mathcal{A}}), Z_0, F_\mathcal{A})$
der PDA
und $\mathcal{B} = (Q_\mathcal{B}, \Sigma, \delta, q_{0_\mathcal{B}}, F_\mathcal{B})$ der DFA.\\
Dann definiere den PDA
$$
    \mathcal{P} := (Q_\mathcal{A} \times Q_\mathcal{B}, \Sigma, \Gamma, \Delta', (q_{0_\mathcal{A}}, q_{0_\mathcal{B}}), Z_0, F')
$$
wobei
$$
    \Delta' := \{((q_\mathcal{A}, q_\mathcal{B}), a, Z, (q'_\mathcal{A}, q'_\mathcal{B}), \gamma)
    \mid (q_\mathcal{A}, a, Z, q'_\mathcal{A}, \gamma) \in \Delta \land \delta(q_\mathcal{B}, a) = q'_\mathcal{B}\}
$$
für $a \in \Sigma, Z \in \Gamma, \gamma \in \Gamma^*$.
Weiter ist
$$
    F' := \{(f_\mathcal{A},f_\mathcal{B}) \mid f_\mathcal{A} \in F_\mathcal{A} \land f_\mathcal{B} \in F_\mathcal{B}\}
$$

\end{document}
