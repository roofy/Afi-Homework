\documentclass[a4paper,graphics,11pt]{article}
\pagenumbering{arabic}

\usepackage[margin=1in]{geometry}
\usepackage[utf8]{inputenc}
\usepackage[T1]{fontenc}
\usepackage{lmodern}
\usepackage[ngerman]{babel}
\usepackage{amsmath, tabu}
\usepackage{amsthm}
\usepackage{amssymb}
\usepackage{complexity}
\usepackage{mathtools}
\usepackage{setspace}
\usepackage{graphicx,color,curves,epsf,float,rotating}
\usepackage{tasks}
\setlength{\parindent}{0em}
\setlength{\parskip}{1em}
\usepackage{tikz}
\usetikzlibrary{automata, arrows}

\newcommand{\aufgabe}[1]{\subsection*{Aufgabe #1}}
\newcommand{\up}[2]{\mathrel{\overset{\makebox[0pt]{\mbox{\normalfont\tiny #2}}}{#1}}}

\begin{document}
\noindent Gruppe \fbox{\textbf{3}}             \hfill Phil Pützstück, 377247\\
\noindent Formale Systeme und Automaten \hfill Benedikt Gerlach, 376944\\
\strut\hfill Sebastian Hackenberg, 377550\\
\begin{center}
	\LARGE{\textbf{Hausaufgabe 9}}
\end{center}
\begin{center}
\rule[0.1ex]{\textwidth}{1pt}
\end{center}



\aufgabe{6}

 \begin{tikzpicture}[>=stealth',shorten >=1pt,auto,node distance=4cm]
        \node[state, initial]   (q0)                        {$q_0$};
        \node[state]            (q1)    [above right of=q0] {$q_1$};
        \node[state]            (q2)    [ right of=q1] {$q_2$};
        \node[state]            (q3)    [below right of=q2] {$q_3$};
        \node[state]            (q4)    [ below  of=q3] {$q_4$};
        \node[state]            (q5)    [ left of=q4] {$q_5$};
        \node[state,accepting]            (q6)    [left  of=q5] {$q_6$};
       

        \path[->] (q0)  edge                node {$a,Z_0,Z_0$} (q1);
        
        \path[->] (q1)  edge                node {$b,Z_0,ZZ_0\newline b,Z,ZZ$} (q2);
        \path[->] (q1)  edge [loop above]                node {$a,Z_0,Z_0\newline  a,Z,Z$} (q1);
        \path[->] (q1)  edge [bend right]             node {$c,Z,Z \newline c,Z_0,Z_0$} (q3);
        
        \path[->] (q2)  edge                node {$c,Z,Z$} (q3);
        \path[->] (q2)  edge [loop above]              node {$b,Z,Z$} (q2);
        \path[->] (q2)  edge [bend left]                node {$a,Z,Z$} (q1);
        
        \path[->] (q3)  edge                node {$a,Z,Z$} (q4);
        
        \path[->] (q4)  edge    [loop right]            node {$a,Z,Z$} (q4);
        \path[->] (q4)  edge                node {$b,Z,\varepsilon$} (q5);
        \path[->] (q4)  edge  [bend left]              node {$\varepsilon,Z_0,\varepsilon$} (q6);
        
        \path[->] (q5)  edge                node {$\varepsilon,Z_0,\varepsilon$} (q6);
        \path[->] (q5)  edge [bend left]               node {$a,Z,Z$} (q4);
        
         
       
    \end{tikzpicture}
    
Der Automat fügt ein Z in den Kellerspeicher für jedes vollständige Infix $ab$ in $u$ ein und entfernt ein Z aus dem Kellerspeicher für jedes vollständige $ab$ in v.





\end{document}
