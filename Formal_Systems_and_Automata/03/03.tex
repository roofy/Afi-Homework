\documentclass[a4paper,graphics,11pt]{article}
\pagenumbering{arabic}

\usepackage[margin=1in]{geometry}
\usepackage[utf8]{inputenc}
\usepackage[T1]{fontenc}
\usepackage{lmodern}
\usepackage[ngerman]{babel}
\usepackage{amsmath, tabu}
\usepackage{amsthm}
\usepackage{amssymb}
\usepackage{complexity}
\usepackage{mathtools}
\usepackage{setspace}
\usepackage{graphicx,color,curves,epsf,float,rotating}
\usepackage{tasks}
\setlength{\parindent}{0em}
\setlength{\parskip}{1em}
\usepackage{tikz}
\usetikzlibrary{automata, arrows}

\newcommand{\aufgabe}[1]{\subsection*{Aufgabe #1}}
\newcommand{\up}[2]{\mathrel{\overset{\makebox[0pt]{\mbox{\normalfont\tiny #2}}}{#1}}}

\begin{document}
\noindent Gruppe \fbox{\textbf{3}}             \hfill Phil Pützstück, 377247\\
\noindent Formale Systeme und Automaten \hfill Benedikt Gerlach, 376944\\
\strut\hfill Sebastian Hackenberg, 377550\\
\begin{center}
	\LARGE{\textbf{Hausaufgabe 3}}
\end{center}
\begin{center}
\rule[0.1ex]{\textwidth}{1pt}
\end{center}


\aufgabe{5}

\begin{tikzpicture}[>=stealth',shorten >=1pt,auto,node distance=4cm]
	\node[state, initial]							(q0)						{$\big\{q_0\big\}$};
	\node[state,accepting]						(q1q2) 	[right of=q0]			{$\big\{q_1,q_2\big\}$};
	\node[state,accepting]						(q1q3) 	[above of=q1q2]		{$\big\{q_1,q_3\big\}$};
	\node[state]								(q2)	 	[below of=q1q2]		{$\big\{q_2\big\}$};
	\node[state]								(q3) 		[right of=q1q2]		{$\big\{q_3\big\}$};
	\node[state]								(qs) 		[right of=q3]			{$\big\{SENKE\big\}$};
	
	
	

	\path[->] (q0) edge [loop above]		node {a,b} (q0);
	\path[->] (q0) edge 				node {c} (q1q2);
	
	\path[->] (q1q2) edge [loop above]		node {c} (q1q2);
	\path[->] (q1q2) edge 				node {a} (q3);
	\path[->] (q1q2) edge [bend left]		node {b} (qs);
	
	\path[->] (q1q3) edge [loop above]		node {b} (q1q3);
	\path[->] (q1q3) edge [bend right]		node {a} (q3);
	\path[->] (q1q3) edge [bend right]		node {c} (q2);
	
	\path[->] (q3) edge 				node {a,c} (qs);
	\path[->] (q3) edge [bend right]			node {b} (q1q3);	
		
	
	\path[->] (q2) edge 				node {a} (q3);
	\path[->] (q2) edge 				node {b} (qs);
	\path[->] (q2) edge 				node {c} (q1q2);
	
	\path[->] (qs) edge [loop above]		node {a,b,c} (qs);
	
	
	
	
	
	
\end{tikzpicture}

\newpage
\aufgabe{6}
\newpage
\aufgabe{7}
\textbf{a)}\\
von $ q_0$ mit a: -\\
von $q_0$ mit b:$q_0,q_1$\\
von $q_1$ mit a:$q_0$\\
von $q_1 $mit b:$q_0,q_1,q_3$\\
von $q_2$ mit a:-\\
von $q_2$ mit b:-\\
von $q_3$ mit a:$q_0$\\
von $q_3$ mit b:$q_0,q_1,q_3$\\
von $q_4$ mit a:-\\
von $q_4$ mit b:$q_2$\\
von $q_5$ mit a:$q_4$\\
von $q_5$ mit b:$q_2$\\

\textbf{b)}\\

\begin{tikzpicture}[>=stealth',shorten >=1pt,auto,node distance=5cm]
	\node[state, initial]							(q0)						{${q_0}$};
	\node[state,accepting]								(q1)	[right of = q0]			{${q_1}$};
	\node[state]								(q2)	[right of = q1]			{${q_2}$};
	\node[state]								(q3)	[below of = q0]			{${q_3}$};
	\node[state,accepting]								(q4)	[right of = q3]			{${q_4}$};
	\node[state]								(q5)	[right of = q4]			{${q_5}$};
	
	
	\path[->] (q0) edge [loop above]		node {a,b} (q0);
	\path[->] (q0) edge [bend left]				node {b} (q1);
	\path[->] (q0) edge [bend left]				node {b} (q3);
	
	\path[->] (q1) edge [loop above]		node {b} (q1);
	\path[->] (q1) edge [bend left]				node {a,b} (q0);
	\path[->] (q1) edge [bend left]				node {b} (q3);
	\path[->] (q1) edge 				node {a} (q4);
	
	\path[->] (q3) edge [loop below]				node {b} (q3);
	\path[->] (q3) edge [bend left]				node {a,b} (q0);
	\path[->] (q3) edge [bend left]				node {b} (q1);
	
	\path[->] (q4) edge [bend left] 				node {b} (q2);
	\path[->] (q4) edge [bend left] 				node {b} (q5);
	
	\path[->] (q2) edge [loop above]				node {b} (q2);
	\path[->] (q2) edge [bend left]				node {a} (q4);
	
	\path[->] (q5) edge [bend left]				node {a} (q4);
	\path[->] (q5) edge 				node {b} (q2);
\end{tikzpicture}

\newpage
\aufgabe{8}
Es gibt einen Automaten $\mathcal{A}$, welcher binäre, durch 3 teilbare Zahlen erkennt (Bsp 1.19 Skript).
Des weiteren wissen wir, dass es einen FA gibt welcher alle Wörter akzeptieren,
in denen ein gegebenes Symbol weniger als eine feste Grenze (hier 42) vorkommt (HA02). So ein FA $\mathcal{B}$
soll im folgenden alle Wörter mit höchstens 42 Einsen erkennen.
Zu guter letzt haben wir einen FA zur Erkennung Infixen in der letzten Abgabe definiert bzw gezeigt, dass diese Sprache FA-erkennbar ist (HA02 Nr9). Der FA $\mathcal{C}$ soll dies hier für das Infix 101010 tun.

Wir haben also, dass $L := L(\mathcal{A}), K := L(\mathcal{B})$ und $M := L(\mathcal{C})$ FA-erkennbar sind.
Nach Satz 2.40 folgt nun, dass ebenso $\overline{M}, L \cap K$ und $(\overline{M}) \cup (L \cap K)$ FA-erkennbar
sind.
Dies ist für ein Wort $w \in \{0, 1\}^*$ logisch zu interpretieren als
$$
    (w \in \overline{M}) \lor (w \in L \cap K)
    \equiv \lnot (w \in M) \lor (w \in L \land w \in K)
    \equiv w \in M \,\Longrightarrow\, w \in L \land w \in K
$$
Dies entspricht eben genaus der Aussage; Wenn 101010 als Infix in $w$ vorkommt ($w \in M$), dann muss 
$w$ höchstens 42 Einsen haben ($w \in K$) und durch 3 teilbar sein ($w \in L$).

Insgesamt ist dies also FA-erkennbar.
\end{document}
