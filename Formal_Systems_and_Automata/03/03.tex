\documentclass[a4paper,graphics,11pt]{article}
\pagenumbering{arabic}

\usepackage[margin=1in]{geometry}
\usepackage[utf8]{inputenc}
\usepackage[T1]{fontenc}
\usepackage{lmodern}
\usepackage[ngerman]{babel}
\usepackage{amsmath, tabu}
\usepackage{amsthm}
\usepackage{amssymb}
\usepackage{complexity}
\usepackage{mathtools}
\usepackage{setspace}
\usepackage{graphicx,color,curves,epsf,float,rotating}
\usepackage{tasks}
\setlength{\parindent}{0em}
\setlength{\parskip}{1em}
\usepackage{tikz}
\usetikzlibrary{automata, arrows}

\newcommand{\aufgabe}[1]{\subsection*{Aufgabe #1}}
\newcommand{\up}[2]{\mathrel{\overset{\makebox[0pt]{\mbox{\normalfont\tiny #2}}}{#1}}}

\begin{document}
\noindent Gruppe \fbox{\textbf{3}}             \hfill Phil Pützstück, 377247\\
\noindent Formale Systeme und Automaten \hfill Benedikt Gerlach, 376944\\
\strut\hfill Sebastian Hackenberg, 377550\\
\begin{center}
	\LARGE{\textbf{Hausaufgabe 3}}
\end{center}
\begin{center}
\rule[0.1ex]{\textwidth}{1pt}
\end{center}


\aufgabe{5}
\newpage
\aufgabe{6}
Es sei o.B.d.A ein DFA $\mathcal{A}$ gegeben (sonst machen wir den gegebenen Automaten zum DFA nach Methoden
der Vorlesung).

Es sei $\mathcal{A} := (Q, \Sigma, \delta, q_0, F)$.
Wir definieren den $\varepsilon$-NFA $\mathcal{B} := (Q', \Sigma', \Delta, q_0', F')$ mit
$$
    Q' := Q\ \dot\cup\ \{q_0'\}\ \dot\cup\ \{(q, \varepsilon) \mid q \in Q\}
    \qquad \Sigma' := \Sigma\ \dot\cup\ \{\varepsilon\}
    \qquad F' := \{(q, \varepsilon) \mid q \in F)
$$$$
    \Delta := \{(q, a, q') \mid \exists q,q' \in Q, a \in \Sigma : \delta(q, a) = q'\}\ \dot\cup\ \Delta_{Q \to \varepsilon}\ \dot\cup\ \Delta_\varepsilon
$$
Wobei
$$
    \Delta_{Q \to \varepsilon} := \{(q, \varepsilon, (q', \varepsilon)) \mid \exists q,q' \in Q, a \in \Sigma : \delta(q, a) = q'\}
$$$$
    \Delta_\varepsilon := \{((q, \varepsilon), a, (q', \varepsilon)) \mid \exists q,q' \in Q, a \in \Sigma : \delta(q, a) = q'\}
$$
Die Idee ist, dass wir in jedem Zustand $q \in Q$ von $\mathcal{A}$ die Möglichkeit bieten, durch eine\\
$\varepsilon$-Transition zu dem nächsten Zustand zu kommen. Da wir dies jedoch nur einmal machen dürfen
(Es ex. genau 1 Lücke pro Wort), gehen wir dann zu einer Kopie der Zustände aus $Q$, hier mit $(q, \varepsilon)$
betitelt über, in denen diese $\varepsilon$-Transitionen nicht mehr möglich sind. Sobald wir uns einmal in diesem
Bereich der $\varepsilon$-Zustände befinden verhält sich der Automat wie vorher und akzeptiert die Kopien der 
eigentlichen Endzustände.

Wir zeigen $L(\mathcal{B}) \subseteq (L(A))^-$.

\newpage
\aufgabe{7}
\newpage
\aufgabe{8}
Es gibt einen Automaten $\mathcal{A}$, welcher binäre, durch 3 teilbare Zahlen erkennt (Bsp 1.19 Skript).
Des weiteren wissen wir, dass es einen FA gibt welcher alle Wörter akzeptiert,
in denen ein gegebenes Symbol weniger als eine feste Grenze (hier 42) vorkommt (HA02). So ein FA $\mathcal{B}$
soll im folgenden alle Wörter mit höchstens 42 Einsen erkennen.
Zu guter letzt haben wir einen FA zur Erkennung Infixen in der letzten Abgabe definiert bzw gezeigt, dass diese Sprache FA-erkennbar ist (HA02 Nr9). Der FA $\mathcal{C}$ soll dies hier für das Infix 101010 tun.

Wir haben also, dass $L := L(\mathcal{A}), K := L(\mathcal{B})$ und $M := L(\mathcal{C})$ FA-erkennbar sind.
Nach Satz 2.40 folgt nun, dass ebenso $\overline{M}, L \cap K$ und $(\overline{M}) \cup (L \cap K)$ FA-erkennbar
sind.
Dies ist für ein Wort $w \in \{0, 1\}^*$ logisch zu interpretieren als
$$
    (w \in \overline{M}) \lor (w \in L \cap K)\
    \equiv\ \lnot (w \in M) \lor (w \in L \land w \in K)\
    \equiv\ w \in M \,\Longrightarrow\, (w \in L \land w \in K)
$$
Dies entspricht eben genau der Aussage; Wenn 101010 als Infix in $w$ vorkommt ($w \in M$), dann muss 
$w$ höchstens 42 Einsen haben ($w \in K$) und durch 3 teilbar sein ($w \in L$).

Insgesamt ist dies also FA-erkennbar.
\end{document}
