\documentclass[a4paper,graphics,11pt]{article}
\pagenumbering{arabic}

\usepackage[margin=1in]{geometry}
\usepackage[utf8]{inputenc}
\usepackage[T1]{fontenc}
\usepackage{lmodern}
\usepackage[ngerman]{babel}
\usepackage{amsmath, tabu}
\usepackage{amsthm}
\usepackage{amssymb}
\usepackage{complexity}
\usepackage{mathtools}
\usepackage{setspace}
\usepackage{graphicx,color,curves,epsf,float,rotating}
\usepackage{tasks}
\setlength{\parindent}{0em}
\setlength{\parskip}{1em}
\usepackage{tikz}
\usetikzlibrary{automata, arrows}

\tikzset{
  heap/.style={
    every node/.style={circle, draw},
    every rectangle node/.style={draw},
    level 1/.style={sibling distance=20mm},
    level 2/.style={sibling distance=10mm}
  }
}

\newcommand{\aufgabe}[1]{\subsection*{Aufgabe #1}}
\newcommand{\up}[2]{\mathrel{\overset{\makebox[0pt]{\mbox{\normalfont\tiny #2}}}{#1}}}

\begin{document}
\noindent Gruppe \fbox{\textbf{3}}             \hfill Phil Pützstück, 377247\\
\noindent Formale Systeme und Automaten \hfill Benedikt Gerlach, 376944\\
\strut\hfill Sebastian Hackenberg, 377550\\
\begin{center}
	\LARGE{\textbf{Hausaufgabe 3}}
\end{center}
\begin{center}
\rule[0.1ex]{\textwidth}{1pt}
\end{center}


\aufgabe{5}
\textbf{a)}
\underline{S} $\to$ b\underline{S}SS $\to$ ba\underline{S}S $\to$ ban\underline{S}S $\to$ bana\underline{S}
$\to$ banan\underline{S} $\to$ banana
\begin{center}
    \begin{tikzpicture}[heap]
        \node {S}
        child{node[rectangle]{b}}
        child{node{S}
            child{node[rectangle]{a}}}
        child{node{S}
            child{node[rectangle]{n}}
            child{node{S}
                child{node[rectangle]{a}}}}
        child{node{S}
            child{node[rectangle]{n}}
            child{node{S}
                child{node[rectangle]{a}}}}
      ;
    \end{tikzpicture}
\end{center}
\textbf{b)}
$\mathcal{G}$ ist nicht eindeutig. Wir haben für das Wort baaaa folgende verschiedene Ableitungsbäume:

\begin{minipage}{0.5\textwidth}
    \begin{tikzpicture}[heap]
        \node {S}
        child{node[rectangle]{b}}
        child{node{S}
            child{node[rectangle]{a}}}
        child{node{S}
            child{node[rectangle]{a}}}
        child{node{S}
            child{node[rectangle]{a}}
            child{node{S}
                child{node[rectangle]{a}}}}
      ;
    \end{tikzpicture}
\end{minipage}
\begin{minipage}{0.5\textwidth}
    \begin{tikzpicture}[heap]
        \node {S}
        child{node[rectangle]{b}}
        child{node{S}
            child{node[rectangle]{a}}
            child{node{S}
                child{node[rectangle]{a}}}}
        child{node{S}
            child{node[rectangle]{a}}}
        child{node{S}
            child{node[rectangle]{a}}}
    ;
    \end{tikzpicture}
\end{minipage}

\aufgabe{6}
\textbf{a)}\\
Es wird folgende reguläre Sprach erzeugt: $L(a(ba)^+\ +\ (ba)^+ b)$.


\textbf{b)}\\
Diese Sprache ist nicht regulär. Es wird erzeugt: $L := \{w \in \Sigma^* \mid |w|_1 \geq |w|_0\}$, also
alle Wörter über dem gegebenen Alphabet, welche mehr 1 als 0 beinhalten.

\newpage

\textbf{c)}

\aufgabe{7}
Die Grammatik der Listensprache \texttt{LIST(BIN)} ist
$\mathcal{G} := (\{S, B, D\}, \Sigma^*_{\texttt{ASCII}}\}, P, S)$ wobei $P$\\
folgende Produktionsregeln hat:
$$
    S \to \texttt{nil} \mid \texttt{cons(}B\texttt{,}S\texttt{)}
    \qquad
    B \to \texttt{0} \mid \texttt{1}D
    \qquad
    D \to \texttt{0}D \mid \texttt{1}D \mid \varepsilon
$$

\aufgabe{8}
Da die Grammatik schon frei von jeglichen $\varepsilon$-Regeln ist, können wir uns den ersten Schritt sparen.

Nach dem 2. Schritt erhalten wir $\mathcal{G}_2 := (\{S, A, B, C, D, T_a, T_b\}, \{a,b\}, P, S)$ mit
$$
    S \to AB \mid BA \mid C
    \qquad
    A \to DAD \mid T_a
    \qquad
    B \to DBD \mid T_b
$$$$
    C \to A \mid B \mid T_aT_bC \mid D
    \qquad
    D \to T_a \mid T_b
    \qquad
    T_a \to a
    \qquad
    T_b \to b
$$

Nach dem 3. Schritt erhalten wir $\mathcal{G}_3 := (\{S, A, B, C, D, T_a, T_b\}, \{a,b\}, P, S)$ mit
$$
    S \to AB \mid BA \mid T_aT_bC \mid DAD \mid DBD \mid a \mid b
    \qquad
    A \to DAD \mid a
$$$$
    B \to DBD \mid b
    \qquad
    C \to T_aT_bC \mid DAD \mid DBD \mid a \mid b
$$$$
    D \to a \mid b
    \qquad
    T_a \to a
    \qquad
    T_b \to b
$$

Nach dem 4. Schritt erhalten wir $\mathcal{G}_4 := (\{S, A, B, C, D, T_a, T_b, X, Y, Z\}, \{a,b\}, P, S)$ mit
$$
    S \to AB \mid BA \mid T_aX \mid DY \mid DZ \mid a \mid b
    \qquad
    A \to DY \mid a
$$$$
    B \to DZ \mid b
    \qquad
    C \to T_aX \mid DY \mid DZ \mid a \mid b
    \qquad
    D \to a \mid b
$$$$
    X \to T_bC
    \qquad
    Y \to AD
    \qquad
    Z \to BD
    \qquad
    T_a \to a
    \qquad
    T_b \to b
$$

Diese Grammatik ist nach 7.20 äquivalent zur gegebenen und offensichtlich auch in CNF.
\end{document}
