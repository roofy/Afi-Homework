\documentclass[a4paper,graphics,11pt]{article}
\pagenumbering{arabic}

\usepackage{algpseudocode}
\usepackage{algorithm}
\usepackage[margin=1in]{geometry}
\usepackage[utf8]{inputenc}
\usepackage[T1]{fontenc}
\usepackage{lmodern}
\usepackage[ngerman]{babel}
\usepackage{amsmath, tabu}
\usepackage{amsthm}
\usepackage{amssymb}
\usepackage{complexity}
\usepackage{mathtools}
\usepackage{setspace}
\usepackage{graphicx,color,curves,epsf,float,rotating}
\usepackage{tasks}
\setlength{\parindent}{0em}
\setlength{\parskip}{1em}
\usepackage{tikz}
\usetikzlibrary{automata, arrows, shapes.geometric,positioning}

\tikzset{
    heap/.style={
    every node/.style={circle, draw},
    every rectangle node/.style={draw},
    level 1/.style={sibling distance=20mm},
    level 2/.style={sibling distance=10mm}
    }
}

\newcommand{\aufgabe}[1]{\subsection*{Aufgabe #1}}
\newcommand{\up}[2]{\mathrel{\overset{\makebox[0pt]{\mbox{\normalfont\tiny #2}}}{#1}}}

\begin{document}
\noindent Gruppe \fbox{\textbf{3}}             \hfill Phil Pützstück, 377247\\
\noindent Formale Systeme und Automaten \hfill Benedikt Gerlach, 376944\\
\strut\hfill Sebastian Hackenberg, 377550\\
\begin{center}
    \LARGE{\textbf{Hausaufgabe 11}}
\end{center}
\begin{center}
    \rule[0.1ex]{\textwidth}{1pt}
\end{center}

\aufgabe{5}
\textbf{a)}
\begin{center}
    \begin{tikzpicture}[heap]
        \node {S}
        child{node[rectangle]{t}}
        child{node[rectangle]{(}}
        child{node{S}
            child{node[rectangle]{l}}
            child{node[rectangle]{(}}
            child{node{Z}
                child{node[rectangle]{0}}}
            child{node[rectangle]{)}}}
        child{node[rectangle]{,}}
        child{node{S}
            child{node[rectangle]{t}}
            child{node[rectangle]{(}}
            child{node{S}
                child{node[rectangle]{l}}
                child{node[rectangle]{(}}
                child{node{Z}
                    child{node[rectangle]{1}}}
                child{node[rectangle]{)}}
            }
            child{node[rectangle]{)}}}
        child{node[rectangle]{)}}
        ;
    \end{tikzpicture}
\end{center}
\textbf{c)}
\begin{center}
    \begin{tikzpicture}[>=stealth',shorten >=1pt,auto,node distance=4cm]
        \node[state, initial]   (q0)     {$q_0$};
        \node[state] (q1)  [below left of=q0]            {$q_1$};
        \node[state,accepting] (q2)  [below right of=q0]            {$q_2$};

        \path[->] (q0)  edge [loop above] node
        {\Large$\substack{
            t,\gamma,X_t\gamma \\[2pt]
            \texttt{(},X_t,X_{t\texttt{(}}
        }$} (q0);
        \path[->] (q0) edge node [sloped,above] {$l,\gamma',X_l\gamma'$} (q1);

        \path[->] (q1)  edge [loop left] node 
        {\Large$\substack{
            \texttt{(},X_l,X_{l\texttt{(}}\\[2pt]
            i,X_{l\texttt{(}},X_{l\texttt{(}Z}
        }$} (q1);

        \path[->] (q1) edge node {$\texttt{)},X_{l\texttt{(}Z},\varepsilon$} (q2);

        \path[->] (q2)  edge [loop right] node
        {\Large$\substack{
            \texttt{)},X_{t\texttt{(}},\varepsilon\\[2pt]
            \texttt{)},X_{t\texttt{(}2},\varepsilon\\[2pt]
        }$} (q2);

        \path[->] (q2) edge node [sloped,above] {$\texttt{,},X_{t\texttt{(}},X_{t\texttt{(}2}$} (q0);
    \end{tikzpicture}
\end{center}

\begin{algorithm}
    \begin{algorithmic}[1]
        \Function{ParseTree}{$text$}
            \State $a \leftarrow$ first symbol of $text$
            \State initialise Stack $S$
            \While{$a \neq \texttt{EOF}$}
                \If{$a \in \{0,1\}$}
                    \State $T \leftarrow$ \Call{NumTree}{$a$}; \Call{Push}{$S,T$}
                \ElsIf{$a \in \{\texttt{t},\texttt{l}, \texttt{(}, \texttt{,}\}$}
                    \State \Call{Push}{$S,a$}
                \ElsIf{$a = \texttt{)}$}
                    \State $T \leftarrow$ \Call{Reduce}{$S$}; \Call{Push}{$S,T$}
                \EndIf
            \EndWhile
            \State $T \leftarrow$ \Call{Pop}{$S$}
            \If{$T$ is a tree and $S = \varnothing$}
                \State \Return $T$
            \EndIf
        \EndFunction\\

        \Function{Reduce}{$S$}
            \State $T_1 \leftarrow$ \Call{Pop}{$S$}
            \If{$T_1$ is a NumTree}
                \State $X_\texttt{(} \leftarrow$ \Call{Pop}{$S$}
                \State $X_\texttt{l} \leftarrow$ \Call{Pop}{$S$}
                \If{$X_\texttt{(} = \texttt{(}$ and $X_\texttt{l} = \texttt{l}$}
                    \State \Return \Call{Leaf}{$T$}
                \EndIf
            \ElsIf{$T_1$ is a S-Tree}
                \State $k \leftarrow$ \Call{Pop}{$S$}
                \If{$k = \texttt{,}$}
                    \State $T_2 \leftarrow$ \Call{Pop}{$S$}
                    \State $X_\texttt{(} \leftarrow$ \Call{Pop}{$S$}
                    \State $X_\texttt{t} \leftarrow$ \Call{Pop}{$S$}
                    \If{$T_2$ is a S-Tree and $X_\texttt{(} = \texttt{(}$ and $X_{\texttt{t}} = \texttt{t}$}
                        \State \Return \Call{BinaryTree}{$T_2,T_1$}
                    \EndIf
                \ElsIf{$k = \texttt{(}$}
                    \State $X_\texttt{t} \leftarrow$ \Call{Pop}{$S$}
                    \If{$X_\texttt{t} = \texttt{t}$}
                        \State \Return \Call{SingleTree}{$T_1$}
                    \EndIf
                \EndIf
            \EndIf
        \EndFunction
    \end{algorithmic}
\end{algorithm}

\newpage
Dabei werden folgende Ableitungsbäume widergegeben:

\begin{minipage}{0.5\textwidth}
    NumTree($a$):

    \begin{tikzpicture}[heap]
        \node {$Z$}
            child{node[rectangle]{$a$}}
        ;
    \end{tikzpicture}
\end{minipage}
\begin{minipage}{0.5\textwidth}
    Leaf($Z$):

    \begin{tikzpicture} [heap,
        triangle/.style = {draw, regular polygon, regular polygon sides=3},
        node rotated/.style = {rotate=180},
        border rotated/.style = {shape border rotate=180}
    ]
        \node {S}
            child{node[rectangle]{l}}
            child{node[rectangle]{(}}
            child{node[triangle] {$Z$}}
            child{node[rectangle]{)}}
            ;
    \end{tikzpicture}
\end{minipage}

    BinaryTree($T_1,T_2$):

    \begin{tikzpicture} [heap,
        triangle/.style = {draw, regular polygon, regular polygon sides=3},
        node rotated/.style = {rotate=180},
        border rotated/.style = {shape border rotate=180}
    ]
        \node {S}
            child{node[rectangle]{t}}
            child{node[rectangle]{(}}
            child{node[triangle] {$T_1$}}
            child{node[rectangle]{,}}
            child{node[triangle] {$T_2$}}
            child{node[rectangle]{)}}
            ;
    \end{tikzpicture}


\begin{minipage}{0.5\textwidth}
    SingleTree($T$):

    \begin{tikzpicture} [heap,
        triangle/.style = {draw, regular polygon, regular polygon sides=3},
        node rotated/.style = {rotate=180},
        border rotated/.style = {shape border rotate=180}
    ]
        \node {S}
            child{node[rectangle]{t}}
            child{node[rectangle]{(}}
            child{node[triangle] {$T$}}
            child{node[rectangle]{)}}
            ;
    \end{tikzpicture}
\end{minipage}

\aufgabe{6}
\textbf{a)}

Wir markieren folgende Nichtterminale in angegebener Reihenfolge:
C, A, D, G\\
Wir können keine weitern Nichtterminale markieren, da sie eine zirkuläre
Abhängigkeit haben:\\
Um S zu markieren, müssten wir B oder F markiert haben.\\
Um B zu markieren, müssten wir S oder F markiert haben.\\
Um F zu markieren, müssten wir B markiert haben.

\textbf{b)}
Insbesondere ist die Sprache damit leer, da S kein terminierendes Nichtterminal ist.

\aufgabe{7}
\begin{minipage}{0.5\textwidth}
    \begin{tabular}{*{7}{c|}}
        \cline{2-7}
                            a & A,B,C   & S & A & S,B & A & S,B   \\  \cline{2-7}
        \multicolumn{1}{c} {} & b & B,C     &   & A & S & A       \\  \cline{3-7}
        \multicolumn{2}{c} {} & c & C       & S & A & S           \\  \cline{4-7}
        \multicolumn{3}{c} {} & a & A,B,C   & S & A               \\  \cline{5-7}
        \multicolumn{4}{c} {} & b & B,C     &                     \\  \cline{6-7}
        \multicolumn{5}{c} {} & c & C                       \\  \cline{7-7}
    \end{tabular}
\end{minipage}
\begin{minipage}{0.5\textwidth}
    \begin{tabular}{*{6}{c|}}
        \cline{2-6}
                            c & C &   &   & S & A   \\  \cline{2-6}
        \multicolumn{1}{c} {} & c & C & S & A & S       \\  \cline{3-6}
        \multicolumn{2}{c} {} & a & A,B,C & S & A           \\  \cline{4-6}
        \multicolumn{3}{c} {} & c & C &                 \\  \cline{5-6}
        \multicolumn{4}{c} {} & c & C                   \\  \cline{6-6}
    \end{tabular}
\end{minipage}

Damit gilt $abcabc \in L(\mathcal{G})$ und $ccacc \notin L(\mathcal{G})$.

\aufgabe{8}
Sei eine kontextfreie Grammatik $\mathcal{G} = (N, \Sigma, P, S)$ gegeben.
Wenn $|L(\mathcal{G})| = \infty$, so muss ein Wort $z \in L(\mathcal{G})$
mit $|z| \geq 2^{|N|+1}$ existieren, sodass der Ableitungsbaum
von $z$ mindestens Höhe $h \geq |N| + 1$ hat, also eins der endlich vielen
Nichtterminale mehr als einmal in der Ableitung von $z$ vorkommt (Idee Pumping-Lemma).
Andererseits haben wir nun endlich viele Nichtterminale und damit endlich viele
Ableitungen, in denen jedes Nichtterminal höchstens ein mal vorkommt.\\
Weiter gilt auch, dass wenn ein $z \in L(\mathcal{G})$ mit $|z| \geq n = 2^{|N|+1}$
existiert (wobei $n$ eben genau das $n$ aus dem Pumping-Lemma ist), dass
dann das Pumping Lemma uns einer Zerlegung liefert, welche wir insebsondere beliebig
oft ''pumpen'' können, um unendlich viele verschiedene Wörter in $L(\mathcal{G})$ 
zu erhalten.\\
Insgesamt haben wir also gezeigt, dass:
$$
    \exists z \in L(\mathcal{G}) : |z| \geq 2^{|N|+1}
    \qquad \,\Longleftrightarrow\,\qquad
    |L(\mathcal{G})| = \infty
$$
Nun können wir wie folgt entscheiden, ob $L(\mathcal{G})$ zu einer gegebenen 
Grammatik $\mathcal{G}$ unendlich ist:
Es ist $\Sigma$ endlich, also ist $\Sigma^n\Sigma^*$ ein regulärer Ausdruck,
welcher alle Wörter über dem Eingabealphabeit mit mindestens Länge $n := 2^{|N|+1}$
erkennt. Dann können wir einen DFA $\mathcal{A}$ erstellen mit $L(\mathcal{A}) =
L(\Sigma^n\Sigma^*)$. Weiter können wir aus der gegebenen Grammatik $\mathcal{G}$
einen PDA $\mathcal{B}$ mit $L(\mathcal{B}) = L(\mathcal{G})$ erstellen, welcher
mit leerem Zustand akzeptiert. Dann können wir den PDA $\mathcal{C}$ erstellen, welcher
die Produktkonstruktion der beiden darstellt.
Also $L(\mathcal{C}) = L(\mathcal{A}) \cap L(\mathcal{B})$.
Zu diesem können wir nun wieder eine kontextfreie Grammatik $\mathcal{G}'$ erstellen,
sodass $L(\mathcal{G}') = L(\mathcal{C}) $. Ferner ist also
$L(\mathcal{G}') = \{z \mid z \in L(\mathcal{G}) \land |z| \geq n\}$.
Wenn wir also mit dem Markierungsalgoritmus das Leerheitsproblem auf $\mathcal{G}'$
lösen, so wissen wir, ob die Sprache $L(\mathcal{G})$ ein Wort der Länge $\geq n$ enthält.
Nach dem obigen Beweis ist dann
$
    L(\mathcal{G}') \neq \varnothing \,\Longrightarrow\, |L(\mathcal{G})| = \infty
$.\\
Somit können wir mit dem angegebenen Algorithmus testen, ob die Sprache einer
gegebenen kontextfreien Grammatik unendlich ist.
\end{document}
