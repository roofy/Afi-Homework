\documentclass[a4paper,graphics,11pt]{article}
\pagenumbering{arabic}

\usepackage[margin=1in]{geometry}
\usepackage[utf8]{inputenc}
\usepackage[T1]{fontenc}
\usepackage{lmodern}
\usepackage[ngerman]{babel}
\usepackage{amsmath, tabu}
\usepackage{amsthm}
\usepackage{amssymb}
\usepackage{complexity}
\usepackage{mathtools}
\usepackage{setspace}
\usepackage{graphicx,color,curves,epsf,float,rotating}
\usepackage{tasks}
\setlength{\parindent}{0em}
\setlength{\parskip}{1em}
\usepackage{tikz}
\usetikzlibrary{automata, arrows}

\newcommand{\aufgabe}[1]{\subsection*{Aufgabe #1}}
\newcommand{\up}[2]{\mathrel{\overset{\makebox[0pt]{\mbox{\normalfont\tiny #2}}}{#1}}}

\begin{document}
\noindent Gruppe \fbox{\textbf{3}}             \hfill Phil Pützstück, 377247\\
\noindent Formale Systeme und Automaten \hfill Benedikt Gerlach, 376944\\
\strut\hfill Sebastian Hackenberg, 377550\\
\begin{center}
	\LARGE{\textbf{Hausaufgabe 11}}
\end{center}
\begin{center}
\rule[0.1ex]{\textwidth}{1pt}
\end{center}



\aufgabe{5}
\newpage
\aufgabe{6}
\textbf{a)}

Wir markieren folgende Nichtterminale in angegebener Reihenfolge:
C, A, D, G\\
Wir können keine weitern Nichtterminale markieren, da sie eine zirkuläre
Abhängigkeit haben:\\
Um S zu markieren, müssten wir B oder F markiert haben.\\
Um B zu markieren, müssten wir S oder F markiert haben.\\
Um F zu markieren, müssten wir B markiert haben.

\textbf{b)}
Insbesondere ist die Sprache damit leer, da S kein terminierendes Nichtterminal ist.

\aufgabe{7}
\begin{minipage}{0.5\textwidth}
\begin{tabular}{*{7}{c|}}
                                                            \cline{2-7}
                            a & A,B,C   & S & A & S,B & A & S,B   \\  \cline{2-7}
    \multicolumn{1}{c} {} & b & B,C     &   & A & S & A       \\  \cline{3-7}
    \multicolumn{2}{c} {} & c & C       & S & A & S           \\  \cline{4-7}
    \multicolumn{3}{c} {} & a & A,B,C   & S & A               \\  \cline{5-7}
    \multicolumn{4}{c} {} & b & B,C     &                     \\  \cline{6-7}
    \multicolumn{5}{c} {} & c & C                       \\  \cline{7-7}
\end{tabular}
\end{minipage}
\begin{minipage}{0.5\textwidth}
\begin{tabular}{*{6}{c|}}
                                                        \cline{2-6}
                            c & C &   &   & S & A   \\  \cline{2-6}
    \multicolumn{1}{c} {} & c & C & S & A & S       \\  \cline{3-6}
    \multicolumn{2}{c} {} & a & A,B,C & S & A           \\  \cline{4-6}
    \multicolumn{3}{c} {} & c & C &                 \\  \cline{5-6}
    \multicolumn{4}{c} {} & c & C                   \\  \cline{6-6}
\end{tabular}
\end{minipage}

Damit gilt $abcabc \in L(\mathcal{G})$ und $ccacc \notin L(\mathcal{G})$.

\aufgabe{8}
Sei eine kontextfreie Grammatik $\mathcal{G} = (N, \Sigma, P, S)$ gegeben.
Wenn $|L(\mathcal{G})| = \infty$, so muss ein Wort $z \in L(\mathcal{G})$
mit $|z| \geq 2^{|N|+1}$ existieren, sodass der Ableitungsbaum
von $z$ mindestens Höhe $h \geq |N| + 1$ hat, also eins der endlich vielen
Nichtterminale mehr als einmal in der Ableitung von $z$ vorkommt (Idee Pumping-Lemma).
Andererseits haben wir nun endlich viele Nichtterminale und damit endlich viele
Ableitungen, in denen jedes Nichtterminal höchstens ein mal vorkommt.\\
Weiter gilt auch, dass wenn ein $z \in L(\mathcal{G})$ mit $|z| \geq n = 2^{|N|+1}$
existiert (wobei $n$ eben genau das $n$ aus dem Pumping-Lemma ist), dass
dann das Pumping Lemma uns einer Zerlegung liefert, welche wir insebsondere beliebig
oft ''pumpen'' können, um unendlich viele verschiedene Wörter in $L(\mathcal{G})$ 
zu erhalten.\\
Insgesamt haben wir also gezeigt, dass:
$$
    \exists z \in L(\mathcal{G}) : |z| \geq 2^{|N|+1}
    \qquad \,\Longleftrightarrow\,\qquad
    |L(\mathcal{G})| = \infty
$$
Nun können wir wie folgt entscheiden, ob $L(\mathcal{G})$ zu einer gegebenen 
Grammatik $\mathcal{G}$ unendlich ist:
Es ist $\Sigma$ endlich, also ist $\Sigma^n\Sigma^*$ ein regulärer Ausdruck,
welcher alle Wörter über dem Eingabealphabeit mit mindestens Länge $n := 2^{|N|+1}$
erkennt. Dann können wir einen DFA $\mathcal{A}$ erstellen mit $L(\mathcal{A}) =
L(\Sigma^n\Sigma^*)$. Weiter können wir aus der gegebenen Grammatik $\mathcal{G}$
einen PDA $\mathcal{B}$ mit $L(\mathcal{B}) = L(\mathcal{G})$ erstellen, welcher
mit leerem Zustand akzeptiert. Dann können wir den PDA $\mathcal{C}$ erstellen, welcher
die Produktkonstruktion der beiden darstellt.
Also $L(\mathcal{C}) = L(\mathcal{A}) \cap L(\mathcal{B})$.
Zu diesem können wir nun wieder eine kontextfreie Grammatik $\mathcal{G}'$ erstellen,
sodass $L(\mathcal{G}') = L(\mathcal{C}) $. Ferner ist also
$L(\mathcal{G}') = \{z \mid z \in L(\mathcal{G}) \land |z| \geq n\}$.
Wenn wir also mit dem Markierungsalgoritmus das Leerheitsproblem auf $\mathcal{G}'$
lösen, so wissen wir, ob die Sprache $L(\mathcal{G})$ ein Wort der Länge $\geq n$ enthält.
Nach dem obigen Beweis ist dann
$
    L(\mathcal{G}') \neq \varnothing \,\Longrightarrow\, |L(\mathcal{G})| = \infty
$.\\
Somit können wir mit dem angegebenen Algorithmus testen, ob die Sprache einer
gegebenen kontextfreien Grammatik unendlich ist.
\end{document}
