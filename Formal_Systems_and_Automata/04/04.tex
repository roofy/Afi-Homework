\documentclass[a4paper,graphics,11pt]{article}
\pagenumbering{arabic}

\usepackage[margin=1in]{geometry}
\usepackage[utf8]{inputenc}
\usepackage[T1]{fontenc}
\usepackage{lmodern}
\usepackage[ngerman]{babel}
\usepackage{amsmath, tabu}
\usepackage{amsthm}
\usepackage{amssymb}
\usepackage{complexity}
\usepackage{mathtools}
\usepackage{setspace}
\usepackage{graphicx,color,curves,epsf,float,rotating}
\usepackage{tasks}
\setlength{\parindent}{0em}
\setlength{\parskip}{1em}
\usepackage{tikz}
\usetikzlibrary{automata, arrows}

\newcommand{\aufgabe}[1]{\subsection*{Aufgabe #1}}
\newcommand{\up}[2]{\mathrel{\overset{\makebox[0pt]{\mbox{\normalfont\tiny #2}}}{#1}}}

\begin{document}
\noindent Gruppe \fbox{\textbf{3}}             \hfill Phil Pützstück, 377247\\
\noindent Formale Systeme und Automaten \hfill Benedikt Gerlach, 376944\\
\strut\hfill Sebastian Hackenberg, 377550\\
\begin{center}
	\LARGE{\textbf{Hausaufgabe 4}}
\end{center}
\begin{center}
\rule[0.1ex]{\textwidth}{1pt}
\end{center}

\aufgabe{5}
\newpage
\aufgabe{6}
\newpage
\aufgabe{7}
$$
    L_1 = (a(b+c)^*a)^*
    \qquad
    L_2 = (c^*a+b^*a)^*+(c^*+b^*)
$$$$
    L_3 = 
    \qquad
    L_4 = (b+c)^* + (a+b+c)^*bc(a+b+c)^*
$$

\newpage
\aufgabe{8}
\textbf{a)}\\
\begin{tikzpicture}[>=stealth',shorten >=1pt,auto,node distance=4cm]

    \node[state, initial]   (q0)                    {$+$};
    \node[state]  (q1)  [above of=q0]   {};
    \node[state]  (q2)  [right of=q1] {};
    \node[state]  (q3)    [below of=q0] {$*$};
    \node[state]  (q4)    [right of=q3] {};
    \node[state]  (q5)    [right of=q4]   {};
    \node[state]  (q6)    [above of=q5]   {};
    \node[state,accepting]  (q7)    [right of=q6]   {};


    \path[->] (q0)      edge                node {$\varepsilon$} (q1);
    \path[->] (q0)      edge                node {$\varepsilon$} (q3);
    
    \path[->] (q1)      edge                node {b} (q2);
    
    \path[->] (q2)      edge                node {$\varepsilon$} (q6);
    
    \path[->] (q3)      edge                node {$\varepsilon$} (q4);
   
    \path[->] (q4)      edge                node {c} (q5);
    
    \path[->] (q5)      edge[bend left]                node {$\varepsilon$} (q3);
    \path[->] (q5)      edge                node {$\varepsilon$} (q6);
    
    \path[->] (q6)      edge                node {a} (q7);
 
\end{tikzpicture}


\textbf{b)}\\
\begin{tikzpicture}[>=stealth',shorten >=1pt,auto,node distance=2cm]

    \node[state, initial]   (q0)                    {$*$};
    \node[state]  (q1)  			[right of=q0]   {};
    \node[state]  (q2)  			[right of=q1] {};
    \node[state]  (q3)   			[right of=q2] {};
    \node[state]  (q4)   			[right of=q3] {};
    \node[state]  (q5)    			[right of=q4]   {};
    \node[state,accepting]  (q6)    			[right of=q5]   {};
    


    \path[->] (q0)      edge                node {$\varepsilon$} (q1);
    \path[->] (q0)      edge [bend left]              node {$\varepsilon$} (q6);
    
    
    \path[->] (q1)      edge                node {a} (q2);
    
    \path[->] (q2)      edge                node {$\varepsilon$} (q3);
    \path[->] (q2)      edge[bend left]                node {$\varepsilon$} (q0);
    
    \path[->] (q3)      edge                node {b} (q4);
   
    \path[->] (q4)      edge                node {$\varepsilon$} (q5);
    \path[->] (q4)      edge [bend left]                node {$\varepsilon$} (q0);
    
    \path[->] (q5)      edge               node {c} (q6);
    
    \path[->] (q6)      edge [bend left]              node {$\varepsilon$} (q0);
    
 
\end{tikzpicture}


\end{document}
