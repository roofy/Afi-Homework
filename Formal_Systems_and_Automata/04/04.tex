\documentclass[a4paper,graphics,11pt]{article}
\pagenumbering{arabic}

\usepackage[margin=1in]{geometry}
\usepackage[utf8]{inputenc}
\usepackage[T1]{fontenc}
\usepackage{lmodern}
\usepackage[ngerman]{babel}
\usepackage{amsmath, tabu}
\usepackage{amsthm}
\usepackage{amssymb}
\usepackage{complexity}
\usepackage{mathtools}
\usepackage{setspace}
\usepackage{graphicx,color,curves,epsf,float,rotating}
\usepackage{tasks}
\setlength{\parindent}{0em}
\setlength{\parskip}{1em}
\usepackage{tikz}
\usetikzlibrary{automata, arrows}

\newcommand{\aufgabe}[1]{\subsection*{Aufgabe #1}}
\newcommand{\up}[2]{\mathrel{\overset{\makebox[0pt]{\mbox{\normalfont\tiny #2}}}{#1}}}

\begin{document}
\noindent Gruppe \fbox{\textbf{3}}             \hfill Phil Pützstück, 377247\\
\noindent Formale Systeme und Automaten \hfill Benedikt Gerlach, 376944\\
\strut\hfill Sebastian Hackenberg, 377550\\
\begin{center}
	\LARGE{\textbf{Hausaufgabe 4}}
\end{center}
\begin{center}
\rule[0.1ex]{\textwidth}{1pt}
\end{center}

\aufgabe{5}
\textbf{a)}
\begin{align*}
    &A: \qquad \varepsilon \in L\\
    &B: \qquad A \land (1):\quad x=\varepsilon, &y&=\varepsilon, &z&=\varepsilon &\Longrightarrow&\quad abc \in L\\
    &C: \qquad B \land (1):\quad x=a, &y&=b, &z&=c &\Longrightarrow&\quad aabbcc \in L\\
    &D: \qquad C \land (1):\quad x=aa, &y&=bb, &z&=cc &\Longrightarrow&\quad aaabbbccc \in L\\
    &E: \qquad D \land (2):\quad x=aaa,\ &y&=bbb,\ &z&=cc\quad &\Longrightarrow&\quad cbbbccaaa \in L\\
\end{align*}
\textbf{b)}
Sei
$$
    L' = \{a^nb^nc^n \mid n \in \mathbb{N}_{>0}\} \cup \{cb^nc^{n-1}a^n\ \mid n \in \mathbb{N}_{>0}\} \cup \{\varepsilon\}
$$
Dann ist $L = L'$. Wir zeigen zuerst $L \subseteq L'$.

Wir führen Induktion über den Aufbau von $L$.
Wir fangen mit der Basisregel an. Sei also $w = \varepsilon$. Dann gilt trivialerweise $w \in L'$.

Sei nun $w \in L \land w \in L'$ (IV). Wir überprüfen die Rekursionregeln:

\textbf{Fall 1:} Anwenden der Rekursionregel (1).\\
Dann ist $w = xyz$ mit $x = a^n, y= b^n$  und $z = c^n$ für ein $n \in \mathbb{N}$, da $w \in L'$ und
Rekursionregel (1) diesen Aufbau von Wort benötigt.
Durch anwenden der Rekursionregel (1) auf $w$ erhalten wir ein $w'$ und es folgt
$$
    w' = axbycz = aa^nbb^ncc^n = a^{n+1}b^{n+1}c^{n+1} \in L'
$$

\textbf{Fall 2:} Anwenden der Rekursionregel (2).\\
Dann ist $w = xyzc$ mit $x = a^n, y=b^n$ und $z = c^{n-1}$ für ein $n \in \mathbb{N}$, da die Rekursionregel (2)
benötigt, dass das gegebene Wort diese Form hat. Weiter ist ja nach (IV) auch $w \in L'$, also insgesamt
$w = xyzc = a^nb^nc^{n-1}c = a^nb^nc^n$. Durch anwenden der Rekursionsregel (2) auf $w$ erhalten wir ein
$w'$ und es folgt:
$$
    w' = cyzx = cb^nc^{n-1}a^n \in L'
$$

Insgesamt folgt für jedes ableitbare Wort $w \in L$, dass auch $w \in L'$. Also $L \subseteq L'$.
\newpage

Wir zeigen nun $L' \subseteq L$.
Offensichtlich gilt $\varepsilon \in L'$ und $\varepsilon \in L$ nach Basisregel.\\
Wir notieren das anwenden der Rekursionregel auf ein wort $w$ als $R_1(w)$ bzw. $R_2(w)$.\\
Wir führen Induktion über $n \in \mathbb{N}_{> 0}$ und zeigen, dass stets $a^nb^nc^n \in L$ sowie $cb^nc^{n-1}a^n \in L$

Sei also $n = 1$. Es gilt
$
    a^nb^nc^n = abc = R_1(\varepsilon)
$
und $\varepsilon \in L$ also auch $abc \in L$. Weiter ist
$cb^nc^{n-1}a^n = cba = R_2(abc) = R_2(R_1(\varepsilon))$ also auch $cba \in L$.

Sei nun $n \in \mathbb{N}_{>0}$ gegeben, sodass $a^nb^nc^n \in L$ (IV). Es folgt
$$
    a^{n+1}b^{n+1}c^{n+1} = R_1(a^nb^nc^n)
    \quad\text{sowie}\quad
    cb^{n+1}c^na^{n+1} = R_2(a^{n+1}b^{n+1}c^{n+1}) = R_2(R_1(a^nb^nc^n))
$$
Da $a^nb^nc^n \in L$ nach IV, folgt also auch $a^{n+1}b^{n+1}c^{n+1} \in L$ sowie $cb^{n+1}c^na^{n+1} \in L$.
Nach Prinzip der vollständigen Induktion gilt also, dass $a^nb^nc^n \in L$ und $cb^nc^{n-1}a^n \in L$ für
alle $n \in \mathbb{N}_{>0}$. Insgesamt gilt also nach Konstruktion von $L'$, dass $L' \subseteq L$.

Aus beiden Induktionen folgt dann $L = L'.\hfill\square$

\aufgabe{6}
\textbf{a)}
Wir definieren $q$ rekursiv wie folgt für ein $w \in \Sigma^*$
$$
    q(w) = \begin{cases}
        0 & \text{für}\ w = \varepsilon\\
        a + q(v) & \text{für}\ w = av\ \text{mit}\ a \in \Sigma\ \text{und}\ v \in \Sigma^*
    \end{cases}
$$
\textbf{b)}

Wir zeigen $q(vw) = q(v) + q(w)$ für $v,w \in \Sigma^*$ mittels Induktion über $v$. Sei also $w$ beliebig aber fest.
Für $v = \varepsilon$ ist
$$
    q(vw)
    = q(\varepsilon w)
    = q(w)
    = 0 + q(w)
    = q(\varepsilon) + q(w)
    = q(v) + q(w)
$$
Sei also nun $v$ so dass $q(vw) = q(v) + q(w)$ (IV). Wir verlängern also $v$ um ein Präfix $a \in \Sigma$:
$$
    q(avw)
    = a + q(vw)
    \up{=}{IV} a + q(v) + q(w)
    = q(av) + q(w)
$$
Folglich gilt für alle $v,w \in \Sigma^*$, dass $q(vw) = q(v) + q(w)$. Da wir aber mindestens im abelschen
Monoid der natürlichen Zahlen (da hier $0 \in \mathbb{N}$) rechnen, ist die Addition kommutativ.\\
Folglich gilt für $v,w \in \Sigma^*$
$$
    q(vw)
    = q(v) + q(w)
    = q(w) + q(v)
    = q(wv)
$$
\strut$\hfill\square$


\aufgabe{7}
Es sei im Sinne der Lesbarkeit im folgenden $\Sigma^*$ als $(\sum_{a \in \Sigma} a)^*$, also konkret $(a+b+c)^*$ zu interpretieren.
$$
    r_1 = ( \Sigma \Sigma)^*
    \qquad
    r_2 = (c^*a+b^*a)^*+(c^*+b^*)
$$$$
    r_3 = \Sigma^*(ab (c + \Sigma^*bc) + bc \Sigma^*ab) \Sigma^*
    \qquad
    r_4 = (b+c)^* +  (\Sigma^*bc \Sigma^*)
$$
\newpage
\aufgabe{8}
\textbf{a)}\\
\begin{tikzpicture}[>=stealth',shorten >=1pt,auto,node distance=4cm]

    \node[state, initial]   (q0)                    {$+$};
    \node[state]  (q1)  [above of=q0]   {};
    \node[state]  (q2)  [right of=q1] {};
    \node[state]  (q3)    [below of=q0] {$*$};
    \node[state]  (q4)    [right of=q3] {};
    \node[state]  (q5)    [right of=q4]   {};
    \node[state]  (q6)    [above of=q5]   {};
    \node[state,accepting]  (q7)    [right of=q6]   {};


    \path[->] (q0)      edge                node {$\varepsilon$} (q1);
    \path[->] (q0)      edge                node {$\varepsilon$} (q3);

    \path[->] (q1)      edge                node {b} (q2);

    \path[->] (q2)      edge                node {$\varepsilon$} (q6);

    \path[->] (q3)      edge                node {$\varepsilon$} (q4);
    \path[->] (q3)      edge                node {$\varepsilon$} (q6);

    \path[->] (q4)      edge                node {c} (q5);

    \path[->] (q5)      edge[bend left]                node {$\varepsilon$} (q3);

    \path[->] (q6)      edge                node {a} (q7);

\end{tikzpicture}


\textbf{b)}\\
\begin{tikzpicture}[>=stealth',shorten >=1pt,auto,node distance=2cm]

    \node[state, initial]   (q0)                    {$*$};
    \node[state]  (q1)  			[right of=q0]   {};
    \node[state]  (q2)  			[right of=q1] {};
    \node[state]  (q3)   			[right of=q2] {};
    \node[state]  (q4)   			[right of=q3] {};
    \node[state]  (q5)    			[right of=q4]   {};
    \node[state,accepting]  (q6)    			[right of=q5]   {};



    \path[->] (q0)      edge                node {$\varepsilon$} (q1);
    \path[->] (q0)      edge [bend left]              node {$\varepsilon$} (q6);


    \path[->] (q1)      edge                node {a} (q2);

    \path[->] (q2)      edge                node {$\varepsilon$} (q3);
    \path[->] (q2)      edge[bend left]                node {$\varepsilon$} (q0);

    \path[->] (q3)      edge                node {b} (q4);

    \path[->] (q4)      edge                node {$\varepsilon$} (q5);
    \path[->] (q4)      edge [bend left=40]                node {$\varepsilon$} (q0);

    \path[->] (q5)      edge               node {c} (q6);

    \path[->] (q6)      edge [bend left=45]              node {$\varepsilon$} (q0);
\end{tikzpicture}

\end{document}
