\documentclass[a4paper,graphics,11pt]{article}
\pagenumbering{arabic}

\usepackage[margin=1in]{geometry}
\usepackage[utf8]{inputenc}
\usepackage[T1]{fontenc}
\usepackage{lmodern}
\usepackage[ngerman]{babel}
\usepackage{amsmath, tabu}
\usepackage{amsthm}
\usepackage{amssymb}
\usepackage{complexity}
\usepackage{mathtools}
\usepackage{setspace}
\usepackage{graphicx,color,curves,epsf,float,rotating}
\usepackage{tasks}
\setlength{\parindent}{0em}
\setlength{\parskip}{1em}
\usepackage{tikz}
\usetikzlibrary{automata, arrows}

\newcommand{\aufgabe}[1]{\subsection*{Aufgabe #1}}
\newcommand{\up}[2]{\mathrel{\overset{\makebox[0pt]{\mbox{\normalfont\tiny #2}}}{#1}}}

\begin{document}
\noindent Gruppe \fbox{\textbf{3}}             \hfill Phil Pützstück, 377247\\
\noindent Formale Systeme und Automaten \hfill Benedikt Gerlach, 376944\\
\strut\hfill Sebastian Hackenberg, 377550\\
\begin{center}
	\LARGE{\textbf{Hausaufgabe 4}}
\end{center}
\begin{center}
\rule[0.1ex]{\textwidth}{1pt}
\end{center}

\aufgabe{5}
\newpage
\aufgabe{6}
\textbf{a)}
Wir definieren $q$ rekursiv wie folgt für ein $w \in \Sigma^*$
$$
    q(w) = \begin{cases}
        0 & \text{für}\ w = \varepsilon\\
        a + q(v) & \text{für}\ w = av\ \text{mit}\ a \in \Sigma\ \text{und}\ v \in \Sigma^*
    \end{cases}
$$
\textbf{b)}

Wir zeigen $q(vw) = q(v) + q(w)$ für $v,w \in \Sigma^*$ mittels Induktion über $v$. Sei also $w$ beliebig aber fest.
Für $v = \varepsilon$ ist
$$
    q(vw)
    = q(\varepsilon w)
    = q(w)
    = 0 + q(w)
    = q(\varepsilon) + q(w)
    = q(v) + q(w)
$$
Sei also nun $v$ so dass $q(vw) = q(v) + q(w)$ (IV). Wir verlängern also $v$ um ein Präfix $a \in \Sigma$:
$$
    q(avw)
    = a + q(vw)
    \up{=}{IV} a + q(v) + q(w)
    = q(av) + q(w)
$$
Folglich gilt für alle $v,w \in \Sigma^*$, dass $q(vw) = q(v) + q(w)$. Da wir aber mindestens im abelschen
Monoid der natürlichen Zahlen (da hier $0 \in \mathbb{N}$) rechnen, ist die Addition kommutativ.\\
Folglich gilt für $v,w \in \Sigma^*$
$$
    q(vw)
    = q(v) + q(w)
    = q(w) + q(v)
    = q(wv)
$$
\strut$\hfill\square$

\aufgabe{7}
Es sei im Sinne der Lesbarkeit im folgenden $\Sigma^*$ als $(\sum_{a \in \Sigma} a)^*$, also konkret $(a+b+c)^*$ zu interpretieren.
$$
    L_1 =
    \qquad
    L_2 =
$$$$
    L_3 =
    \qquad
    L_4 =
$$
\newpage
\aufgabe{8}


\end{document}
