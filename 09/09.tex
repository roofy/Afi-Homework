\documentclass[a4paper,graphics,11pt]{article}
\pagenumbering{arabic}

\usepackage[margin=1in]{geometry}
\usepackage[utf8]{inputenc}
\usepackage[T1]{fontenc}
\usepackage{lmodern}
\usepackage[ngerman]{babel}
\usepackage{amsmath, tabu}
\usepackage{amsthm}
\usepackage{amssymb}
\usepackage{mathtools}
\usepackage{setspace}
\usepackage{graphicx,color,curves,epsf,float,rotating}
\usepackage{tasks}
\setlength{\parindent}{0em}
\setlength{\parskip}{1em}

\newcommand{\aufgabe}[1]{\subsection*{Aufgabe #1}}
\newcommand{\up}[2]{\mathrel{\overset{\makebox[0pt]{\mbox{\normalfont\tiny #2}}}{#1}}}

\begin{document}
\noindent Gruppe \fbox{\textbf{11}}             \hfill Tobias Riedel, 379133 \\
\noindent Analysis für Informatiker             \hfill Phil Pützstück, 377247 \\
\strut\hfill Kevin Holzmann, 371116\\
\strut\hfill Gurvinderjit Singh, 369227
\begin{center}
	\LARGE{\textbf{Hausaufgabe 9}}
\end{center}
\begin{center}
\rule[0.1ex]{\textwidth}{1pt}
\end{center}



\aufgabe{1}
\textbf{a)}\\
Die Exponentialfunktion $\exp(x)$ ist grundsätzlich für ganz $\mathbb{R}$ definiert.
Wir können in der\\
gegebenen Funktion nie durch 0 teilen noch die Wurzel einer
negativen Zahl ziehen. Es folgt, dass der maximale Definitionsbereich der
reellen Funktion $\exp(1-4x^2)$ durch $\mathbb{R}$ gegeben ist.

\textbf{b) (i)}\\
Das größte offene Intervall in dem die Funktion streng monoton steigend ist, ist
$(-\infty, 0)$. Das größte offene Intervall in dem die Funktion streng monoton
fallend ist, ist $(0, \infty)$.

\textbf{(ii)}\\
Seien $x,y \in \mathbb{R}$ mit $x<y$ gegeben.

Falls $x,y \in (-\infty, 0)$, dann gilt $1-4x^2 < 1-4y^2$ und da die Exponentialfunktion
streng monoton wachsend ist (III 3.21 c) auch $\exp(1-4x^2) < \exp(1-4y^2)$.
Also ist $\exp(1-4x^2)$ streng monoton wachsend im Intervall $(-\infty, 0)$ (V 1.7a).

Falls $x,y \in (0, \infty)$, dann gilt $1-4x^2 > 1-4y^2$ und da die Exponentialfunktion
streng monoton wachsend ist (III 3.21 c) auch $\exp(1-4x^2) > \exp(1-4y^2)$.
Also ist $\exp(1-4x^2)$ streng monoton fallend im Intervall $(-\infty, 0)$ (V 1.7b).

\textbf{(iii)}\\
Wenn man nun die Intervalle $(-\infty, 0]$ und $[0, \infty)$ betrachtet, ändert sich
nichts daran, dass $f$ in diesen streng monoton wächst bzw. fällt, denn für
$|x| > 0$ folgt $x^2 > 0^2 = 0$ und damit $1-4\cdot0^2 > 1-4x^2$, d.h.
$f(0) > f(x)$ für $x \in (-\infty, 0)$ oder $x \in (0, \infty)$. Also
bleibt $f$ in beiden halboffenen Intervallen streng monoton steigend bzw. fallend.
\textbf{EVTL. VERBESSERUNG}

\textbf{c)}

Für $x \in \mathbb{R}$ gilt grundsätzlich $\exp(x) > 0$ (III 3.21 b), also kann die
Exponentialfunktion keine Nullstellen besitzen. Damit kann die gegebene Funktion
ebenfalls keine Nullstellen besitzen.

Da wir in Aufgabenteil b) festgestellt haben, dass $f$ im Intervall $(-\infty, 0]$
streng monoton wächst und im Intervall $[0, \infty)$ streng monoton fällt, folgt,
dass $\forall x \in \mathbb{R}\colon f(0) \geq f(x)$ gilt. Damit ist 0 eine Extremal-
und insbesondere auch Maximalstelle von $f$. Hinzukommend kann es keine weiteren
(lokalen) Maximal- bzw. Minimalstellen geben. Denn es folgt aus der strengen Monotonie:

Es gilt für jedes $x \in (-\infty, 0)$, dass $y, z \in (-\infty, 0)$ mit $y<x<z$ existieren.
Analog gilt für jedes $x \in (0, \infty)$, dass $y,z \in (0, \infty)$ mit
$y<x<z$ existieren.\\
Insgesamt gilt aber $\{0\}\cup(-\infty,0)\cup(0,\infty) = \mathbb{R}$.
Somit kann es abgesehen von 0 keine weiteren Extremalwerte für $f$ geben.

\newpage
\aufgabe{2}
\textbf{a)}\\[5pt]
Zuerst sollten wir die Gleichung so umformen, dass nur noch ein $x$ vorkommt:
\begin{alignat*}{4}
    &ax^2+bx+c\ &=\ & 0 &|&\div a\\[2pt]
    &x^2+\frac{b}{a}x + \frac{c}{a}\ &=\ & 0 \quad &|& - \frac{c}{a}\\[2pt]
    &x^2+\frac{b}{a}x\ &=\ & -\frac{c}{a}&|& + \left(\frac{b}{2a}\right)^2\\[2pt]
    &x^2+\frac{b}{a}x\ + \left(\frac{b}{2a}\right)^2 &=\ & \left(\frac{b}{2a}\right)^2-\frac{c}{a}\quad&|&\ \text{quad. Ergänzung}\\[2pt]
    &\left(x+ \frac{b}{2a}\right)^2&=\ & \left(\frac{b}{2a}\right)^2-\frac{c}{a}\quad&&
\end{alignat*}
Nun lässt sich die Gleichung nach $x$ lösen:
\begin{alignat*}{4}
    &\left(x+\frac{b}{2a}\right)^2 = \left(\frac{b}{2a}\right)^2 - \frac{c}{a}\quad&|& \sqrt{\ }\\[2pt]
    &x+\frac{b}{2a} = \pm\sqrt{\left(\frac{b}{2a}\right)^2 - \frac{c}{a}}\quad&|& - \frac{b}{2a}\\[2pt]
    &x = -\frac{b}{2a}\pm\sqrt{\left(\frac{b}{2a}\right)^2 - \frac{c}{a}} \\[2pt]
\end{alignat*}
Dies lässt sich weiter vereinfachen:
$$
    x = -\frac{b}{2a} \pm \sqrt{\left(\frac{b}{2a}\right)^2 -\frac{c}{a}}
    \ =\ -\frac{b}{2a} \pm \sqrt{\left(\frac{b}{2a}\right)^2-\frac{c}{a}} \cdot \frac{2a}{2a}
    \ =\ \frac{-b\pm \sqrt{\left(\frac{b}{2a}\right)^2\cdot (2a)^2-\frac{c}{a}\cdot (2a)^2}}{2a}
$$$$
    = \frac{-b\pm \sqrt{b^2-4ac}}{2a}
$$
Damit wären wir bei der allbekannten Mitternachtsformel. Die sogenannte Diskriminante,\\
$b^2-4ac$, welche unter der Wurzel steht, bestimmt die Anzahl der Lösungen (Nullstellen).
Gilt $b^2-4ac < 0$, so gibt es keine Lösungen (Nullstellen) in $\mathbb{R}$, gilt $b^2-4ac = 0$ so
gibt es genau eine Lösung (Nullstelle) in $\mathbb{R}$, da stets $\pm\sqrt{0}= 0$ gilt. Ist $b^2-4ac > 0$ so gibt
es genau 2 Lösungen (Nullstellen) in $\mathbb{R}$.
\newpage
\textbf{b)}

Hier lässt sich einfach $z = x^2$ substituieren. Dann lassen sich die Nullstellen dieses
Polynoms in $z$ wie in a) beschrieben finden:
$$
    ax^4+bx^2+c = 0
    \,\Longrightarrow\, az^2+bz+c = 0
    \,\Longrightarrow\, z = \frac{-b \pm \sqrt{b^2-4ac}}{2a}
$$ 
Dann kann man wieder resubstituieren um die Nullstellen in des ursprünglichen Polynoms
in $x$ zu erhalten. Es gilt $x = \pm \sqrt{z}$. Seien $z_1, z_2$ die möglichen Nullstellen
des Polynoms in $z$. Es gilt nun
$$
    x_{1,2} = \pm\sqrt{z_1}\quad\text{und}\quad x_{3,4} = \pm \sqrt{z_2}
$$
Insofern das Polynom in $x$ genau 4 Nullstellen in $\mathbb{R}$ besitzt.
Insgesamt lässt sich dies auch alles in einem schreiben:
$$
    x = \pm\sqrt{\frac{-b\pm \sqrt{b^2-4ac}}{2a}}
$$

\textbf{c)}

Wählt man nun $a,b,c \in \mathbb{C}$ und sucht nach komplexen Nullstellen,
so wird man nach dem Fundamentalsatz der Algebra (IV 1.7) auch stets mindestens eine finden.
Dies liegt grundlegend daran, dass die Gleichung $x = \sqrt{z}$ für $x,z \in \mathbb{C}$
stets in $\mathbb{C}$ lösbar ist, auch wenn $z \in \mathbb{R}$ mit $z < 0$, da $\sqrt{-1} = i$
gilt, wo $i$ die imaginäre Einheit von $\mathbb{C}$ ist.


\aufgabe{3}
\textbf{a)}

Zu gegebenem $\varepsilon > 0$ wählen wir $\delta = \varepsilon$. Da stets
$x^2 > 0$  sowie $|x| > 0$ für $x \in \mathbb{R}\setminus\{0\}$ gilt\\
(II 2.8 a4 und II 2.12), folgt:
$$
    \forall x \in \mathbb{R}\setminus \{0\},\ |x-0| < \delta \colon
    |f(x)-0| = \left|\frac{x^2}{|x|}\right| = \frac{x^2}{|x|} < \frac{\delta^2}{\delta} 
    = \delta = \varepsilon
$$
Somit gilt $\lim_{x \to 0}\limits f(x) = 0$.

\textbf{b)}

Wir benutzen Polynomdivision (*) und die gegebenen Eigenschaften, dass stets $0\leq x \leq 2$ (**),
sowie die Betragssätze (II 2.12) (***).
Zu gegebenem $\varepsilon > 0$ wähle $\delta = \dfrac{\varepsilon}{5}$. Es folgt:
$$
    \forall x \in [0,1)\cup(1,2],\ |x-1| < \delta \colon
    |f(x) - 4|
    \up{=}{*} |x^2+2x+1 - 4|
    = |(x-1)(x+3)|
$$$$
    \up{=}{***} |x-1|\cdot|x+3|
    < \delta|x+3|
    \up{=}{**} x\delta + 3\delta
    \up{\leq}{**} 5\delta
    = \varepsilon
$$
Somit gilt $\lim_{x \to 1}\limits f(x) = 4$.

\newpage
\aufgabe{4}
Wir benutzen im folgenden, dass $\lim_{x \to x_0}\limits f(x) = f(x_0)$ gilt, insofern $f$ an
der Stelle $x_0$ definiert ist. Dies ist zu interpretieren als eine unendlich nahe
Annäherung von $x$ an $x_0$, was in diesen Fällen eben äquivalent zum Einsetzen von $x_0$ in
$f$ ist (Siehe Satz 1.6).

\textbf{a)}

Da $f$ im Punkt $-1$ definiert ist, lässt sich $x\to-1$ durch Einsetzen von -1 unendlich
nahe approximieren:
$$
    \lim_{x \to -1} f(x)
    = \lim_{x \to -1} \frac{3x-1}{2x-2}
    = \frac{3(-1)-1}{2(-1)-2} 
    = \frac{-4}{-4}
    = 1
$$

\textbf{b)}

Die gegebene Funktion lässt sich wie folgt umformen:
$$
    g(x) = \frac{1}{2-x} - \frac{12}{8-x^3}
    = \frac{1}{2-x} - \frac{12}{(2-x)(x^2+2x+4)} 
    = \frac{x^2+2x+4-12}{(2-x)(x^2+2x+4)}
$$$$
    = \frac{(x-2)(x+4)}{-(x-2)(x^2+2x+4)}
    = -\frac{x+4}{x^2+2x+4}
$$
Nun lässt sich 2 einsetzen, um 2 mit $x$ unendlich nahe zu approximieren:
$$
    \lim_{x \to 2} g(x)
    = \lim_{x \to 2} \frac{1}{2-x} - \frac{12}{8-x^3}
    = \lim_{x \to 2} -\frac{x+4}{x^2+2x+4}
    = - \frac{2+4}{2^2+2\cdot 2 +4}
    = -\frac{1}{2}
$$

\textbf{c)}

Die gegebene Funktion lässt sich wieder umformen, sodass der Grenzwert wie in den vorherigen
Aufgabenteilen durch Einsetzen bestimmt werden kann. Mit Polynomdivision folgt:
$$
    h(x) = \frac{2x^4-6x^3+x^2+3}{x-1} = 2x^3-4x^2-3x-3
$$
Dieses Polynom ist für $x=1$ definiert. Somit gilt:
$$
    \lim_{x \to 1} h(x)
    = \lim_{x \to 1} \frac{2x^4-6x^3+x^2+3}{x-1}
    = \lim_{x \to 1} 2x^3-4x^2-3x-3
$$$$
    = 2(1)^3-4(1)^2-3(1)-3 = -8
$$

\newpage

\aufgabe{5}
\textbf{(i)}

Die gegebene Funktion lässt sich wie folgt umschreiben:
$$
    f(x) = \frac{x^4-2x}{2x^4+1} = \frac{1-\dfrac{2}{x^3}}{2+\dfrac{1}{x^4}}
$$
Wir zeigen, dass $\dfrac{c}{x^n}$ mit $c \in \mathbb{R}$ und $n\in \mathbb{N}$ für $x\to \infty$ gegen 0 konvergiert.\\
Nach Beispiel 1.13 a) ist $f(x) = \frac{1}{x}$ konvergent gegen 0 (*). Nun lassen sich
die Grenzwertsätze\\
(Satz 1.5) anwenden (**). Für $\frac{1}{x^n}$ mit $n \in \mathbb{N}$ folgt:
$$
    \lim_{x \to \infty} \frac{1}{x^n}
    = \lim_{x \to \infty} \prod_{k=1}^{n} \frac{1}{x}
    \up{=}{**} \prod_{k=1}^{n} \lim_{x \to \infty} \frac{1}{x}
    \up{=}{*} \prod_{k=1}^{n} 0 = 0
$$
Damit ist $f_n(x) = \frac{1}{x^n}$ mit $n \in \mathbb{N}$ für $x\to \infty$ kovergent gegen
0. Sei nun ein beliebiges $c \in \mathbb{R}$\\
gegeben. Wir wissen $f_n(x) = \dfrac{1}{x^n}$ konvergiert für
$n\in \mathbb{N}$ (*), die konstante Funktion ist grundsätzlich konvergent und können
wieder die Grenzwertsätze anwenden (**).
Für $\dfrac{c}{x^n}$ mit $n \in \mathbb{N}$ folgt:
$$
    \lim_{x \to \infty} \frac{c}{x^n}
    = \lim_{x \to \infty} c\cdot \frac{1}{x^n}
    = \lim_{x \to \infty} c \cdot \lim_{x \to \infty} \frac{1}{x^n}
    = c \cdot 0 = 0
$$
Insgesamt ist $\dfrac{c}{x^n}$ mit beliebigen $c \in \mathbb{R}, n \in \mathbb{N}$
für $x \to \infty$ konvegent gegen 0. Damit sind Nenner und Zähler unserer umgeformten
Funktion ebenfalls konvergent und die Grenzwertsätze lassen sich wie folgt anwenden:
$$
    \lim_{x \to \infty} f(x)
    = \lim_{x \to \infty} \frac{1-\dfrac{2}{x^3}}{2+\dfrac{1}{x^4}}
    = \frac{\lim_{x \to \infty}\limits 1 - \dfrac{2}{x^3}}{\lim_{x \to \infty}\limits 2+\dfrac{1}{x^4}} 
    = \frac{\lim_{x \to \infty}\limits 1 - \lim_{x \to \infty}\limits \dfrac{2}{x^3}}
        {\lim_{x \to \infty}\limits 2 + \lim_{x \to \infty}\limits \dfrac{1}{x^4}}
    = \frac{1-0}{2+0} = \frac{1}{2}
$$

\textbf{(ii)}

Dies lässt sich analog zu den Teilaufgaben aus Nr. 4 umformen, um dann durch Einsetzen,
unendlich nahe zu approximieren. Die gegebene Funktion lässt sich wie folgt umformen:
$$
    g(x) = \frac{1}{2-x} - \frac{4}{4-x^2}
    = \frac{1}{2-x} - \frac{4}{(2-x)(2+x)}
    = \frac{2+x-4}{(2-x)(2+x)}
    = -\frac{1}{2+x}
$$
Die umgeformte Funktion ist für $x=2$ definiert. Es folgt:
$$
    \lim_{x \to 2} g(x)
    = \lim_{x \to 2} \frac{1}{2-x} - \frac{4}{4-x^2}
    = \lim_{x \to 2} - \frac{1}{2+x}
    = - \frac{1}{2+2}
    = -\frac{1}{4}
$$
\end{document}
