\documentclass[a4paper,graphics,11pt]{article}
\pagenumbering{arabic}

\usepackage[margin=1in]{geometry}
\usepackage[utf8]{inputenc}
\usepackage[T1]{fontenc}
\usepackage{lmodern}
\usepackage[ngerman]{babel}
\usepackage{amsmath, tabu}
\usepackage{amsthm}
\usepackage{amssymb}
\usepackage{complexity}
\usepackage{mathtools}
\usepackage{setspace}
\usepackage{graphicx,color,curves,epsf,float,rotating}
\usepackage{tasks}
\setlength{\parindent}{0em}
\setlength{\parskip}{1em}

\newcommand{\aufgabe}[1]{\subsection*{Aufgabe #1}}
\newcommand{\up}[2]{\mathrel{\overset{\makebox[0pt]{\mbox{\normalfont\tiny #2}}}{#1}}}

\begin{document}
\noindent Gruppe \fbox{\textbf{11}}             \hfill Tobias Riedel, 379133 \\
\noindent Analysis für Informatiker             \hfill Phil Pützstück, 377247 \\
\strut\hfill Kevin Holzmann, 371116\\
\strut\hfill Gurvinderjit Singh, 369227
\begin{center}
	\LARGE{\textbf{Hausaufgabe 9}}
\end{center}
\begin{center}
\rule[0.1ex]{\textwidth}{1pt}
\end{center}



\aufgabe{2}
\textbf{a)}\\[5pt]
Zuerst sollten wir die Gleichung so umformen, dass nur noch ein $x$ vorkommt:
\begin{alignat*}{4}
    &ax^2+bx+c\ &=\ & 0 &|&\div a\\[2pt]
    &x^2+\frac{b}{a}x + \frac{c}{a}\ &=\ & 0 \quad &|& - \frac{c}{a}\\[2pt]
    &x^2+\frac{b}{a}x\ &=\ & -\frac{c}{a}&|& + \left(\frac{b}{2a}\right)^2\\[2pt]
    &x^2+\frac{b}{a}x\ + \left(\frac{b}{2a}\right)^2 &=\ & \left(\frac{b}{2a}\right)^2-\frac{c}{a}\quad&|&\ \text{quad. Ergänzung}\\[2pt]
    &\left(x+ \frac{b}{2a}\right)^2&=\ & \left(\frac{b}{2a}\right)^2-\frac{c}{a}\quad&&
\end{alignat*}
Nun lässt sich die Gleichung nach $x$ lösen:
\begin{alignat*}{4}
    &\left(x+\frac{b}{2a}\right)^2 = \left(\frac{b}{2a}\right)^2 - \frac{c}{a}\quad&|& \sqrt{\ }\\[2pt]
    &x+\frac{b}{2a} = \pm\sqrt{\left(\frac{b}{2a}\right)^2 - \frac{c}{a}}\quad&|& - \frac{b}{2a}\\[2pt]
    &x = -\frac{b}{2a}\pm\sqrt{\left(\frac{b}{2a}\right)^2 - \frac{c}{a}} \\[2pt]
\end{alignat*}
Dies lässt sich weiter vereinfachen:
$$
    x = -\frac{b}{2a} \pm \sqrt{\left(\frac{b}{2a}\right)^2 -\frac{c}{a}}
    \ =\ -\frac{b}{2a} \pm \sqrt{\left(\frac{b}{2a}\right)^2-\frac{c}{a}} \cdot \frac{2a}{2a}
    \ =\ \frac{-b\pm \sqrt{\left(\frac{b}{2a}\right)^2\cdot (2a)^2-\frac{c}{a}\cdot (2a)^2}}{2a}
$$$$
    = \frac{-b\pm \sqrt{b^2-4ac}}{2a}
$$
Damit wären wir bei der allbekannten Mitternachtsformel. Die sogenannte Diskriminante,\\
$b^2-4ac$, welche unter der Wurzel steht, bestimmt die Anzahl der Lösungen (Nullstellen).
Gilt $b^2-4ac < 0$, so gibt es keine Lösungen (Nullstellen) in $\mathbb{R}$, gilt $b^2-4ac = 0$ so
gibt es genau eine Lösung (Nullstelle) in $\mathbb{R}$, da stets $\pm\sqrt{0}= 0$ gilt. Ist $b^2-4ac > 0$ so gibt
es genau 2 Lösungen (Nullstellen) in $\mathbb{R}$.
\newpage
\textbf{b)}

Hier lässt sich einfach $z = x^2$ substituieren. Dann lassen sich die Nullstellen dieses
Polynoms in $z$ wie in a) beschrieben finden:
$$
    ax^4+bx^2+c = 0
    \,\Longrightarrow\, az^2+bz+c = 0
    \,\Longrightarrow\, z = \frac{-b \pm \sqrt{b^2-4ac}}{2a}
$$ 
Dann kann man wieder resubstituieren um die Nullstellen in des ursprünglichen Polynoms
in $x$ zu erhalten. Es gilt $x = \pm \sqrt{z}$. Seien $z_1, z_2$ die möglichen Nullstellen
des Polynoms in $z$. Es gilt nun
$$
    x_{1,2} = \pm\sqrt{z_1}\quad\text{und}\quad x_{3,4} = \pm \sqrt{z_2}
$$
Insofern das Polynom in $x$ genau 4 Nullstellen in $\mathbb{R}$ besitzt.
Insgesamt lässt sich dies auch alles in einem schreiben:
$$
    x = \pm\sqrt{\frac{-b\pm \sqrt{b^2-4ac}}{2a}}
$$

\textbf{c)}

Oft gibt es für Polynome dieser Form (quadratisch bzw. biquadratisch) weniger als 2 bzw 4
reelle Lösungen, da evtl. die Wurzel einer negativen Zahl gezogen wird.
Wählt man jedoch $a,b,c \in \mathbb{C}$ und sucht nach komplexen Nullstellen,
so wird man nach dem Fundamentalsatz der Algebra (IV 1.7) auch stets 2 bzw. 4, genauso viele
wie der Grad des Polynoms, finden. Dies liegt grundlegend daran, dass die Gleichung
$x = \sqrt{z}$ für $x,z \in \mathbb{C}$ stets in $\mathbb{C}$ lösbar ist, auch wenn $z \in \mathbb{R}$ mit $z < 0$, da $\sqrt{-1} = i$ gilt, wo $i$ die imaginäre Einheit von $\mathbb{C}$ ist.

\end{document}
