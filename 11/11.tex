\documentclass[a4paper,graphics,11pt]{article}
\pagenumbering{arabic}

\usepackage[margin=1in]{geometry}
\usepackage[utf8]{inputenc}
\usepackage[T1]{fontenc}
\usepackage{lmodern}
\usepackage[ngerman]{babel}
\usepackage{amsmath, tabu}
\usepackage{amsthm}
\usepackage{amssymb}
\usepackage{complexity}
\usepackage{mathtools}
\usepackage{setspace}
\usepackage{graphicx,color,curves,epsf,float,rotating}
\usepackage{tasks}
\setlength{\parindent}{0em}
\setlength{\parskip}{1em}

\newcommand{\aufgabe}[1]{\subsection*{Aufgabe #1}}
\newcommand{\up}[2]{\mathrel{\overset{\makebox[0pt]{\mbox{\normalfont\tiny #2}}}{#1}}}

\begin{document}
\noindent Gruppe \fbox{\textbf{11}}             \hfill Tobias Riedel, 379133 \\
\noindent Analysis für Informatiker             \hfill Phil Pützstück, 377247 \\
\strut\hfill Kevin Holzmann, 371116\\
\strut\hfill Gurvinderjit Singh, 369227
\begin{center}
	\LARGE{\textbf{Hausaufgabe 11}}
\end{center}
\begin{center}
\rule[0.1ex]{\textwidth}{1pt}
\end{center}



\aufgabe{1}
\textbf{a)}

\textbf{(i)}
Mit (VII 1.11) folgt:
$$
    \frac{df}{dx} = \left[x^{\alpha} \cdot \ln\left(x+|\alpha-1|\right)\right]'
    = \left[x^{\alpha}\right]'\cdot \ln(x+|\alpha-1|) + x^{\alpha} \cdot [\ln(x+|\alpha-1|)]'
$$$$
    = \alpha x^{\alpha-1}\cdot \ln(x+|\alpha-1|)
        + x^{\alpha} \cdot \left(\frac{[x+|\alpha-1|]'}{x+|\alpha-1|}\right)
$$$$
    = \alpha x^{\alpha-1}\cdot \ln(x+|\alpha-1|)
        + x^{\alpha} \cdot \frac{1}{x+|\alpha-1|}
$$



\textbf{(ii)}
Es ist $g(x) = \cos\left(\sin\left(x^2+\frac{1}{x}\right)\right)$ nicht für $x = 0$ definiert,
da sonst durch $0$ geteilt werden würde. $\cos$ und $\sin$ sind wie Polynome sonst für ganz
$\mathbb{R}$ definiert, also ist $D_g =  \mathbb{R}\setminus \{0\}$ der Definitionsbereich
von $g$. Weiterhin sind nach (VI 2.2) $\cos, \sin$  und Polynome, sowie rationale Funktionen
auf ihrem Definitionsbereich stetig. Also ist $g$ stetig auf $\mathbb{R}\setminus\{0\}$.
Mit der Kettenregel und (VII 1.5, 1.7) folgt nun:
$$
    \frac{dg}{dx} = \left[\cos\left(\sin\left(x^2+\frac{1}{x}\right)\right)\right]'
    = \cos'\left(\sin\left(x^2+\frac{1}{x}\right)\right)\cdot \left[\sin\left(x^2+\frac{1}{x}\right)\right]'
$$$$
    = -\sin\left(\sin\left(x^2+\frac{1}{x}\right)\right) \cdot \left(\sin'\left(x^2+\frac{1}{x}\right)\cdot\left[x^2+\frac{1}{x}\right]'\right)
$$$$
    = -\sin\left(\sin\left(x^2+\frac{1}{x}\right)\right) \cdot \left(\cos\left(x^2+\frac{1}{x}\right)\cdot\left(\left[x^2\right]'+\left[\frac{1}{x}\right]'\right)\right)
$$$$
    = -\sin\left(\sin\left(x^2+\frac{1}{x}\right)\right) \cdot \cos\left(x^2+\frac{1}{x}\right)\cdot\left(2x-\frac{1}{x^2}\right)
$$
\newpage
\textbf{b)}

\textbf{(iii)}

$$
    \frac{dh}{dx} = \left[\cos\left(\frac{e^x+\sqrt{1+x^2}}{\arctan x}\right)\right]'
    = -\sin\left(\frac{e^x+\sqrt{1+x^2}}{\arctan x}\right) \cdot
        \left[\frac{e^x+\sqrt{1+x^2}}{\arctan x}\right]'
$$$$
    = -\sin\left(\frac{e^x+\sqrt{1+x^2}}{\arctan x}\right) \cdot
        \left(\frac{\left[e^x+\sqrt{1+x^2}\right]'\cdot (\arctan x) -
        (e^x+\sqrt{1+x^2}) \cdot [\arctan x]'}{(\arctan x)^2} \right)
$$$$
    = -\sin\left(\frac{e^x+\sqrt{1+x^2}}{\arctan x}\right) \cdot
        \left(\frac{\left(e^x+\left(\dfrac{2x}{2\sqrt{1+x^2}}\right)\right)\cdot (\arctan x) -
        (e^x+\sqrt{1+x^2}) \cdot \left(\dfrac{1}{\tan^{-1}(\arctan x)}\right)}{(\arctan x)^2} \right)
$$$$
    = -\sin\left(\frac{e^x+\sqrt{1+x^2}}{\arctan x}\right) \cdot
        \left(\frac{\left(e^x+\dfrac{x}{\sqrt{1+x^2}}\right)\cdot (\arctan x) -
        (e^x+\sqrt{1+x^2}) \cdot \left(\dfrac{1}{x^2+1}\right)}{(\arctan x)^2}
        \right)
$$

\textbf{(iv)}

$$
    \frac{di}{dx} = [\cosh x]'
    = \left[\frac{e^x+e^{-x}}{2}\right]'
    = \frac{1}{2} \left[e^x+e^{-x}\right]'
    = \frac{1}{2} \left([e^x]' + [e^{(-1)x}]'\right)
    = \frac{e^x-e^{-x}}{2}
    = \sinh x
$$


\textbf{c)}

\textbf{(v)}

$$
    \frac{dj}{dx} = [x^x]' = \left[e^{x\ln(x)}\right]'
    = e^{x\ln(x)} \cdot [x\ln(x)]'
    = e^{x\ln(x)} \cdot ([x]' \cdot \ln(x) + [\ln(x)]' \cdot x)
$$$$
    = e^{x\ln(x)} \cdot (1 \cdot \ln(x) + \frac{1}{x} \cdot x)
    = e^{x\ln(x)} \cdot (\ln(x) + 1)
$$

\textbf{(vi)}
$$
    \frac{dk}{dx} = \left[\left(1 + \sqrt{x} + x^2\right)^{\sin x}\right]'
    = \left[\exp\left(\sin(x)\cdot \ln\left(1 + \sqrt{x} + x^2\right)\right) \right]'
$$$$
    = \exp\left(\sin(x)\cdot \ln\left(1 + \sqrt{x} + x^2\right)\right)
        \cdot \left[\sin(x)\cdot \ln\left(1 + \sqrt{x} + x^2\right)\right]'
$$$$
    = \left(1+\sqrt{x} + x^2\right)^{\sin x}
        \cdot \left[\sin(x)\cdot \ln\left(1 + \sqrt{x} + x^2\right)\right]'
$$$$
    = \left(1+\sqrt{x} + x^2\right)^{\sin x}
        \cdot \left([\sin(x)]'\cdot \ln\left(1 + \sqrt{x} + x^2\right)
        + \sin(x) \cdot \left[\ln\left(1+\sqrt{x} + x^2\right)\right]'\right)
$$$$
    = \left(1+\sqrt{x} + x^2\right)^{\sin x}
        \cdot \left(\cos(x)\cdot \ln\left(1 + \sqrt{x} + x^2\right)
        + \sin(x) \cdot \frac{[1+\sqrt{x}+x^2]'}{1+\sqrt{x} + x^2}\right)
$$$$
    = \left(1+\sqrt{x} + x^2\right)^{\sin x}
        \cdot \left(\cos(x)\cdot \ln\left(1 + \sqrt{x} + x^2\right)
        + \sin(x) \cdot \frac{\frac{1}{2\sqrt{x}}+2x}{1+\sqrt{x} + x^2}\right)
$$
\newpage

\aufgabe{2}
Der maximal Def. Bereich ist $\left[-\frac{3}{2} , 3\right)$, da für $x < - \frac{3}{2}$ dann
$3 - 2x < 0$ währe und wir dann die Wurzel einer negativen Zahl ziehen würden.
Des weiteren gilt für $x \geq 3$,\\ dass $\sqrt{3+2x} - x \leq \sqrt{9} - 3= 0$, und da
der natürliche Logarithmus nur für $x > 0$ definiert ist, gibt dies eine Definitionslücke.
Insgesamt ist also $D_f = \left[-\frac{3}{2}, 3\right)$.

Für die Nullstellen setzen wir $f = 0$:
$$
    f = 0 \,\Longrightarrow\, \ln\left(\sqrt{3+2x}-x\right) = 0
    \,\Longrightarrow\, \exp\left(\ln\left(\sqrt{3+2x}-x\right)\right) = \exp(0)
$$$$
    \,\Longrightarrow\, \sqrt{3+2x} -x = 1 \,\Longrightarrow\, \sqrt{3+2x} = x+1
    \,\Longrightarrow\, 3+2x = x^2+2x+1
$$$$
    \,\Longrightarrow\, 2 = x^2 \,\Longrightarrow\, x = \pm\sqrt{2}
$$
Wir testen nun:
$$
    f(\sqrt{2}) = \ln\left(\sqrt{3+2\sqrt{2}}-\sqrt{2}\right)
    = \ln\left(\sqrt{(1+\sqrt{2})^2} -\sqrt{2}\right)
    = \ln(1) = 0
$$$$
    f(-\sqrt{2}) = \ln\left(\sqrt{3-2\sqrt{2}}+\sqrt{2}\right)
    = \ln\left(\sqrt{(1-\sqrt{2})^2} +\sqrt{2}\right)
    = \ln(2\sqrt{2}-1) \neq 0
$$
Also hat $f$ eine Nullstelle bei $\sqrt{2} \in D_f$, aber nicht bei $-\sqrt{2}$.

Für Extrema bestimmen wir zuerst $f'\colon$
$$
    \frac{df}{dx} = \left[\ln\left(\sqrt{3+2x}-x\right)\right]'
    = \frac{\left[\sqrt{3+2x}-x\right]'}{\sqrt{3+2x} -x}
    = \frac{\frac{2}{2\sqrt{3+2x}}-1}{\sqrt{3+2x} -x}
    = \frac{\frac{1}{\sqrt{3+2x}}-1}{\sqrt{3+2x}-x}
$$

Für die Extrema setzen wir $f' = 0\colon$
$$
    f'=0 \,\Longrightarrow\, \frac{\frac{1}{\sqrt{3+2x}} -1}{\sqrt{3+2x} -x} = 0
    \,\Longrightarrow\, \frac{1}{\sqrt{3+2x}} -1 = 0
    \,\Longrightarrow\, 1 = \sqrt{3+2x}
$$$$
    \,\Longrightarrow\, 3+2x = 1
    \,\Longrightarrow\, x = -1
$$

Wir prüfen $x=-1$ auf die hinreicheinde Bedingung der Extremalstelle und bestimmten $f''$:
$$
    f'' = \left[\frac{\frac{1}{\sqrt{3+2x}}-1}{\sqrt{3+2x} -x}\right]'
$$$$
    = \frac{\left[\frac{1}{\sqrt{3+2x}} -1\right]'\cdot \left(\sqrt{3+2x}-x\right)
        - \left(\frac{1}{\sqrt{3+2x}-1}\right) \cdot \left[\sqrt{3+2x}-x\right]'}
        {\left(\sqrt{3+2x} -x\right)^2}
$$$$
    = \frac{-\dfrac{\sqrt{2x+3} -x}{\sqrt{(2x+3)^3}}
        - \left(\dfrac{1}{\sqrt{2x+3}}-1\right)^2}{\left(\sqrt{2x+3}-x\right)^2}
$$
Es gilt nun:
$$
    f''(-1) = \frac{-\dfrac{\sqrt{-2+3} +1}{\sqrt{(-2+3)^3}}
        - \left(\dfrac{1}{\sqrt{-2+3}}-1\right)^2}{\left(\sqrt{-2+3}+1\right)^2}
    = \frac{-\frac{1+1}{1} - 0^2}{4} = -\frac{1}{2} < 0
$$
Damit ist $x = -1$ eine strikte lokale Maximalstelle von $f$.

\newpage

\aufgabe{3}
\end{document}
