\documentclass[a4paper,graphics,11pt]{article}
\pagenumbering{arabic}

\usepackage[margin=1in]{geometry}
\usepackage[utf8]{inputenc}
\usepackage[T1]{fontenc}
\usepackage{lmodern}
\usepackage[ngerman]{babel}
\usepackage{amsmath, tabu}
\usepackage{amsthm}
\usepackage{amssymb}
\usepackage{complexity}
\usepackage{mathtools}
\usepackage{setspace}
\usepackage{graphicx,color,curves,epsf,float,rotating}
\usepackage{tasks}
\setlength{\parindent}{0em}
\setlength{\parskip}{1em}

\newcommand{\aufgabe}[1]{\subsection*{Aufgabe #1}}
\newcommand{\up}[2]{\mathrel{\overset{\makebox[0pt]{\mbox{\normalfont\tiny #2}}}{#1}}}

\begin{document}
\noindent Gruppe \fbox{\textbf{11}}             \hfill Tobias Riedel, 379133 \\
\noindent Analysis für Informatiker             \hfill Phil Pützstück, 377247 \\
\strut\hfill Kevin Holzmann, 371116\\
\strut\hfill Gurvinderjit Singh, 369227
\begin{center}
	\LARGE{\textbf{Hausaufgabe 11}}
\end{center}
\begin{center}
\rule[0.1ex]{\textwidth}{1pt}
\end{center}



\aufgabe{1}
In dieser Aufgabe steht ''Verkettung'' für Addition, Subtraktion, Multiplikation, Skalierung oder Komponierung
von Funktionen.
\textbf{a)}

\textbf{(i)}
Nach Skript ist jedes Polynom und $\ln x$ als Umkehrfunktion von $e^x$ differenzierbar und Verkettung dieser Funktionen erhält dies.
$|a-1|$ ist hier nur eine konstante, deswegen ist $f$ auf ihrem Definitionsbereich differenzierbar.\\
Mit VII 1.11 sowie der Produkt und Kettenregel folgt:
$$
    \frac{df}{dx} = \left[x^{\alpha} \cdot \ln\left(x+|\alpha-1|\right)\right]'
    \up{=}{P} \left[x^{\alpha}\right]'\cdot \ln(x+|\alpha-1|) + x^{\alpha} \cdot [\ln(x+|\alpha-1|)]'
$$$$
    \up{=}{*K} \alpha x^{\alpha-1}\cdot \ln(x+|\alpha-1|)
        + x^{\alpha} \cdot \left(\frac{[x+|\alpha-1|]'}{x+|\alpha-1|}\right)
$$$$
    = \alpha x^{\alpha-1}\cdot \ln(x+|\alpha-1|)
        + x^{\alpha} \cdot \frac{1}{x+|\alpha-1|}
$$



\textbf{(ii)}
Nach Skript ist jedes Polynom, $\sin x$, $\cos x$ auf $\mathbb{R}$ und jede rationale Funktion auf ihrem
Definitionsbereich differenzierbar. Verkettung dieser Funktionen erhält Differenzierbarkeit, also ist $g$
auf ihrem Definitionsbereich differenzierbar.\\
Mit Satz 1.7 und der Kettenregel folgt:
$$
    \frac{dg}{dx} = \left[\cos\left(\sin\left(x^2+\frac{1}{x}\right)\right)\right]'
    = \cos'\left(\sin\left(x^2+\frac{1}{x}\right)\right)\cdot \left[\sin\left(x^2+\frac{1}{x}\right)\right]'
$$$$
    = -\sin\left(\sin\left(x^2+\frac{1}{x}\right)\right) \cdot \left(\sin'\left(x^2+\frac{1}{x}\right)\cdot\left[x^2+\frac{1}{x}\right]'\right)
$$$$
    = -\sin\left(\sin\left(x^2+\frac{1}{x}\right)\right) \cdot \left(\cos\left(x^2+\frac{1}{x}\right)\cdot\left(\left[x^2\right]'+\left[\frac{1}{x}\right]'\right)\right)
$$$$
    = -\sin\left(\sin\left(x^2+\frac{1}{x}\right)\right) \cdot \cos\left(x^2+\frac{1}{x}\right)\cdot\left(2x-\frac{1}{x^2}\right)
$$
\newpage
\textbf{b)}

\textbf{(iii)}
Nach Skript ist jedes Polynom, $\sin x$, $\cos x$, $e^x$ auf $\mathbb{R}$ und jede rationale Funktion auf ihrem
Definitionsbereich differenzierbar. Ebenso ist die Umkehfunktion $f^{-1}$ einer differenzierbaren, stetigen und
injektiven Funktion $f$ ebenfalls differenzierbar in den Punkten $x$ wo $f'(x) \neq 0$ ist. Verkettung dieser
Funktionen erhält Differenzierbarkeit.\\
Es ist $\sqrt{x}$ die Umkehrfunktion des Polynoms $x^2$ und damit auch differenzierbar auch ihrem Definitionsbereich.
Analog dazu ist $\arctan$ die Umkehrfunktion von $\tan = \frac{\sin}{\cos}$ und damit auch differenzierbar
auf ihrem Definitionsbereich. Insgesamt ist also $h$ auf ihrem Definitionsbereich differenzierbar.\\
Mit Satz 1.7, 1.10 und der Ketten und Quotientenregel folgt:

$$
    \frac{dh}{dx} = \left[\cos\left(\frac{e^x+\sqrt{1+x^2}}{\arctan x}\right)\right]'
    = -\sin\left(\frac{e^x+\sqrt{1+x^2}}{\arctan x}\right) \cdot
        \left[\frac{e^x+\sqrt{1+x^2}}{\arctan x}\right]'
$$$$
    = -\sin\left(\frac{e^x+\sqrt{1+x^2}}{\arctan x}\right) \cdot
        \left(\frac{\left[e^x+\sqrt{1+x^2}\right]'\cdot (\arctan x) -
        (e^x+\sqrt{1+x^2}) \cdot [\arctan x]'}{(\arctan x)^2} \right)
$$$$
    = -\sin\left(\frac{e^x+\sqrt{1+x^2}}{\arctan x}\right) \cdot
        \left(\frac{\left(e^x+\left(\dfrac{2x}{2\sqrt{1+x^2}}\right)\right)\cdot (\arctan x) -
        (e^x+\sqrt{1+x^2}) \cdot \left(\dfrac{1}{\tan^{-1}(\arctan x)}\right)}{(\arctan x)^2} \right)
$$$$
    = -\sin\left(\frac{e^x+\sqrt{1+x^2}}{\arctan x}\right) \cdot
        \left(\frac{\left(e^x+\dfrac{x}{\sqrt{1+x^2}}\right)\cdot (\arctan x) -
        (e^x+\sqrt{1+x^2}) \cdot \left(\dfrac{1}{x^2+1}\right)}{(\arctan x)^2}
        \right)
$$

\textbf{(iv)}
Nach Skript ist $e^x$ auf $\mathbb{R}$ differenzierbar, also ist $i$ auf ihrem Definitionsbereich\\
differenzierbar.\\
Mit Satz 1.7 und der Kettenregel folgt:
$$
    \frac{di}{dx} = [\cosh x]'
    = \left[\frac{e^x+e^{-x}}{2}\right]'
    = \frac{1}{2} \left[e^x+e^{-x}\right]'
    = \frac{1}{2} \left([e^x]' + [e^{(-1)x}]'\right)
    = \frac{e^x-e^{-x}}{2}
    = \sinh x
$$


\textbf{c)}
\textbf{(v)}
Nach Skript ist jedes Polynom sowie $e^x$ und $\ln x$ als dessen Umkehrfunktion differenzierbar auf ihrem
Definitionsbereich. Also ist auch $j$ als Verkettung dieser differenzierbar.\\
Mit Satz 1.7, 1.11 und der Kettenregel folgt:
$$
    \frac{dj}{dx} = [x^x]' = \left[e^{x\ln(x)}\right]'
    = e^{x\ln(x)} \cdot [x\ln(x)]'
    = e^{x\ln(x)} \cdot ([x]' \cdot \ln(x) + [\ln(x)]' \cdot x)
$$$$
    = e^{x\ln(x)} \cdot (1 \cdot \ln(x) + \frac{1}{x} \cdot x)
    = e^{x\ln(x)} \cdot (\ln(x) + 1)
$$$$
    = x^x \cdot (\ln(x) + 1)
$$
\newpage
\textbf{(vi)}
Wie zuvor erwähnt sind alle Polynome, $e^x$, $\sin x$, $\cos x$ und auch $\ln x$ als Umkehrfunktion von $e^x$
sowie $\sqrt{x}$ als Umkehrfunktion von $x^2$ auf ihrem Definitionsbereich differenzierbar.
Damit ist $k$ als Verkettung dieser Funktionen ebenfalls auf ihrem Definitionsbereich differenzierbar.\\
Mit Satz 1.7, 1.10, 1.11 und der Produkt und Kettenregel folgt:
$$
    \frac{dk}{dx} = \left[\left(1 + \sqrt{x} + x^2\right)^{\sin x}\right]'
    = \left[\exp\left(\sin(x)\cdot \ln\left(1 + \sqrt{x} + x^2\right)\right) \right]'
$$$$
    = \exp\left(\sin(x)\cdot \ln\left(1 + \sqrt{x} + x^2\right)\right)
        \cdot \left[\sin(x)\cdot \ln\left(1 + \sqrt{x} + x^2\right)\right]'
$$$$
    = \left(1+\sqrt{x} + x^2\right)^{\sin x}
        \cdot \left[\sin(x)\cdot \ln\left(1 + \sqrt{x} + x^2\right)\right]'
$$$$
    = \left(1+\sqrt{x} + x^2\right)^{\sin x}
        \cdot \left([\sin(x)]'\cdot \ln\left(1 + \sqrt{x} + x^2\right)
        + \sin(x) \cdot \left[\ln\left(1+\sqrt{x} + x^2\right)\right]'\right)
$$$$
    = \left(1+\sqrt{x} + x^2\right)^{\sin x}
        \cdot \left(\cos(x)\cdot \ln\left(1 + \sqrt{x} + x^2\right)
        + \sin(x) \cdot \frac{[1+\sqrt{x}+x^2]'}{1+\sqrt{x} + x^2}\right)
$$$$
    = \left(1+\sqrt{x} + x^2\right)^{\sin x}
        \cdot \left(\cos(x)\cdot \ln\left(1 + \sqrt{x} + x^2\right)
        + \sin(x) \cdot \frac{\frac{1}{2\sqrt{x}}+2x}{1+\sqrt{x} + x^2}\right)
$$
\newpage

\aufgabe{2}
Der maximale reelle Def. Bereich ist $\left[-\frac{3}{2} , 3\right)$, da für $x < - \frac{3}{2}$ dann
$3 - 2x < 0$ währe und wir dann die Wurzel einer negativen Zahl ziehen würden.
Des weiteren gilt für $x \geq 3$,\\ dass $\sqrt{3+2x} - x \leq \sqrt{9} - 3= 0$, und da
der natürliche Logarithmus nur für $x > 0$ definiert ist, gibt dies eine Definitionslücke.
Insgesamt ist also $D_f = \left[-\frac{3}{2}, 3\right)$.

Für die Nullstellen setzen wir $f = 0$:
$$
    f = 0 \,\Longrightarrow\, \ln\left(\sqrt{3+2x}-x\right) = 0
    \,\Longrightarrow\, \exp\left(\ln\left(\sqrt{3+2x}-x\right)\right) = \exp(0)
$$$$
    \,\Longrightarrow\, \sqrt{3+2x} -x = 1 \,\Longrightarrow\, \sqrt{3+2x} = x+1
    \,\Longrightarrow\, 3+2x = x^2+2x+1
$$$$
    \,\Longrightarrow\, 2 = x^2 \,\Longrightarrow\, x = \pm\sqrt{2}
$$
Wir testen nun:
$$
    f(\sqrt{2}) = \ln\left(\sqrt{3+2\sqrt{2}}-\sqrt{2}\right)
    = \ln\left(\sqrt{(1+\sqrt{2})^2} -\sqrt{2}\right)
    = \ln(1) = 0
$$$$
    f(-\sqrt{2}) = \ln\left(\sqrt{3-2\sqrt{2}}+\sqrt{2}\right)
    = \ln\left(\sqrt{(1-\sqrt{2})^2} +\sqrt{2}\right)
    = \ln(2\sqrt{2}-1) \neq 0
$$
Also hat $f$ eine Nullstelle bei $\sqrt{2} \in D_f$, aber nicht bei $-\sqrt{2}$.

Für mögliche Extrema bestimmen wir zuerst $f'\colon$
Nach Satz 1.10 und wie in Aufgabe 1 bereits erwähnt sind $\ln x$, $\sqrt{x}$ und jedes Polynom auf ihrem
Definitionsbereich differenzierbar. $f$ ist eine Verkettung dieser Funktionen und erhält damit die Differenzierbarkeit.
$$
    \frac{df}{dx} = \left[\ln\left(\sqrt{3+2x}-x\right)\right]'
    = \frac{\left[\sqrt{3+2x}-x\right]'}{\sqrt{3+2x} -x}
    = \frac{\frac{2}{2\sqrt{3+2x}}-1}{\sqrt{3+2x} -x}
    = \frac{\frac{1}{\sqrt{3+2x}}-1}{\sqrt{3+2x}-x}
$$
Nun setzen wir $f' = 0\colon$
$$
    f'=0 \,\Longrightarrow\, \frac{\frac{1}{\sqrt{3+2x}} -1}{\sqrt{3+2x} -x} = 0
    \,\Longrightarrow\, \frac{1}{\sqrt{3+2x}} -1 = 0
    \,\Longrightarrow\, 1 = \sqrt{3+2x}
$$$$
    \,\Longrightarrow\, 3+2x = 1
    \,\Longrightarrow\, x = -1
$$
Damit ist $x = -1$ ein möglicher Kandidat für Extrema von $f$.

\newpage

\aufgabe{3}

\textbf{a)}
Da jedes Polynom und $e^x$ sowie deren Komponierung auf ganz $\mathbb{R}$ differenzierbar sind, ist auch $f$
differenzierbar.
$$
    f' = \frac{df}{dx} = \left[\exp(2x^4-x^2-1)\right]'
    = \exp(2x^4-x^2-1)\cdot \left[2x^4-x^2-1\right]'
$$$$
    = \exp(2x^4-x^2-1) \cdot (8x^3-2x)
$$
$f$ immernoch differenzierbar, da es wieder nur eine Verkettung von Polynomen und Exponentialfunktionen ist.
$$
    f'' = \frac{df'}{dx} = \left[\exp(2x^4-x^2-1)\cdot (8x^3-2x)\right]'
$$$$
    = [\exp(2x^4-x^2-1)]'\cdot(8x^3-2x) + \exp(2x^4-x^2-1)\cdot[8x^3-2x]'
$$$$
    = \left(\exp(2x^4-x^2-1) \cdot (8x^3-2x)\right) \cdot(8x^3-2x) + \exp(2x^4-x^2-1)\cdot(24x^2-2)
$$$$
    = \exp(2x^4-x^2-1)(64x^6-32x^4+28x^2-2)
$$

Wir setzen $f'=0$. Da nach (III 3.21) stets $\exp(x) \neq 0$ ist und $\mathbb{R}$ ein Körper,
also ein Integritätsbereich mit Nullteilerfreiheit ist, genügt es hier das Polynom
$8x^3-2x$ gleich 0 zu setzen. Weiterhin ist $8x^3-2x = x(8x^2-2)$ also ist $x=0$ eine
Nullstelle von $8x^3-2x$. Wir lösen also $8x^2-2$ mit der quadratischen Formel:
$$
    8x^2-2 = 0
    \,\Longrightarrow\, x = \frac{\pm\sqrt{-4\cdot8\cdot(-2)}}{2\cdot 8}
    = \frac{\pm \sqrt{64}}{16} = \frac{\pm8}{16} = \pm \frac{1}{2}
$$
Somit haben wir mögliche Extremalstellen $x_1 = 0, x_2 = \frac{1}{2}, x_3 = - \frac{1}{2}$.
Wir überprüfen dies durch einsetzen in $f''$. Wir verwenden wieder, dass stets $\exp(x) > 0$ (*):
$$
    f''(0) = \exp(-1)(-2) \up{<}{*} 0
$$$$
    f''\left(\frac{1}{2}\right)
    = \exp\left(\frac{1}{8}- \frac{1}{4} -1\right)\left(1- 2 + 7 - 2\right)
    = 4\exp\left(-\frac{9}{8}\right) \up{>}{*} 0
$$$$
     f''\left(-\frac{1}{2}\right)
    = \exp\left(\frac{1}{8}- \frac{1}{4} -1\right)\left(1- 2 + 7 - 2\right)
    = 4\exp\left(-\frac{9}{8}\right) \up{>}{*} 0
$$
Also gilt nach (VII 2.9), dass $x_1 = 0$ eine strikte lokale Maximalstelle ist,
und $x_2=\frac{1}{2}$ und $x_3 = - \frac{1}{2}$ strikte lokale Minimalstellen sind.
Weiterhin ist $e^x$ sowie $2x^4-x^2-1$ nach oben unberschränkt, daher kann es kein
globales Maximum geben. Durch $\lim_{x \to -\infty}\limits 2x^4-x^2-1 = \infty$ sowie
$\lim_{x \to \infty}\limits 2x^4-x^2-1 = \infty$ folgt auch $\lim_{x \to -\infty}\limits
f = \infty$ und $\lim_{x \to \infty}\limits f = \infty$. Da also $f$ keine weiteren
Extremalstellen besitzt und für $x\to \infty$ und $x\to -\infty$ bestimmt gegen $\infty$ 
divergiert, folgt, dass $x_2 = \frac{1}{2}$ und $x_3 = -\frac{1}{2}$
mit $f(x_3) = f(x_2) = \exp\left(\frac{1}{8}-\frac{1}{4} -1\right)
= \exp\left(-\frac{9}{8}\right)$ wegen der strengen Monotonie der Exponentialfunktion
(III 3.21) auch globale Minimalstellen von $f$ sind.

\newpage

\textbf{b)}
Nach Satz 1.10 und wie zuvor erwähnt ist $\ln x$ auf ihrem Definitionsbereich differenzierbar. Da also $g$ eine
Verkettung von Polynomen und dem natürlichen Logarithmus ist, ist auch $g$ differenzierbar.
$$
    g' = \frac{dg}{dx} = [3\ln(3x^2+1)]'
    = 3[\ln(3x^2+1)]'
    = 3\left(\frac{[3x^2+1]'}{3x^2+1}\right)
    = \frac{18x}{3x^2+1}
$$
Nach Skript ist jede rationale Funktion auf ihrem Definitionsbereich differnzierbar, also $g''$.
$$
    g'' = \frac{dg'}{dx} = \left[\frac{18x}{3x^2+1}\right]'
    = \frac{[18x]'(3x^2+1)-(18x)[3x^2+1]'}{(3x^2+1)^2}
    = \frac{54x^2-108x+18}{(3x^2+1)^2}
$$

Wir setzen also $g' = 0$:
$$
    g' = 0 \,\Longrightarrow\, \frac{18x}{3x^2+1} = 0
    \,\Longrightarrow\, 18x = 0 \,\Longrightarrow\, x_1 = 0
$$
Es ist $0 \in D_g = [-1, 1]$, also überprüfen wir durch Einsetzen in $g''$:
$$
    g''(0) = \frac{18}{1} = 18 > 0
$$
Damit ist nach (VII 2.9) $x_1=0$ eine lokale strikte Minimalstelle von $g$.
Wir überprüfen die Randwerte und den Wert von $g(0)$:
$$
    g(0) = 3\ln(1) = 0
$$$$
    g(-1) = 3\ln(4) > 0 \qquad\text{und}\qquad g(1) = 3\ln(4) >0
$$
Da $g$ keine weitern Extremalstellen besitzt, folgt, dass $x_1=0$ mit $g(x_1) = 0$ auch das
globale Minimum von $g$ darstellt.

\aufgabe{4}

\end{document}
